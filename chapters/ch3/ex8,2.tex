%\documentclass[a4paper, 14pt]{extreport}

%\usepackage{StyleMMF}

%\setcounter{chapter}{7}

%\begin{document}

%\chapter{Метод характеристик і формула Даламбера: нескінченна пряма, півнескінченна пряма та відрізок. Метод непарного продовження.}

\section[Задача №8.2]{8.2}

\textit{Зобразити графічно поле зміщень і поле швидкостей нескінченної струни в характерні послідовні моменти часу, якщо початкове відхилення (зміщення) дорівнює нулю, початкова швидкість всіх точок струни на деякому відрізку довжиною $2l$ однакова і дорівнює $\nu_0$, а в усіх інших точках дорівнює нулю. У який кінцевий стан переходить струна в результаті такого процесу? З точки зору механіки системи частинок результат є парадоксальним: у початковий момент тілу був переданий імпульс (у поперечному напрямі до струни), а в кінцевому стані струна нерухома, замість того щоб рухатись рівномірно. Зобразіть також вигляд поля зміщень і поля швидкостей при наближенні до границі: $l \to 0$, $\nu_0 \to \infty$ при фіксованому $t$, якщо переданий струні імпульс залишається сталим.}


%\end{document}