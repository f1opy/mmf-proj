%\documentclass[a4paper, 14pt]{extreport}

%\usepackage{StyleMMF}

%\setcounter{chapter}{7}

%\begin{document}

%\chapter{Метод характеристик і формула Даламбера: нескінченна пряма, півнескінченна пряма та відрізок. Метод непарного продовження.}

\textbf{\large Метод непарного продовження для півнескінченної та скінченної струни}

\section[Задача №8.3]{8.3}

\textit{Зобразити графічно поле зміщень півнескінченної струни у характерні послідовні моменти часу. Початкове відхилення має форму прямокутного трикутника, більшим катетом служить положення рівноваги струни, а вершина гострого кута орієнтована в бік кінця струни. Початкова швидкість дорівнює нулю. Кінець струни а) вільний (відносно поперечних зміщень), б) закріплений нерухомо.}


%\end{document}