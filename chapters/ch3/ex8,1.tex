%\documentclass[a4paper, 14pt]{extreport}

%\usepackage{StyleMMF}

%\setcounter{chapter}{7}

%\begin{document}

%\chapter{Метод характеристик і формула Даламбера: нескінченна пряма, півнескінченна пряма та відрізок. Метод непарного продовження.}

\textbf{\large Вільні коливання нескінченної струни.}

\section[Задача №8.1]{8.1}

\textit{Зобразити графічно поле зміщень і поле швидкостей нескінченної струни в характерні послідовні моменти часу, якщо початковий відхил (зміщення) має форму рівнобедреного трикутника висотою $h$ і основою $2L$, а початкова швидкість дорівнює нулю. Чи всі частини трикутника приходять у рух одразу? Відповідь поясніть.}


%\end{document}