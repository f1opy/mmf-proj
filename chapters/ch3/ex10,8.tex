%\documentclass[a4paper, 14pt]{extreport}
%
%\usepackage{../../main/StyleMMF}
%
%\setcounter{chapter}{9}
%
%\begin{document}
%
%\chapter{Приведення лінійних рівнянь у частинних похідних 2-го порядку з двома змінними до заданого вигляду}

\section[Задача №10.8]{10.8}

\textit{Привести рівняння $u_{tt} = v^2 \big(u_{rr} + (2/r) u_r\big) + cu$ до самоспряженого вигляду: \[\rho(r)u_{tt} = \frac{\partial\;}{\partial r} \left(k(r) \frac{\partial u}{\partial r}\right) - q(r)u.\]}

\begin{center}
    \large{\textbf{Розв'язок}}
\end{center}

Запишемо рівняння, розкриваючи дужки 
\begin{equation} \label{eq10,8}
    u_{tt} = v^2u_{rr} + \frac{2v^2}{r}u_r + cu
\end{equation}
Домножимо його на довільну функцію $\rho(r)$, яку ми визначимо далі
\begin{equation*}
    \rho(r)u_{tt} = v^2\rho(r)u_{rr} + \frac{2v^2\rho(r)}{r}u_r + c\rho(r)u
\end{equation*}

Порівнюючи його з рівнянням у самоспряженому вигляді, позначимо \[k(r) = v^2\rho(r),\quad k'(r) = \frac{2v^2\rho(r)}{r},\quad q(r) = - c\rho(r).\] Ми отримали вирази для функції $k(r)$ та її похідної. Звідси і знайдемо рівняння для $\rho(r)$: диференцюємо вираз для $k(r)$ і прирівнюємо до $k'(r)$.
\begin{equation}
    \big(v^2\rho(r)\big)' = \frac{2v^2\rho(r)}{r}
\end{equation}
Розв'яжемо отримане рівняння
\begin{equation*}
    \rho'(r) = \frac{2\rho(r)}{r}
    \;\Rightarrow\;
    \frac{\mathrm{d}\rho}{\rho} = \frac{2\mathrm{d}r}{r}
    \;\Rightarrow\;
    \ln\rho = 2\ln r + \ln K = \ln(Kr^2)
    \;\Rightarrow\;
    \rho(r) = Kr^2
\end{equation*}
Оскільки на $\rho(r)$ ми домножили все рівняння і враховуючи, що рівняння є однорідним, то можна покласти $K = 1$.

Маємо вихідне рівняння (\ref{eq10,8}) в самоспряженому вигляді:
\begin{equation}
    r^2u_{tt} = \frac{\partial\;}{\partial r} \left(v^2r^2 \frac{\partial u}{\partial r}\right) + cr^2u
\end{equation}

%\end{document}