%\documentclass[a4paper, 14pt]{extreport}

%\usepackage{StyleMMF}

%\setcounter{chapter}{9}

%\begin{document}

%\chapter{Приведення лінійних рівнянь у частинних похідних 2-го порядку з двома змінними до заданого вигляду}

\section[Задача №10.1]{10.1}

\textit{Визначити тип рівняння $u_{xx} + 4u_{xy} + cu_{yy} + u_x = 0$, привести його до канонічного вигляду для $c = 0$ і знайти загальний розв’язок.}

\begin{center}
    \large{\textbf{Розв'язок}}
\end{center}

Загальний вид рівняння:
\begin{equation}
    a_{11}u_{xx} + 2a_{12}u_{xy} + a_{22}u_{yy} + b_1u_x + b_2u_y + cu = 0, \quad \text{або} \quad \hat{L}u + cu = 0
\end{equation}
Тип рівняння визначається визначником матриці, яка складається з коефіцієнтів перед другими похідними. \textit{Фактично оператор $\hat{L}$ є білінійною формою з лінійної алгебри, де замість змінних будуть похідні.}
\begin{equation}
    \Delta = -
    \begin{vmatrix}
        a_{11} & a_{12}\\
        a_{12} & a_{22}
    \end{vmatrix} 
    = a_{12}^2 - a_{11}a_{22} = 2^2 - 1\cdot c = 4 - c 
\end{equation}
При $c = 0$ визначник $\Delta > 0$, тому маємо рівняння гіперболічного типу. 

Суть канонізації -- перейти до нових змінних для яких рівняння прийматиме канонічний вид. Для визначення таких змінних записуємо  спочатку характеристичне рівняння:
\begin{equation}
    a_{11} (\mathrm{d}y)^2 + 2a_{12} \mathrm{d}x\mathrm{d}y + a_{22} (\mathrm{d}x)^2 = 0, \quad \text{або} \quad \frac{\mathrm{d}y}{\mathrm{d}x} = \frac{a_{12} \pm \sqrt{a_{12}^2 - a_{11}a_{22}}}{a_{12}}
\end{equation}
Обидва рівняння приводять до 
\begin{equation}
    y'_1 = 4, \qquad y'_2 = 0.
\end{equation}
Звідси маємо перші інтеграли
\begin{equation}
    \begin{gathered}
        y'_1 = 4 
        \quad\Rightarrow\quad
        y_1 = 4x
        \quad\Rightarrow\quad
        \Phi(x,y) = y - 4x = C_1\\
        y'_2 = 0 
        \quad\Rightarrow\quad
        \Psi(x,y) = y = C_2
    \end{gathered}
\end{equation}
З теорії нові змінні отримаємо формальною заміною $C_1 \to \xi$, $C_2 \to \eta$. Отже, нові змінні
\begin{equation}
    \left\{ \begin{aligned}
        \xi = y - 4x,\\
        \eta = y.
    \end{aligned} \right.
\end{equation}

Далі треба зробити заміну змінних. Для цього окремо випишемо похідні від нових змінних
\begin{equation*}
    \xi_x = -4,\, \xi_y = 1,\, \eta_x = 0,\, \eta_y = 1,\, \xi_{xy} = \eta_{xy} = 0,\, \xi_{xx} = \eta_{xx} = 0,\, \xi_{yy} = \eta_{yy} = 0.
\end{equation*}  

Тепер не важко виконати заміну змінних 
\begin{equation*}
    \begin{gathered}
        u_x = u_\xi \xi_x + u_\eta \eta_x = -4 u_\xi,\\
        u_y = u_\xi \xi_y + u_\eta \eta_y = u_\xi + u_\eta,\\
        u_{xx} = (-4 u_\xi)'_x = -4(u_{\xi\xi} \xi_x + u_{\eta\eta} \eta_x) = 16 u_{\xi\xi},\\
        u_{xy} = (-4 u_\xi)'_y = -4(u_{\xi\xi} \xi_y + u_{\eta\eta} \eta_y) = -4(u_{\xi\xi} + u_{\xi\eta}).
    \end{gathered}
\end{equation*}
Підставляємо отримані вирази в рівняння 
\begin{equation*}
    u_{xx} + 4u_{xy} + u_x = 16 u_{\xi\xi} - 16(u_{\xi\xi} + u_{\xi\eta}) - 4u_\xi = 0
    \quad\Rightarrow\quad
    u_{\xi\eta} + \frac{1}{4}u_\xi = 0
\end{equation*}
Отже, отримали рівняння в канонічному виді
\begin{equation}
    u_{\xi\eta} + \frac{1}{4}u_\xi = 0
\end{equation}

Розв'яжемо отримане рівняння. Легко побачити, що по $\xi$ можна проінтегрувати. 
\begin{equation*}
    u_{\xi\eta} + \frac{1}{4}u_\xi = 0
    \quad\Rightarrow\quad
    \left(u_\eta + \frac{1}{4}u\right)'_\xi = 0
    \quad\Rightarrow\quad
    u_\eta + \frac{1}{4}u = f(\eta)
\end{equation*}
Звідки ми отримали лінійне неоднорідне диференційне рівняння однієї змінної. Розв'яжемо спочатку однорідне рівняння
\begin{equation*}
    \tilde{u}_\eta + \frac{1}{4}\tilde{u} = 0
    \quad\Rightarrow\quad
    \ln\tilde{u} = -\frac{1}{4}\eta + \ln C 
    \quad\Rightarrow\quad
    \tilde{u} = Ce^{-\eta/4} 
\end{equation*}
Варіюєму змінну $C \to C(\eta)$
\begin{equation*}
    u = C(\eta)e^{-\eta/4} 
    \quad\Rightarrow\quad
    C'e^{-\eta/4} = f(\eta)
    \quad\Rightarrow\quad
    C(\eta) = \int f(\eta) e^{\eta/4} \;\mathrm{d}\eta + \gamma
\end{equation*}

Отже, маємо розв'язок рівняння
\begin{equation}
    u(\xi,\eta) = \gamma e^{-\eta/4} + e^{-\eta/4} \cdot \int^\eta f(z) e^{z/4} \;\mathrm{d}z
\end{equation} 


%\end{document}