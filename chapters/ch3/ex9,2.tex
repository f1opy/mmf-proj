%\documentclass[a4paper, 14pt]{extreport}

%\usepackage{StyleMMF}

%\setcounter{chapter}{8}

%\begin{document}

%\chapter{Використання загального розв’язку хвильового рівняння у вигляді суперпозиції зустрічних хвиль. Нестаціонарна задача розсіяння.}

\section[Задача №9.2]{9.2}

\textit{При $t < t_0$ по півнескінченній струні $0 \geq x < \infty$ у напрямі її кінця поширюється хвиля заданої форми (падаючий «імпульс»), причому передній фронт хвилі при $t \geq t_0$ не досягає кінця струни. Знайти коливання струни при $t > t_0$ і форму відбитого імпульсу для скінченного $t_0$ і $t_0 \to -\infty$. Кінець струни: а) закріплений жорстко; б) зазнає дії сили тертя, пропорційної швидкості. Як пояснити відсутність відбивання при певному значенні коефіцієнта тертя?\\
Указівка: звести до задачі про поширення межового режиму типу задачі 9.1, використати вказівку до цієї задачі та умову, що при $t < t_0$ фронт хвилі не досягає кінця струни.}

\begin{center}
    \large{\textbf{Розв'язок}}
\end{center}

\noindent Формальна постановка задачі:
\begin{equation} \label{withered}
    \left\{ \begin{aligned} 
            \;&u = u(x,t), \\
            &u_{tt} = v^2 u_{xx}, \\
            &0 \leq x \leq \infty, t \geq t_0 \\
            &\text{а)}\; u(0,t) = 0,\\
            &\text{б)}\; \mu u_x(0,t) = u_t(0,t),\\
            &u(x,t) = F_0(vt + x) - \text{падаюча хвиля, для  } t<t_0\\
            &F(t) = 0,\;t < t_0 \text{  - падаюча хвиля не досягне кінця струни до }t_0.\\
    \end{aligned} \right.
\end{equation}
Де $\mu$ - коефіцієнт тертя. 

Повний розв'язок хвильового рівняння представляється комбінацією двох збурень, що поширюються у протилежних напрямках та є фукнціями однієї змінної. У нашому випадку це падаюча та відбита хвиля у відповідному порядку:

\begin{equation}
    u(x,t) = u_{\text{пад}}(x,t) + u_{\text{від}}(x,t) = F(x + tv) + f(x - tv)
\end{equation}

З постановки (\ref{withered}) ми знаємо частину, що відповідає падаючій хвилі:
\begin{equation*}
    u_{\text{пад}}(x,t) = F_0(x + tv)
\end{equation*}

Звідси маємо умову на $u_{\text{від}}$ при $t<t_0$:

\begin{equation} \label{hunger}
    u_{\text{від}}(x,t)=0
\end{equation}

Для випадку \textit{а)} можна було б cкористатися доведенною в лекціях вимогою на непарність $u(x,t)$ по змінній $x$ для приведенної граничної умови. Але розв'яжемо обидва випадки одним методом. Розглянемо точку $x=0$ у момент часу $t>t_0$:

\begin{equation}  \label{slave}
    \begin{aligned} 
            &\text{а)}\;u_{\text{від}}(0,t) + u_{\text{пад}}(0,t) = 0 \quad\Rightarrow\\
            &\Rightarrow\quad u_{\text{від}}(0,t) = f(x - tv)|_{x=0} = f(-tv) = -F_0(tv), \\
            & \\
            &\text{б)}\; \mu u_x(0,t) - u_t(0,t) = 0 \quad\Rightarrow\\
            &\Rightarrow\quad \mu \frac{\partial u_{\text{від}}}{\partial x}\bigg|_{x=0} +\frac{\partial u_{\text{від}}}{\partial t}\bigg|_{x=0} = (\mu + v)f'(-tv) =\\
            &= -(\mu - v)F_0'(tv)
    \end{aligned} 
\end{equation}

З отриманих рівнянь (\ref{slave}) та умови (\ref{hunger}) можна одразу отримати відповіді:

\begin{equation} 
    \begin{aligned} 
            &\text{а)}\;u_{\text{від}}(x,t) =
                \begin{cases}
                    0 & \text{, якщо } t < t_0, \text{ або } x > vt \\
                    -F_0(tv - x) & \text{, якщо } t > t_0,\;x < vt.
                \end{cases}
            & \\
            & \\
            &\text{б)}\;u_{\text{від}}(x,t)=
                \begin{cases}
                    0 & \text{, якщо } t < t_0, \text{ або } x > vt \\
                    \frac{\mu - v}{\mu + v}(F_0(tv - x) + c) & \text{, якщо } t > t_0,\;x < vt.
                \end{cases}
    \end{aligned} 
\end{equation}

Де \textit{c} - константа інтегрування. Накладаючи вимогу неперервності $u_{\text{від}}(0,t_0) = 0$ маємо $c = -F_0(0)$. Якщо $F(\xi)$ є неперервною функцією, то $c = 0$. Ці результати можна переписати у більш винтонченній формі за допомогою тета-функції:

\begin{equation} 
    \begin{aligned} 
            &\text{а)}\;u_{\text{від}}(x,t)=-F_0(tv-x) \Theta\big(v(t - t_0) - x\big)\\
            & \\
            &\text{б)}\;u_{\text{від}}(x,t)=\frac{\mu - v}{\mu + v} \big(F_0(tv-x)-F_0(0)\big) \Theta\big(v(t - t_0) - x\big).
    \end{aligned} 
\end{equation}

При $t_0 \rightarrow - \infty, \quad \Theta\big(v(t - t_0) - x\big) = 1$, тож поле матиме форму: 

\begin{equation} 
    \begin{aligned} 
            &\text{а)}\;u_{\text{від}}(x,t)=-F_0(tv - x)\\
            & \\
            &\text{б)}\;u_{\text{від}}(x,t) = \frac{\mu - v}{\mu + v} (F_0(tv - x) - F_0(0)).
    \end{aligned} 
\end{equation}


%\end{document}