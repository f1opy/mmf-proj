%\documentclass[a4paper, 14pt]{extreport}

%\usepackage{StyleMMF}

%\setcounter{chapter}{8}

%\begin{document}

%\chapter{Використання загального розв’язку хвильового рівняння у вигляді суперпозиції зустрічних хвиль. Нестаціонарна задача розсіяння.}

\section[Задача №9.2]{9.2}

\textit{При $t < t_0$ по півнескінченній струні $0 \geq x < \infty$ у напрямі її кінця поширюється хвиля заданої форми (падаючий «імпульс»), причому передній фронт хвилі при $t \geq t_0$ не досягає кінця струни. Знайти коливання струни при $t > t_0$ і форму відбитого імпульсу для скінченного $t_0$ і $t_0 \to -\infty$. Кінець струни: а) закріплений жорстко; б) зазнає дії сили тертя, пропорційної швидкості. Як пояснити відсутність відбивання при певному значенні коефіцієнта тертя?\\
Указівка: звести до задачі про поширення межового режиму типу задачі 9.1, використати вказівку до цієї задачі та умову, що при $t < t_0$ фронт хвилі не досягає кінця струни.}


%\end{document}