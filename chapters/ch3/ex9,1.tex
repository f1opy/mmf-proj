%\documentclass[a4paper, 14pt]{extreport}

%\usepackage{../StyleMMF}

%\setcounter{chapter}{8}

%\begin{document}

%\chapter{Використання загального розв’язку хвильового рівняння у вигляді суперпозиції зустрічних хвиль. Нестаціонарна задача розсіяння.}

\section[Задача №9.1]{9.1}

\textit{Півнескінченна струна (сила натягу $T_0$, швидкість хвиль $v$) з вільним кінцем перебувала у стані рівноваги. Починаючи з моменту часу $t=0$, на її кінець діє у поперечному напрямі задана сила $F(t)$. Знайти розв’язок задачі про вимушені коливання струни у квадратурах, а також знайти поле зміщень у явному вигляді і зобразити графічно форму струни, якщо: а) $F(t) = F_0$, б) $F(t) = F_0 \cos\omega t$, в) $F(t) = F_0 \sin\omega t$\\
Задача є прикладом так званої задачі про поширення межового режиму: задачі для півнескінченної струни з неоднорідною межовою умовою. Указівка: задача відшукання форми хвилі, створеної таким джерелом, зводиться до диференціального рівняння першого порядку; проблема знаходження сталої інтегрування вирішується, якщо врахувати умову неперервності хвильового поля на передньому фронті хвилі, тобто на межі областей $x > vt$ й $x < vt$.}

\begin{center}
    \large{\textbf{Розв'язок}}
\end{center}

\noindent Формальна постановка задачі:
\begin{equation} %\label{probcond12}
    \left\{ \begin{aligned} %%
            \;&u = u(x,t), \\
            &u_{tt} = v^2 u_{xx}, \\
            &0 \leq x < \infty, t \geq 0 \\
            &u_x(0,t) = F(t)/\beta = f(t),\\
            &u(x,0) = 0, \, u_t(x,0) = 0.
    \end{aligned} \right.
\end{equation}

Шукаємо розв'язок у вигляді:
\begin{equation}
    u(x,t) = g(t - x/v) + h(t + x/v)
\end{equation}

Із початкових умов маємо систему:
\begin{equation}
    \left\{ \begin{aligned}
        \;&u(x,0) = g(-x/v) + h(x/v) = 0,\\
        &u_t(x,0) = g'(-x/v) + h'(x/v) = 0.
\end{aligned} \right.
\end{equation}
Інтегруємо друге рівняння та розв'язуємо лінійну систему
\begin{equation*}
    \begin{gathered}
        \left\{ \begin{aligned}
            \;&g(-\xi) + h(\xi) = 0,\\
            &\int g'(-\xi)\;\mathrm{d}\xi + \int h'(\xi)\;\mathrm{d}\xi = 0;
        \end{aligned} \right.
        \quad\Rightarrow\quad
        \left\{ \begin{aligned}
            \;&g(-\xi) + h(\xi) = 0,\\
            &h(\xi) + C_2 - g(-\xi) + C_1 = 0;
        \end{aligned} \right.
        \quad\Rightarrow\\
        \Rightarrow\quad
        \left\{ \begin{aligned}
            \;&g(-\xi) + h(\xi) = 0,\\
            &-g(-\xi) + h(\xi) = 2\widetilde{C};
        \end{aligned} \right.
        \quad\Rightarrow\quad
        \left\{ \begin{aligned}
            \;&g(-\xi) = -\widetilde{C},\\
            &h(\xi) = \widetilde{C}.
        \end{aligned} \right.
    \end{gathered}
\end{equation*}
Обираємо константу інтегрування $\widetilde{C}$ рівною нулю, тоді 
\begin{equation*}
    \left\{ \begin{aligned}
        \;&g(-x/v) = 0,\\
        &h(x/v) = 0;
    \end{aligned} \right.
    \quad\Rightarrow\quad
    \left\{ \begin{aligned}
        \;&g(\xi) = 0, \text{ при } \xi < 0,\\
        &h(\eta)= 0, \text{ при } \eta > 0.
    \end{aligned} \right.
\end{equation*}

Отже, $h(t + x/v) = 0$ в нашій задачі, адже $x \geq 0$ та $t > 0$, що фізично означає відсутність хвилі, яка поширюється з нескінченность до краю струни (падаючої хвилі).\\
Маємо розв'язок у виді біжучої хвилі, яка створюється межовою умовою.
\begin{equation}
    u(x,t) = g(t - x/v)
\end{equation}

З межової умови визначимо розв'язок
\begin{equation}
    u_x(0,t) = f(t)
    \quad\Rightarrow\quad
    -\frac{1}{v}g'(t) = f(t)
    \quad\Rightarrow\quad
    g(t) = - v \int\limits_0^t f(\tau) \;\mathrm{d}\tau
\end{equation}

Загальний вид розв'язку
\begin{equation}
    u(x,t) = - v \int\limits_0^{t-x/v} f(\tau) \;\mathrm{d}\tau
\end{equation}

Обчислимо розв'язки для визначених межових умов:
\begin{enumerate}
    \item[\text{а})] \[u(x,t) = - \frac{vF_0}{\beta} \int\limits_0^{t-x/v}  \;\mathrm{d}\tau = -\frac{vF_0}{\beta} \left(t - \frac{x}{v}\right),\]
    \item[\text{б})] \[u(x,t) = - \frac{vF_0}{\beta} \int\limits_0^{t-x/v} \sin\omega\tau \;\mathrm{d}\tau = -\frac{vF_0}{\omega\beta} \cos\omega(t - x/v),\]
    \item[\text{в})] \[u(x,t) = - \frac{vF_0}{\beta} \int\limits_0^{t-x/v} \cos\omega\tau \;\mathrm{d}\tau = \frac{vF_0}{\omega\beta} \sin\omega(t - x/v).\]
\end{enumerate} 

%\end{document}