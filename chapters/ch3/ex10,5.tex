%\documentclass[a4paper, 14pt]{extreport}
%
%\usepackage{../../main/StyleMMF}
%
%\setcounter{chapter}{9}
%
%\begin{document}
%
%\chapter{Приведення лінійних рівнянь у частинних похідних 2-го порядку з двома змінними до заданого вигляду}

\section[Задача №10.5]{10.5}

\textit{Привести до простішого вигляду рівняння $u_t = a^2(u_{xx} + \alpha u_x) + cu$.}

\begin{center}
    \large{\textbf{Розв'язок}}
\end{center}

Перенесемо всі доданки на одну сторону та поділимо на $a^2$
\begin{equation} \label{eq10,5}
    u_{xx} + \alpha u_x - a^{-2}u_t + a^{-2}cu = 0
\end{equation}
Маємо рівняння параболічного типу в канонічному виді.\\
Спростимо його позбавившись якнайбільше від похідних першого порядку. Це зробимо використовуючи наступну заміну змінних та функції
\begin{equation}
    u(x,t) =  e^{\lambda x + \mu t} v(x,t)
\end{equation}

Обчислимо перші похідні та другу похідну по просторовый змінній
\begin{equation*}
    u_x =  e^{\lambda x + \mu t} (v_x + \lambda v), \; u_t =  e^{\lambda x + \mu t} (v_t + \mu v), \; u_{xx} =  e^{\lambda x + \mu t} (v_{xx} + 2\lambda v_x + \lambda^2 v)
\end{equation*}
Підставляємо їх в рівняння (\ref{eq10,5}) та ділимо його на експоненту
\begin{equation*}
    v_{xx} + 2\lambda v_x + \lambda^2 v + \alpha (v_x + \lambda v) - a^{-2}(v_t + \mu v) + a^{-2}cv = 0
\end{equation*}
Зводимо подібні доданки 
\begin{equation}
    v_{xx} + (2\lambda + \alpha)v_x - a^{-2}v_t + (\lambda^2 + \alpha \lambda - a^{-2}\mu  + a^{-2}c)v = 0
\end{equation}
Звідси визначимо $\lambda$ та $\mu$, прирівнючи вирази перед $v_x$ та $v$ до нуля.
\begin{equation}
    \left\{ \begin{aligned}
        \;& 2\lambda + \alpha = 0,\\
          & \lambda^2 + \alpha \lambda - a^{-2}\mu  + a^{-2}c = 0;
    \end{aligned} \right.
    \quad\Rightarrow\quad
    \left\{ \begin{aligned}
        \;& \lambda = -\alpha/2,\\
          & \mu = c - a^2\alpha^2/4;
    \end{aligned} \right.
\end{equation}
Таким чином, заміна
\begin{equation}
    u(x,t) =  \exp\bigg[\bigg(c - \frac{a^2\alpha^2}{4}\bigg)t - \frac{\alpha t}{2}\bigg] v(x,t)
\end{equation}
спрощує вихідне рівняння (\ref{eq10,5}) до вигляду
\begin{equation}
    v_t = a^2v_{xx}
\end{equation}

%\end{document}