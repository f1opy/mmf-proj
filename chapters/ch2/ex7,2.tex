%\documentclass[a4paper, 14pt]{extreport}

%\usepackage{StyleMMF}
%\usepackage{bookmark}

%\begin{document}

%\setcounter{chapter}{6}
%\chapter{Задачі з неоднорідними межовими умовами загального вигляду}

\section[Задача №7.2]{7.2}

\textit{Розв’язати задачу №7.1 методом розкладання по власних функціях.}

\begin{center}
    \large{\textbf{Розв'язок}}
\end{center}

\noindent Формальна постановка задачі:
\begin{equation} \label{cond7,2}
    \left\{ \begin{aligned} 
            \;&u = u(x,t), \\
            &u_{tt} = v^2 u_{xx}, \\
            &0 \leq x \leq l, t \geq 0 \\
            &u_x(0,t) = \frac{F_0}{\beta} e^{-\alpha t} = f_0 e^{-\alpha t},\\
            &u(l,t) = 0, \\
            &u(x,0) = 0, \, u_t(x,0) = 0.
    \end{aligned} \right.
\end{equation}

Розв'язок шукаємо у вигляді:
\begin{equation}
    u(x,t) = \sum\limits_{n = 0}^\infty T_n(t) X_n(x),
\end{equation}
де $X_n(x)$ -- власні функції задачі. Їх визначаємо, розв'язуючи задачу Штурма-Ліувілля з однорідними межовими умовами.
\begin{equation*}
    \begin{gathered}
        \left\{ \begin{aligned} 
            \;&u_{tt} = v^2 u_{xx}, \\
            &u_x(0,t) = 0, \, u(l,t) = 0.
        \end{aligned} \right.
        \;\Rightarrow\;
        \left\{ \begin{aligned} 
            \;&X_m'' + k_m^2 X_m = 0, \\
            &X_m'(0) = 0, \, X_m(l) = 0.
        \end{aligned} \right.
        \;\Rightarrow\\
        \vspace{10pt}
        \Rightarrow\;
        \left\{ \begin{aligned} 
            \;&k_m = \frac{\pi}{l}(m + 1/2) \text{ -- хвильві числа,} \\
            &X_m(x) = \cos k_mx,\, m = 0,1,2\ldots
        \end{aligned} \right.
    \end{gathered}
\end{equation*}

Розкладемо рівняння по власним функціям. Для цього домножимо йього на $X_m / \left\lVert X_m\right\rVert^2$ та проінтегруємо по $x$.
\begin{equation} \label{series-exp7,2}
    \frac{1}{\left\lVert X_m\right\rVert^2} \int\limits_0^l u_{tt} X_m \;\mathrm{d}x = \frac{v^2}{\left\lVert X_m\right\rVert^2} \int\limits_0^l u_{xx} X_m \;\mathrm{d}x
\end{equation}
Випишемо окремо перетворення для лівої та правої частини рівняня. Почнемо з лівої
\begin{equation*}
    \frac{1}{\left\lVert X_m\right\rVert^2} \int\limits_0^l u_{tt} X_m \;\mathrm{d}x =  \sum\limits_{n=0}^\infty \frac{T_n''(t)}{\left\lVert X_m\right\rVert^2} \int\limits_0^l X_n(x) X_m(x) \;\mathrm{d}x = \sum\limits_{n=0}^\infty T_n''(t) \delta_{n,m} = T_m''(t)
\end{equation*}

Тепер розглянемо праву частину рівняння, двічі використаємо інтегрування частинами
\begin{equation*}
    \frac{1}{\left\lVert X_m\right\rVert^2} \int\limits_0^l u_{xx} X_m \;\mathrm{d}x = \sum\limits_{n=0}^\infty \frac{T_n(t)}{\left\lVert X_m\right\rVert^2} \int\limits_0^l X_n''(x) X_m(x) \;\mathrm{d}x \  \textcolor{red}{=}
\end{equation*}
Випишемо окремо інтегрування під сумою
\begin{equation*}
    \begin{gathered}
        \int\limits_0^l X_n''(x) X_m(x) \;\mathrm{d}x = X_n'(x) X_m(x) \bigg|_0^l - \int\limits_0^l X_n'(x) X_m'(x) \;\mathrm{d}x = \\
        = X_n'(x) X_m(x) \bigg|_0^l - X_n(x) X_m'(x) \bigg|_0^l + \int\limits_0^l X_n(x) X_m''(x) \;\mathrm{d}x
    \end{gathered}
\end{equation*}
Скористаємося рівнянням для $X_n(x)$:
\begin{equation*}
    \begin{gathered}
        X_n'(x) X_m(x) \bigg|_0^l - X_n(x) X_m'(x) \bigg|_0^l - k_n^2 \int\limits_0^l X_n(x) X_m(x) \;\mathrm{d}x =\\
        = X_n'(x) X_m(x) \bigg|_0^l - X_n(x) X_m'(x) \bigg|_0^l - k_n^2 \left\lVert X_m\right\rVert^2 \delta_{n,m}
    \end{gathered}
\end{equation*}
І повернемося до початкового виразу та запишемо його через зміщення $u(x,t)$
\begin{equation*}
    \begin{gathered}
        \textcolor{red}{=}\ \sum\limits_{n=0}^\infty \frac{T_n(t)}{\left\lVert X_m\right\rVert^2} \bigg[X_n'(x) X_m(x) \bigg|_0^l - X_n(x) X_m'(x) \bigg|_0^l - k_n^2 \left\lVert X_m\right\rVert^2 \delta_{n,m}\bigg] =\\
        = - k_m^2 T_m(t) + \frac{1}{\left\lVert X_m\right\rVert^2} \left[ u_x(x,t) X_m(x) \bigg|_0^l - u(x,t) X_m'(x) \bigg|_0^l \right] \ \textcolor{red}{=}
    \end{gathered}
\end{equation*}
Скористаємося межовими умовами початкової задачі та задачі Штурма-Ліувілля для власних функцій системи при обчисленні значень отриманих виразів після інтегрування
\begin{equation*}
    \begin{gathered}
        \textcolor{red}{= } -k_n^2 T_m(t) + \frac{2}{l} \bigg[ u_x(l,t) X_m(l) - u_x(0,t) X_m(0) -\\
        - u(x,l) X_m'(l) + u(x,0) X_m'(0) \bigg] = - k_n^2 T_m(t) - \frac{2 f_0}{l} e^{-\alpha t},
    \end{gathered}
\end{equation*}
тут для $X_m(x)$ дивимось на межові умови задачі Штурма-Ліувілля, а для $u(x,t)$ -- межові умови початкової задачі.

Повертаємося до розкладу рівняння (\ref{series-exp7,2}), збираючи разом результати обчислень для лівої та правої частинами
\begin{equation}
    T_m''(t) = - k_m^2v^2 T_m(t)  - \frac{2v^2 f_0}{l} e^{-\alpha t}, \quad \text{або} \quad T_n''(t) + \omega_n^2 T_n(t) = - \kappa_n \omega_n^2  e^{-\alpha t}
\end{equation}

Отримали лінійне неожнорідне рівняння для $T_n(t)$, його розв'язок шукаємо у вигляді
\begin{equation}
    T_n(t) = A_n \cos\omega_nt + B_n \sin\omega_nt + \gamma e^{-\alpha t},
\end{equation}
а $\gamma$, коефіцієнт частинного розв'язку, визначаємо підстановкою в рівняння
\begin{equation*}
    \widetilde{T}_n(t) = \gamma e^{-\alpha t} 
    \quad\Rightarrow\;
    \alpha^2\gamma + \omega_n^2\gamma = - \kappa_n \omega_n^2
    \;\Rightarrow\;
    \gamma = - \frac{\kappa_n \omega_n^2}{\omega_n^2 + \alpha^2}
\end{equation*} 
Маємо
\begin{equation}
    T_n(t) = A_n \cos\omega_nt + B_n \sin\omega_nt - \frac{\kappa_n \omega_n^2}{\omega_n^2 + \alpha^2} e^{-\alpha t}
\end{equation}

Залишається визначити коефіцієнти $A_n$ та $B_n$. Оскільки, обидві умови однорідні можна одразу записати:
\begin{equation*}
    \left\{ \begin{aligned}
        \;&T_n(0) = 0,\\
        &T_n'(0) = 0; 
    \end{aligned} \right.
    \quad\Rightarrow\;
    \left\{ \begin{aligned}
        \;&T_n(0) = A_n - \frac{\kappa_n \omega_n^2}{\omega_n^2 + \alpha^2} = 0,\\
        &T_n'(0) = B_n\omega_n + \frac{\alpha \kappa_n \omega_n^2}{\omega_n^2 + \alpha^2} = 0; 
    \end{aligned} \right.
    \;\Rightarrow\;
    \left\{ \begin{aligned}
        \;&A_n = \frac{\kappa_n \omega_n^2}{\omega_n^2 + \alpha^2},\\
        &B_n = - \frac{\kappa_n \alpha \omega_n}{\omega_n^2 + \alpha^2}. 
    \end{aligned} \right.
\end{equation*}

Наш остаточний розв'язок 
\begin{equation}
    u(x,t) = \sum\limits_0^l \frac{\kappa_n \cos k_nx}{\omega_n^2 + \alpha^2} \left(\omega_n^2 \cos\omega_nt - \alpha \omega_n \sin\omega_nt - \omega_n^2 e^{-\alpha t} \right)
\end{equation}

%\end{document}