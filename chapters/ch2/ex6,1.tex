%\documentclass[a4paper, 14pt]{extreport}

%\usepackage{StyleMMF}

%\begin{document}

%\setcounter{chapter}{5}

%\chapter{Задачі з неоднорідним рівнянням або неоднорідними межовими умовами}

\textbf{\large Джерела з гармонічною залежністю від часу.}

\section[Задача №6.1]{6.1}

\textit{Знайти коливання струни $0 \leq x \leq l$, лівий кінець якої закріплений, а правий вільний, при $t > 0$ під дією розподіленої сили $f(x,t) = f(x)\cos\omega t$. При $t < 0$ струна перебувала в положенні рівноваги. Розглянути окремий випадок $f(x) = f_0$. Виділити складову розв’язку, яка відповідає усталеним вимушеним коливанням і проаналізувати картину резонансу. Перевірити, чи переходить одержаний розв’язок у розв’язок задачі 5.1 за відповідних умов.}

\begin{center}
    \large{\textbf{Розв'язок}}
\end{center}

\noindent Спершу розберемося з розмірностями. Під розподіленою силою слід розуміти величину $\vec{f}=\frac{\Delta \vec{F}}{\Delta m} = \frac{d\vec{F}}{dV} \nu$, де $\nu =\frac 1\rho$ - питомий об'єм. 

\noindent Формальна постановка задачі:
\begin{equation} \label{cond6,1}
    \left\{ \begin{aligned} %%
            \;&u = u(x,t), \\
            &0 \leq x \leq l, t \geq 0, \\
            &u_{tt}=v^2u_{xx}+f(x,t), \\
            &f(x,t)=f(x)\cos\omega t\\
            &\text{окремий випадок: }\;\; f(x)= f_0,\\
            &u(x,0)=0,\\
            &u_t(x,0)=0,\\
            &u(0,t) =0, \\
            &u_x(l,t) =0 . 
    \end{aligned} \right.
\end{equation}

Виділимо складову розв'язку $\tilde{u}$, яка відповідає вимушеним коливанням. Шукатимемо її у формі $\tilde{u}(x,t) = \widetilde{X}(x) \cos\omega t$. Після підстановки отримуємо наступну задачу (нам не принципово, щоб вона задовільняла ще й початкові умови):
\begin{equation} 
    \left\{ \begin{aligned} 
        &v^2\widetilde{X}'' + \omega^2\widetilde{X} + f(x) = 0,\\
        &\widetilde{X}(0)=0,\\
        &\widetilde{X}'(l)=0.
    \end{aligned} \right.
\end{equation}

Найпростішим методом її розв'язання є метод функцій Гріна, який ви вивчали на курсі диференціальних рівнянь. Спочатку знаходимо функцію Гріна $G(x,s)$ до цієї задачі:

\begin{enumerate} 
  \item При $x \neq s$, вона має задовільняти однорідній частині рівняння, тобто 
  \begin{equation*}
    v^2 G'' + \omega^2G = 0.    
  \end{equation*}
  Тоді, при вказаних вище умовах, маємо для кожної з областей $(x<s) \;\;\text{і} \;\;(x>s)$:
  \begin{equation*}
    G(x,s) = C_i(s) \cos\left(\frac{\omega x}{v}\right) + C_j(s) \sin\left(\frac{\omega x}{v}\right),    
  \end{equation*}
  \begin{equation*} 
    \left\{ \begin{aligned}
            & i=1,\, j=2,\, x<s;\\
            & i=3,\, j=4,\, x>s.
    \end{aligned} \right.
\end{equation*}

  \item Вона має задовільняти крайовим умовам.
  \begin{equation*}
    G(x,s) = C_2(s) \sin\left(\frac{\omega x}{v}\right), \, x<s 
  \end{equation*}
  \begin{equation*}
    G(x,s) = C_3(s) \left[\cos\left(\frac{\omega x}{v}\right) + \sin\left(\frac{\omega x}{v}\right) \tan\left(\frac{\omega l}{v}\right)\right], \, x>s 
  \end{equation*}
  
  \item При $x=s$ вона неперервна по $x$, а її похідна має скачок, що дорівнює оберненій величині до коефіцієнта при другій похідній та залежить від $s$, замість $x$. У нашому випадку цей скачок дорівнює $\frac{1}{v^2}$.
  \begin{equation*}
    C_2(s) \sin\left(\frac{\omega s}{v}\right) = C_3(s) \left[\cos\left(\frac{\omega s}{v}\right) + \sin\left(\frac{\omega s}{v}\right) \tan\left(\frac{\omega l}{v}\right)\right] 
  \end{equation*}

  \begin{equation*}
    C_2(s) \cos\left(\frac{\omega s}{v}\right) + \frac{1}{v\omega} = C_3(s) \left[\cos\left(\frac{\omega s}{v}\right) \tan\left(\frac{\omega l}{v}\right) - \sin\left(\frac{\omega s}{v}\right)\right]
  \end{equation*}
  Після спрощень отримуємо:
  \begin{equation*}
    C_3 = -\frac{\sin\left(\frac{\omega s}{v}\right)}{v\omega}
  \end{equation*}
  \begin{equation*}
    C_2 = -\frac{\cos\left(\frac{\omega(s-l)}{v}\right)}{v\omega \cos\left(\frac{\omega l}{v}\right)}
  \end{equation*}
  \begin{equation*}
    G(x,s) = -\frac{\cos\left(\frac{\omega(s-l)}{v}\right)}{v\omega\cos\left(\frac{\omega l}{v}\right)} \sin\left(\frac{\omega x}{v}\right), \, x<s 
  \end{equation*}
  \begin{equation*}
    G(x,s) = -\frac{\cos\left(\frac{\omega(x-l)}{v}\right)}{v\omega\cos\left(\frac{\omega l}{v}\right)} \sin\left(\frac{\omega s}{v}\right), \, x>s 
  \end{equation*}
\end{enumerate}

Звідси отримуємо просторову складову, що відповідає усталеним коливанням, для загальної функції $f$:
\begin{equation} \label{Green-sol}
    \tilde{u} = \int\limits_0^x \frac{\cos\left(\frac{\omega(x-l)}{v}\right) \sin\left(\frac{\omega s}{v}\right) f(s)}{v\omega\cos\left(\frac{\omega l}{v}\right)} \; \mathrm{d}s + \int\limits_x^l \frac{\cos\left(\frac{\omega (s-l)}{v}\right)\sin\left(\frac{\omega x}{v}\right) f(s)}{v\omega\cos\left(\frac{\omega l}{v}\right)} \; \mathrm{d}s
\end{equation}

Підставивши у (\ref{Green-sol}) окремий випадок і виконавши інтегрування, отримаємо:
\begin{equation}
    \tilde{u} = \frac {f_0\cos(\omega t)}{\omega^2}\left(\frac{\cos\left(\frac{\omega (x-l)}{v}\right)}{\cos\left(\frac{\omega l}{v}\right)}-1\right)
\end{equation}

Повний розв'язок є комбінацією однорідної $u_0$ і вимушеної $\tilde{u}$ частини: $u = u_0 + \tilde{u}$. Задача на $u_0$ отримується підстановкою цієї комбінації у (\ref{cond6,1}) і виглядає наступним чином: 
\begin{equation} 
    \left\{ \begin{aligned} %%
            \;&u_0 = u_0(x,t), \\
            &0 \leq x \leq l, t \geq 0, \\
            &{(u_0)}_{tt}=v^2{(u_0)}_{xx}, \\
            &u_0(0,t) = 0, {(u_0)}_x(l,t) = 0,\\  
            &u_0(x,0) = -\tilde{u}(x,0),\\
            &{(u_0)}_t(x,0) = 0.\\
    \end{aligned} \right.
\end{equation}

Маємо просту задачу на хвильове рівняння з початковими умовами. Пошук власних мод для таких крайових умов вже виконаний у домашній задачі №1,2. Випишемо результат:

\begin{equation*} 
    \left\{ \begin{aligned}
            \;&\lambda_n = \frac{\pi^2 (2n-1)^2}{4l^2} - \text{власнi числа},\\
            &k_n = \frac{\pi (2n-1)}{2l} - \text{власнi хвильові вектори},\\
            &\omega_n=vk_n - \text{власнi частоти},\\
            &\text{де } n \in \mathbb{N},\\ 
            &{(u_0)}_n = \left(A_n\sin\omega_nt + B_n\cos\omega_nt\right)\sin k_nx - \text{власнi моди}.       
    \end{aligned} \right.
\end{equation*}

Одразу можемо побачити, що оскільки початковий розподіл швидкостей нульовий, то $\forall n \in \mathbb{N}, A_n = 0$. Тоді однорідна частина розв'язку матиме вигляд:

\begin{equation} \label{hom-modes6,1}
    u = \sum_{n=1}^\infty B_n\cos\omega_nt\sin k_nx
\end{equation}

Залишається підставити (\ref{hom-modes6,1}) в умову на початкове зміщення:

\begin{equation} 
    u_0(x,0) = \sum_{n=1}^\infty B_n\sin k_nx = -\tilde{u} = \frac{f_0}{\omega^2}\left(1 - \frac{\cos\left(\frac{\omega (x-l)}{v}\right)}{\cos\left(\frac{\omega l}{v}\right)}\right)
\end{equation}
 
Коефіцієнти $B_n$ для просторового розподілу сили загального вигляду знаходяться наступним чином:

\begin{equation} 
    B_n = -\frac{2}{l} \int\limits_0^l \tilde{u} \sin k_nx \;\mathrm{d}x
\end{equation}

І після всіх підстановок загальна задача буде розв'язана. Знайдемо ці коефіцієнти для окремого випадку:

\begin{equation} 
    \begin{aligned}
        &B_n = \frac{2f_0}{l\omega^2} \int\limits_0^l \left(1 - \frac{\cos\left(\frac{\omega(x-l)}{v}\right)}{\cos\left(\frac{\omega l}{v}\right)}\right)\sin(k_nx)dx = \frac{2f_0}{l\omega^2} \left(\frac{1}{k_n} - \frac{k_n}{k_n^2 - \left(\frac{\omega}{v}\right)^2}\right)=\\
        &= -\frac{2f_0}{lk_n \left(\omega_n^2 - \omega^2\right)}
    \end{aligned}    
\end{equation}

Повний розв'язок задачі матиме вигляд:

\begin{equation}
    u = \tilde{u}\cos{\omega t} + \sum_{n=1}^\infty B_n\cos\omega_nt\sin k_nx
\end{equation}

Для окремого випадку:

\begin{equation} 
    u = \frac{f_0}{\omega^2} \left(\frac{\cos\left(\frac{\omega (x-l)}{v}\right)}{\cos\left(\frac{\omega l}{v}\right)}-1\right) \cos{\omega t} - \sum_{n=1}^\infty \frac{2f_0}{lk_n\left(\omega_n^2-\omega^2\right)} \cos\omega_nt\sin k_nx
\end{equation}

\textit{Добре видно, що при наближенні частоти $\omega$, з якою діє сила, до власної частоти $\omega_n$, амплітуда відповідної власної моди нескінченно зростатиме. Це явище відповідає визначенню резонансу.}

Тепер спрямуємо $\omega$ до 0. При цьому вимушений розв'язок матиме вигляд:

\begin{equation*}
    \begin{aligned}
        &\frac {f_0}{\omega^2}\left(\frac{\cos\left(\frac{\omega (x-l)}{v}\right)}{\cos\left(\frac{\omega l}{v}\right)}-1\right)\cos{\omega t}=\left[ \cos\left(\frac{\omega (x-l)}{v}\right)=1-\frac{\omega^2 (x-l)^2}{v^2}+O(\omega^4)\right]=\\
        &\frac {f_0\left(1-\frac{\omega^2 (x-l)^2}{v^2}-1+\frac{\omega^2 l^2}{v^2}+O(\omega^4)\right)}{\omega^2-\frac{\omega^4 l^2}{v^2}+O(\omega^6)}=\frac {f_0\left( 2xl-x^2+O(\omega^2)\right)}{v^2(1+O(\omega^2))}.
    \end{aligned}
\end{equation*}

І повний розв'язок матиме вигляд:
\begin{equation} 
    u = \frac{f_0\left( 2xl-x^2\right)}{v^2} - \sum_{n=1}^\infty \frac{2f_0}{lk_n\omega_n^2}\cos(\omega_n t)\sin(k_n x)
\end{equation}

Що повністю відповідає розв'язку (\ref{Cauchy-sol5,1}) задачі  №5,1.

%\end{document}