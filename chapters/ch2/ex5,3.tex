%\documentclass[a4paper, 14pt]{extreport}

%\usepackage{StyleMMF}

%\begin{document}

%\setcounter{chapter}{4}

%\chapter{Еволюційні задачі з неоднорідним рівнянням або неоднорідними межовими умовами: стаціонарні неоднорідності}

\section[Задача №5.3]{5.3}


\textbf{У стержні довжиною $l$ з непроникною бічною поверхнею відбувається дифузія частинок (коефіцієнт дифузії $D$), що мають час життя $\tau$. Через правий кінець всередину стержня подається постійний потік частинок $I_0$. Знайти стаціонарний розподіл концентрації та розв’язок, що задовольняє нульову початкову умову, якщо через лівий кінець частинки вільно виходять назовні й назад не вертаються. Знайти вигляд стаціонарного розв’язку в граничних випадках великих і малих $\tau$ та нарисувати графіки.}
\textit{ Указівка. Рівняння дифузії частинок зі скінченним часом життя має вигляд: $u_t = Du_{xx} - \frac{1}{\tau}u$. Його зручно переписати через так звану довжину дифузійного зміщення $L=\sqrt{D\tau}$:}
\begin{equation*}
    \tau u_t=L^2u_{xx}-u.
\end{equation*}
\textit{Величина $L$ має смисл характерної відстані, на яку частинки встигають зміститися (в середньому) за час свого життя. «Великі» й «малі» $\tau$ означають у дійсності $L\;\gg l$ i $L\;\ll l$ відповідно. Останній випадок фактично означає перехід до наближення півнескінченного стержня $-\infty < x \leq l$.}


\begin{center}
    \large{\textbf{Розв'язок}}
\end{center}

\noindent Формальна постановка задачі:
\begin{equation} \label{cond5,3}
    \left\{ \begin{aligned} %%
            \;&u = u(x,t), \\
            &\tau u_t=L^2u_{xx}-u, \\
            &0 \leq x \leq l, t \geq 0, \\
            &u(x,0)=0,\\
            &u(0,t) = 0, \\
            &u_x(l,t) = I_0. 
    \end{aligned} \right.
\end{equation}

Спочатку спробуємо знайти неоднорідну частину розв'язку $u_{\text{неодн}}$, яка задовільнить рівняння і межові умови (але не обов'язково задовільнятиме початкові). Шукатимемо її методом підбору. Оскільки межові умови не залежать від часу, то намагатимемося знайти його у вигляді функції $f(x)$ лише від $x$. Підставляючи такий вигляд неоднорідного розв'язку у рівняння, отримуємо:

\begin{equation*}
L^2\frac{d^2f}{dx^2} - f = 0
\end{equation*}

Звідси маємо $u_{\text{неодн}} = f(x) = C_1\sinh(x/L) + C_2\cosh(x/L)$. Підставляючи отриманий розв'язок у межові умови, отримуємо \[C_1 = \frac{I_0L}{\cosh(l/L)}, \quad C_2 = 0.\] Остаточно маємо:
\begin{equation}
u_{\text{неодн}}=\frac{I_0L\sinh(x/L)}{\cosh(l/L)}
\end{equation}

Далі шукаємо поний розв'язок у вигляді $u = u_{\text{неодн}} + u_{\text{одн}}$. Підставивши $u$ в задачу (\ref{cond5,3}) i отримаємо однорідну задачу на $u_{\text{одн}}$:

\begin{equation*} 
    \left\{ \begin{aligned}
            \;&u_{\text{одн}} = u_{\text{одн}}(x,t), \\
            &\tau (u_{\text{одн}})_t=L^2(u_{\text{одн}})_{xx}-u_{\text{одн}}, \\
            &u_{\text{одн}}(x,0)=-\frac{I_0L\sinh{(\frac{x}{L})}}{\cosh{(\frac{l}{L})}},\\
            &u_{\text{одн}}(0,t) = 0, \\
            &(u_{\text{одн}})_x(l,t) = 0. 
    \end{aligned} \right.
\end{equation*}

Перш ніж розділити змінні, спростимо рівняння класичним прийомом, який непогано було б запам'ятати. Якщо ми маємо рівння типу 
\begin{equation}
U_t = D U_{xx} + aU    
\end{equation}
з однорідними межовими умовами, то підстановка $U = u\exp(at)$ перетворить його рівняння на $u_t = D u_{xx}$, а межові та початкові умови для $u$ будуть такими самими як були на $U$.\\

Тепер використаємо цей прийом. Підставимо $u_{\text{одн}} = \tilde{u}\exp(-\frac{t}{\tau})$ та отримаэмо задачу на $\tilde{u}$:

\begin{equation} \label{new-cond5,3}
    \left\{ \begin{aligned}
            \;&\tilde{u} = \tilde{u}(x,t), \\
            &\tilde{u}_t = D \tilde{u}_{xx}, \\
            &0 \leq x \leq l, t \geq 0, \\
            &\tilde{u}(x,0) = -\frac{I_0L \sinh(x/L)}{\cosh(l/L)},\\
            &\tilde{u}(0,t) = 0, \\
            &\tilde{u}_x(l,t) = 0. 
    \end{aligned} \right.
\end{equation}
\\

А такі задачі ми вже добре вміємо розв'язувати. Потрібно знайти розв'язки (\ref{new-cond5,3})
наступного вигляду:
\begin{equation} \label{subst5,3}
    \tilde{u}(x,t) = X(x) \cdot T(t) \neq 0 
\end{equation}

Після підстановки відокремлення змінних (див. задачу 1.1) маємо:
\begin{equation}
    \frac{T'}{DT} = \frac{X^{\prime\prime}}{X} = -\lambda
\end{equation}

Отримуємо задачу Штурма-Ліувіля на $X(x)$ яка розв'язана у домашній задачі №1,2 з посібника. Випишемо результат:
  \begin{equation} 
        \left\{ \begin{aligned}
            \;&\lambda_n = \frac{\pi^2 (2n-1)^2}{4l^2} - \text{власнi числа},\\
            &k_n = \frac{\pi (2n-1)}{2l} - \text{власнi хвильові вектори},\\
            &\text{де } n \in \mathbb{N},\\ 
            &X_n(x) = \sin k_nx - \text{власнi функції}.
        \end{aligned} \right.
    \end{equation}

\textit{Оскільки всі власні числа додатні, то після підстановки $\tilde{u}$ в $u_{\text{одн}}$ показники часових експонент не зможуть скомпенсуватися, адже будуть одного знаку, отже $u_{\text{одн}}$ є залежним від часу і виділити незалежну частину неможливо. А це означає, що знайдене раніше $u_{\text{неодн}}$ і є шуканим стаціонарним розв'язком задачі.}

Тепер розв'яжемо рівняння на $T_n(t)$:
\begin{equation} 
    \left\{ \begin{aligned}
        &\dot{T_n}+\lambda_n DT_n=0,\\
        &\frac1{\tau_n} = D\lambda_n - \text{власний характерний час},\\
        &T_n=C_n\exp{\left(-\frac t{\tau_n}\right)}.
    \end{aligned} \right.
\end{equation}

Збираючи по отриманим функціям $T(t)$ і $X(x)$ власні функції задачі й просумувавши їх, отримуємо загальний розв'язок:
\begin{equation} \label{mode5,3}
    \tilde{u}(x,t) = \sum_{n=1}^{\infty} C_n\exp{\left(-\frac t{\tau_n}\right)}\sin\left(k_n x\right)
\end{equation}

Тепер підставимо (\ref{mode5,3}) у початкові умови:
\begin{equation} \label{init-con5,3}
    \tilde{u}(x,0) = \sum_{n=1}^{\infty} C_n\sin\left(k_n x\right) = -\frac{I_0L\sinh{(\frac{x}{L})}}{\cosh{(\frac{l}{L})}}
\end{equation}

Ліва сторона рівності є розкладом Фур'є правої сторони по синусах. Знайдемо коефіцієнти цього розкладу:

\begin{equation*}
    \begin{aligned} 
        &C_n\int_{0}^{l} \left( \sin\left(k_n x\right)\right)^2\;dx= \frac{lC_n}{2}
        =-\frac{I_0L}{\cosh{(\frac{l}{L})}}\int_{0}^{l}\sinh{\left(\frac{x}{L}\right)}\sin\left(k_n x\right)\;dx=\\
        &=\left[\cos(k_n l)=(-1)^n\right]=-\frac{I_0L}{\cosh{(\frac{l}{L})}}\frac{4l^2L(-1)^{n+1}\cosh{(\frac{l}{L})}}{4l^2+\pi^2L^2(2n-1)^2}=\frac{L^2I_0(-1)^n}{1+L^2\lambda_n}.
    \end{aligned}
\end{equation*}

Підставляючи отримані коефіцієнти у (\ref{init-con5,3}), а його у $u_{\text{одн}}$ і сумуючи з $u_{\text{неодн}}$, отримуємо відповідь:
\begin{equation*} 
    %\begin{aligned}
        u(x,t)=\frac{I_0L\sinh{(\frac{x}{L})}}{\cosh{(\frac{l}{L})}}+\frac{2L^2I_0}{l}\sum_{n=1}^{\infty}\frac{L^2I_0(-1)^n}{1+L^2\lambda_n}\exp{\left(-\left(\frac t{\tau_n}+\frac{t}{\tau}\right) \right)}\sin\left(k_n x\right)
    %\end{aligned}
\end{equation*}

%\end{document}