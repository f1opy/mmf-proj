\documentclass[a4paper, 14pt]{extreport}

\usepackage{StyleMMF}

\begin{document}

\section{Еволюційні задачі з неоднорідним рівнянням або неоднорідними межовими умовами: стаціонарні неоднорідності}

\subsubsection{Задача №3}

\textit{У стержні довжиною $l$ з непроникною бічною поверхнею відбувається дифузія частинок (коефіцієнт дифузії $D$), що мають час життя $\tau$. Через правий кінець всередину стержня подається постійний потік частинок $I_0$. Знайти стаціонарний розподіл концентрації та розв’язок, що задовольняє нульову початкову умову, якщо через лівий кінець частинки вільно виходять назовні й назад не вертаються. Знайти вигляд стаціонарного розв’язку в граничних випадках великих і малих $\tau$ та нарисувати графіки.\\
Указівка. Рівняння дифузії частинок зі скінченним часом життя має вигляд:
$u_t = D u_{xx} - u/\tau$. Його зручно переписати через так звану
довжину дифузійного зміщення $L = \sqrt{D\tau}$: \[\tau u_t = L^2 u_{xx} - u\] Величина $L$ має смисл характерної відстані, на яку частинки встигають зміститися (в середньому) за час свого життя. «Великі» й «малі» $\tau$ означають у дійсності $L \gg l$ і $L \ll l$ відповідно. Останній випадок фактично означає перехід до наближення півнескінченного стержня $-\infty < x \leq l$}

\end{document}