%\documentclass[a4paper, 14pt]{extreport}

%\usepackage{StyleMMF}

%\setcounter{chapter}{4}

%\begin{document}

%\chapter{Еволюційні задачі з неоднорідним рівнянням або неоднорідними межовими умовами: стаціонарні неоднорідності}

\section[Задача №5.1]{5.1}

\textit{Знайти коливання вертикально розташованого пружного стержня під дією сили тяжіння для $t > 0$. Верхній кінець стержня закріплений, а нижній вільний. При $t < 0$ стержень був нерухомим і деформацій не було. Знайти спочатку стаціонарний розв’язок, що відповідає положенню рівноваги стержня в полі тяжіння, а потім знайти відхилення від нього, що відповідає коливанням навколо нового положення рівноваги. Намалювати графіки розподілу поля зміщень та поля напружень у положенні рівноваги.}

\begin{center}
    \large{\textbf{Розв'язок}}
\end{center}

\noindent Формальна постановка задачі:
\begin{equation} %\label{probcond12}
    \left\{ \begin{aligned}
            \;&u = u(x,t), \\
            &u_{tt} = v^2 u_{xx} + g, \\
            &0 \leq x \leq l, t \geq 0 \\
            &u(0,t) = 0, \, u_x(l,t) = 0, \\
            &u(x,0) = u_t(x,0) = 0.
    \end{aligned} \right.
\end{equation}

Шукаємо розв'язок у вигляді:
\begin{equation}
    u = u_{\text{ст}}(x) + w(x,t)
\end{equation}

Перепишемо задачу для стаціонарної частини розв'язку
\begin{equation} 
    \left\{ \begin{aligned}
            \;&u = u_{\text{ст}}(x), \\
            &v^2 u_{xx} + g = 0, \\
            &0 \leq x \leq l, t \geq 0 \\
            &u(0) = 0, \, u_x(l) = 0
    \end{aligned} \right.
\end{equation}
Рівняння двічі інтегрується і маємо:
\begin{equation}
    u_{\text{ст}}(x) = - \frac{g x^2}{2v^2} + C_1 x + C_2,
\end{equation}
а з межових умов визначимо константи інтегрування
\begin{equation}
    \begin{aligned}
        u_{\text{ст}}(0) = C_2 = 0,\\
        (u_{\text{ст}})_x(l) = -\frac{gl}{v^2} + C_1 = 0;
    \end{aligned}
    \;\Rightarrow\quad
    C_2 = 0, \quad C_1 = \frac{gl}{v^2}
\end{equation}
Отже, стаціонарний розв'язок
\begin{equation}
    u_{\text{ст}}(x) = - \frac{g x^2}{2v^2} + \frac{glx}{v^2} = - \frac{gl^2}{2v^2} \cdot \frac{x^2 - 2lx}{l^2} = -A \cdot \frac{x^2 - 2lx}{l^2}
\end{equation}

Тепер необхідно записати задачу для нестаціонарної частини $w(x,t)$
Рівняння:
\begin{equation*}
    u_{tt} = v^2 u_{xx} + g
    \;\Rightarrow\;
    w_{tt} = v^2w_{xx} - v^2 \frac{gl^2}{2v^2}\frac{2}{l^2} + g = v^2w_{xx}
\end{equation*}
Межові умови:
\begin{equation*}
    \begin{aligned}
        u(0,t) = -\frac{gl^2}{2v^2} \cdot \frac{x^2 - 2lx}{l^2}\bigg|_{x=0} + w(l,t) = 0,\\
        u_x(l,t) = -\frac{gl^2}{v^2} \cdot \frac{x - l}{l^2}\bigg|_{x=l} + w_x(l,t) = 0
    \end{aligned}
        \quad\Rightarrow\; 
        w(0,t) = 0,\, w_x(l,t) = 0
\end{equation*}
Початкові умови:
\begin{equation*}
    u(x,0) = -A \cdot \frac{x^2 - 2lx}{l^2} + w(x,t) = 0
    \;\Rightarrow\;
    w(x,0) = A \cdot \frac{x^2 - 2lx}{l^2}
\end{equation*}
\begin{equation*}
    u_t(x,0) = 0
    \;\Rightarrow\;
    w_t(x,0) = 0
\end{equation*}
Отже, отримуємо задачу для $w(x,t)$ з однорідними межовими умова, але з неоднорідними рівнянням та початковими умовами.
\begin{equation} 
    \left\{ \begin{aligned} 
            \;&w = w(x,t), \\
            &w_{tt} = v^2 w_{xx}, \\
            &0 \leq x \leq l, t \geq 0 \\
            &w(0,t) = 0, \, w_x(l,t) = 0, \\
            &w(x,0) = A \cdot \frac{x^2 - 2lx}{l^2},\\
            &w_t(x,0) = 0.
    \end{aligned} \right.
\end{equation}

Розв'язок такої задачі був знайдений раніше (див. задачу 1.2):
\begin{equation}
    \begin{aligned}
        &w(x,t) = \sum_{n=0}^{\infty} \big(A_n\sin\omega_nt + B_n\cos\omega_nt \big) \sin k_nx,\\
        &k_n = \frac{\pi}{l}(n + 1/2) \text{ -- хвильове число},\\
        &\omega_n = v k_n \text{ -- частота коливання},\\
        &n = 0, 1, 2, \ldots        
    \end{aligned}
\end{equation}

Залишається визначити коефіцієнти $A_n$ та $B_n$ з початкових умов:
\begin{equation*}
    w_t(x,0) = \sum_{n=0}^{\infty} A_n\omega_n\sin k_nx = 0
    \;\Rightarrow\;
    A_n = 0,\, \forall n
\end{equation*}
\begin{equation*}
    w_t(x,0) = \sum_{n=0}^{\infty} B_n\sin k_nx = A \cdot \frac{x^2 - 2lx}{l^2} 
    \;\Rightarrow\;
    B_n = \frac{2A}{l^3} \int\limits_0^l (x^2 - 2lx) \sin k_nx \;\mathrm{d}x
\end{equation*}
Обчислимо отриманий інтеграл 
\begin{equation*}
    \begin{gathered}
        \int\limits_0^l (x^2 - 2lx) \sin k_nx \;\mathrm{d}x = \frac{1}{k_n} (x^2 - 2lx) \cos k_nx \bigg|_0^l -\\
        - \frac{2}{k_n} \int\limits_0^l (x-l) \cos k_nx \;\mathrm{d}x = \frac{2}{k_n^2} \bigg[(x-l)\sin k_nx \bigg|_0^l - \int\limits_0^l \sin k_nx \;\mathrm{d}x \bigg] =\\
        = \frac{2}{k_n^3} \big(\cos k_nl - \cos 0\big) = \bigg|\cos k_nl = 0\bigg| = -\frac{2}{k_n^3}
    \end{gathered}
\end{equation*}
Прим.: \textit{можна було спростити інтегрування, зсунувши межі інтегрування на $-l/2$, адже тоді можна скористатися тим фактом, що непарна підінтегральна функція по симетричним межам дає нуль}

Отже, розв'язок 
\begin{equation}
    \begin{gathered}
        u(x,t) = - A \cdot \frac{x^2 - 2lx}{l^2} - 4A\sum_{n=0}^{\infty} \frac{\cos\omega_nt \sin k_nx}{l^3k_n^3} =\\
        = -A \left(\frac{x^2 - 2lx}{l^2} + 4\sum_{n=0}^{\infty} \frac{\cos\omega_nt \sin k_nx}{(lk_n)^3}\right)
    \end{gathered}
\end{equation}

%\end{document}