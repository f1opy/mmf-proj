\documentclass[a4paper, 14pt]{extreport}

\usepackage{StyleMMF}

\begin{document}

\chapter{Еволюційні задачі з неоднорідним рівнянням або неоднорідними межовими умовами: стаціонарні неоднорідності}

\section[Задача №5.1]{5.1}

\textit{Знайти коливання вертикально розташованого пружного стержня під дією сили тяжіння для $t > 0$. Верхній кінець стержня закріплений, а нижній вільний. При $t < 0$ стержень був нерухомим і деформацій не було. Знайти спочатку стаціонарний розв’язок, що відповідає положенню рівноваги стержня в полі тяжіння, а потім знайти відхилення від нього, що відповідає коливанням навколо нового положення рівноваги. Намалювати графіки розподілу поля зміщень та поля напружень у положенні рівноваги.}

\end{document}