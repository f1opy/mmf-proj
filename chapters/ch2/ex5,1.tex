%\documentclass[a4paper, 14pt]{extreport}

%\usepackage{../../main/StyleMMF}

%\setcounter{chapter}{4}

%\begin{document}

%\chapter{Еволюційні задачі з неоднорідним рівнянням або неоднорідними межовими умовами: стаціонарні неоднорідності}

\section[Задача №5.1]{5.1}

\textit{Знайти коливання вертикально розташованого пружного стержня під дією сили тяжіння для $t > 0$. Верхній кінець стержня закріплений, а нижній вільний. При $t < 0$ стержень був нерухомим і деформацій не було. Знайти спочатку стаціонарний розв’язок, що відповідає положенню рівноваги стержня в полі тяжіння, а потім знайти відхилення від нього, що відповідає коливанням навколо нового положення рівноваги. Намалювати графіки розподілу поля зміщень та поля напружень у положенні рівноваги.}

\begin{center}
    \large{\textbf{Розв'язок}}
\end{center}

\textit{Знайти коливання вертикально розташованого пружного стержня під дією сили тяжіння для $t > 0$. Верхній кінець стержня закріплений, а нижній вільний. При $t < 0$ стержень був нерухомим і деформацій не було. Знайти спочатку стаціонарний розв’язок, що відповідає положенню рівноваги стержня в полі тяжіння, а потім знайти відхилення від нього, що відповідає коливанням навколо нового положення рівноваги. Намалювати графіки розподілу поля зміщень та сили натягу стержня у положенні рівноваги.}

\begin{center}
    \large{\textbf{Розв'язок}}
\end{center}

\noindent Формальна постановка задачі:
\begin{equation} \label{cond5,1}
    \left\{ \begin{aligned}
            \;&u = u(x,t), \\
            &u_{tt} = v^2 u_{xx} + g, \\
            &0 \leq x \leq l, t \geq 0 \\
            &u(0,t) = 0, \, u_x(l,t) = 0, \\
            &u(x,0) = u_t(x,0) = 0.
    \end{aligned} \right.
\end{equation}

На відміну від попередніх задач, рівняння тут неоднорідне. Якщо рівняння і/або межові умови неоднорідні, то змінні не розділяються, і починати з відокремлення змінних не можна. Проте у даній задачі йдеться про систему, яка занходиться у стаціонарних (тобто незмінних з часом) зовнішніх умовах: стержень (певним чином закрілений) знаходиться у стаціонарному зовнішньому полі сили тяжіння. Математично це проявляється у тому, що неоднорідний член у рівнянні не залежить від часу.  Завдяки такій особливості задачу можна розв'язати відносно просто, не звертаючись до загальних методів рорзв'язання задач з неоднорідними рівнянням чи межовими умовами. Ключик до задщачі можна знайти з фізичних міркувань. 
Простим аналогом задачі є задача про пружинний маятник (частинку, прикріплену до невагомої пружинки) у полі тяжіння. Уявляємо, що стердень перебував у положенні рівноваги, і сила тяжіння включається у початковий момент часу. Під дією сили тяжіння стерень почне розтягуватись і потім коливатися. Якщо врахувати мале тертя, то коливання згодом затухнуть, і стержень зупинится у положенні, в якому він буде  розтягнутий під дією власної ваги. Це положення  рівноваги стержня у полі тяжіння.  Відповідний розв'язок рівняння з межовими умовами називають стаціонарним розв'язком $u = u_{\text{ст}}$. За смислом стаціонарний розв'язок не залежить від часу $u = u_{\text{ст}}(x)$ і задовольняє неоднорідне рівняння і межові умови задачі (\ref{cond5,1}). Отже, $u_{\text{ст}}(x)$ можна знайти як розв'язок неоднорідної крайової задачі 
\begin{equation} \label{stat sol}
    \left\{ \begin{aligned}
            \;&u = u_{\text{ст}}(x), \\
            &v^2 u'' + g = 0, \\
            &0 \leq x \leq l, t \geq 0 \\
            &u(0) = 0, \, u'(l) = 0
    \end{aligned} \right.
\end{equation}
 Фізично це задача на статичну рівновагу стержня у полі тяжіння. 
 
 Щоб розв'язати вихідну задачу, робимо заміну невідомої функції 
\begin{equation} \label{subst w}
    u(x,t) = u_{\text{ст}}(x) + w(x,t)
\end{equation}
 Нове невідоме $w(x,t)$ відповідає вільним коливанням відносно нового положення рівноваги. Тому $w(x,t)$ задовольнятиме однорідні рівняння і межові умови, а такі задачі ми вже вміємо розв'язувати. 

1) Cтаціонарний розв'язок. 
Рівняння задачі (\ref{stat sol}) інтегруємо двічі. Маємо:
\begin{equation}
    u_{\text{ст}}(x) = - \frac{g x^2}{2v^2} + C_1 x + C_2
\end{equation}
Cталі інтегрування визначаємо з крайових умов 
\begin{equation}
    \begin{aligned}
        u_{\text{ст}}(0) = C_2 = 0,\\
        u_{\text{ст}}'(l) = -\frac{gl}{v^2} + C_1 = 0;
    \end{aligned}
    \;\Rightarrow\quad
    C_2 = 0, \quad C_1 = \frac{gl}{v^2}
\end{equation}
Отже, стаціонарний розв'язок має вигляд
\begin{equation}
    u_{\text{ст}}(x) = - \frac{g x^2}{2v^2} + \frac{glx}{v^2} = A \cdot \frac{2lx - x^2}{l^2}
\end{equation}
де позначено $A = g l^2/2v^2$.  Графік поля зміщень має вигляд парболи з максимумом у точці кінця стержня $x = l$. Згідно закону Гука пружна сила $F(x) = \beta u_x$ , де $\beta$  - пружна стала, а $v^2 = \rho/\beta$ , де $\rho$ - лінійна густина маси. Звідси маємо 
\begin{equation}
    F(x) = \beta \frac{g l^2 \rho}{2 \beta} \cdot \frac{2l - 2 x}{l^2} = M g \frac{l - x}{l}
\end{equation}
 $M = \rho l$ - повна маса стержня. Отже, сила натягу максимальна і дорівнює вазі всього стержня у точці його закріплення і лінійно спадає до нуля у точці його нижнього кінця, що повністю відповідає фізиці ситуації. 
 
 
 2) Нестаціонарна частина розв'язку.
 Зробимо заміну (\ref{subst w}): перепишемо умови задачі через  нове невідоме $w(x,t)$, враховуючи умови (\ref{stat sol}), які задовольняє стаціонарний розв'язок.
Рівняння:
\begin{equation*}
    u_{tt} = v^2 u_{xx} + g
    \;\Rightarrow\;
    w_{tt} = v^2w_{xx} + v^2 u''_{\text{ст}} + g = v^2w_{xx}
\end{equation*}
Межові умови:
\begin{equation*}
    \begin{aligned}
        u(0,t) = u_{\text{ст}} (0) + w(0,t) = 0,\\
        u_x(l,t) = u'_{\text{ст}} (0) + w_x(l,t) = 0
    \end{aligned}
        \quad\Rightarrow\; 
        w(0,t) = 0,\, w_x(l,t) = 0
\end{equation*}
Початкові умови:
\begin{equation*}
    u(x,0) = u_{\text{ст}} (x) + w(x,0) = 0
    \;\Rightarrow\;
    w(x,0) = - u_{\text{ст}} (x) = A \cdot \frac{x^2 - 2lx}{l^2}
\end{equation*}
\begin{equation*}
    u_t(x,0) = 0
    \;\Rightarrow\;
    w_t(x,0) = 0
\end{equation*}

Отже, ми одержали задачу для $w(x,t)$ з однорідними межовими умовами і однорідними рівнянням та зміненими початковими умовами:
\begin{equation} 
    \left\{ \begin{aligned} 
            \;&w = w(x,t), \\
            &w_{tt} = v^2 w_{xx}, \\
            &0 \leq x \leq l, t \geq 0 \\
            &w(0,t) = 0, \, w_x(l,t) = 0, \\
            &w(x,0) = A \frac{x^2 - 2lx}{l^2},\\
            &w_t(x,0) = 0.
    \end{aligned} \right.
\end{equation}

 Власні моди для такої задачі були знайдені у задачі 1.2. Загальний розв'язок має вигляд:
\begin{equation}
    \begin{aligned}
        &w(x,t) = \sum_{n=0}^{\infty} \big(A_n\sin\omega_nt + B_n\cos\omega_nt \big) \sin k_nx,\\
        &k_n = \frac{\pi}{l}(n + 1/2) \text{ -- хвильове число},\\
        &\omega_n = v k_n \text{ -- частота коливання},\\
        &n = 0, 1, 2, \ldots        
    \end{aligned}
\end{equation}

Залишається визначити коефіцієнти $A_n$ та $B_n$ з початкових умов:
\begin{equation*}
    w_t(x,0) = \sum_{n=0}^{\infty} A_n\omega_n\sin k_nx = 0
    \;\Rightarrow\;
    A_n = 0,\, \forall n
\end{equation*}
\begin{equation*}
    w_t(x,0) = \sum_{n=0}^{\infty} B_n\sin k_nx = A \cdot \frac{x^2 - 2lx}{l^2} 
    \;\Rightarrow\;
    B_n = \frac{2A}{l^3} \int\limits_0^l (x^2 - 2lx) \sin k_nx \;\mathrm{d}x
\end{equation*}
Обчислимо отриманий інтеграл 
\begin{equation*}
    \begin{gathered}
        \int\limits_0^l (x^2 - 2lx) \sin k_nx \;\mathrm{d}x = \frac{1}{k_n} (x^2 - 2lx) \cos k_nx \bigg|_0^l -\\
        - \frac{2}{k_n} \int\limits_0^l (x-l) \cos k_nx \;\mathrm{d}x = \frac{2}{k_n^2} \bigg[(x-l)\sin k_nx \bigg|_0^l - \int\limits_0^l \sin k_nx \;\mathrm{d}x \bigg] =\\
        = \frac{2}{k_n^3} \big(\cos k_nl - \cos 0\big) = \bigg|\cos k_nl = 0\bigg| = -\frac{2}{k_n^3}
    \end{gathered}
\end{equation*}

Отже, розв'язок 
\begin{equation} \label{Cauchy-sol5,1}
    \begin{gathered}
        u(x,t) = \frac{g l^2}{2v^2} \left(\frac{2lx - x^2}{l^2} - 4\sum_{n=0}^{\infty} \frac{\cos\omega_nt \sin k_nx}{(lk_n)^3}\right)
    \end{gathered}
\end{equation} 

З відповіді видно, що стаціонарний розв'язок (це перший доданок, якщо розкрити дужки) об'єктивно відрізняється від інших складових розв'язку задачі, оскільки вони залежать від часу. Стаціонарний розв'язок відповідає положенню рівноваги стержня у полі тяжіння, і тому представляє самостійний інтерес з фізичної фізичної точки зору. Якщо врахувати затухання коливань, то при великих \textit{t}  поле зміщень прямуватиме до стаціонарного розв'язку. 

Зверніть увагу, що одержаний ряд Фур'є збігається швидше, ніж у попередніх задачах: коефіцієнти ряду спадають як $1/n^3$ при великих \textit{n}. Це пов'язано з тим, що стаціонарний розв'язок задовольняє такі ж крайові умови, як і власні функції задачі Штурма-Ліувілля, по яких він розкладається у ряд.  

Зауважимо, що стаціонарний розв'язок існує не завжди. Може бути, що неоднорідні члени у рівнянні і/або межових умовах не залежать від часу, але задача на стаціонарний розв'язок розв'язку не має.  Наприклад, якщо у розглянутій вище задачі закріплений кінець зробити вільним, то стержень буде вільно падати. Отже фізично це випадки, коли статична рівновага у системі неможлива. Тоді метод необхідно модифікувати.


%\end{document}