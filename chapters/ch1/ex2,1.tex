\documentclass[a4paper, 14pt]{extreport}

\usepackage{StyleMMF}

\begin{document}

\subsection{Стержень з вільними та пружно закріпленими кінцями; системи, описувані іншими рівняннями.}

\subsubsection{Задача №1}

\textit{Знайти власні моди повздовжніх рухів тонкого стержня $0 \leq x \leq l$ із вільними кінцями  (задача для хвильового рівняння з межовими умовами $u_x(0,t) = 0, u_x(l,t) = 0$).\\
Результат перевірити аналітично й графічно (див. заняття №6, зразок модульної контрольної роботи №1) та проаналізувати його фізичний смисл. Чим відрізняється від інших основна (нульова) мода? Якому рухові стержня вона відповідає?}

\begin{center}
    \large{\textbf{Розв'язок}}
\end{center}

\noindent Формальна постановка задачі:
\begin{equation} \label{probcond2}
    \left\{ \begin{aligned} %%
            \;&u = u(x,t), \\
            &u_{tt} = v^2 u_{xx}, \\
            &0 \leq x \leq l, t \in \mathbb{R} \\
            &u_x(0,t) = 0, \\
            &u_x(l,t) = 0. 
    \end{aligned} \right.
\end{equation}
Необхідно знайти розв'язки (\ref{probcond2}) вигляду:
\begin{equation} \label{subst2}
    u(x,t) = X(x) \cdot T(t) \neq 0 
\end{equation}

Від задачі №1 попереднього заняття задача відрізняється тільки межовою умовою, тому підставляємо розв'язок у вигляді добутку (\ref{subst2}) тільки у межові умови (\ref{probcond2}):
\begin{equation*}
    \begin{aligned}
        \;u_x(0,t) = X'(0) \cdot T(t) = 0
        \;\Rightarrow\;
        \left\{ \begin{aligned}
            &T(t) \neq 0, \forall t, \\  &X'(0) = 0; 
        \end{aligned} \right.\\
        u_x(l,t) = X'(l) \cdot T(t) = 0
        \;\Rightarrow\;
        \left\{ \begin{aligned}
            &T(t) \neq 0, \forall t, \\  &X'(l) = 0; 
        \end{aligned} \right.\\
    \end{aligned}
\end{equation*}
Тут ми врахували, що умови на кінцях струни виконуються при всіх $t$, тому $T(t)$ не може бути рівним нулю.\\

Виписуємо результат відокремлення змінних:
\begin{equation} \label{sepvar2}
    \left\{ \begin{aligned}
        \;&X = X(x), \\
          &X^{\prime\prime} = -\lambda X, \\
          &0 \leq x \leq l, \\
          &X'(0) = 0, \\ 
          &X'(l) = 0. 
    \end{aligned} \right.
    \qquad\qquad
    T^{\prime\prime} + \lambda v^2 T = 0
\end{equation}

\begin{enumerate}
    \item[] Розв'язуємо задачу Штурма-Ліувілля (\ref{sepvar2}). Знову скористаємося результатами попередньої задачі та одразу запишемо якого типу отримуємо розв'язки для різних $\lambda$.
    \begin{enumerate}[wide, labelindent=0pt]
        
        \item Випадок $\lambda < 0$. 
        \begin{equation*}
            X(x) = C_1 sh(\sqrt{|\lambda|}x) + C_2 ch({\sqrt{|\lambda|}x})
        \end{equation*}
        Знаходимо константи з межових умов:
        \begin{equation*}
            \begin{aligned}
                &X'(0) = C_1\sqrt{|\lambda|}
                \;\Rightarrow\;
                X(x) = C_2 &ch(\sqrt{|\lambda|}x)\\
                &\left\{ \begin{aligned}
                    &X'(l) = C_2\sqrt{|\lambda|} sh(\sqrt{|\lambda|}l) = 0, \\
                    &sh(\sqrt{|\lambda|}l) \neq 0;
                \end{aligned} \right.&\\
                &\left\{ \begin{aligned}
                    C_1 = 0, \\ 
                    C_2 = 0;
                \end{aligned} \right. \qquad\qquad\qquad\qquad&
            \end{aligned}
            \;\Rightarrow\;
            \begin{aligned}
                \text{розв'язок тривівльний,}\\
                \text{немає від'ємних}\\
                \text{власних значень.}
            \end{aligned}
        \end{equation*}

        \item Випадок $\lambda = 0$:
        \begin{equation*}
            X(x) = C_1 + C_2 x
        \end{equation*}
        Знаходимо константи з межових умов:
        \begin{equation*}
            \begin{aligned}
                &\left\{ \begin{aligned}
                    &X'(0) = C_2 = 0, \\ 
                    &X'(l) = C_2 = 0;
                \end{aligned} \right.
                \\   
                &\left\{ \begin{aligned}
                    C_1 \in \mathbb{R}, \\ 
                    C_2 = 0;
                \end{aligned} \right.
            \end{aligned}
            \quad\Rightarrow\;
            \begin{aligned}
                X(x) = 0 \text{ -- розв'язок нетривівльний,}\\
                \lambda = 0 \text{ є власним значенням.}
            \end{aligned}
        \end{equation*}

        \item Випадок $\lambda > 0$
        \begin{equation*}
            X(x) = C_1 \sin(\sqrt{\lambda}x) + C_2 \cos({\sqrt{\lambda}x})
        \end{equation*}
        Знаходимо константи з межових умов:
        \begin{equation*}
            \begin{aligned}
                \left\{ \begin{aligned}
                    &X'(0) = C_1\sqrt{\lambda} = 0, \\ 
                    &X'(l) = \sqrt{\lambda}\left(\textcolor{red}{\begin{xy}*{\textcolor{black}{C_1}};p+LU;+RD**h@{}+//**h@{-}*h@{>}*h!LD{\scriptstyle 0}\end{xy}} \cos({\sqrt{\lambda}l}) - C_2 \sin(\sqrt{\lambda}l)\right) = 0;
                \end{aligned} \right.
                \;\Rightarrow\;
                \left\{ \begin{aligned}
                    &C_2 \neq 0, \\ 
                    &\sin(\sqrt{\lambda}l) = 0;
                \end{aligned} \right.
            \end{aligned}
        \end{equation*}
        Отже, нетривіальні розв'язки існують при значеннях параметра $\lambda$, які задовольняють характеристичне рівняння :
        \begin{equation*}
            \sin(\sqrt{\lambda}l) = 0
            \;\Rightarrow\;
            \sqrt{\lambda_n}l = \pi n, \, n \in \mathbb{Z}
            \;\Rightarrow\;
            \lambda_n = \frac{\pi^2 n^2}{l^2}.
        \end{equation*}
    \end{enumerate}
\end{enumerate} 
Випишемо тепер розв'язки для всіх $n$ і визначимо, які з них необхідно залишити:
    \begin{equation*}
        \left\{ \begin{aligned}
            &X_0(x) = C_0,\\
            &\lambda_0 = 0;
        \end{aligned} \right.
        \qquad
        \left\{ \begin{aligned}
            &X_n(x) = C_n \sin\left(\frac{\pi n x}{l}\right),\\
            &\lambda_n = \frac{\pi^2 n^2}{l^2}, n \in \mathbb{N}.
        \end{aligned} \right.
    \end{equation*}

    \textbf{\Large Необхідно відредагувати наступний текст}
Видно, що $n = 0$ відповідає тривіальному розв'язку. Видно також, що всі інші розв'язки визначені з точністю до довільного множника.\\
Тому власні функції, які співпадають з точністю до множника, вважають однаковими. У загальному випадку різними вважають лише лінійно незалежні власні функції, а розвя'зати задачу Штурма-Ліувілля означає знайти всі різні власні функції і відповідні власні значення. Отже, різним власним функціям відповідають лише натуральні $n$, а коефіцієнти $C_n$ можна покласти рівними одиниці.\\
    Власними значеннями і власними функціями є
    \begin{equation} %\label{ShLsol}
        \left\{ \begin{aligned}
            \;&\lambda_n = \frac{\pi^2 n^2}{l^2},\\ 
            &X_n(x) = \sin\left(\frac{\pi n x}{l}\right),
        \end{aligned} \right.
        \quad \text{де } n \in \mathbb{N}.
    \end{equation}

Повертаємося до рівняння для $T(t)$ (\ref{sepvar2}). Підставляємо знайдені власні значення та знаходимо $T_n(t)$:
\begin{equation*}
    \left. \begin{aligned}
        \lambda_n = \frac{\pi^2 n^2}{l^2},&\;\\ 
        T^{\prime\prime} + \lambda v^2T = 0,&
    \end{aligned} \right\}
    \;\Rightarrow\;
    T_n(t) = A\cos(\omega_n t) + B\sin(\omega_n t),
\end{equation*}
де $\omega_n^2 = \lambda_n v^2, \, n \in \mathbb{N}.$\\
\begin{equation*}
    \left. \begin{aligned}
        \lambda_0 = 0,&\;\\ 
        T^{\prime\prime} = 0,&
    \end{aligned} \right\}
    \;\Rightarrow\;
    T_0(t) = A_0 + B_0 t,
\end{equation*}
Власними модами коливань струни будуть всі розв'язки вигляду:
\begin{equation*}
    u_n(x,t) = X_n(x) \cdot T_n(t)
\end{equation*}
Виконаємо перепозначення і запишемо остаточний розв'язок:
\begin{equation}
    \left\{ \begin{aligned} \label{sol2}
        \;&u_0(x,t) = A_0 + B_0 t, \\
        &u_n(x,t) = \left[A_n\cos(\omega_n t) + B_n\sin(\omega_n t)\right] \sin(k_n x), \\
        &k_n = \frac{\pi n}{l} - \text{ хвильові вектори}, \\
        &\omega_n = vk_n = \frac{v \pi n}{l} - \text{ власні частоти}, \\
        &n = 1, 2,\ldots
    \end{aligned}\right.
\end{equation}

\end{document}