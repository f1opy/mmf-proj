%\documentclass[a4paper, 14pt]{extreport}

%\usepackage{StyleMMF}

%\begin{document}

%\setcounter{chapter}{1}
%\chapter{Власні моди інших систем. Вільні коливання для заданих початкових умов.}

\textbf{\large Стержень з вільними та пружно закріпленими кінцями; системи, описувані іншими рівняннями.}

\section[Задача №2.1]{2.1}

\textit{Знайти власні моди повздовжніх рухів тонкого стержня $0 \leq x \leq l$ із вільними кінцями  (задача для хвильового рівняння з межовими умовами $u_x(0,t) = 0, u_x(l,t) = 0$).\\
Результат перевірити аналітично й графічно (див. заняття №6, зразок модульної контрольної роботи №1) та проаналізувати його фізичний смисл. Чим відрізняється від інших основна (нульова) мода? Якому рухові стержня вона відповідає?}

\begin{center}
    \large{\textbf{Розв'язок}}
\end{center}

\noindent Формальна постановка задачі:
\begin{equation} \label{cond2,1}
    \left\{ \begin{aligned} %%
            \;&u = u(x,t), \\
            &u_{tt} = v^2 u_{xx}, \\
            &0 \leq x \leq l, t \in \mathbb{R} \\
            &u_x(0,t) = 0, \\
            &u_x(l,t) = 0. 
    \end{aligned} \right.
\end{equation}
Необхідно знайти розв'язки (\ref{cond2,1}) вигляду:
\begin{equation} \label{subst2,1}
    u(x,t) = X(x) \cdot T(t) \neq 0 
\end{equation}

Від задачі №1.1 попереднього заняття задача відрізняється тільки межовою умовою, тому підставляємо розв'язок у вигляді добутку (\ref{subst2,1}) тільки у межові умови (\ref{cond2,1}):
\begin{equation*}
    \begin{aligned}
        \;u_x(0,t) = X'(0) \cdot T(t) = 0
        \;\Rightarrow\;
        \left\{ \begin{aligned}
            &T(t) \neq 0, \forall t, \\  &X'(0) = 0; 
        \end{aligned} \right.\\
        u_x(l,t) = X'(l) \cdot T(t) = 0
        \;\Rightarrow\;
        \left\{ \begin{aligned}
            &T(t) \neq 0, \forall t, \\  &X'(l) = 0; 
        \end{aligned} \right.\\
    \end{aligned}
\end{equation*}
Тут ми врахували, що умови на кінцях струни виконуються при всіх $t$, тому $T(t)$ не може бути рівним нулю.\\

Виписуємо результат відокремлення змінних:
\begin{equation} \label{sepvar2,1}
    \left\{ \begin{aligned}
        \;&X = X(x), \\
          &X^{\prime\prime} = -\lambda X, \\
          &0 \leq x \leq l, \\
          &X'(0) = 0, \\ 
          &X'(l) = 0. 
    \end{aligned} \right.
    \qquad\qquad
    T^{\prime\prime} + \lambda v^2 T = 0
\end{equation}

\begin{enumerate}
    \item[] Розв'язуємо задачу Штурма-Ліувілля (\ref{sepvar2,1}). Розв'язки рівняння задачі для різних $\lambda$ є такими ж, як у задачі 1.1, відмінність полягає у крайових умовах.
    \begin{enumerate}[wide, labelindent=0pt]
        
        \item Випадок $\lambda < 0$. 
        \begin{equation*}
            X(x) = C_1 sh(\sqrt{|\lambda|}x) + C_2 ch({\sqrt{|\lambda|}x})
        \end{equation*}
        Знаходимо константи з межових умов:
        \begin{equation*}
            \begin{aligned}
                &X'(0) = C_1\sqrt{|\lambda|}
                \;\Rightarrow\;
                X(x) = C_2 &ch(\sqrt{|\lambda|}x)\\
                &\left\{ \begin{aligned}
                    &X'(l) = C_2\sqrt{|\lambda|} sh(\sqrt{|\lambda|}l) = 0, \\
                    &sh(\sqrt{|\lambda|}l) \neq 0;
                \end{aligned} \right.&\\
                &\left\{ \begin{aligned}
                    C_1 = 0, \\ 
                    C_2 = 0;
                \end{aligned} \right. \qquad\qquad\qquad\qquad&
            \end{aligned}
            \;\Rightarrow\;
            \begin{aligned}
                \text{розв'язок тривівльний,}\\
                \text{немає від'ємних}\\
                \text{власних значень.}
            \end{aligned}
        \end{equation*}

        \item Випадок $\lambda = 0$:
        \begin{equation*}
            X(x) = C_1 + C_2 x
        \end{equation*}
        Знаходимо константи з межових умов:
        \begin{equation*}
            \begin{aligned}
                &\left\{ \begin{aligned}
                    &X'(0) = C_2 = 0, \\ 
                    &X'(l) = C_2 = 0;
                \end{aligned} \right.
                \\   
                &\left\{ \begin{aligned}
                    C_1 \in \mathbb{R}, \\ 
                    C_2 = 0;
                \end{aligned} \right.
            \end{aligned}
            \quad\Rightarrow\;
            \begin{aligned}
                X(x) = C \text{ -- розв'язок нетривівльний,}\\
                \lambda = 0 \text{ є власним значенням.}
            \end{aligned}
        \end{equation*}

        \item Випадок $\lambda > 0$
        \begin{equation*}
            X(x) = C_1 \sin(\sqrt{\lambda}x) + C_2 \cos({\sqrt{\lambda}x})
        \end{equation*}
        Знаходимо константи з межових умов:
        \begin{equation*}
            \begin{aligned}
                \left\{ \begin{aligned}
                    &X'(0) = C_1\sqrt{\lambda} = 0, \\ 
                    &X(x) = C_2 \cos({\sqrt{\lambda}x}), \\
                    &X'(l) = - C_2 \sin(\sqrt{\lambda}l) = 0;
                \end{aligned} \right.
                \;\Rightarrow\;
                \left\{ \begin{aligned}
                    &C_2 \neq 0, \\ 
                    &\sin(\sqrt{\lambda}l) = 0;
                \end{aligned} \right.
            \end{aligned}
        \end{equation*}
        Отже, нетривіальні розв'язки існують при значеннях параметра $\lambda$, які задовольняють характеристичне рівняння :
        \begin{equation*}
            \sin(\sqrt{\lambda}l) = 0
            \;\Rightarrow\;
            \sqrt{\lambda_n}l = \pi n, \, n \in \mathbb{Z}
            \;\Rightarrow\;
            \lambda_n = \frac{\pi^2 n^2}{l^2}.
        \end{equation*}
    \end{enumerate}
\end{enumerate} 
Випишемо тепер розв'язки для всіх $n$, поклавши всі довільні сталі рівними $1$. Залишаємо з них лише нетривіальні розв'язки для тих $n$, які відповідають різним власним функціям:
    \begin{equation*}
        \left\{ \begin{aligned}
            &X_0(x) = 1,\\
            &\lambda_0 = 0;
        \end{aligned} \right.
        \qquad
        \left\{ \begin{aligned}
            &X_n(x) = \cos\left(\frac{\pi n x}{l}\right),\\
            &\lambda_n = \frac{\pi^2 n^2}{l^2}, n \in \mathbb{N}.
        \end{aligned} \right.
    \end{equation*}


На відміну від задачі 1 з попереднього заняття тут $n = 0$ відповідає нетривіальному розв'язку. Випадок $\lambda > 0$ знову приводить до набору власних функцій занумерованих натуральними числами.\\
Отже, різним власним функціям відповідають натуральні $n$ та $0$.\\
    Власними значеннями і власними функціями є
    \begin{equation} \label{ShLsol2,1}
        \left\{ \begin{aligned}
            &\lambda_0 = 0,\\
            &X_0(x) = 1;\\
            \;&\lambda_n = \frac{\pi^2 n^2}{l^2}, \text{ де } n \in \mathbb{N}\\ 
            &X_n(x) = \cos\left(\frac{\pi n x}{l}\right),
        \end{aligned} \right.
    \end{equation}

Повертаємося до рівняння для $T(t)$ (\ref{sepvar2,1}). Підставляємо знайдені власні значення та знаходимо $T_n(t)$:
\begin{equation*}
    \left. \begin{aligned}
        \lambda_n = \frac{\pi^2 n^2}{l^2},&\;\\ 
        T^{\prime\prime} + \lambda v^2T = 0,&
    \end{aligned} \right\}
    \;\Rightarrow\;
    T_n(t) = A\cos(\omega_n t) + B\sin(\omega_n t),
\end{equation*}
де $\omega_n^2 = \lambda_n v^2, \, n \in \mathbb{N}.$\\
\begin{equation*}
    \left. \begin{aligned}
        \lambda_0 = 0,&\;\\ 
        T^{\prime\prime} = 0,&
    \end{aligned} \right\}
    \;\Rightarrow\;
    T_0(t) = A_0 + B_0 t,
\end{equation*}
Власними модами коливань струни будуть всі розв'язки вигляду:
\begin{equation*}
    u_n(x,t) = X_n(x) \cdot T_n(t)
\end{equation*}
Виконаємо перепозначення і запишемо остаточний розв'язок:
\begin{equation}
    \left\{ \begin{aligned} \label{mode2,1}
        \;&u_0(x,t) = A_0 + B_0 t, \\
        &u_n(x,t) = \left[A_n\cos(\omega_n t) + B_n\sin(\omega_n t)\right] \sin(k_n x), \\
        &k_n = \frac{\pi n}{l} - \text{ хвильові вектори}, \\
        &\omega_n = vk_n = \frac{v \pi n}{l} - \text{ власні частоти}, \\
        &n = 1, 2,\ldots
    \end{aligned}\right.
\end{equation}

\begin{center}
    \large{\textbf{Перевірка розв'язку задачі Штурма-Ліувілля}}
\end{center}

\begin{enumerate}[wide, labelindent=0pt]
    \item Аналітична перевірка
    \begin{enumerate}
        \item[1)] Перевіряємо, чи виконуються крайові умови задачі:
        \begin{equation*}
            \begin{aligned}
                X'(0) &=  0:\\
                &\begin{aligned}
                    &X_0'(0) = 0,\\
                    &X_n'(0) = \sin(\sqrt{\lambda_n} \cdot 0) = 0 \text{ -- виконується,}\\
                    &\text{ причому незалежно від }\lambda_n
                \end{aligned}\\
                \\
                X'(l) &= 0:\\
                &\begin{aligned}
                    &X_0'(l) = 0,\\
                    &X_n'(l) = \sin\left(\frac{\pi n}{l} \cdot l\right) = \sin(\pi n) = 0 \text{ -- виконується}\\
                    &\text{ причому саме для знайдених значень }\lambda_n.
                \end{aligned}\\
            \end{aligned}
        \end{equation*}
        \item[2)] Перевіряємо, чи задовольняють знайдені функції рівняння на власні значення $X^{\prime\prime} = -\lambda X$, і якщо так, то знаходимо відповідне значення спектрального параметра $\lambda$:
        \begin{equation*}
            \begin{aligned}
                &X_0^{''} = \left(1\right)^{''} = 0 \cdot 1 = 0 \cdot X_0 \\
                &X_n^{''} = -\frac{\pi n}{l} \left(\sin\left(\frac{\pi n x}{l}\right)\right)^{'} = -\left(\frac{\pi n}{l}\right)^2 \cos\left(\frac{\pi n x}{l}\right) = -\left(\frac{\pi n}{l}\right)^2 X_n
            \end{aligned} 
        \end{equation*}
        Отже знайдені функції задовольняють і крайові умови, і рівняння задачі Штурма-Ліувілля, причому для значень спектрального параметра \begin{equation}
        \lambda_0 = 0 \text{ i }\lambda_n = \frac{\pi^2 n^2}{l^2}, \text{ де } n \in \mathbb{N}
        \end{equation}які співпадають з раніше знайденими. Звідси робимо висновок, що вказані у відповіді (\ref{ShLsol2,1}) функції та значення спектрального параметра дійсно є власними функціями і відповідними їм власними значеннями задачі Штурма-Ліувілля.
    \end{enumerate}
    \item Графічна перевірка.\\
    Будуємо графіки кількох перших власних функцій. Масштаб по вертикалі може бути довільним і різним для різних функцій, оскільки значення він не має.
    \begin{figure}[h]
        \centering
        %\large \textbf{Graph under comment}%
        \begin{tikzpicture}
    \begin{axis}
        [width = 0.85\textwidth, height = 0.4\textwidth,
         axis x line = center, axis y line = center,
         ylabel = $X(x)$, xlabel = $x$,
         xmin = -0.3, xmax = 5.7, ymin = -5.3, ymax = 7.3,
         axis line style = thin, xtick = {0}, ytick = {0}]   
        
        \tikzmath{\A1 = 5; \l = 5; \k = pi/\l;}
        
        \addplot [black, domain=0:\l, samples = 1000] {0}
        node[anchor=130, pos=0] {0} 
        node[anchor=130, pos=1] {$l$};
        
        \addplot [black, domain=0:\l, samples = 1000] {\A1}
        node[anchor=south, pos=0.55, fill=white] {$X_0(x)$}; 


        \addplot [black, domain=0:\l, samples = 1000] {\A1 * cos(deg(\k*x))}
        node[anchor=south, pos=0.31, fill=white] {$X_1(x)$}; 

        \addplot [black, domain=0:\l, samples = 1000] {\A1/2 * cos(deg(2*\k*x))}
        node[anchor=south, pos=0.15, fill=white] {$X_2(x)$};
        
    \end{axis}
\end{tikzpicture} %%for compiling main 
        %\begin{tikzpicture}
    \begin{axis}
        [width = 0.85\textwidth, height = 0.4\textwidth,
         axis x line = center, axis y line = center,
         ylabel = $X(x)$, xlabel = $x$,
         xmin = -0.3, xmax = 5.7, ymin = -5.3, ymax = 7.3,
         axis line style = thin, xtick = {0}, ytick = {0}]   
        
        \tikzmath{\A1 = 5; \l = 5; \k = pi/\l;}
        
        \addplot [black, domain=0:\l, samples = 1000] {0}
        node[anchor=130, pos=0] {0} 
        node[anchor=130, pos=1] {$l$};
        
        \addplot [black, domain=0:\l, samples = 1000] {\A1}
        node[anchor=south, pos=0.55, fill=white] {$X_0(x)$}; 


        \addplot [black, domain=0:\l, samples = 1000] {\A1 * cos(deg(\k*x))}
        node[anchor=south, pos=0.31, fill=white] {$X_1(x)$}; 

        \addplot [black, domain=0:\l, samples = 1000] {\A1/2 * cos(deg(2*\k*x))}
        node[anchor=south, pos=0.15, fill=white] {$X_2(x)$};
        
    \end{axis}
\end{tikzpicture} %%for compiling only this file
        \caption{Графічний розв'язок задачі, наведені три власні функції}        
    \end{figure}\\
     З рисунку бачимо, що дотичні до всіх графіків у точках $x = 0$ та $x = l$ горизонтальні, тобто крайові умови (\ref{sepvar2,1}) виконуються на обох кінцях проміжку. Далі перевіряємо, чи виконується осциляційна теорема. Власні функції занумеровані у нас у порядку зростання власних значень, а мінімальному власному значенню відповідає функція $X_0(x) = 1$. Як видно з рисунка, вона не має нулів всередині проміжку, як і має бути для основної моди; кожна наступна власна функція має рівно на один нуль більше. Тобто осциляційна теорема виконується. Звідси робимо висновок, що ми знайшли всі власні функції і власні значення задачі.  \\
   
\end{enumerate}

\begin{center}
    \large{\textbf{Аналіз результату}}
\end{center}
Моди $n=1,2,\ldots$ відповідають стоячим хвилям. Коливання стержня повздовжні, і тому показані на рисунку графіки власних функцій пов'язані з реальним рухом стержня не настульки очевидним чином, як для поперечних коливань струни. У процесі коливань певні частини стердня зміщуються поперемінно праворуч і ліворуч, а між ним виникають області розтягу і стиснення.  Області максимального відносного стиснення і розтягу (екстремуми похідної по координаті) припадають на вузли поля зміщень. Оскільки стежень з обома вільними кінцями симетричний відносно сердини, то власні функції почергово є або симетричними (парними), або антисиметричними (непарними) відносно середини проміжку (див. рисунок). Аналогічну картину ми спостерігали і у задачі №1.1, для поперечних коливань струни з обома закріпленими кінцями, яка теж є симетричною відносно середини. Проте, якщо врахувати векторний характер поля зміщень, то для поперечних коливань (струна) симетричній власній функції (див. рисунок вище) відповідає симетричне поле зміщень, а для повздовжніх коливань (стержнь) - антисеместричне! Нарисуйте самостійно векторне поле зміщень, яке відповідає моді $n = 1$, наприклад. \\
Основна мода у даній задачі одночасно є нульовою модою, оскільки вона відповідає нульовому власному значенню. Нульові моди, як правило, є особливими і відрізняються від інших. Так, у даній задачі нульова мода відповідає не коливанню, а стану спокою або рівномірного прямолінійного руху стержня як цілого, залежно від початкових умов, які у задачі на власні моди не задається. Кожній власній можі можна поставити у відповідність окремий ступінь вільності. Нульова мода відповідає рухові центра мас стержня, а інші - коливанням різних типів відносно нерухомого центра мас. 

%\end{document}