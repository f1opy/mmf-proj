%\documentclass[a4paper, 14pt]{extreport}

%\usepackage{StyleMMF}

%\begin{document}

%\chapter{Другий спосіб знаходження коефіцієнтів. Коливання стержня з вільними кінцями, неповнота базису.}

\section[Задача №3.1]{3.1}

\textit{Знайти коливання пружного стержня $0 \leq x \leq l$, лівий кінець якого закріплений, а правий вільний, якщо початкове відхилення $\varphi(x) = h \sin(3\pi x/2l)$, а початкова швидкість $\psi(x) = \nu_0 \sin(\pi x/2l)$.}

%\end{document}