%\documentclass[a4paper, 14pt]{extreport}

%\usepackage{StyleMMF}

%\begin{document}

%\chapter{Рівняння теплопровідності з однорідними межовими умовами}

\section[Задача №4.4]{4.4}

\textit{Початкова температура повністю теплоізольованого тонкого стержня\\ $0 \leq x \leq l$ дорівнює $T_1 \cos(\pi x/2l) + T_2 \cos(2\pi x/l)$ . Знайти поле температур при $t > 0$. Перевірити виконання початкових умови при $T_1 = 0$ і $T_2 = 0$.}

\begin{center}
    \large{\textbf{Розв'язок}}
\end{center}

\noindent Формальна постановка задачі:
\begin{equation} \label{probcond8}
    \left\{ \begin{aligned} %%
            \;&u = u(x,t), \\
            &u_t = D u_{xx}, \\
            &0 \leq x \leq l, t \geq 0, \\
            &u_x(0,t) = 0, \, u_x(l,t) = 0,\\ 
            &u(x,0) = T_1 \cos(\pi x/2l) + T_2 \cos(2\pi x/l).
    \end{aligned} \right.
\end{equation}

Виконуючи розділення змінних ми отримаємо дві попереднбо розв'язані задачі. Задачу Штурма-Ліувілля з задачі №2.1 та часове диференціальне рівняня з задачі №4.2. Отже, загальний розв'язок можна одразу записати комбінуюці відомі.

\begin{equation} \label{gen-sol8}
    u(x,t) = C_0 + \sum_{n=1}^{\infty}C_n e^{-t/\tau_n} \cos k_n x,
\end{equation}
\begin{equation*}
    \begin{aligned}
        &k_n = \frac{\pi n}{l} - \text{ хвильові вектори}, \\
        &\tau_n = \frac{1}{D k_n^2} - \text{ характерний час зміни температури}, \\
        &n = 0, 1, 2,\ldots
    \end{aligned}
\end{equation*}

З початковї умови визначимо невіомі коефіцієнти. Для цього треба розкласти $\cos(\pi x/2l)$ по набору власних функцій задачі Ш.-Л. 
\begin{equation*}
    \begin{gathered}
        \cos(\pi x/2l) = a_0 + \sum_{n=1}^{\infty} a_n \cos k_nx \\
        a_0 = \frac{1}{l}\int\limits_0^l \cos(\pi x/2l) \;\mathrm{d}x = \frac{2}{\pi} \sin(\pi x/2l) \bigg|_0^l = \frac{2}{\pi}\\
        a_n = \frac{2}{l}\int\limits_0^l \cos(\pi x/2l)\cos k_nx \;\mathrm{d}x = \frac{1}{l}\bigg(\int\limits_0^l \cos((k_n + \pi/2l)x) \;\mathrm{d}x +\\
        + \int\limits_0^l \cos((k_n - \pi/2l)x) \;\mathrm{d}x\bigg) = \frac{1}{l}\left(\frac{\sin((k_n + \pi/2l)x)}{k_n + \pi/2l}\bigg|_0^l + \frac{\sin((k_n - \pi/2l)x)}{k_n - \pi/2l}\bigg|_0^l\right) =\\
        = \left(\frac{\sin(k_nl + \pi/2)}{k_nl + \pi/2} + \frac{\sin(k_nl - \pi/2)}{k_nl - \pi/2}\right) = \left(\frac{1}{k_nl + \pi/2} - \frac{1}{k_nl - \pi/2}\right)\cos k_nl =\\
        = \big|\cos k_nl = (-1)^n\big| = \frac{(-1)^{n+1} \pi}{(k_nl + \pi/2)(k_nl - \pi/2)} =\\
        = (-1)^{n+1} \cdot \frac{4\pi}{4k_n^2l^2 - \pi^2} = \frac{4}{\pi} \cdot \frac{(-1)^{n+1}}{4n^2 - 1}
    \end{gathered}
\end{equation*}
Тепер підставимо (\ref{gen-sol8}) в початкову умову і отримаємо:
\begin{equation}
    \begin{gathered}
        u(x,0) = C_0 + \sum_{n=1}^{\infty}C_n \cos k_n x = T_1 \cos(\pi x/2l) + T_2 \cos(2\pi x/l) =\\
        = \frac{2T_1}{\pi} + T_2 \cos k_2x + \frac{4T_1}{\pi} \sum_{n=1}^{\infty} \frac{(-1)^{n+1} \cos k_nx}{4n^2 - 1}
    \end{gathered}
\end{equation} 
З чого слідує 
\begin{equation*}
    C_0 = \frac{2T_1}{\pi},\, C_2 = T_2 - \frac{4T_1}{15\pi},\, C_n = \frac{4T_1}{\pi} \cdot \frac{(-1)^{n+1}}{4n^2 - 1}, \text{ де } n \neq 2
\end{equation*}
Отже, остаточним розв'язком буде 
\begin{equation}
    u(x,t) = \frac{2T_1}{\pi} + T_2 e^{-t/\tau_2}\cos k_2x + \frac{4T_1}{\pi} \sum_{n=1}^{\infty} \frac{(-1)^{n+1} \cos k_nx}{4n^2 - 1}
\end{equation}

Прямою підстановкою можна переконатися, що при $T_1 = 0$ та $T_2 = 0$ початкові умови виконуються.


%\end{document}