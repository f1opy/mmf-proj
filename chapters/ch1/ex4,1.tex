\documentclass[a4paper, 14pt]{extreport}

\usepackage{StyleMMF}

\usetikzlibrary{fillbetween}

\begin{document}

\setcounter{chapter}{3}

\chapter{Рівняння теплопровідності з однорідними межовими умовами}

\section[Задача №4.1]{4.1}

\textit{Одну і ту ж функцію, наприклад $f(x) = \alpha x$, можна представити на проміжку $0 \leq x \leq l$ узагальненим рядом Фур’є по кожній із систем власних функцій чотирьох задач Штурма-Ліувілля, одержаних у задачах 1.1, 1.2, 1.3, 2.1. Користуючись явним виглядом власних функцій і не обчислюючи коєфіцієнтів рядів, дайте відповіді на такі запитання.
\begin{enumerate}
    \item Який вигляд матиме графік суми кожного з таких рядів на всій числовій осі? Якою є парність суми ряду відносно точок $x = nl$, де $n$ – ціле число, і як це пов’язано з виглядом крайових умов задачі Штурма-Ліувілля?
    \item Покажіть, що кожний з рядів є частинним випадком класичного тригонометричного ряду Фур’є, сума якого є періодичною функцією. Які саме періоди відповідають кожному з рядів? Яка саме частина повного тригонометричного базису використовується в кожному з розкладань, а які коефіцієнти Фур’є дорівнюють нулю і чому?
    \item Як пов’язаний характер збіжності вказаних рядів з крайовими умовами, які задовольняє функція $f(x)$ у точках $x = 0, l$ ? Чи дорівнює сума ряду Фур’є функції $f(x)$ на відкритому проміжку $0 < x < l$? на закритому проміжку
    $0 \leq x \leq l$
\end{enumerate}}

\begin{center}
    \large \textbf{Розв'язок}
\end{center}
\begin{figure}
    \begin{tikzpicture}
        \begin{axis}
            [width = \textwidth, height = 0.7\textwidth,
             axis x line = center, axis y line = center,
             ylabel = $X(x)$, xlabel = $x$,
             xmin = -19, xmax = 19, ymin = -3, ymax = 3,
             axis line style = thin, xtick = {0}, ytick = {0}]   
            
            \tikzmath{\l = 5; \alp = 2/5; \k = pi/\l;}
            
            \addplot[gray, dashed, samples=50, domain=-20:20, name path=three] coordinates {(0,-3)(0,3)}
            node[anchor=130, pos=0.5] {\footnotesize\textcolor{black}{0}};

            \addplot[gray, dashed, samples=50, domain=-20:20, name path=three] coordinates {(\l,-3)(\l,3)}
            node[anchor=130, pos=0.5] {\footnotesize$\textcolor{black}{l}$};
            \addplot[gray, dashed, samples=50, domain=-20:20, name path=three] coordinates {(2*\l,-3)(2*\l,3)}
            node[anchor=130, pos=0.5] {\footnotesize$\textcolor{black}{2l}$};
            \addplot[gray, dashed, samples=50, domain=-20:20, name path=three] coordinates {(3*\l,-3)(3*\l,3)}
            node[anchor=130, pos=0.5] {\footnotesize$\textcolor{black}{3l}$};
            
            \addplot[gray, dashed, samples=50, domain=-20:20, name path=three] coordinates {(-\l,-3)(-\l,3)}
            node[anchor=100, pos=0.5] {\footnotesize$\textcolor{black}{-l}$};
            \addplot[gray, dashed, samples=50, domain=-20:20, name path=three] coordinates {(-2*\l,-3)(-2*\l,3)}
            node[anchor=100, pos=0.5] {\footnotesize$\textcolor{black}{-2l}$};
            \addplot[gray, dashed, samples=50, domain=-20:20, name path=three] coordinates {(-3*\l,-3)(-3*\l,3)}
            node[anchor=100, pos=0.5] {\footnotesize$\textcolor{black}{-3l}$};
            
            \addplot [black, domain=0:\l, samples = 1000] {sin(deg(\k*x))};


            \addplot [red, thick, domain=0:0.99*\l, samples=150] {\alp*x};
            \node (mark) [draw, red, fill=red, circle, minimum size = 2pt, inner sep=0.5pt] at (axis cs: 0, 0) {};
            \node (mark) [draw, red, circle, minimum size = 2pt, inner sep=0.5pt] at (axis cs: \l, \alp*\l) {};
            \node (mark) [draw, red, fill=red, circle, minimum size = 2pt, inner sep=0.5pt] at (axis cs: \l, 0) {};

            \addplot [red, dashed, thick, domain=-0.99*\l:0, samples=150] {\alp*x};
            \node (mark) [draw, red, circle, minimum size = 2pt, inner sep=0.5pt] at (axis cs: -\l, -\alp*\l) {};
            \node (mark) [draw, red, fill=red, circle, minimum size = 2pt, inner sep=0.5pt] at (axis cs: -\l, 0) {};

            
            \addplot [black, thick, dashed, domain=1.01*\l:3*0.998*\l, samples=150] {\alp*(x - 2*\l)};
            \node (mark) [draw, black, circle, minimum size = 2pt, inner sep=0.5pt] at (axis cs: \l, -\alp*\l) {};
            %\node (mark) [draw, black, fill=black, circle, minimum size = 2pt, inner sep=0.5pt] at (axis cs: 2*\l, 0) {};
            \node (mark) [draw, black, circle, minimum size = 2pt, inner sep=0.5pt] at (axis cs: 3*\l, \alp*\l) {};
            
            \addplot [black, thick, dashed, domain=-3*0.998*\l:-0.99*\l, samples=150] {\alp*(x + 2*\l)};
            \node (mark) [draw, black, circle, minimum size = 2pt, inner sep=0.5pt] at (axis cs: -3*\l, -\alp*\l) {};
            %\node (mark) [draw, black, fill=black, circle, minimum size = 2pt, inner sep=0.5pt] at (axis cs: -2*\l, 0) {};
            \node (mark) [draw, black, circle, minimum size = 2pt, inner sep=0.5pt] at (axis cs: -\l, \alp*\l) {};
            

            %node[anchor=130, pos=0] {0} 
            %node[pos=0.75, fill=white] {$X_1(x)$} 
            %node[anchor=130, pos=1] {$l$};
            
            %\addplot [black, domain=0:\l, samples = 1000] {\A1/2.5 * sin(deg(2*\k*x))}
            %node[pos=0.83, fill=white] {$X_2(x)$};
            
            %\addplot [black, domain=0:\l, samples = 1000] {\A1/3.7 * sin(deg(3*\k*x))}
            %node[pos=0.76, fill=white] {$X_3(x)$};
            
        \end{axis}
    \end{tikzpicture}
    \caption{Розклад по системі власних функцій із задачі 1.1}
\end{figure}



\end{document}