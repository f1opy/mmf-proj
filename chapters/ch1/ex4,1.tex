%\documentclass[a4paper, 14pt]{extreport}

%\usepackage{StyleMMF}

%\begin{document}

%\setcounter{chapter}{3}
%\chapter{Рівняння теплопровідності з однорідними межовими умовами}

\section[Задача №4.1]{4.1}

\textit{Одну і ту ж функцію, наприклад $f(x) = \alpha x$, можна представити на проміжку $0 \leq x \leq l$ узагальненим рядом Фур’є по кожній із систем власних функцій чотирьох задач Штурма-Ліувілля, одержаних у задачах 1.1, 1.2, 1.3, 2.1. Користуючись явним виглядом власних функцій і не обчислюючи коєфіцієнтів рядів, дайте відповіді на такі запитання.
\begin{enumerate}
    \item Який вигляд матиме графік суми кожного з таких рядів на всій числовій осі? Якою є парність суми ряду відносно точок $x = nl$, де $n$ – ціле число, і як це пов’язано з виглядом крайових умов задачі Штурма-Ліувілля?
    \item Покажіть, що кожний з рядів є частинним випадком класичного тригонометричного ряду Фур’є, сума якого є періодичною функцією. Які саме періоди відповідають кожному з рядів? Яка саме частина повного тригонометричного базису використовується в кожному з розкладань, а які коефіцієнти Фур’є дорівнюють нулю і чому?
    \item Як пов’язаний характер збіжності вказаних рядів з крайовими умовами, які задовольняє функція $f(x)$ у точках $x = 0, l$ ? Чи дорівнює сума ряду Фур’є функції $f(x)$ на відкритому проміжку $0 < x < l$? на закритому проміжку
    $0 \leq x \leq l$
\end{enumerate}}

\begin{center}
    \large \textbf{Розв'язок}
\end{center}

\subsection*{Запитання №1} 
Графік суми ряду буде періодично повторювати розкладену в ряд Фур'є функцію з періодом $l$. Відносно точок $x = nl$ сума ряду буде або парною, або непарною, функцією. Для задачі, де кінці струни закріплені, відносто всіх цих точок буде непарною. Це випливає із умов при яких можливий розклад в ряд Фур'є -- кусково-гладкість та періодичність функції, тобто щоб задовольняти всім цим умовам і отримували в точках $x = nl$ значення функції рівним нулю, необхідно щоб функція була непарна відносно ціє точки. Аналогічна, ситуація для межової умови, де кінець струни вільний ($X'(nl) = 0$), тільки треба згадати, що похідна змінює парність функції (звісно якщо функція має первну парність), тому відносно вільних кінців сума ряду буде парною функцією.     

\subsection*{Запитання №2}


\subsection*{Запитання №3} 
Графіки суми рядів Фур'є буде періодичною функцією, що на відкритому проміжку $(0,l)$ співпадає з нашою функцією, а на в точках може бути розрив першого роду.


%% малюнки треба якось розподілити по тексту задачі 
\begin{figure} \label{fourier1}
    %\begin{tikzpicture}
    \begin{axis}
        [width = \textwidth, height = 0.7\textwidth,
         axis x line = center, axis y line = center,
         ylabel = $X(x)$, xlabel = $x$,
         xmin = -19, xmax = 19, ymin = -3, ymax = 3,
         axis line style = thin, xtick = {0}, ytick = {0}]   
        
        \tikzmath{\l = 5; \alp = 2/5;}
        
        \addplot[gray, dashed, samples=50, domain=-20:20, name path=three] coordinates {(0,-3)(0,3)}
        node[anchor=130, pos=0.5] {\footnotesize\textcolor{black}{0}};

        % вертикальні пунктирні лінії для x > 0
        \addplot[gray, dashed, samples=50, domain=-20:20, name path=three] coordinates {(\l,-3)(\l,3)}
        node[anchor=130, pos=0.5] {\footnotesize$\textcolor{black}{l}$};
        \addplot[gray, dashed, samples=50, domain=-20:20, name path=three] coordinates {(2*\l,-3)(2*\l,3)}
        node[anchor=130, pos=0.5] {\footnotesize$\textcolor{black}{2l}$};
        \addplot[gray, dashed, samples=50, domain=-20:20, name path=three] coordinates {(3*\l,-3)(3*\l,3)}
        node[anchor=130, pos=0.5] {\footnotesize$\textcolor{black}{3l}$};
        
        % вертикальні пунктирні лінії для x < 0
        \addplot[gray, dashed, samples=50, domain=-20:20, name path=three] coordinates {(-\l,-3)(-\l,3)}
        node[anchor=100, pos=0.5] {\footnotesize$\textcolor{black}{-l}$};
        \addplot[gray, dashed, samples=50, domain=-20:20, name path=three] coordinates {(-2*\l,-3)(-2*\l,3)}
        node[anchor=100, pos=0.5] {\footnotesize$\textcolor{black}{-2l}$};
        \addplot[gray, dashed, samples=50, domain=-20:20, name path=three] coordinates {(-3*\l,-3)(-3*\l,3)}
        node[anchor=100, pos=0.5] {\footnotesize$\textcolor{black}{-3l}$};
        
        % функція, яку розкладаємо
        \addplot [red, thick, domain=0:0.99*\l, samples=150] {\alp*x};
        % точки закріплення струни
        \node (mark) [draw, red, fill=red, circle, minimum size = 2pt, inner sep=0.5pt] at (axis cs: 0, 0) {};
        \node (mark) [draw, red, fill=red, circle, minimum size = 2pt, inner sep=0.5pt] at (axis cs: \l, 0) {};
        % виколата точка (l,αl)
        \node (mark) [draw, red, circle, minimum size = 2pt, inner sep=0.5pt] at (axis cs: \l, \alp*\l) {};

        % непарне продовження відносно нуля
        \addplot [red, dashed, thick, domain=-0.99*\l:0, samples=150] {\alp*x};
        % виколата точка (-l,-αl)
        \node (mark) [draw, red, circle, minimum size = 2pt, inner sep=0.5pt] at (axis cs: -\l, -\alp*\l) {};
        % ізольована точка 
        \node (mark) [draw, black, fill=black, circle, minimum size = 2pt, inner sep=0.5pt] at (axis cs: -\l, 0) {};

        % непарне продовження відносно точки x = l 
        \addplot [black, thick, dashed, domain=1.01*\l:3*0.998*\l, samples=150] {\alp*(x - 2*\l)};
        % виколоті точки
        \node (mark) [draw, black, circle, minimum size = 2pt, inner sep=0.5pt] at (axis cs: \l, -\alp*\l) {};
        \node (mark) [draw, black, circle, minimum size = 2pt, inner sep=0.5pt] at (axis cs: 3*\l, \alp*\l) {};
        % ізольована точка
        \node (mark) [draw, black, fill=black, circle, minimum size = 2pt, inner sep=0.5pt] at (axis cs: 3*\l, 0) {};
        
        % непарне продовження відносно точки x = -l 
        \addplot [black, thick, dashed, domain=-3*0.998*\l:-0.99*\l, samples=150] {\alp*(x + 2*\l)};
        \node (mark) [draw, black, circle, minimum size = 2pt, inner sep=0.5pt] at (axis cs: -3*\l, -\alp*\l) {};
        \node (mark) [draw, black, fill=black, circle, minimum size = 2pt, inner sep=0.5pt] at (axis cs: -3*\l, 0) {};
        \node (mark) [draw, black, circle, minimum size = 2pt, inner sep=0.5pt] at (axis cs: -\l, \alp*\l) {};
        
        \tikzmath{\k = pi/\l; \A = 2*\alp/\k;}

        \addplot [black, domain=-19:19, samples = 1500] {\A*sin(deg(\k*x))};

    \end{axis}
\end{tikzpicture} %% this file compilation
    \begin{tikzpicture}
    \begin{axis}
        [width = \textwidth, height = 0.7\textwidth,
         axis x line = center, axis y line = center,
         ylabel = $X(x)$, xlabel = $x$,
         xmin = -19, xmax = 19, ymin = -3, ymax = 3,
         axis line style = thin, xtick = {0}, ytick = {0}]   
        
        \tikzmath{\l = 5; \alp = 2/5;}
        
        \addplot[gray, dashed, samples=50, domain=-20:20, name path=three] coordinates {(0,-3)(0,3)}
        node[anchor=130, pos=0.5] {\footnotesize\textcolor{black}{0}};

        % вертикальні пунктирні лінії для x > 0
        \addplot[gray, dashed, samples=50, domain=-20:20, name path=three] coordinates {(\l,-3)(\l,3)}
        node[anchor=130, pos=0.5] {\footnotesize$\textcolor{black}{l}$};
        \addplot[gray, dashed, samples=50, domain=-20:20, name path=three] coordinates {(2*\l,-3)(2*\l,3)}
        node[anchor=130, pos=0.5] {\footnotesize$\textcolor{black}{2l}$};
        \addplot[gray, dashed, samples=50, domain=-20:20, name path=three] coordinates {(3*\l,-3)(3*\l,3)}
        node[anchor=130, pos=0.5] {\footnotesize$\textcolor{black}{3l}$};
        
        % вертикальні пунктирні лінії для x < 0
        \addplot[gray, dashed, samples=50, domain=-20:20, name path=three] coordinates {(-\l,-3)(-\l,3)}
        node[anchor=100, pos=0.5] {\footnotesize$\textcolor{black}{-l}$};
        \addplot[gray, dashed, samples=50, domain=-20:20, name path=three] coordinates {(-2*\l,-3)(-2*\l,3)}
        node[anchor=100, pos=0.5] {\footnotesize$\textcolor{black}{-2l}$};
        \addplot[gray, dashed, samples=50, domain=-20:20, name path=three] coordinates {(-3*\l,-3)(-3*\l,3)}
        node[anchor=100, pos=0.5] {\footnotesize$\textcolor{black}{-3l}$};
        
        % функція, яку розкладаємо
        \addplot [red, thick, domain=0:0.99*\l, samples=150] {\alp*x};
        % точки закріплення струни
        \node (mark) [draw, red, fill=red, circle, minimum size = 2pt, inner sep=0.5pt] at (axis cs: 0, 0) {};
        \node (mark) [draw, red, fill=red, circle, minimum size = 2pt, inner sep=0.5pt] at (axis cs: \l, 0) {};
        % виколата точка (l,αl)
        \node (mark) [draw, red, circle, minimum size = 2pt, inner sep=0.5pt] at (axis cs: \l, \alp*\l) {};

        % непарне продовження відносно нуля
        \addplot [red, dashed, thick, domain=-0.99*\l:0, samples=150] {\alp*x};
        % виколата точка (-l,-αl)
        \node (mark) [draw, red, circle, minimum size = 2pt, inner sep=0.5pt] at (axis cs: -\l, -\alp*\l) {};
        % ізольована точка 
        \node (mark) [draw, black, fill=black, circle, minimum size = 2pt, inner sep=0.5pt] at (axis cs: -\l, 0) {};

        % непарне продовження відносно точки x = l 
        \addplot [black, thick, dashed, domain=1.01*\l:3*0.998*\l, samples=150] {\alp*(x - 2*\l)};
        % виколоті точки
        \node (mark) [draw, black, circle, minimum size = 2pt, inner sep=0.5pt] at (axis cs: \l, -\alp*\l) {};
        \node (mark) [draw, black, circle, minimum size = 2pt, inner sep=0.5pt] at (axis cs: 3*\l, \alp*\l) {};
        % ізольована точка
        \node (mark) [draw, black, fill=black, circle, minimum size = 2pt, inner sep=0.5pt] at (axis cs: 3*\l, 0) {};
        
        % непарне продовження відносно точки x = -l 
        \addplot [black, thick, dashed, domain=-3*0.998*\l:-0.99*\l, samples=150] {\alp*(x + 2*\l)};
        \node (mark) [draw, black, circle, minimum size = 2pt, inner sep=0.5pt] at (axis cs: -3*\l, -\alp*\l) {};
        \node (mark) [draw, black, fill=black, circle, minimum size = 2pt, inner sep=0.5pt] at (axis cs: -3*\l, 0) {};
        \node (mark) [draw, black, circle, minimum size = 2pt, inner sep=0.5pt] at (axis cs: -\l, \alp*\l) {};
        
        \tikzmath{\k = pi/\l; \A = 2*\alp/\k;}

        \addplot [black, domain=-19:19, samples = 1500] {\A*sin(deg(\k*x))};

    \end{axis}
\end{tikzpicture} %% main compilation
    \caption{Аналітичне продовження функції $\alpha x$ для розкладу системі власних функцій із задачі 1.1}
\end{figure}

Червоним позначена функція, яку ми розкладуємо в ряд Фур'є. Пунктина лінія, знову червона, показує відповідне симетрійне продовження відносно нуля. Чорною пунктирною лінією позначено симетрійні відображення відносно інших точкок.

\begin{figure} \label{fourier2}
    %\begin{tikzpicture}
    \begin{axis}
        [width = \textwidth, height = 0.7\textwidth,
         axis x line = center, axis y line = center,
         ylabel = $X(x)$, xlabel = $x$,
         xmin = -19, xmax = 19, ymin = -3, ymax = 3,
         axis line style = thin, xtick = {0}, ytick = {0}]   
        
        \tikzmath{\l = 5; \alp = 2/5;}
        
        \addplot[gray, dashed, samples=50, domain=-20:20, name path=three] coordinates {(0,-3)(0,3)}
        node[anchor=130, pos=0.5] {\footnotesize\textcolor{black}{0}};

        % вертикальні пунктирні лінії для x > 0
        \addplot[gray, dashed, samples=50, domain=-20:20, name path=three] coordinates {(\l,-3)(\l,3)}
        node[anchor=130, pos=0.5] {\footnotesize$\textcolor{black}{l}$};
        \addplot[gray, dashed, samples=50, domain=-20:20, name path=three] coordinates {(2*\l,-3)(2*\l,3)}
        node[anchor=130, pos=0.5] {\footnotesize$\textcolor{black}{2l}$};
        \addplot[gray, dashed, samples=50, domain=-20:20, name path=three] coordinates {(3*\l,-3)(3*\l,3)}
        node[anchor=130, pos=0.5] {\footnotesize$\textcolor{black}{3l}$};
        
        % вертикальні пунктирні лінії для x < 0
        \addplot[gray, dashed, samples=50, domain=-20:20, name path=three] coordinates {(-\l,-3)(-\l,3)}
        node[anchor=100, pos=0.5] {\footnotesize$\textcolor{black}{-l}$};
        \addplot[gray, dashed, samples=50, domain=-20:20, name path=three] coordinates {(-2*\l,-3)(-2*\l,3)}
        node[anchor=100, pos=0.5] {\footnotesize$\textcolor{black}{-2l}$};
        \addplot[gray, dashed, samples=50, domain=-20:20, name path=three] coordinates {(-3*\l,-3)(-3*\l,3)}
        node[anchor=100, pos=0.5] {\footnotesize$\textcolor{black}{-3l}$};
        
        % функція, яку розкладаємо
        \addplot [red, thick, domain=0:0.99*\l, samples=150] {\alp*x};
        
        % парне продовження відносно нуля
        \addplot [red, dashed, thick, domain=-0.99*\l:0, samples=150] {-\alp*x};
        
        % парне продовження відносно точки x = l 
        \addplot [black, thick, dashed, domain=\l:2*\l, samples=150] {-\alp*(x - 2*\l)};
        \addplot [black, thick, dashed, domain=2*\l:3*\l, samples=150] {\alp*(x - 2*\l)};
        
        \node (mark) [draw, black, fill=black, circle, minimum size = 2pt, inner sep=0.5pt] at (axis cs: 3*\l, 0) {};
        
        % парне продовження відносно точки x = -l 
        \addplot [black, thick, dashed, domain=-2*\l:-1*\l, samples=150] {\alp*(x + 2*\l)};
        \addplot [black, thick, dashed, domain=-3*\l:-2*\l, samples=150] {-\alp*(x + 2*\l)};
        
        \tikzmath{\k = pi/\l; \A = 4*\alp*\l/pi^2;}

        \addplot [black, domain=-19:19, samples = 1000] {\alp*\l/2};
        \addplot [black, domain=-19:19, samples = 1500] {\A*cos(deg(\k*x))};
        
    \end{axis}
\end{tikzpicture}
 %% this file compilation
    \begin{tikzpicture}
    \begin{axis}
        [width = \textwidth, height = 0.7\textwidth,
         axis x line = center, axis y line = center,
         ylabel = $X(x)$, xlabel = $x$,
         xmin = -19, xmax = 19, ymin = -3, ymax = 3,
         axis line style = thin, xtick = {0}, ytick = {0}]   
        
        \tikzmath{\l = 5; \alp = 2/5;}
        
        \addplot[gray, dashed, samples=50, domain=-20:20, name path=three] coordinates {(0,-3)(0,3)}
        node[anchor=130, pos=0.5] {\footnotesize\textcolor{black}{0}};

        % вертикальні пунктирні лінії для x > 0
        \addplot[gray, dashed, samples=50, domain=-20:20, name path=three] coordinates {(\l,-3)(\l,3)}
        node[anchor=130, pos=0.5] {\footnotesize$\textcolor{black}{l}$};
        \addplot[gray, dashed, samples=50, domain=-20:20, name path=three] coordinates {(2*\l,-3)(2*\l,3)}
        node[anchor=130, pos=0.5] {\footnotesize$\textcolor{black}{2l}$};
        \addplot[gray, dashed, samples=50, domain=-20:20, name path=three] coordinates {(3*\l,-3)(3*\l,3)}
        node[anchor=130, pos=0.5] {\footnotesize$\textcolor{black}{3l}$};
        
        % вертикальні пунктирні лінії для x < 0
        \addplot[gray, dashed, samples=50, domain=-20:20, name path=three] coordinates {(-\l,-3)(-\l,3)}
        node[anchor=100, pos=0.5] {\footnotesize$\textcolor{black}{-l}$};
        \addplot[gray, dashed, samples=50, domain=-20:20, name path=three] coordinates {(-2*\l,-3)(-2*\l,3)}
        node[anchor=100, pos=0.5] {\footnotesize$\textcolor{black}{-2l}$};
        \addplot[gray, dashed, samples=50, domain=-20:20, name path=three] coordinates {(-3*\l,-3)(-3*\l,3)}
        node[anchor=100, pos=0.5] {\footnotesize$\textcolor{black}{-3l}$};
        
        % функція, яку розкладаємо
        \addplot [red, thick, domain=0:0.99*\l, samples=150] {\alp*x};
        
        % парне продовження відносно нуля
        \addplot [red, dashed, thick, domain=-0.99*\l:0, samples=150] {-\alp*x};
        
        % парне продовження відносно точки x = l 
        \addplot [black, thick, dashed, domain=\l:2*\l, samples=150] {-\alp*(x - 2*\l)};
        \addplot [black, thick, dashed, domain=2*\l:3*\l, samples=150] {\alp*(x - 2*\l)};
        
        \node (mark) [draw, black, fill=black, circle, minimum size = 2pt, inner sep=0.5pt] at (axis cs: 3*\l, 0) {};
        
        % парне продовження відносно точки x = -l 
        \addplot [black, thick, dashed, domain=-2*\l:-1*\l, samples=150] {\alp*(x + 2*\l)};
        \addplot [black, thick, dashed, domain=-3*\l:-2*\l, samples=150] {-\alp*(x + 2*\l)};
        
        \tikzmath{\k = pi/\l; \A = 4*\alp*\l/pi^2;}

        \addplot [black, domain=-19:19, samples = 1000] {\alp*\l/2};
        \addplot [black, domain=-19:19, samples = 1500] {\A*cos(deg(\k*x))};
        
    \end{axis}
\end{tikzpicture}
 %% main compilation
    \caption{Аналітичне продовження функції $\alpha x$ для розкладу системі власних функцій із задачі 2.1}
\end{figure}

%\end{document}