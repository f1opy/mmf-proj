\documentclass[a4paper, 14pt]{extreport}

\usepackage{StyleMMF}

\begin{document}

\section{Рівняння теплопровідності з однорідними межовими умовами}

\subsubsection{Задача №1}

\textit{Одну і ту ж функцію, наприклад $f(x) = \alpha x$, можна представити на проміжку $0 \leq x \leq l$ узагальненим рядом Фур’є по кожній із систем власних функцій чотирьох задач Штурма-Ліувілля, одержаних у задачах 1.1, 1.2, 1.3, 2.1. Користуючись явним виглядом власних функцій і не обчислюючи коєфіцієнтів рядів, дайте відповіді на такі запитання.
\begin{enumerate}
    \item Який вигляд матиме графік суми кожного з таких рядів на всій числовій осі? Якою є парність суми ряду відносно точок $x = nl$, де $n$ – ціле число, і як це пов’язано з виглядом крайових умов задачі Штурма-Ліувілля?
    \item Покажіть, що кожний з рядів є частинним випадком класичного тригонометричного ряду Фур’є, сума якого є періодичною функцією. Які саме періоди відповідають кожному з рядів? Яка саме частина повного тригонометричного базису використовується в кожному з розкладань, а які коефіцієнти Фур’є дорівнюють нулю і чому?
    \item Як пов’язаний характер збіжності вказаних рядів з крайовими умовами, які задовольняє функція $f(x)$ у точках $x = 0, l$ ? Чи дорівнює сума ряду Фур’є функції $f(x)$ на відкритому проміжку $0 < x < l$? на закритому проміжку
    $0 \leq x \leq l$
\end{enumerate}}


\end{document}