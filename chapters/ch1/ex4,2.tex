%\documentclass[a4paper, 14pt]{extreport}

%\usepackage{StyleMMF}

%\begin{document}

%\setcounter{chapter}{3}
%\chapter{Рівняння теплопровідності з однорідними межовими умовами}

\clearpage

\section[Задача №4.2]{4.2}

\textit{У початковий момент часу ліва половина стержня з теплоізольованою бічною поверхнею має температуру $T_1$ , а права -- температуру $T_2$ . Знайти розподіл температури при $t> 0$, якщо кінці стержня підтримуються при температурі $T_0$. Указівка: подумайте, що означає «температура дорівнює нулю», що це за нуль? Покладіть у кінцевому результаті $T_0 = 0$ і розгляньте частинні випадки: $T_1 = T_2$ та $T_1 = -T_2$. Які члени ряду при цьому обертаються в нуль? Чому? Нарисуйте графіки та порівняйте часову залежність температури для     різних мод. Нарисуйте (якісно) графіки розподілу     температури вдовж стержня у різні характерні послідовні моменти часу. Що таке «малий» і «великий» проміжок часу для цієї задачі? Як характерні часи залежать від розмірів системи?}

\begin{center}
    \textbf{Розв'язок}
\end{center}

\begin{wrapfigure}{r}{0.31\textwidth}
    \centering
    %Graph under comment
    \begin{tikzpicture}
    \begin{axis}
        [width = 0.35\textwidth, height = 0.3\textwidth,
         axis x line = center, axis y line = center,
         ylabel = $T(x)$, xlabel = $x$,
         xmin = -0.3, xmax = 3.6, ymin = -0.5, ymax = 3.1,
         axis line style = thin, xtick = {0}, ytick = {0}]   
        
        \tikzmath{\T2 = 2.5; \T1 = 2*\T2/3; \T0 = \T2/3; \l = 3;}
        
        \addplot [black, domain=0:\l, samples = 1000] {0}
        node[anchor=130, pos=0] {0} 
        node[anchor=90, pos=0.5] {$l/2$} 
        node[anchor=130, pos=1] {$l$};
        
        % вертикальні пунктирні лінії для x > 0
        \addplot[black, dashed, samples=50, domain=-0.2:3.3, name path=three] coordinates {(0,0)(0,\T2)}
        node[anchor=130, pos=1/3] {\footnotesize$T_0$}
        node[anchor=130, pos=2/3] {\footnotesize$T_1$}
        node[anchor=130, pos=1] {\footnotesize$T_2$};
        \addplot[black, dashed, samples=50, domain=-0.2:3.3, name path=three] coordinates {(\l,0)(\l,\T2)};
        \addplot[black, dashed, samples=50, domain=-0.2:3.3, name path=three] coordinates {(\l/2,0)(\l/2,\T2)};


        \addplot [black, domain=0.02:\l/2, samples = 300] {\T1};
        \addplot [black, domain=\l/2-0.02:\l-0.02, samples = 300] {\T2};


        \node (mark) [draw, black, fill=black, circle, minimum size = 2pt, inner sep=0.5pt] at (axis cs: 0, \T0) {};
        \node (mark) [draw, black, circle, minimum size = 2pt, inner sep=0.5pt] at (axis cs: 0, \T0) {};
        \node (mark) [draw, black, circle, minimum size = 2pt, inner sep=0.5pt] at (axis cs: 0, \T1) {};
        \node (mark) [draw, black, fill=black, circle, minimum size = 2pt, inner sep=0.5pt] at (axis cs: \l/2, \T1) {};
        \node (mark) [draw, black, circle, minimum size = 2pt, inner sep=0.5pt] at (axis cs: \l/2, \T2) {};
        \node (mark) [draw, black, circle, minimum size = 2pt, inner sep=0.5pt] at (axis cs: \l, \T2) {};
        \node (mark) [draw, black, fill=black, circle, minimum size = 2pt, inner sep=0.5pt] at (axis cs: \l, \T0) {};
    \end{axis}
\end{tikzpicture}  % main compilation
    %\begin{tikzpicture}
    \begin{axis}
        [width = 0.35\textwidth, height = 0.3\textwidth,
         axis x line = center, axis y line = center,
         ylabel = $T(x)$, xlabel = $x$,
         xmin = -0.3, xmax = 3.6, ymin = -0.5, ymax = 3.1,
         axis line style = thin, xtick = {0}, ytick = {0}]   
        
        \tikzmath{\T2 = 2.5; \T1 = 2*\T2/3; \T0 = \T2/3; \l = 3;}
        
        \addplot [black, domain=0:\l, samples = 1000] {0}
        node[anchor=130, pos=0] {0} 
        node[anchor=90, pos=0.5] {$l/2$} 
        node[anchor=130, pos=1] {$l$};
        
        % вертикальні пунктирні лінії для x > 0
        \addplot[black, dashed, samples=50, domain=-0.2:3.3, name path=three] coordinates {(0,0)(0,\T2)}
        node[anchor=130, pos=1/3] {\footnotesize$T_0$}
        node[anchor=130, pos=2/3] {\footnotesize$T_1$}
        node[anchor=130, pos=1] {\footnotesize$T_2$};
        \addplot[black, dashed, samples=50, domain=-0.2:3.3, name path=three] coordinates {(\l,0)(\l,\T2)};
        \addplot[black, dashed, samples=50, domain=-0.2:3.3, name path=three] coordinates {(\l/2,0)(\l/2,\T2)};


        \addplot [black, domain=0.02:\l/2, samples = 300] {\T1};
        \addplot [black, domain=\l/2-0.02:\l-0.02, samples = 300] {\T2};


        \node (mark) [draw, black, fill=black, circle, minimum size = 2pt, inner sep=0.5pt] at (axis cs: 0, \T0) {};
        \node (mark) [draw, black, circle, minimum size = 2pt, inner sep=0.5pt] at (axis cs: 0, \T0) {};
        \node (mark) [draw, black, circle, minimum size = 2pt, inner sep=0.5pt] at (axis cs: 0, \T1) {};
        \node (mark) [draw, black, fill=black, circle, minimum size = 2pt, inner sep=0.5pt] at (axis cs: \l/2, \T1) {};
        \node (mark) [draw, black, circle, minimum size = 2pt, inner sep=0.5pt] at (axis cs: \l/2, \T2) {};
        \node (mark) [draw, black, circle, minimum size = 2pt, inner sep=0.5pt] at (axis cs: \l, \T2) {};
        \node (mark) [draw, black, fill=black, circle, minimum size = 2pt, inner sep=0.5pt] at (axis cs: \l, \T0) {};
    \end{axis}
\end{tikzpicture}  % this compilation
\end{wrapfigure}


Формальна постановка задачі:
\begin{equation} \label{cond4,2}
    \left\{ \begin{aligned}
        &\;u = u(x,t), \\
        &\;u_{t} = D u_{xx}, \\
        &\;0 \leq x \leq l, t \geq 0, \\
        &\;u(0,t) = u(l,t) = T_0,\\
        &u(x,0) = \varphi(x) = T_1 - (T_1 - T_2)\Theta(x - l/2).
    \end{aligned} \right.
\end{equation}
\vspace{1cm}

Тут використана тета-функція Хевісайта (або функція сходинки). Вона задається таким чином:
\begin{equation*}
    \Theta(x - x_0) = 
    \begin{cases}
        \;0, \text{ при } x < x_0\\
        \;1, \text{ при } x > x_0
    \end{cases}    
\end{equation*}
Зробимо заміну, яка приведе до однорідних крайових умовами\[u(x,t) = \widetilde{u}(x,t) + T_0 \] і тепер можемо скористатися розділенням змінних.
\begin{equation} \label{subst4,2}
    \widetilde{u}(x,t) = \widetilde{X}(x)\cdot\widetilde{T}(t) 
\end{equation}
Рівняння для нової функції не змінить свого виду, тому процедура відокремлення змінних виконується аналогічно до проведених раніше. Результат відокремлення змінних:
\begin{equation} 
    \left\{ \begin{aligned}
        \;&\widetilde{X} = \widetilde{X}(x), \\  &\widetilde{X}'' = -\lambda \widetilde{X}, \\ &0 \leq x \leq l, \\  &\widetilde{X}(0) = 0, \\ &\widetilde{X}(l) = 0. 
    \end{aligned} \right.
    \qquad\qquad
    \widetilde{T}' + \lambda D \widetilde{T} = 0
\end{equation}
Розв'язок цієї задачі Штурма-Ліувілля (\ref{ShLsol1,1}) отриманий в задачі №1,1.
\begin{equation} 
    \left\{ \begin{aligned}
        \;&\lambda_n = \frac{\pi^2 n^2}{l^2},\\ 
        &\widetilde{X}_n(x) = \sin\left(\frac{\pi n x}{l}\right),
    \end{aligned} \right.
    \quad \text{де } n \in \mathbb{N}.
\end{equation}

У часовому рівнянні змінні розділяються, тому його легко проінтегруємо:
\begin{equation} \label{time-eq4,2}
    \begin{aligned}
        \frac{\widetilde{T}'}{\widetilde{T}} = -\lambda_n D = -\tau_n^{-1}
        \;&\Rightarrow\;
        \int \frac{\mathrm{d} \widetilde{T}}{\widetilde{T}} = - \int \frac{\mathrm{d} t}{\tau_n}
        \;\Rightarrow\\
        \Rightarrow\;
        \ln{\widetilde{T}_n} = \ln{C_n} - t/\tau_n
        \;&\Rightarrow\;
        \widetilde{T}_n = C_n e^{-t/\tau_n}, \text{ де } n \in \mathbb{N}
    \end{aligned}
\end{equation}

Отже, отримаємо остаточний розв'язок
\begin{equation}
    \begin{aligned}
        \;&u_n(x,t) = T_0 + \widetilde{X}_n \cdot \widetilde{T}_n = T_0 + C_n e^{-t/\tau_n} \sin(k_n x),\\
        &k_n = \frac{\pi n}{l} - \text{ хвильові вектори}, \\
        &\tau_n = \frac{1}{D k_n^2} = \frac{l^2}{D \pi^2 n^2} - \text{ характерний час зміни температури}, \\
        &n = 1, 2,\ldots
    \end{aligned}
\end{equation}

Запишемо загальний розв'язок задачі
\begin{equation}
    u(x,n) = T_0 + \sum_{n=1}^{\infty}C_n e^{-t/\tau_n} \sin(k_n x)
\end{equation}

Із початкової умови (\ref{cond4,2}) визначимо коефіцієнти $C_n$:
\begin{equation}
    u(x,0) = T_0 + \sum_{n=1}^{\infty}C_n \sin(k_n x) =  T_1 - (T_1 - T_2)\Theta(x - l/2)
\end{equation}

Розкладемо за синусами модифіковану початкову умову $u(x,0) - T_0$ 
\begin{equation*}
    \begin{gathered}
        T_1 - T_0 - (T_1 - T_2)\Theta(x - l/2) = - \sum_{n=1}^{\infty}C_n \sin(k_n x)\\
        C_n = -\frac{2}{l}\left[\int\limits_0^{l/2} (T_1 - T_0)\sin(k_n x) \;\mathrm{d}x + \int\limits_{l/2}^l (T_2 - T_0)\sin(k_n x) \;\mathrm{d}x\right] =\\
        = \frac{2}{lk_n}\left[(T_1 - T_0)\cos(k_n x)\biggr\rvert_0^{l/2} + (T_2 - T_0)\cos(k_n x)\biggr\rvert_0^{l/2}\right] =\\
        = \frac{2}{lk_n}\left[(T_1 - T_0)(1 - \cos(k_n l/2)) + (T_2 - T_0)(\cos(k_n l/2) - (-1)^n)\right] =\\
        = \frac{2}{lk_n}\left[T_1(1 - \cos(k_n l/2)) + T_2(\cos(k_n l/2) - (-1)^n) + T_0 ((-1)^n - 1)\right]
    \end{gathered}
\end{equation*}
Випишемо декілька перших множників $C_n$ для визначення поведінки їх множини
\begin{equation*}
    \begin{aligned}
        &n=1: \left[T_1 + T_2 - 2T_0 \right] &\Rightarrow&\ C_{4m+1} = \frac{2}{k_{4m+1}l}\left[T_1 + T_2 - 2T_0 \right]\\
        &n=2: \left[T_1 - T_2 \right] &\Rightarrow&\ C_{4m+2} = \frac{2}{k_{4m+2}l}\left[T_1 - T_2 \right]\\
        &n=3: \left[T_1 + T_2 - 2T_0 \right] &\Rightarrow&\ C_{4m+3} = \frac{2}{k_{4m+3}l}\left[T_1 + T_2 - 2T_0 \right]\\
        &n=4: 0 &\Rightarrow&\ C_{4m} = 0
    \end{aligned}    
\end{equation*}
Покладемо тут $T_0 = 0$ та $T_1 = T_2$ з чого видно, що коефіцієнти з парними індексами зануляються. Для випадку $T_1 = -T_2$ навпаки -- з непарними.

Отже, розв'язком буде
\begin{equation}
    \begin{aligned}
        u(x,t) = T_0 + \frac{2}{l}\sum_{n=1}^{\infty}\big[&T_1(1 - \cos(k_n l/2)) +\\
        + T_2(\cos(k_n l/2) - (-1)^n) + &T_0 ((-1)^n - 1)\big] e^{-t/\tau_n} \frac{\sin(k_n x)}{k_n}
    \end{aligned}
\end{equation}

\begin{center}
    \textbf{Графіки розв'язків}
\end{center}

Поведінку знайдених груп доданків можна побачити на наступному графіку. Тут наведено по одному з кожної групи. %% графік не хоче підтягуватись вгоду сторінки, бо на новій сторінці немає тексту, а на стару він не вміщається 
\begin{figure}[h]
    \centering
    %\large \textbf{Graph under comment}%
    \begin{tikzpicture}
    \begin{axis}
        [width = 0.95\textwidth, height = 0.6\textwidth,
         axis x line = center, axis y line = center,
         ylabel = $u(x_0; t)$, xlabel = $t$,
         xmin = -0.3, xmax = 5.7, ymin = -1.5, ymax = 3.8,
         axis line style = thin, xtick = {0}, ytick = {0}]   


        \tikzmath{\T2 = 3.5; \T1 = 4*\T2/5; \T0 = \T2/2; \l = 5.5; \t = (\l/pi)^2;}

        \addplot [black, domain=0:\l, samples = 300] {(\T1 + \T2 - 2*\T0) * exp(-x/\t)}
        node[anchor=50, pos=0] {$C_1$}
        node[anchor=south, pos=1/5] {$u_1$};

        \addplot [red, domain=0:\l, samples = 300] {(\T1 - \T2) * exp(-x*2^2/\t)}
        node[anchor=50, pos=0] {$C_2$}
        node[anchor=130, pos=1/4] {$u_2$};

        \addplot [green, domain=0:\l, samples = 300] {0}
        node[anchor=60, pos=0] {$\textcolor{black}{0}$}
        node[anchor=south, pos=1/3] {$u_4$};

        \addplot [blue, domain=0:\l, samples = 300] {\T0}
        node[anchor=50, pos=0] {$T_0$}
        node[anchor=south, pos=2/5] {$u_0$};

    \end{axis}
\end{tikzpicture} %%for compiling main 
    %\begin{tikzpicture}
    \begin{axis}
        [width = 0.95\textwidth, height = 0.6\textwidth,
         axis x line = center, axis y line = center,
         ylabel = $u(x_0; t)$, xlabel = $t$,
         xmin = -0.3, xmax = 5.7, ymin = -1.5, ymax = 3.8,
         axis line style = thin, xtick = {0}, ytick = {0}]   


        \tikzmath{\T2 = 3.5; \T1 = 4*\T2/5; \T0 = \T2/2; \l = 5.5; \t = (\l/pi)^2;}

        \addplot [black, domain=0:\l, samples = 300] {(\T1 + \T2 - 2*\T0) * exp(-x/\t)}
        node[anchor=50, pos=0] {$C_1$}
        node[anchor=south, pos=1/5] {$u_1$};

        \addplot [red, domain=0:\l, samples = 300] {(\T1 - \T2) * exp(-x*2^2/\t)}
        node[anchor=50, pos=0] {$C_2$}
        node[anchor=130, pos=1/4] {$u_2$};

        \addplot [green, domain=0:\l, samples = 300] {0}
        node[anchor=60, pos=0] {$\textcolor{black}{0}$}
        node[anchor=south, pos=1/3] {$u_4$};

        \addplot [blue, domain=0:\l, samples = 300] {\T0}
        node[anchor=50, pos=0] {$T_0$}
        node[anchor=south, pos=2/5] {$u_0$};

    \end{axis}
\end{tikzpicture} %%for compiling this file
\caption{Перші чотири розв'язки}
\end{figure}

Проміжок часу називається малим, коли $t \ll \tau$, а великим -- $t > \tau$

%\end{document}