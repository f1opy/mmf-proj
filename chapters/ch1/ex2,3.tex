\documentclass[a4paper, 14pt]{extreport}

\usepackage{StyleMMF}

\begin{document}

\subsection{Вільні коливання поля в резонаторі для заданих початкових умов. Ряд Фур'є по системі ортогональних функцій.}

\subsubsection{Задача №3}

\textit{Знайти коливання струни завдовжки $0 \leq x \leq l$ із закріпленими кінцями, якщо початкове відхил є $\varphi(x) = hx/l$, а початкова швидкість $\psi(x) = \nu_0$. Обчислити інтеграл ортогональності власних функцій і знайти квадрат норми. Чи є рух струни періодичним (тобто повторюється початковий стан струни через деякий проміжок часу?) Чи буде рух періодичним, якщо він описується рівнянням $u_{tt} = v^2 u_{xx} - \omega_0^2 u$}

\begin{center}
    \textbf{Розв'язок}
\end{center}
Формальна постановка задачі:
\begin{equation} \label{probcond3}
    \left\{ \begin{aligned} %%
        &\;u = u(x,t), \\
        &\;u_{tt} = v^2 u_{xx}, \\
        &\;0 \leq x \leq l, t \geq 0, \\
        &\;u(0,t) = u(l,t) = 0,\\
        &\left.\begin{aligned}
            &u(x,0) = \varphi(x) = \frac{hx}{l}, \\ 
            &u_t(x,0) = \psi(x) = \nu_0.
        \end{aligned}\right\} \; 
        \begin{aligned}
            &\text{ початкові умови задають} \\
          - &\text{ механічний стан} \\
            &\text{ системи при } t = 0
        \end{aligned}
    \end{aligned} \right.
\end{equation}

Задача з заданими початковими умовами має єдиний розв'язок. Скористаємося результатами задачі 1 попередньго заняття (\ref{sol1}).
\begin{equation*}
    \left\{ \begin{aligned} \label{fullsol}
        \;&u_n(x,t) = \left[A_n\cos(\omega_n t) + B_n\sin(\omega_n t)\right] \sin(k_n x), \\
        &k_n = \frac{\pi n}{l}, \, n = 1, 2,\ldots\\
        &\omega_n = vk_n = \frac{v \pi n}{l} - \text{ власні частоти}.
    \end{aligned}\right.
\end{equation*}

Запишемо загальний розв'язок задачі:
\begin{equation} \label{gensol}
    u(x,t) = \sum^{\infty}_{n=1} u_n(x,t) = \sum^{\infty}_{n=1} \left[A_n\cos(\omega_n t) + B_n\sin(\omega_n t)\right] \sin(k_n x)
\end{equation}
Коефіцієнти $A_n$ та $B_n$ визначаємо із початкових умов. Підставляємо (\ref{gensol}) в початкові умови (\ref{probcond3}):
\begin{equation} \label{sol-init-cond}
    \begin{aligned}
        &u(x,0) = \varphi(x)
        \;\Rightarrow\;
        \sum^{\infty}_{n=1} A_n\sin(k_n x) = \varphi(x)\\
        &\begin{aligned}
            u_t(x,0) = \psi(x)
            \;&\Rightarrow\\
            \Rightarrow \left(\sum^{\infty}_{n=1}\right.&\left.\left. \left[-A_n\omega_n\sin(\omega_n t) + B_n\omega_n\cos(\omega_n t)\right] \sin(k_n x)\right)\right|_{t=0} =\\
            &= \sum^{\infty}_{n=1} B_n\omega_n\sin(k_n x) = \psi(x)
        \end{aligned}
    \end{aligned}
\end{equation}
Отже, ми отримали дві умови для визначення $A_n$, $B_n$.\\
Далі скористаємося ортогональністю власних функцій задачі Штурма-Ліувілля.
\begin{equation} \label{orth}
    \int_0^l X_n(x) \cdot X_m(x) \,\mathrm{d}x = ||X_n||^2\delta_{n,m},
\end{equation}
де $||X_n||$ -- норма власної функції.

Доможуємо отримані вирази в (\ref{sol-init-cond}) на $m$-ту власну функції $\sin(k_m x)$ та інтегруємо від $0$ до $l$. 
\begin{equation*}
    \begin{aligned}
        \begin{aligned}
            &\int\limits_0^l \varphi(x) \sin(k_m x) \,\mathrm{d}x = \sum^{\infty}_{n=1} A_n \int\limits_0^l \sin(k_n x) \sin(k_m x) \,\mathrm{d}x =\\
            &= \sum^{\infty}_{n=1} A_n \cdot \frac{l}{2} \delta_{n,m} = \frac{A_m l}{2}
            \;\Rightarrow\;
            A_n = \frac{2}{l} \int\limits_0^l \varphi(x) \sin(k_n x) \,\mathrm{d}x =\\
            &= \frac{2}{l} \int\limits_0^l \frac{hx}{l} \sin(k_n x) \,\mathrm{d}x = \frac{2h}{l^2} \left(\left.-\frac{1}{k_n} x \cos(k_n x)\right|_0^l + \int\limits_0^l \frac{\cos(k_n x)}{k_n} \,\mathrm{d}t\right) =\\
            &= \left| k_n l = \frac{\pi n}{l} l = \pi n \Rightarrow \sin(k_n l) = 0,\, \cos(k_n l) = (-1)^n \right| =\\
            &= \frac{2h}{l^2} \left(-\frac{l}{k_n}(-1)^n + \left.\frac{\sin(k_n x)}{k_n^2}\right|_0^l \right) = \frac{2h}{l} \frac{(-1)^{n+1}}{k_n} \equiv A_n
        \end{aligned}\\
        \begin{aligned}
            \int\limits_0^l \psi(x) \sin(k_m x) \,\mathrm{d}x &= \sum^{\infty}_{n=1} B_n\omega_n \cdot \frac{l}{2} \delta_{n,m} = \frac{B_m \omega_m l}{2}
            \;\Rightarrow\\
            \Rightarrow\;
            B_n =&\ \frac{2}{\omega_n l} \int\limits_0^l \psi(x) \sin(k_n x) \,\mathrm{d}x = \frac{2\nu_0}{\omega_n l} \int\limits_0^l \sin(k_n x) \,\mathrm{d}x =\\
            =&\ \left.\frac{2\nu_0}{k_n\omega_n l} \cos(k_n x)\right|_l^0 = \frac{2\nu_0}{l} \frac{1 - (-1)^n}{k_n\omega_n} \equiv B_n
        \end{aligned}
    \end{aligned}
\end{equation*} 
Підставляємо визначені константи в (\ref{gensol})
\begin{equation} \label{sol3}
    u(x,t) = \frac{2}{l}\sum^{\infty}_{n=1} \left[\nu_0 (1 - (-1)^n)\frac{\sin(\omega_n t)}{\omega_n} - h(-1)^n\cos(\omega_n t)\right] \frac{\sin(k_n x)}{k_n}
\end{equation}

Перевіримо періодичність розв'язку. Період коливання визначається за відомою формулою \[T_n = \frac{2\pi}{\omega_n},\] де $n$ - номер власної моди. Підставимо в (\ref{sol3}) $t = t + T_n$
\begin{equation*}
    \begin{aligned}
        u(x,t+T_n) & = \frac{2}{l}\sum^{\infty}_{n=1} \left[\nu_0 (1 - (-1)^n)\frac{\sin(\omega_n t + \omega_n \cdot \frac{2\pi}{\omega_n})}{\omega_n} -\right.\\
        &\qquad\qquad\qquad\qquad\quad\left.- h(-1)^n\cos(\omega_n t + \omega_n \cdot \frac{2\pi}{\omega_n})\right] \frac{\sin(k_n x)}{k_n} =\\
        = \frac{2}{l}\sum^{\infty}_{n=1} & \left[\nu_0 (1 - (-1)^n)\frac{\sin(\omega_n t + 2\pi)}{\omega_n} - h(-1)^n\cos(\omega_n t + 2\pi)\right] \frac{\sin(k_n x)}{k_n} =\\ 
        = \frac{2}{l}\sum^{\infty}_{n=1} & \left[\nu_0 (1 - (-1)^n)\frac{\sin(\omega_n t)}{\omega_n} - h(-1)^n\cos(\omega_n t)\right] \frac{\sin(k_n x)}{k_n} = u(x,t)
    \end{aligned}
\end{equation*} 
Тобто коливання струни буде періодичним.

\end{document}