%\documentclass[a4paper, 14pt]{extreport}

%\usepackage{../StyleMMF}

%\begin{document}

\section[Задача №1.1]{1.1}

\textit{\textbf{Знайти власні моди коливань струни завдовжки $l$ із закріпленими кінцями (знайти функції вигляду $u(x,t) = X(x) \cdot T(t)$, визначені і достатньо гладкі в області $0 \leq x \leq l, -\infty \leq t \leq \infty$, не рівні тотожно нулю, які задовольняють одновимірне хвильове рівняння $u_{tt} = v^2 u_{xx}$ на проміжку $0 \leq x \leq l$ і межові умови $u(0,t) = 0, u(l,t) = 0$ на його кінцях).} Результат перевірити аналітично й графічно (див. текст до модульної контрольної роботи №1, с. 25) та проаналізувати його фізичний смисл. Знайти початкові умови (початкове відхилення і початкову швидкість) для кожної з мод.}

\begin{center}
    \large{\textbf{Розв'язок}}
\end{center}

\noindent Формальна постановка задачі:
\begin{equation} \label{cond1,1}
    \left\{ \begin{aligned} %%
        \;&u = u(x,t), \\
          &u_{tt} = v^2 u_{xx}, \\
          &0 \leq x \leq l, t \in \mathbb{R}, \\
          &u(0,t) = 0, \\
          &u(l,t) = 0. 
    \end{aligned} \right.
\end{equation}
Необхідно знайти нетривіальні (тобто не рівні тотожно нулю) розв'язки (\ref{cond1,1}) вигляду:
\begin{equation} \label{subst1,1}
    u(x,t) = X(x) \cdot T(t) \neq 0 
\end{equation}

Хвильове рівняння з двома межовими умовами (\ref{cond1,1}) на кінцях проміжку по координаті $x$ описує малі поперечні коливання струни із закріпленими кінцями, її довільний вільний рух. Струна має попередній натяг, і у положенні рівноваги  всі її точки знаходяться на осі $x$, а при коливаннях відхиляться у напрямку осі $y$; $u(x,t)$ - це відповідне зміщення точки струни з координатою $x$ в напрямку $y$ відносно її рівноважного положення у даний момент часу $t$. Власні моди струни - це особливі рухи струни, які описуються розв'язками у вигляді добутків (\ref{subst1,1}). 

Підставляємо розв'язок у вигляді добутку (\ref{subst1,1})  у рівняння й умови (\ref{cond1,1}) Почнемо з межових умов:
\begin{equation*}
    \begin{aligned}
        \;u(0,t) = X(0) \cdot T(t) = 0
        \;\Rightarrow\;
        \left\{ \begin{aligned}
            &T(t) \neq 0, \forall t, \\  &X(0) = 0; 
        \end{aligned} \right.\\
        u(l,t) = X(l) \cdot T(t) = 0
        \;\Rightarrow\;
        \left\{ \begin{aligned}
            &T(t) \neq 0, \forall t, \\  &X(l) = 0; 
        \end{aligned} \right.\\
    \end{aligned}
\end{equation*}
Тут ми врахували, що умови на кінцях струни виконуються при всіх $t$, тому $T(t)$ не може бути рівним нулю.\\
Далі підставимо (\ref{subst1,1}) у рівняння:
\begin{equation*}
    \frac{\partial^2}{\partial t^2}\left[X(x)T(t)\right] = v^2 \frac{\partial^2}{\partial x^2}\left[X(x)T(t)\right]
    \;\Rightarrow\; 
    X T^{\prime\prime} = v^2 X^{\prime\prime} T 
\end{equation*}
Звідси переходимо до рівності двох функцій від різних змінних:
\begin{equation}
    \frac{T^{\prime\prime}}{v^2T} = \frac{X^{\prime\prime}}{X}
\end{equation}
Це і є ситуація відокремлення змінних: функція від $x$ має дорівнювати функції від $t$ при всіх $x$ і $t$. Це можливо тільки у випадку, якщо обидві ці функції є сталими. Тому маємо
\begin{equation}
    \frac{T^{\prime\prime}}{v^2T} = \frac{X^{\prime\prime}}{X} = - \lambda,
\end{equation}
де $\lambda$ -- стала відокремлення. Її можливі значення необхідно буде знайти.

Виписуємо результат відокремлення змінних:
\begin{equation} \label{sepvar1,1}
    \left\{ \begin{aligned}
        \;&X = X(x), \\  &X^{\prime\prime} = -\lambda X, \\ &0 \leq x \leq l, \\  &X(0) = 0, \\ &X(l) = 0. 
    \end{aligned} \right.
    \qquad\qquad
    T^{\prime\prime} + \lambda v^2 T = 0
\end{equation}

Задача для $X = X(x)$ є так званою Штурма-Ліувілля. Необхідно знайти нетривіальні розв'язки цієї задачі і значення параметра   $\lambda$, при яких вони існують; їх називають, відповідно, власними функціями і власними значеннями задачі. З умов задачі можна показати (див. лекції), що її власні значення є дійсними.\\
\begin{enumerate}
    \item[] Розв'язуємо задачу Штурма-Ліувілля (\ref{sepvar1,1}):
    \begin{enumerate}[wide, labelindent=0pt]
        \item Розглянемо випадок $\lambda = 0$:
        \begin{equation*}
            X^{\prime\prime} = -\lambda X
            \;\Rightarrow\;
            X^{\prime\prime} = 0
            \;\Rightarrow\;
            X(x) = C_1 + C_2 x
        \end{equation*}
        Знаходимо константи з межових умов:
        \begin{equation*}
            \begin{aligned}
                &\left\{ \begin{aligned}
                    &X(0) = C_1 = 0, \\ 
                    &X(l) = C_1 + C_2 l = 0;
                \end{aligned} \right.
                \\   
                &\left\{ \begin{aligned}
                    C_1 = 0, \\ 
                    C_2 = 0;
                \end{aligned} \right.
            \end{aligned}
            \quad\Rightarrow\;
            \begin{aligned}
                X(x) = 0 \text{ -- розв'язок тривівльний,}\\
                \lambda = 0 \text{ не є власним значенням.}
            \end{aligned}
        \end{equation*}
    
        \item Розглянемо випадок $\lambda < 0$. Розв'язок рівняння шукаємо у вигляді $X(x) = e^{\alpha x}$: 
        \begin{equation*}
            \begin{aligned}
                &X^{\prime\prime} = -\lambda X
                \quad\Rightarrow\quad
                \alpha^2 \textcolor{red}{\begin{xy}*{\textcolor{black}{e^{\alpha x}}};p+LU;+RD**h@{}+/\jot/**h@{-}\end{xy}} = +|\lambda| \textcolor{red}{\begin{xy}*{\textcolor{black}{e^{\alpha x}}};p+LU;+RD**h@{}+/\jot/**h@{-}\end{xy}}
                \quad\Rightarrow\quad
                \alpha = \pm \sqrt{|\lambda|}
                \;\Rightarrow\\
                \Rightarrow\;
                &X(x) = \widetilde{C}_1 e^{\sqrt{|\lambda|}x} + \widetilde{C}_2 e^{-\sqrt{|\lambda|}x} = C_1 sh(\sqrt{|\lambda|}x) + C_2 ch({\sqrt{|\lambda|}x})
            \end{aligned}
        \end{equation*}
        Знаходимо константи з межових умов:
        \begin{equation*}
            \begin{aligned}
                X(0) = C_2
                \;\Rightarrow\;
                X(x) = C_1 &sh(\sqrt{|\lambda|}x)\\
                \left\{ \begin{aligned}
                    &X(l) = C_1 sh(\sqrt{|\lambda|}l) = 0, \\
                    &sh(\sqrt{|\lambda|}l) \neq 0;
                \end{aligned} \right.&\\
                \left\{ \begin{aligned}
                    C_1 = 0, \\ 
                    C_2 = 0;
                \end{aligned} \right. \qquad\qquad\qquad\qquad&
            \end{aligned}
            \;\Rightarrow\;
            \begin{aligned}
                \text{розв'язок тривівльний,}\\
                \text{немає від'ємних}\\
                \text{власних значень.}
            \end{aligned}
        \end{equation*}

        \item Розглянемо випадок $\lambda > 0$. Розв'язок рівняння шукаємо у вигляді $X(x) = e^{\alpha x}$: 
        \begin{equation*}
            \begin{aligned}
                &X^{\prime\prime} = -\lambda X
                \quad\Rightarrow\quad
                \alpha^2 \textcolor{red}{\begin{xy}*{\textcolor{black}{e^{\alpha x}}};p+LD;+RU**h@{}+/\jot/**h@{-}\end{xy}} = -\lambda \textcolor{red}{\begin{xy}*{\textcolor{black}{e^{\alpha x}}};p+LD;+RU**h@{}+/\jot/**h@{-}\end{xy}}
                \quad\Rightarrow\quad
                \alpha = \pm i\sqrt{\lambda}
                \;\Rightarrow\\
                \Rightarrow\;
                &X(x) = \widetilde{C}_1 e^{i\sqrt{\lambda}x} + \widetilde{C}_2 e^{-\sqrt{\lambda}x} = C_1 \sin(\sqrt{\lambda}x) + C_2 \cos({\sqrt{\lambda}x})
            \end{aligned}
        \end{equation*}
        Знаходимо константи з межових умов:
        \begin{equation*}
            \begin{aligned}
                \left\{ \begin{aligned}
                    &X(0) = C_2 = 0, \\ 
                    &X(l) = C_1 \sin(\sqrt{\lambda}l) + \textcolor{red}{\begin{xy}*{\textcolor{black}{C_2}};p+LU;+RD**h@{}+//**h@{-}*h@{>}*h!LD{\scriptstyle 0}\end{xy}} \cos({\sqrt{\lambda}l}) = 0;
                \end{aligned} \right.
                \;\Rightarrow\;
                \left\{ \begin{aligned}
                    &C_1 \neq 0, \\ 
                    &\sin(\sqrt{\lambda}l) = 0;
                \end{aligned} \right.
            \end{aligned}
        \end{equation*}
        Отже, нетривіальні розв'язки існують при значеннях параметра $\lambda$, які задовольняють характеристичне рівняння :
        \begin{equation*}
            \sin(\sqrt{\lambda}l) = 0
            \;\Rightarrow\;
            \sqrt{\lambda_n}l = \pi n, \, n \in \mathbb{Z}
            \;\Rightarrow\;
            \lambda_n = \frac{\pi^2 n^2}{l^2}.
        \end{equation*}
    \end{enumerate}
\end{enumerate} 
Випишемо тепер розв'язки для всіх $n$ і визначимо, які з них необхідно залишити:
    \begin{equation*}
        X_n(x) = C_n \sin\left(\frac{\pi n x}{l}\right)
    \end{equation*}
    Видно, що $n = 0$ відповідає тривіальному розв'язку. Видно також, що всі інші розв'язки визначені з точністю до довільного множника.\\
    Тому власні функції, які співпадають з точністю до множника, вважають однаковими. У загальному випадку різними вважають лише лінійно незалежні власні функції, а розвя'зати задачу Штурма-Ліувілля означає знайти всі різні власні функції і відповідні власні значення. Отже, різним власним функціям відповідають лише натуральні $n$, а коефіцієнти $C_n$ можна покласти рівними одиниці.\\
    Власними значеннями і власними функціями є
    \begin{equation} \label{ShLsol1,1}
        \left\{ \begin{aligned}
            \;&\lambda_n = \frac{\pi^2 n^2}{l^2},\\ 
            &X_n(x) = \sin\left(\frac{\pi n x}{l}\right),
        \end{aligned} \right.
        \quad \text{де } n \in \mathbb{N}.
    \end{equation}

Повертаємося до рівняння для $T(t)$ (\ref{sepvar1,1}). Підставляємо знайдені власні значення та знаходимо $T_n(t)$:
\begin{equation*}
    \left. \begin{aligned}
        \lambda_n = \frac{\pi^2 n^2}{l^2},&\;\\ 
        T^{\prime\prime} + \lambda v^2T = 0,&
    \end{aligned} \right\}
    \;\Rightarrow\;
    T_n(t) = A\cos(\omega_n t) + B\sin(\omega_n t),
\end{equation*}
де $\omega_n^2 = \lambda_n v^2, \, n \in \mathbb{N}.$\\
Власними модами коливань струни будуть всі розв'язки вигляду:
\begin{equation*}
    u_n(x,t) = X_n(x) \cdot T_n(t)
\end{equation*}
Виконаємо перепозначення і запишемо остаточний розв'язок:
\begin{equation}
    \left\{ \begin{aligned} \label{mode1,1}
        \;&u_n(x,t) = \left[A_n\cos(\omega_n t) + B_n\sin(\omega_n t)\right] \sin(k_n x), \\
        &k_n = \frac{\pi n}{l} - \text{ хвильові вектори}, \\
        &\omega_n = vk_n = \frac{v \pi n}{l} - \text{ власні частоти}, \\
        &n = 1, 2,\ldots
    \end{aligned}\right.
\end{equation}

\begin{center}
    \large{\textbf{Перевірка розв'язку задачі Штурма-Ліувілля}}
\end{center}

\noindent Перевірка результату (\ref{ShLsol1,1}) включає аналітичну і графічну перевірку. Необхідно перевірити, що знайдені розв'язки і числа (\ref{ShLsol1,1}) дійсно є власними функціями і власними значеннями задачі (\ref{sepvar1,1}), а також, що знайдені всі її власні функції і власні значення. Перш за все, перевіряємо виконання всіх умов задачі Штурма-Ліувілля (\ref{sepvar1,1}).
\begin{enumerate}[wide, labelindent=0pt]
    \item Аналітична перевірка 
    Підставляємо знайдені функції у крайові умови і рівняння задачі Штурма-Ліувілля. Відмінні від нуля функції, які задовольняють крайові умови і рівняння, за означенням є власними функціями задачі. Одночасно знаходимо з рівняння відповідне власне значення і звіряємо його з указаним у відповіді.
    \begin{enumerate}
        \item[1)] Перевіряємо виконання крайових умов, підставляємо власні функції в умови (\ref{sepvar1,1}) 
        \begin{equation*}
            \begin{aligned}
                X(0) &=  0:\\
                &\begin{aligned}
                    &X_n(0) = C_n \sin(\sqrt{\lambda_n} \cdot 0) = 0 \text{ -- виконується,}\\
                    &\text{ причому незалежно від }\lambda_n
                \end{aligned}\\
                \\
                X(l) &= 0:\\
                &\begin{aligned}
                    &X_n(l) = C_n \sin\left(\frac{\pi n}{l} \cdot l\right) = C_n \sin(\pi n) = 0 \text{ -- виконується}\\
                    &\text{ причому саме для знайдених значень }\lambda_n.
                \end{aligned}\\
            \end{aligned}
        \end{equation*}
        \item[2)] Перевіряємо рівняння і власні значення. Виконання крайових умов уже перевірено; якщо ми підставимо функцію у рівняння, то одночасно знайдемо і відповідне їй власне значення (якщо рівняння виконується). Це власне значення має співпасти з указаним у відповіді. 
        
        Почергово перевіряємо всі функції, вказані у відповіді. Обчислимо другу похідну  
        \begin{equation*}
            X_n^{''} = \frac{\pi n}{l} \left(C_n\cos\left(\frac{\pi n x}{l}\right)\right)^{'} = -\left(\frac{\pi n}{l}\right)^2 C_n\sin\left(\frac{\pi n x}{l}\right) = -\left(\frac{\pi n}{l}\right)^2 X_n
        \end{equation*}
        Порівнюємо з вихідним рівнянням і робимо перший висновок: кожна з функцій $X_n(x)$ дійсно є розв’язком рівняння (\ref{sepvar1,1}). Одночасно, знаходимо з рівняння відповідне даній функції значення спектрального параметра, - це $\lambda = \left(\frac{\pi n}{l}\right)^2$. Порівняємо це значення з тим, яке вказане у відповіді (\ref{ShLsol1,1}) і робимо другий висновок: знайдені власні значення дійсно відповідають знайденим власним функціям.
    \end{enumerate}
    \item Графічна перевірка.\\
    Будуємо графіки кількох перших власних функцій. Масштаб по вертикалі може бути довільним і різним для різних функцій, оскільки значення він не має.
    \begin{figure}[h]
        \centering
        %\large \textbf{Graph under comment}%
        \input{../graphs/sines} %%for compiling main 
        %\input{../../graphs/sines} %%for compiling this file
    \caption{Графічний розв'язок, наведені три перші власні функції}
    \end{figure}\\
    Функція, що відповідає найменшому власному значенню (у даному випадку це $X_1(x) = \sin\left(\frac{\pi x}{l}\right)$), відповідає так званій \textit{основній моді} резонатора (струна є частинним випадком одномірного резонатора). \textit{Для стаціонарних станів у квантовій механіці це основний стан системи, стан з найменшою можливою енергією.}\\
    Графіки власних функцій відображають їх основні властивості, які і треба перевірити. З рисунку видно, що в точках $x = 0$ та $x = l$ всі графіки проходять через нуль, отже крайова умова (\ref{sepvar1,1}) на обох кінцях виконується. \textit{Графічна перевірка підсилює надійність аналітичної, в якій також іноді припускаються помилок, приймаючи бажане за дійсне.}
\end{enumerate}

Проведені перевірки допомагають позбавитись більшості типових помилок, проте є надзвичайно підступний тип помилки, який проведені перевірки виявити не взмозі. Це випадок, коли ви дійсно знайшли власні функції та власні значення, але \textbf{не всі}, а отже задача розв'язана неправильно.\\
Помітити таку помилку допомагає так звана \textbf{осциляційна теорема}. З рисунку, наведеного в графічному методі, видно, що всі власні функції на проміжку $0 \leq x \leq l$ осцилюють, і при цьому всі вони мають \textbf{різне} число нулів. Число нулів (або "вузлів") всередені проміжку, на якому ророзв’язується задача, є своєрідною унікальною міткою власної функції. Власні значення необхідно розташувати у порядку зростання. Тоді за осциляційною теоремою основна мода одновимірної задачі Штурма-Ліувілля не має нулів у внутрішніх точках проміжку $[0, l]$. Тобто основна мода завжди є безвузловою. Далі, наступному за величиною власному значенню відповідає власна функція, що має один нуль, наступному - два нулі, і так далі. Для кожної наступної моди число вузлів збільшується на одиницю. Іншими словами, число нулів власної функції збігається з порядковим номером відповідного власного значення, якщо нумерувати їх у порядку зростання, починаючи з нуля.

\begin{center}
    \large{\textbf{Аналіз результату}}
\end{center}
З'ясуємо фізичний смисл одержаних розв'язків: яким саме рухам відповідають її власні моди. Розв’язки (\ref{mode1,1}) є частинними розв'язками однорідного хвильового рівняння з однорідними межовими умовами (\ref{cond1,1}). Це означає (див. лекції), що зовнішні сили на систему не діють, тому знайдені розв’язки відповідають вільним коливанням (рухам) струни. Це коливання (рухи) спеціального вигляду, оскільки відповідні розв’язки мають вигляд добутків $X_n(x) \cdot T_n(t)$.\\
Усі розв’язки $u_n(x, t)$ -- дійсні. Візьмемо один із них. Зафіксуємо певний довільний момент часу $t = t_1$, це буде миттєве фото $n$–ї моди струни. Просторовий розподіл зміщень описується формулою \[u_n(x, t_1) = X_n(x) \cdot T_n(t_1).\] Видно, що в будь-який момент часу форма просторового розподілу зміщень (тобто форма струни) залишається однаковою і описується відповідною власною функцією $X_n(x)$; за рахунок множника $T_n(t)$ змінюється лише спільна амплітуда просторового розподілу і його знак. Отже, $X_n(x)$ задає просторовий «профіль» моди, це її унікальне просторове «обличчя», - найперша визначальна характеристика певної моди, за якою можна ідентифікувати відповідний рух струни.\\
Тепер прослідкуємо за рухом певної точки струни $x = x_1$: \[u_n(x_1, t) = X_n(x_1) \cdot T_n(t).\] Видно, що всі точки струни здійснюють один і той же рух, одне і те ж гармонічне коливання з частотою $\omega_n$, але для різних мод частоти коливань різні. Коливання будь-якої точки задається однією функцією $T_n(t)$, але амплітуда визначається величиною $\left|X_n(x)\right|$. У точках, де $X_n(x)$ має нулі, амплітуда коливань дорівнює нулю, це \textbf{вузли} моди. Якщо ж $X_n(x)$ змінює знак, фаза коливання змінюється на $\pi$: частини струни, розділені вузлами, коливаються у протифазі.

Рух струни - це єдиний часово-просторовий процес. Для мод $n= 2$ і $n= 3$ він зображений на рисунку нижче. Усі вузли залишаються нерухомими, тільки якщо рух струни відповідає певній моді. У точках, де $X_n(x)$ максимальне (за модулем), максимальна й амплітуда коливань, -- це пучності.
\begin{figure}[h]
    \centering
    %\large \textbf{Graph under comment}%
    \begin{tikzpicture}
    \begin{axis}
        [width = 0.85\textwidth, height = 0.4\textwidth,
         axis x line = center, axis y line = center,
         ylabel = $X(x)$, xlabel = $x$,
         xmin = -0.3, xmax = 5.7, ymin = -5.3, ymax = 5.3,
         axis line style = thin, xtick = {0}, ytick = {0}]   
        
        \tikzmath{\A = 4; \l = 5; \k = pi/\l;}
        
        \addplot [black, domain=0:\l, samples = 1000] {\A * sin(deg(2*\k*x))}
        node[anchor=50, pos=0] {0} 
        node[anchor=west, pos=0.26, fill=white] {$u_2(x, t)$} 
        node[anchor=130, pos=1] {$l$};

        \addplot [red, domain=0:\l, samples = 1000] {-\A * sin(deg(2*\k*x))}
        node[anchor=north, pos=0.19] {$\textcolor{black}{\omega_2}$};
        
        \tikzmath{\p1 = \l/8; \p2 = \l/4; \p3 = 3*\l/8;
        \y1 = \A * sin(deg(2*\k*\p1)); \y2 = \A * sin(deg(2*\k*\p2)); \y3 = \A * sin(deg(2*\k*\p3));}

        \draw [arrows = {-Stealth[scale=1.5]}] (\p1,-\y1) -- (\p1,\y1);
        \draw [arrows = {-Stealth[scale=1.5]}] (\p2,-\y2) -- (\p2,\y2);
        \draw [arrows = {-Stealth[scale=1.5]}] (\p3,-\y3) -- (\p3,\y3);

        \tikzmath{\p4 = 5*\l/8; \p5 = 3*\l/4; \p6 = 7*\l/8;
        \y4 = \A * sin(deg(2*\k*\p4)); \y5 = \A * sin(deg(2*\k*\p5)); \y6 = \A * sin(deg(2*\k*\p6));}

        \draw [arrows = {-Stealth[scale=1.5]}] (\p4,-\y4) -- (\p4,\y4);
        \draw [arrows = {-Stealth[scale=1.5]}] (\p5,-\y5) -- (\p5,\y5);
        \draw [arrows = {-Stealth[scale=1.5]}] (\p6,-\y6) -- (\p6,\y6);
        
    \end{axis}
\end{tikzpicture}

\vspace{1cm}

\begin{tikzpicture}
    \begin{axis}
        [width = 0.85\textwidth, height = 0.4\textwidth,
         axis x line = center, axis y line = center,
         ylabel = $X(x)$, xlabel = $x$,
         xmin = -0.3, xmax = 5.7, ymin = -5.3, ymax = 5.3,
         axis line style = thin, xtick = {0}, ytick = {0}]   
        
        \tikzmath{\A = 4; \l = 5; \k = pi/\l;}
        
        \addplot [black, domain=0:\l, samples = 1000] {\A * sin(deg(3*\k*x))}
        node[anchor=50, pos=0] {0} 
        node[anchor=south, pos=0.19, fill=white] {$u_3(x, t)$} 
        node[anchor=130, pos=1] {$l$};

        \addplot [red, domain=0:\l, samples = 1000] {-\A * sin(deg(3*\k*x))}
        node[anchor=west, pos=0.23] {$\textcolor{black}{\omega_3}$};
        
        \tikzmath{\p1 = \l/9; \p2 = 2*\l/9; \p3 = 4*\l/9;
        \y1 = \A * sin(deg(3*\k*\p1)); \y2 = \A * sin(deg(3*\k*\p2)); \y3 = \A * sin(deg(3*\k*\p3));}

        \draw [arrows = {-Stealth[scale=1.5]}] (\p1,-\y1) -- (\p1,\y1);
        \draw [arrows = {-Stealth[scale=1.5]}] (\p2,-\y2) -- (\p2,\y2);
        \draw [arrows = {-Stealth[scale=1.5]}] (\p3,-\y3) -- (\p3,\y3);

        \tikzmath{\p4 = 5*\l/9; \p5 = 7*\l/9; \p6 = 8*\l/9;
        \y4 = \A * sin(deg(3*\k*\p4)); \y5 = \A * sin(deg(3*\k*\p5)); \y6 = \A * sin(deg(3*\k*\p6));}

        \draw [arrows = {-Stealth[scale=1.5]}] (\p4,-\y4) -- (\p4,\y4);
        \draw [arrows = {-Stealth[scale=1.5]}] (\p5,-\y5) -- (\p5,\y5);
        \draw [arrows = {-Stealth[scale=1.5]}] (\p6,-\y6) -- (\p6,\y6);
        
    \end{axis}
\end{tikzpicture} %%for compiling main 
    %\begin{tikzpicture}
    \begin{axis}
        [width = 0.85\textwidth, height = 0.4\textwidth,
         axis x line = center, axis y line = center,
         ylabel = $X(x)$, xlabel = $x$,
         xmin = -0.3, xmax = 5.7, ymin = -5.3, ymax = 5.3,
         axis line style = thin, xtick = {0}, ytick = {0}]   
        
        \tikzmath{\A = 4; \l = 5; \k = pi/\l;}
        
        \addplot [black, domain=0:\l, samples = 1000] {\A * sin(deg(2*\k*x))}
        node[anchor=50, pos=0] {0} 
        node[anchor=west, pos=0.26, fill=white] {$u_2(x, t)$} 
        node[anchor=130, pos=1] {$l$};

        \addplot [red, domain=0:\l, samples = 1000] {-\A * sin(deg(2*\k*x))}
        node[anchor=north, pos=0.19] {$\textcolor{black}{\omega_2}$};
        
        \tikzmath{\p1 = \l/8; \p2 = \l/4; \p3 = 3*\l/8;
        \y1 = \A * sin(deg(2*\k*\p1)); \y2 = \A * sin(deg(2*\k*\p2)); \y3 = \A * sin(deg(2*\k*\p3));}

        \draw [arrows = {-Stealth[scale=1.5]}] (\p1,-\y1) -- (\p1,\y1);
        \draw [arrows = {-Stealth[scale=1.5]}] (\p2,-\y2) -- (\p2,\y2);
        \draw [arrows = {-Stealth[scale=1.5]}] (\p3,-\y3) -- (\p3,\y3);

        \tikzmath{\p4 = 5*\l/8; \p5 = 3*\l/4; \p6 = 7*\l/8;
        \y4 = \A * sin(deg(2*\k*\p4)); \y5 = \A * sin(deg(2*\k*\p5)); \y6 = \A * sin(deg(2*\k*\p6));}

        \draw [arrows = {-Stealth[scale=1.5]}] (\p4,-\y4) -- (\p4,\y4);
        \draw [arrows = {-Stealth[scale=1.5]}] (\p5,-\y5) -- (\p5,\y5);
        \draw [arrows = {-Stealth[scale=1.5]}] (\p6,-\y6) -- (\p6,\y6);
        
    \end{axis}
\end{tikzpicture}

\vspace{1cm}

\begin{tikzpicture}
    \begin{axis}
        [width = 0.85\textwidth, height = 0.4\textwidth,
         axis x line = center, axis y line = center,
         ylabel = $X(x)$, xlabel = $x$,
         xmin = -0.3, xmax = 5.7, ymin = -5.3, ymax = 5.3,
         axis line style = thin, xtick = {0}, ytick = {0}]   
        
        \tikzmath{\A = 4; \l = 5; \k = pi/\l;}
        
        \addplot [black, domain=0:\l, samples = 1000] {\A * sin(deg(3*\k*x))}
        node[anchor=50, pos=0] {0} 
        node[anchor=south, pos=0.19, fill=white] {$u_3(x, t)$} 
        node[anchor=130, pos=1] {$l$};

        \addplot [red, domain=0:\l, samples = 1000] {-\A * sin(deg(3*\k*x))}
        node[anchor=west, pos=0.23] {$\textcolor{black}{\omega_3}$};
        
        \tikzmath{\p1 = \l/9; \p2 = 2*\l/9; \p3 = 4*\l/9;
        \y1 = \A * sin(deg(3*\k*\p1)); \y2 = \A * sin(deg(3*\k*\p2)); \y3 = \A * sin(deg(3*\k*\p3));}

        \draw [arrows = {-Stealth[scale=1.5]}] (\p1,-\y1) -- (\p1,\y1);
        \draw [arrows = {-Stealth[scale=1.5]}] (\p2,-\y2) -- (\p2,\y2);
        \draw [arrows = {-Stealth[scale=1.5]}] (\p3,-\y3) -- (\p3,\y3);

        \tikzmath{\p4 = 5*\l/9; \p5 = 7*\l/9; \p6 = 8*\l/9;
        \y4 = \A * sin(deg(3*\k*\p4)); \y5 = \A * sin(deg(3*\k*\p5)); \y6 = \A * sin(deg(3*\k*\p6));}

        \draw [arrows = {-Stealth[scale=1.5]}] (\p4,-\y4) -- (\p4,\y4);
        \draw [arrows = {-Stealth[scale=1.5]}] (\p5,-\y5) -- (\p5,\y5);
        \draw [arrows = {-Stealth[scale=1.5]}] (\p6,-\y6) -- (\p6,\y6);
        
    \end{axis}
\end{tikzpicture} %%for compiling this file
    \caption{Графіки другої та третьої моди}
\end{figure}
У нашому випадку моди занумеровані так, що число вузлів всередині струни на одиницю менше номера моди. При цьому кожна мода має свою частоту коливань, за якою також можна розрізнити різні моди. Для струни з обома закріпленими кінцями частоти всіх мод кратні частоті основної моди (див. (\ref{mode1,1})).\\
У цілому рух у часі і розподіл зміщень у просторі залишаються ніби незалежними. Струна коливається, а просторовий розподіл «стоїть». Подібні рухи прийнято називати \textbf{стоячими хвилями}. Стоячі хвилі описуються добутком вигляду (\ref{subst1,1}), в яких обидва множники є дійсними.

%\end{document}