\documentclass[a4paper, 14pt]{extreport}
\usepackage[top=2cm, bottom=2cm, left=2.5cm, right=1.5cm]{geometry}

\usepackage[utf8]{inputenc}
\usepackage[english, russian, ukrainian]{babel}
\usepackage{amssymb,amsfonts,amsmath,amsthm}

\usepackage[pdftex, unicode, colorlinks=true, linkcolor=black]{hyperref}

%\usepackage{relsize} %%позволяет пользоваться функцией \mathlarger{}
\usepackage{xcolor}
\usepackage[pdf]{xy}

\usepackage{wrapfig}
\usepackage{tikz} 
\usetikzlibrary{math}

\usepackage{pgfplots}
\pgfplotsset{compat=1.18}


\usepackage{titlesec}
\titleformat{\chapter}[display]{\normalfont\Large\bfseries}{\chaptertitlename\ \thechapter}{24pt}{\large\bfseries}
\titleformat{\section}{\normalfont\Large\bfseries}{\thesection}{20pt}{\large\bfseries}

\renewcommand{\labelenumii}{\theenumii)} %% заменяем счёчтик 2 уровня вида (a), (b), (c) и т.д. на русский алфавит а), б), в), и т.д. 


\begin{document}

\tableofcontents
\setcounter{page}{2}

\chapter{ЗАСТОСУВАННЯ ПРОЦЕДУРИ ФУР’Є БЕЗПОСЕРЕДНЬОГО ВІДОКРЕМЛЕННЯ ЗМІННИХ}

\section{Відокремлення змінних, задача Штурма-Ліувілля і власні моди коливань струни для різних межових умов}

\subsection*{Задача №1.1}

\textit{\textbf{Знайти власні моди коливань струни завдовжки $l$ із закріпленими кінцями (знайти функції вигляду $u(x,t) = X(x) \cdot T(t)$, визначені і достатньо гладкі в області $0 \leq x \leq l, -\infty \leq t \leq \infty$, не рівні тотожно нулю, які задовольняють одновимірне хвильове рівняння $u_{tt} = v^2 u_{xx}$ на проміжку $0 \leq x \leq l$ і межові умови $u(0,t) = 0, u(l,t) = 0$ на його кінцях).} Результат перевірити аналітично й графічно (див. текст до модульної контрольної роботи №1, с. 25) та проаналізувати його фізичний смисл. Знайти початкові умови (початкове відхилення і початкову швидкість) для кожної з мод.}

\begin{center}
    \large{\textbf{Розв'язок}}
\end{center}

\noindent Постановка задачі:
\begin{equation}
    \left\{ \begin{aligned} %%
        \;&u = u(x,t), \\  &u_{tt} = v^2 u_{xx}, \\ &0 \leq x \leq l, t \in \mathbb{R}, \\  &u(0,t) = 0, \\ &u(l,t) = 0. 
    \end{aligned} \right.
\end{equation}
Шукаємо нетривіальні розв'язки рівняння у виді:
\begin{equation} \label{subst}
    u(x,t) = X(x) \cdot T(t) \neq 0 
\end{equation}

Тепер можливе відокремлення змінних в задачі. Почнемо з межових умов:
\begin{equation*}
    \begin{aligned}
        \;u(0,t) = X(0) \cdot T(t) = 0
        \;\Rightarrow\;
        \left\{ \begin{aligned}
            &T(t) \neq 0, \forall t, \\  &X(0) = 0; 
        \end{aligned} \right.\\
        u(l,t) = X(l) \cdot T(t) = 0
        \;\Rightarrow\;
        \left\{ \begin{aligned}
            &T(t) \neq 0, \forall t, \\  &X(l) = 0; 
        \end{aligned} \right.\\
    \end{aligned}
\end{equation*}
Далі підставимо (\ref{subst}) в рівняння та виконаємо ряд перетворень:
\begin{equation*}
    \frac{\partial^2}{\partial t^2}\left[X(x)T(t)\right] = v^2 \frac{\partial^2}{\partial x^2}\left[X(x)T(t)\right]
    \;\to\; 
    X T^{\prime\prime} = v^2 X^{\prime\prime} T 
    \;\to\; 
    \frac{T^{\prime\prime}}{v^2T} = \frac{X^{\prime\prime}}{X} = - \lambda,
\end{equation*}
де $\lambda$ -- стала відокремлення.\\
Виписуємо результат відокремлення змінних:
\begin{equation} \label{sepvar}
    \left\{ \begin{aligned}
        \;&X = X(x), \\  &X^{\prime\prime} = -\lambda X, \\ &0 \leq x \leq l, \\  &X(0) = 0, \\ &X(l) = 0. 
    \end{aligned} \right.
    \qquad\qquad
    \begin{aligned}
        T^{\prime\prime} + \lambda v^2 T = 0\\
        \lambda \text{ -- невідома}
    \end{aligned}
\end{equation}

Для $X = X(x)$ отримуємо задачу Штурма-Ліувілля. Розв'яжемо її:
\begin{enumerate}
    \item[] \begin{enumerate}
        \item Розглянемо випадок $\lambda = 0$:
        \begin{equation*}
            X^{\prime\prime} = -\lambda X
            \;\to\;
            X^{\prime\prime} = 0
            \;\to\;
            X(x) = C_1 + C_2 x
        \end{equation*}
        Знаходимо константи з межових умов:
        \begin{equation*}
            \begin{aligned}
                &\left\{ \begin{aligned}
                    &X(0) = C_1 = 0, \\ 
                    &X(l) = C_1 + C_2 l = 0;
                \end{aligned} \right.
                \\   
                &\left\{ \begin{aligned}
                    C_1 = 0, \\ 
                    C_2 = 0;
                \end{aligned} \right.
            \end{aligned}
            \quad\Rightarrow\;
            X(x) = 0 \text{ -- розв'язок тривівльний}
        \end{equation*}
    
        \item Розглянемо випадок $\lambda < 0$. Розв'язок рівняння шукаємо у виді $X(x) = e^{\alpha x}$, підставимо це в рівняння: 
        \begin{equation*}
            \begin{aligned}
                &X^{\prime\prime} = -\lambda X
                \quad\to\quad
                \alpha^2 \textcolor{red}{\begin{xy}*{\textcolor{black}{e^{\alpha x}}};p+LU;+RD**h@{}+/\jot/**h@{-}\end{xy}} = +|\lambda| \textcolor{red}{\begin{xy}*{\textcolor{black}{e^{\alpha x}}};p+LU;+RD**h@{}+/\jot/**h@{-}\end{xy}}
                \quad\to\quad
                \alpha = \pm \sqrt{|\lambda|}
                \;\Rightarrow\\
                \Rightarrow\;
                &X(x) = \widetilde{C}_1 e^{\sqrt{|\lambda|}x} + \widetilde{C}_2 e^{-\sqrt{|\lambda|}x} \equiv C_1 sh(\sqrt{|\lambda|}x) + C_2 ch({\sqrt{|\lambda|}x})
            \end{aligned}
        \end{equation*}
        Знаходимо константи з межових умов:
        \begin{equation*}
            \begin{aligned}
                &\left\{ \begin{aligned}
                    &X(0) = C_2 = 0, \\ 
                    &X(l) = C_1 sh(\sqrt{|\lambda|}l) + C_2 ch({\sqrt{|\lambda|}l}) = 0;
                \end{aligned} \right.
                \;\to\\
                \to\;
                &\left\{ \begin{aligned}
                    &C_2 = 0, \\ 
                    &C_1 sh(\sqrt{|\lambda|}l) = 0, \\
                    &sh(\sqrt{|\lambda|}l) \neq 0;
                \end{aligned} \right.
                \;\to\;
                \left\{ \begin{aligned}
                    C_1 = 0, \\ 
                    C_2 = 0;
                \end{aligned} \right.
                \quad\Rightarrow\;
                \text{розв'язок тривівльний}
            \end{aligned}
        \end{equation*}
    
        \item Розглянемо випадок $\lambda > 0$. Розв'язок рівняння шукаємо у виді $X(x) = e^{\alpha x}$, підставимо це в рівняння: 
        \begin{equation*}
            \begin{aligned}
                &X^{\prime\prime} = -\lambda X
                \quad\to\quad
                \alpha^2 \textcolor{red}{\begin{xy}*{\textcolor{black}{e^{\alpha x}}};p+LD;+RU**h@{}+/\jot/**h@{-}\end{xy}} = -\lambda \textcolor{red}{\begin{xy}*{\textcolor{black}{e^{\alpha x}}};p+LD;+RU**h@{}+/\jot/**h@{-}\end{xy}}
                \quad\to\quad
                \alpha = \pm i\sqrt{\lambda}
                \;\Rightarrow\\
                \Rightarrow\;
                &X(x) = \widetilde{C}_1 e^{i\sqrt{\lambda}x} + \widetilde{C}_2 e^{-\sqrt{\lambda}x} \equiv C_1 \sin(\sqrt{\lambda}x) + C_2 \cos({\sqrt{\lambda}x})
            \end{aligned}
        \end{equation*}
        Знаходимо константи з межових умов:
        \begin{equation*}
            \begin{aligned}
                \left\{ \begin{aligned}
                    &X(0) = C_2 = 0, \\ 
                    &X(l) = C_1 \sin(\sqrt{\lambda}l) + \textcolor{red}{\begin{xy}*{\textcolor{black}{C_2}};p+LU;+RD**h@{}+//**h@{-}*h@{>}*h!LD{\scriptstyle 0}\end{xy}} \cos({\sqrt{\lambda}l}) = 0;
                \end{aligned} \right.
                \;\to\;
                \left\{ \begin{aligned}
                    &C_1 \neq 0, \\ 
                    &\sin(\sqrt{\lambda}l) = 0;
                \end{aligned} \right.
            \end{aligned}
        \end{equation*}
        Маємо нетривіальний розв'язок. Визначимо з характерисичного рівняння при яких значеннях $\lambda$ він можливий:
        \begin{equation*}
            \sin(\sqrt{\lambda}l) = 0
            \;\to\;
            \sqrt{\lambda}l = \pi n, \, n \in \mathbb{Z}
            \;\to\;
            \lambda_n = \frac{\pi^2 n^2}{l^2}, \, n \in \mathbb{N}.
        \end{equation*}
    \end{enumerate}
\end{enumerate} 
Отже, ми визначили всі власні значення та відповідні їм власні функції.
    \begin{equation} \label{ShLsol}
        \left\{ \begin{aligned}
            \;&\lambda_n = \frac{\pi^2 n^2}{l^2},\\ 
            &X_n(x) = C_n \sin\left(\frac{\pi n x}{l}\right),
        \end{aligned} \right.
        \quad \text{де } n \in \mathbb{N}.
    \end{equation}
    
Повертаємося до рівняння для $T(t)$ - (\ref{sepvar}). Підставляємо знайдені значення та знаходимо $T_n(t)$:
\begin{equation*}
    \left. \begin{aligned}
        \lambda_n = \frac{\pi^2 n^2}{l^2},&\;\\ 
        T^{\prime\prime} + \lambda v^2T = 0,&
    \end{aligned} \right\}
    \;\Rightarrow\;
    T_n(t) = A\cos(\omega_n t) + B\sin(\omega_n t),
\end{equation*}
де $\omega_n^2 = \lambda_n v^2, \, n \in \mathbb{N}.$\\
Власними модами коливань струни будуть всі розв'язки вигляду:
\begin{equation*}
    u_n(x,t) = X_n(x) \cdot T_n(t)
\end{equation*}
Виконаємо перепозначення і запишемо остаточний розв'язок:
\begin{equation}
    \left\{ \begin{aligned}
        \;&u_n(x,t) = \left(A_n\cos(\omega_n t) + B_n\sin(\omega_n t)\right) \sin(k_n x), \\
        &k_n = \frac{\pi n}{l} - \text{ хвильові вектори}, \\
        &\omega_n = vk_n = \frac{v \pi n}{l} - \text{ власні частоти}, \\
        &n = 1, 2,\ldots
    \end{aligned}\right.
\end{equation}
\newpage

\begin{center}
    \large{\textbf{Перевірка розв'язку задачі Штурма-Ліувілля}}
\end{center}

\noindent Щоб перевірити правильність отриманого результату, (\ref{ShLsol}), треба використовувати постановки самої задачі -- (\ref{sepvar}). 
\begin{enumerate}
    \item Аналітична перевірка (пряма підстановка результату в рівняння).
    \begin{enumerate}
        \item[1)] Перевіряємо рівняння, для цього обчислимо другу похідну власної функції. 
        \begin{equation*}
            X_n^{''} = \frac{\pi n}{l} \left(C_n\cos\left(\frac{\pi n x}{l}\right)\right)^{'} = -\left(\frac{\pi n}{l}\right)^2 C_n\sin\left(\frac{\pi n x}{l}\right) = -\left(\frac{\pi n}{l}\right)^2 X_n
        \end{equation*}
        Порівнюємо з вихідним рівнянням і робимо перший висновок: кожна з функцій $X_n(x)$ дійсно є розв’язком рівняння (\ref{sepvar}) для значення спектрального параметра $\lambda = \left(\frac{\pi n}{l}\right)^2$. Далі, ці значення $\lambda$ збігаються зі знайденими раніше власними значеннями (\ref{ShLsol}), отже робимо другий висновок: знайдені власні значення дійсно відповідають знайденим власним функціям.
        \item[2)] Перевіряємо виконання межових умов, підставляємо власні функції в межові умови.
        \begin{equation*}
            \begin{aligned}
                X_n(0) &=  0:\\
                &\begin{aligned}
                    X_n(0) = C_n \sin(\sqrt{\lambda_n} \cdot 0) = 0 &\text{ -- виконується,}\\
                    &\text{ причому незалежно від }\lambda_n
                \end{aligned}\\
                \\
                X_n(l) &= 0:\\
                &\begin{aligned}
                    X_n(l) = C_n \sin\left(\frac{\pi n}{l} \cdot l\right) = C_n &\sin(\pi n) = 0 \text{ -- виконується,  але}\\
                    &\text{ саме для знайдених значень }\lambda_n
                \end{aligned}\\
            \end{aligned}
        \end{equation*}
    \end{enumerate}
    \item Графічна перевірка:
    \begin{center}
        \begin{tikzpicture}
            \begin{axis}
                [width = 0.85\textwidth, height = 0.4\textwidth,
                 axis x line = center, axis y line = center,
                 ylabel = $X(x)$, xlabel = $x$,
                 xmin = 0.7, xmax = 6.7, ymin = -3.3, ymax = 5.3,
                 axis line style = thin, xtick = {0}, ytick = {0}]   
                
                \tikzmath{\A1 = 5; \l = 5; \k = pi/\l;}
                
                \addplot [black, domain = 1:(\l+1), samples = 1000] {\A1 * sin(deg(\k*(x-1)))}
                node[anchor=north, pos=0] {0} 
                node[pos=0.75, fill=white] {$X_0(x)$} 
                node[anchor=north, pos=1] {$l$};
                
                \addplot [black, domain = 1:(\l+1), samples = 1000] {\A1/2.5 * sin(deg(2*\k*(x-1)))}
                node[pos=0.83, fill=white] {$X_1(x)$};
                
                \addplot [black, domain = 1:(\l+1), samples = 1000] {\A1/3.7 * sin(deg(3*\k*(x-1)))}
                node[pos=0.74, fill=white] {$X_2(x)$};
                
            \end{axis}
        \end{tikzpicture}
    \end{center}
\end{enumerate}




\end{document}