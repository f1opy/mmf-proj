%\documentclass[a4paper, 14pt]{extreport}

%\usepackage{StyleMMF}

%\begin{document}

%\chapter{Другий спосіб знаходження коефіцієнтів. Коливання стержня з вільними кінцями, неповнота базису.}

\section[Задача №3.3]{3.3}

\textit{Знайти коливання пружного стержня довжиною $l$ з вільними кінцями, якщо початкове відхилення дорівнює нулю, а початкова швидкість $\psi(x) = \nu_0$. Якщо всі знайдені вами коефіцієнти Фур'є (коефіцієнти загального\\ розв’язку) дорівнюють нулю, поясніть, що це означає, і знайдіть, де була допущена помилка.}

%\end{document}