% \documentclass[a4paper, 14pt]{extreport}
%
%\usepackage{../../main/StyleMMF}
%\usepackage{subcaption}
%
%\setcounter{chapter}{12}
%
%\begin{document}
%
%\chapter{Функції Гріна і розв’язки задач для рівнянь у частинних похідних з однорідними межовими умовами}

\section[Задача №13.6]{13.6}

\textit{Поставити задачу на функцію Гріна $G(\vec{r},\vec{r}')$ крайової задачі для 3-D рівняння Гельмгольца $\Delta_3 u - \mu^2 u = -f(\vec{r})$ у необмеженому просторі з умовою прямування розв’язку до нуля на нескінченності і розв’язати її за допомогою інтегрального перетворення Фур’є, дати фізичну інтерпретацію розв’язку у термінах стаціонарної дифузії частинок зі скінченним часом життя. Граничним переходом $\mu \to +0$ перейти до функції Гріна рівняння Лапласа. Записати розв’язок задачі з довільним джерелом $f(\vec{r})$ через функцію Гріна.}

\begin{center}
    \large{\textbf{Розв'язок}}
\end{center}

Постановка задачі на фінкцію Гріна 
\begin{equation}
    \left\{ \begin{aligned}
        \,& u = G(\vec{r}, \vec{r}^{\,\prime}),\\
        & \Delta_3 u - \mu^2 u = -\delta(\vec{r} - \vec{r}^{\,\prime})
    \end{aligned} \right.
\end{equation}

Виконаємо перетворення Фур'є рівняння
\begin{equation*}
    - k_x^2 \hat{u} - k_y^2 \hat{u} - k_z^2 \hat{u} - \mu^2 \hat{u} = -e^{i(\vec{k} \cdot (\vec{r} - \vec{r}^{\,\prime}))}
\end{equation*}
Звідки отримаємо Фур'є-образ функції Гріна
\begin{equation}
    \hat{u}(\vec{k}) = \frac{1}{k^2 + \mu^2} \cdot e^{i(\vec{k} \cdot (\vec{r} - \vec{r}^{\,\prime}))}
\end{equation} 

Тепер функції Гріна знайдемо оберненим перетворенням Фур'є
\begin{equation}
    G(\vec{r} - \vec{r}^{\,\prime}) = \frac{1}{(2\pi)^3} \int_{\mathbb{R}^3} \frac{e^{i(\vec{k} \cdot (\vec{r} - \vec{r}^{\,\prime}))}}{k^2 + \mu^2} \;\mathrm{d}\vec{k} \; \textcolor{red}{=}
\end{equation}  
Залишається обчислити отриманий інтеграл. Це не важко зробити використовуючи лемму Жордана та обчислючи лишки.\\
Переходимо в сферичні координати і позначимо $\rho = |\vec{r} - \vec{r}^{\,\prime}|$ та $\theta$ -- кут між векторами $\vec{k}$ та $\vec{r} - \vec{r}^{\,\prime}$ 
\begin{equation*}
    \begin{gathered}
        \textcolor{red}{=} \; \frac{1}{(2\pi)^3} \int_0^{2\pi} \;\mathrm{d}\phi \int_0^\infty k^2 \;\mathrm{d}k \int_0^\pi \frac{e^{ik\rho \cos\theta}}{k^2 + \mu^2} \;\mathrm{d}(\cos\phi) =\\
        = \frac{2\pi}{(2\pi)^3} \int_0^\infty \frac{k^2}{k^2 + \mu^2} \bigg( \frac{e^{ik\rho \cos\theta}}{ik\rho} \bigg) \bigg|_0^\pi \;\mathrm{d}k =\\
        = \frac{1}{8\pi^2i\rho} \bigg[ \int_{-\infty}^\infty \frac{ke^{-ik\rho}}{k^2 + \mu^2} \;\mathrm{d}k - \int_{-\infty}^\infty \frac{ke^{ik\rho}}{k^2 + \mu^2} \;\mathrm{d}k \bigg] \; \textcolor{red}{=}
    \end{gathered}
\end{equation*}  
Підінтегральній вираз має 2 особливі точки $k = \pm i\mu$. Для першого інтегралу потрібно розглянути контур в комплексній півплощині, де $\mathrm{Im}z > 0$, а для другого навпаки -- $\mathrm{Im}z < 0$. 

\begin{figure}[h]
    \centering
    %Graph under comment
    % Зображення контурів на комплексній площині

  \begin{minipage}{.49\textwidth}
    \centering
    \begin{tikzpicture}
      \begin{axis}
          [width = \textwidth,
          axis lines = center,
          ylabel = $\mathrm{Im}z$, xlabel = $\mathrm{Re}z$,
          xmin = -5, xmax = 5, ymin = -5, ymax = 5,
          axis line style = thin, ticks = none]   
          
          \tikzmath{\R = 3.75;}

          \draw[red, thick] (-\R,0) -- (\R,0) arc(0:180:\R) --cycle;
          \addplot[red, thick, domain=-\R:\R] {0};   
          
          \addplot[mark=*, red] coordinates {(0,1)}
          node[anchor=160, pos=0.5] {\footnotesize{$k = i\mu$}};

          \addplot[mark=*] coordinates {(0,-1)}
          node[anchor=160, pos=0.5] {\footnotesize{$k = -i\mu$}};

      \end{axis}
    \end{tikzpicture}

    \captionof{figure}{Контур для 1 інтегралу}
  \end{minipage}
  \begin{minipage}{.49\textwidth}
    \centering
    \begin{tikzpicture}
      \begin{axis} 
          [width = \textwidth,
          axis lines = center,
          ylabel = $\mathrm{Im}z$, xlabel = $\mathrm{Re}z$,
          xmin = -5, xmax = 5, ymin = -5, ymax = 5,
          axis line style = thin, ticks = none]   
          
          \tikzmath{\R = 3.75;}

          \draw[red, thick] (-\R,0) -- (\R,0) arc(0:-180:\R) --cycle;
          \addplot[red, thick, domain=-\R:\R] {0};   

          \addplot[mark=*] coordinates {(0,1)}
          node[anchor=160, pos=0.5] {\footnotesize{$k = i\mu$}};

          \addplot[mark=*, red] coordinates {(0,-1)}
          node[anchor=160, pos=0.5] {\footnotesize{$k = -i\mu$}};

      \end{axis}
    \end{tikzpicture}
    
    \captionof{figure}{Контур для 2 інтегралу}
  \end{minipage}
  % main compilation
    %% Зображення контурів на комплексній площині

  \begin{minipage}{.49\textwidth}
    \centering
    \begin{tikzpicture}
      \begin{axis}
          [width = \textwidth,
          axis lines = center,
          ylabel = $\mathrm{Im}z$, xlabel = $\mathrm{Re}z$,
          xmin = -5, xmax = 5, ymin = -5, ymax = 5,
          axis line style = thin, ticks = none]   
          
          \tikzmath{\R = 3.75;}

          \draw[red, thick] (-\R,0) -- (\R,0) arc(0:180:\R) --cycle;
          \addplot[red, thick, domain=-\R:\R] {0};   
          
          \addplot[mark=*, red] coordinates {(0,1)}
          node[anchor=160, pos=0.5] {\footnotesize{$k = i\mu$}};

          \addplot[mark=*] coordinates {(0,-1)}
          node[anchor=160, pos=0.5] {\footnotesize{$k = -i\mu$}};

      \end{axis}
    \end{tikzpicture}

    \captionof{figure}{Контур для 1 інтегралу}
  \end{minipage}
  \begin{minipage}{.49\textwidth}
    \centering
    \begin{tikzpicture}
      \begin{axis} 
          [width = \textwidth,
          axis lines = center,
          ylabel = $\mathrm{Im}z$, xlabel = $\mathrm{Re}z$,
          xmin = -5, xmax = 5, ymin = -5, ymax = 5,
          axis line style = thin, ticks = none]   
          
          \tikzmath{\R = 3.75;}

          \draw[red, thick] (-\R,0) -- (\R,0) arc(0:-180:\R) --cycle;
          \addplot[red, thick, domain=-\R:\R] {0};   

          \addplot[mark=*] coordinates {(0,1)}
          node[anchor=160, pos=0.5] {\footnotesize{$k = i\mu$}};

          \addplot[mark=*, red] coordinates {(0,-1)}
          node[anchor=160, pos=0.5] {\footnotesize{$k = -i\mu$}};

      \end{axis}
    \end{tikzpicture}
    
    \captionof{figure}{Контур для 2 інтегралу}
  \end{minipage}
  % this compilation
\end{figure}

За лемою Жордана інтеграл вздовж півкола буде прямувати до нуля при прямуванні його радіуса до нескінченності, тому значення інтегралів, які ми отримали раніше дорівнює 
\begin{equation*}
    \begin{gathered}
        \textcolor{red}{=} \; \frac{2\pi i}{8\pi^2i\rho} \bigg[ \mathop{\mathrm{Res}}_{k = i\mu} \frac{ke^{-ik\rho}}{k^2 + \mu^2} + \mathop{\mathrm{Res}}_{k = -i\mu} \frac{ke^{ik\rho}}{k^2 + \mu^2} \bigg] = \frac{1}{4\pi \rho} \bigg[ \lim_{k = i\mu} \frac{ke^{-ik\rho}}{k + i\mu} + \lim_{k = -i\mu} \frac{ke^{ik\rho}}{k - i\mu} \bigg] =\\
        = \frac{1}{4\pi \rho} \bigg[ \frac{i\mu e^{\mu\rho}}{2i\mu} + \frac{-i\mu e^{\mu\rho}}{-2i\mu} \bigg] = \frac{1}{4\pi\rho} e^{\mu\rho} = \frac{e^{\mu|\vec{r} - \vec{r}^{\,\prime}|}}{4\pi |\vec{r} - \vec{r}^{\,\prime}|}
    \end{gathered}
\end{equation*}  
Отже, маємо функцію Гріна для рівняння Гельмгольца
\begin{equation}
    G(\vec{r} - \vec{r}^{\,\prime}) = \frac{e^{\mu|\vec{r} - \vec{r}^{\,\prime}|}}{4\pi |\vec{r} - \vec{r}^{\,\prime}|}
\end{equation} 

Фізична інтерпритація: ???

Знайдемо функцію Гріна для рівняння Лапласа
\begin{equation}
    G(\vec{r} - \vec{r}^{\,\prime}) = \lim_{\mu \to +0} \frac{e^{\mu|\vec{r} - \vec{r}^{\,\prime}|}}{4\pi |\vec{r} - \vec{r}^{\,\prime}|} = \frac{1}{4\pi |\vec{r} - \vec{r}^{\,\prime}|}
\end{equation} 
Розв'язок задачі для довільного джерела
\begin{equation}
    u(\vec{r}) = \int_{-\infty}^{+\infty} G(\vec{r} - \vec{r}^{\,\prime}) f(\vec{r}^{\,\prime}) \; \mathrm{d}\vec{r}^{\,\prime} = \frac{1}{4\pi} \int_{-\infty}^{+\infty} \frac{f(\vec{r}^{\,\prime})}{|\vec{r} - \vec{r}^{\,\prime}|} \; \mathrm{d}\vec{r}^{\,\prime}
\end{equation}

%\end{document}