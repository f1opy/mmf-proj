\documentclass[a4paper, 14pt]{extreport}

\usepackage{../../main/StyleMMF}

\setcounter{chapter}{12}

\begin{document}

\chapter{Функції Гріна і розв’язки задач для рівнянь у частинних похідних з однорідними межовими умовами}

\section[Задача №13.6]{13.6}

\textit{Поставити задачу на функцію Гріна $G(\vec{r},\vec{r}')$ крайової задачі для 3-D рівняння Гельмгольца $\Delta_3 u - \mu^2 u = -f(\vec{r})$ у необмеженому просторі з умовою прямування розв’язку до нуля на нескінченності і розв’язати її за допомогою інтегрального перетворення Фур’є, дати фізичну інтерпретацію розв’язку у термінах стаціонарної дифузії частинок зі скінченним часом життя. Граничним переходом $\mu \to +0$ перейти до функції Гріна рівняння Лапласа. Записати розв’язок задачі з довільним джерелом $f(\vec{r})$ через функцію Гріна.}


\end{document}