%\documentclass[a4paper, 14pt]{extreport}
%
%\usepackage{../../main/StyleMMF}
%
%\setcounter{chapter}{12}
%
%\begin{document}
%
%\chapter{Функції Гріна і розв’язки задач для рівнянь у частинних похідних з однорідними межовими умовами}

\section[Задача №13.7]{13.7}

\textit{Знайти функцію Гріна $G(x,x')$ крайової задачі для одновимірного рівняння Гельмгольца \[u'' - \mu^2 u = - f(x), \quad -\infty < x < +\infty, \, |u| < \infty \text{ при } x \to \pm\infty \] за допомогою інтегрального перетворення Фур’є. Порівняти результат з розв’язком задачі 12.5б.}

\begin{center}
    \large{\textbf{Розв'язок}}
\end{center}

Постановка задачі на фінкцію Гріна 
\begin{equation}
    \left\{ \begin{aligned}
        \,& u = G(x, x'),\\
          & u'' - \mu^2 u = -\delta(x)
    \end{aligned} \right.
\end{equation}

Перетворимо рівняння за Фур'є 
\begin{equation*}
    - k^2 \hat{u} - \mu^2 \hat{u} = -e^{ikx}
\end{equation*}
Маємо Фур'є-образ
\begin{equation}
    \hat{u}(k) = \frac{e^{ikx}}{k^2 + \mu^2} 
\end{equation} 
Тепер функції Гріна знайдемо оберненим перетворенням Фур'є
\begin{equation}
    G(x) = \frac{1}{2\pi} \int_{-\infty}^{+\infty} \frac{e^{ikx}}{k^2 + \mu^2} \;\mathrm{d}k 
\end{equation}  
Аналогічно до задачі №13,6 будемо шукати лишки, але в цій задачі треба розглянути окремо дві області $x > 0$ та $x < 0$ і зшити їх.\\
При $x > 0$ розглядаємо контур з $\mathrm{Im}z < 0$:
\begin{equation*}
    G(x>0) = \frac{1}{2\pi} \int_{-\infty}^{+\infty} \frac{e^{ikx}}{k^2 + \mu^2} \;\mathrm{d}k = i \mathop{\mathrm{Res}}_{k = -i\mu} \frac{ke^{-ikx}}{k^2 + \mu^2} = i \lim_{k = -i\mu} \frac{ke^{-ikx}}{k - i\mu} = -\frac{e^{\mu x}}{2\mu}
\end{equation*}
При $x < 0$ розглядаємо контур з $\mathrm{Im}z > 0$:
\begin{equation*}
    G(x<0) = i \mathop{\mathrm{Res}}_{k = -i\mu} \frac{ke^{-ikx}}{k^2 + \mu^2} = i \lim_{k = -i\mu} \frac{ke^{-ikx}}{k - i\mu} = -\frac{e^{-\mu x}}{2\mu}
\end{equation*}
Отже, функція Гріна має вигляд
\begin{equation}
    G(x, x') = 
    \left\{ \begin{aligned}
        \;& -\frac{1}{2\mu} e^{-\mu (x - x')} , \; x < 0 \\
          & -\frac{1}{2\mu} e^{\mu (x - x')} , \; x > 0,
    \end{aligned} \right.
\end{equation}
або 
\begin{equation}
    G(x,x') = -\frac{1}{2\mu} e^{-\mu |x - x'|}
\end{equation}

Отриманий вираз выдповідає результату пунтку б) задачі №12,5 

%\end{document}