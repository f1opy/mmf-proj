%\documentclass[a4paper, 14pt]{extreport}

%\usepackage{StyleMMF}

%\setcounter{chapter}{10}

%\begin{document}

%\chapter{Рівняння Лапласа в прямокутній області.}

\section[Задача №11.3]{11.3}

\textit{Знайти електростатичний потенціал всередині області, обмеженої провідними пластинами $y=0, y=b, x=0$, якщо пластина $x=0$ заряджена до потенціалу $V$, а інші -- заземлені. Заряди всередині області відсутні. Розв’язком якої задачі є знайдена функція у півпросторі $x>0$?\\
Вказівка. Це приклад задачі для рівняння Лапласа в необмеженій області. Подумайте, яку умову слід накласти на розв’язок при
$x \to +\infty$, щоб для $V=0$ задача мала лише нульовий розв’язок (в іншому разі розв’язок задачі не буде єдиним).\\
Ряд просумувати.\\
Указівка: скористайтесь формулою для суми геометричної прогресії.}


%\end{document}