%\documentclass[a4paper, 14pt]{extreport}
%
%\usepackage{../../main/StyleMMF}
%
%\setcounter{chapter}{10}
%
%\begin{document}
%
%\chapter{Рівняння Лапласа в прямокутній області.}

\section[Задача №11.1]{11.1}

\textit{Знайти стаціонарний розподіл температури в однорідній прямокутній пластині, якщо вздовж лівої її сторони (довжиною $b$) підтримується заданий розподіл температури, права сторона теплоізольована, а верхня і нижня (довжиною $a$) підтримуються при нульовій температурі. Відповідь запишіть через коефіцієнти Фур’є розподілу температури на лівій стороні, вважаючи їх відомими. Які якісні зміни відбуваються у розв’язку при переході від довгої і вузької пластини ($a \gg b$) до короткої і широкої ($a \ll b$)? Намалюйте для цих випадків графіки функцій, що описують зміну температури в повздовжньому напрямку для кількох перших поперечних мод; функції нормуйте так, щоб на лівій стороні пластини вони приймали однакове значення одиниця. Як змінюється в залежності від співвідношення сторін відносна роль внесків різних поперечних мод у розподіл температури на правій стороні пластини?}

\begin{center}
    \large{\textbf{Розв'язок}}
\end{center}

Для стаціонарної задачі $u \neq u(t)$ рівняння параболічного типу, яке відповідає задачі теплопровідності, перетворюється на на еліптичне. Тобто нам потрібно розглянути задачу, де є дві незалежні просторові змінні. Запишемо постановку задачі:
\begin{equation} \label{cond11,1}
    \left\{ \begin{aligned} 
        \;&u = u(x,y), \\
          &\Delta u = u_{xx} + u_{yy} = 0, \\
          &0 \leq x \leq a,\\ &0 \leq y \leq b, \\
          &u(0,y) = \varphi(y),\\ &u_x(a,y) = 0,\\
          &u(x,0) = 0,\\ &u(x,b) = 0.
    \end{aligned} \right.
\end{equation}

Виконаємо розділення змінних $u(x,y) = X(x)\cdot Y(y)$. Для $Y(y)$ отримаємо багато разів розв'язану задачу Штурма-Ліувілля, а для $X(x)$ -- лінійне рівняння.
\begin{equation} 
    \left\{ \begin{aligned}
        \;&Y'' + \lambda Y = 0, \\ 
          &0 \leq y \leq b, \\
          &Y(0) = 0,\, Y(b) = 0 
    \end{aligned} \right.
    \quad\Rightarrow\quad
    \left\{ \begin{aligned}
        \;& Y_n(y) = \sin k_ny, \\
          & k_n = \sqrt{\lambda_n} = \pi n/b, n \in \mathbb{N} 
    \end{aligned} \right.
\end{equation}
Запишемо розв'язок рівняння для $X(x)$
\begin{equation}
    X'' - k_n^2 X = 0
    \quad\Rightarrow\quad
    X_n(x) = A_n\mathrm{sh}k_nx + B_n\mathrm{ch}k_nx
\end{equation}

Виконуємо зворотню заміну та отримаємо, виконуючи підсумовування по всім модам, загальний розв'язок задачі.
\begin{equation}
    u(x,y) = \sum_{n=1}^{\infty} \left(A_n\mathrm{sh}k_nx + B_n\mathrm{ch}k_nx\right) \sin k_ny
\end{equation}

Залишається із межових умов для змінної $x$ визначити невідомі константи $A_n$ та $B_n$. Маємо
\begin{equation}
    \left\{ \begin{aligned}
        \;&u(0,y) = \sum_{n=1}^{\infty} B_n \sin k_ny = \varphi(y),\\
          &u_x(a,y) = \sum_{n=1}^{\infty} \left(A_nk_n\mathrm{ch}k_na + B_nk_n\mathrm{sh}k_na\right) \sin k_ny = 0.
    \end{aligned} \right.
\end{equation}
В правій частині першого рівняння підставимо розклад межової умови в ряд Фур'є, який вважається відомим, та визначимо $B_n$
\begin{equation}
    \sum_{n=1}^{\infty} B_n \sin k_ny = \frac{2}{b} \sum_{n=1}^{\infty} \varphi_n \sin k_ny
    \quad\Rightarrow\quad 
    B_n = \frac{2}{b} \varphi_n
\end{equation} 
З другої, однорідної, межової умови маємо
\begin{equation}
    A_n\mathrm{ch}k_na + B_n\mathrm{sh}k_na = 0
    \quad\Rightarrow\quad
    A_n = - B_n \mathrm{th}k_na = -\frac{2}{b}\varphi_n \mathrm{th}k_na
\end{equation}
Підставляємо отримані значення в загальний розв'язок
\begin{equation} \label{gensol11,1}
    u(x,y) = \frac{2}{b}\sum_{n=1}^{\infty} \varphi_n \big(\mathrm{ch}k_nx - \mathrm{th}(k_na)\mathrm{sh}k_nx\big) \sin k_ny,
\end{equation}
або, скориставшись однією з властивостей гіперболічних функцій, запишемо
\begin{equation}
    u(x,y) = \frac{2}{b}\sum_{n=1}^{\infty} \varphi_n \cdot \frac{\mathrm{ch}\big(k_n(x-a)\big)}{\mathrm{ch}k_na} \sin k_ny
\end{equation}

Розглянемо, використовуючи формулу (\ref{gensol11,1}), граничні випадки: а) довгої і вузької пластини, б) короткої і широкої.\\
a) $a \gg b$
\begin{equation*}
    \mathrm{th}k_na = \mathrm{th}(\pi na/b)\bigg|_{a \gg b} \to 1
\end{equation*}
Таким чином розв'язок переходить в 
\begin{equation}
    u(x,y) = \frac{2}{b}\sum_{n=1}^{\infty} \varphi_n \big(\mathrm{ch}k_nx - \mathrm{sh}k_nx\big) \sin k_ny = \frac{2}{b}\sum_{n=1}^{\infty} \varphi_n e^{-k_nx} \sin k_ny,
\end{equation}
що відповідає рівнянню теплопровідності для одновимірного випадку; температура спадає за експоненційним законом при віддалені від джерела   

б) $a \ll b$
\begin{equation*}
    \mathrm{th}(\pi na/b)\bigg|_{a \ll b} \to 0, \quad \mathrm{ch}(\pi nx/b)\bigg|_{b\to\infty} \to 1,
\end{equation*}
оскільки $x \leq a$, то умову $a \ll b$ можна замінити на $b\to\infty$

Отже, при зменшенні $b$ зменшуються втрати теплоти, оскільки ширина пластинки набагато менша за довжину джерела.\\
Таким чином розв'язок переходить в 
\begin{equation}
    u(x,y) = \frac{2}{b}\sum_{n=1}^{\infty} \mathrm{ch}k_nx\sin k_ny
\end{equation}

%\end{document}