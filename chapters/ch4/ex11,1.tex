%\documentclass[a4paper, 14pt]{extreport}

%\usepackage{StyleMMF}

%\setcounter{chapter}{10}

%\begin{document}

%\chapter{Рівняння Лапласа в прямокутній області.}

\section[Задача №11.1]{11.1}

\textit{Знайти стаціонарний розподіл температури в однорідній прямокутній пластині, якщо вздовж лівої її сторони (довжиною $b$) підтримується заданий розподіл температури, права сторона теплоізольована, а верхня і нижня (довжиною $a$) підтримуються при нульовій температурі. Відповідь запишіть через коефіцієнти Фур’є розподілу температури на лівій стороні, вважаючи їх відомими. Які якісні зміни відбуваються у розв’язку при переході від довгої і вузької пластини ($a \gg b$) до короткої і широкої ($a \ll b$)? Намалюйте для цих випадків графіки функцій, що описують зміну температури в повздовжньому напрямку для кількох перших поперечних мод; функції нормуйте так, щоб на лівій стороні пластини вони приймали однакове значення одиниця. Як змінюється в залежності від співвідношення сторін відносна роль внесків різних поперечних мод у розподіл температури на правій стороні пластини?}


%\end{document}