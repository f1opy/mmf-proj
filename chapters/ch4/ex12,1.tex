%\documentclass[a4paper, 14pt]{extreport}

%\usepackage{StyleMMF}

%\setcounter{chapter}{11}

%\begin{document}

%\chapter{Функції Гріна звичайних диференціальних задач}

\section[Задача №12.1]{12.1}

\textit{Функція Гріна $G(t)$ задачі Коші для рівняння гармонічного осцилятора \[y'' + \omega^2y = f(t), \, t \geq 0, \, y(0)=y_0, \, y'(0)=\nu_0\] є розв’язком цієї задачі при $\nu_0 = 1 , \, y_0 = 1$ і $f(t) = 0$. Тобто $y = G(t)$ задовольняє умови \[y'' + \omega^2y = 0, \, t \geq 0, \, y(0)=1, \, y'(0)=1\] Знайдіть явний вигляд функції Гріна сцилятора; чи зберігає вона смисл при $\omega \to 0$?}


%\end{document}