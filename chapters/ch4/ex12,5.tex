%\documentclass[a4paper, 14pt]{extreport}
%
%\usepackage{../../main/StyleMMF}
%
%\setcounter{chapter}{11}

%\begin{document}

%\chapter{Функції Гріна звичайних диференціальних задач}

\section[Задача №12.5]{12.5}

\textit{Функція Гріна $G(x,x')$ крайової задачі для одновимірного рівняння Гельмгольца $u'' - \mu^2u = -f(x), \, u(0) = 0, \, |u| < \infty $ при $x \to \infty$ за означенням є неперервним розв’язком цієї задачі для $f(x) = \delta(x-x'), \, 0 < x' < \infty$.\\
а) Знайти функцію Гріна цієї задачі шляхом зшивання розв’язків однорідного рівняння і подальшого нормування (для даної задачі можливі принаймні три різні способи нормування розв’язку, які?).\\
б) Знайти функцію Гріна $G(x,x')$ крайової задачі для одновимірного рівняння Гельмгольца $u'' - \mu^2u = -f(x), \, x \in \mathbb{R}, \, |u| < \pm\infty $ при $x \to \pm\infty$ – шляхом граничного переходу $x, x' \to \infty$ при сталому $x-x'$ у $G(x,x')$, одержаній у пункті а) цієї задачі.\\
Дайте фізичну інтерпретацію знайдених функцій Гріна у термінах стаціонарної дифузії частинок зі скінченним часом життя. Якою є залежність від кожного з аргументів функції Гріна та симетрія відносно їх перестановки? Чому в одних випадках функція Гріна залежить від кожного з аргументів окремо, а в інших – тільки від їх різниці?}

\begin{center}
    \large{\textbf{Розв'язок}}
\end{center}

Постановка задачі:
\begin{equation} \label{cond12,5}
    \left\{ \begin{aligned}
        \;&u'' + \mu^2u = -f(x),\, t \geq 0,\\
          &u(0) = 0,\\
          & |u| < \infty \text{ при } x \to +\infty.
    \end{aligned} \right.
\end{equation}

Розв'язок однорідного рівняння:
\begin{equation}
    u_0(x) = C_1 e^{-\mu x} + C_2 e^{\mu x}
\end{equation}
Знайдемо вигляд розв'язків, що задовольняють також і межовим умовам. Перший розв'язок знаходимо прямою підстановкою:
\begin{equation*}
    u(0) = C_1 + C_2 = 0 \quad\Rightarrow\quad C_1 = - C_2 = 1 \quad\Rightarrow\quad u_1(x) = e^{-\mu x} - e^{\mu x}
\end{equation*}
Для другого розв'язку треба покласти $C_2$ рівним нулю, щоб позбавитись розбіжного доданку.
\begin{equation*}
    |u(+\infty)| < \infty \quad\Rightarrow\quad u_2(x) = e^{-\mu x}
\end{equation*}
Отже, маємо два розв'язки з яких зшиванням побудуємо функцію Гріна
\begin{equation}
    u_1(x) = e^{-\mu x} - e^{\mu x}, \quad u_2(x) = e^{-\mu x}
\end{equation}

Визначимо функцію Гріна за формулою
\begin{equation} \label{green-draft}
    G(x,x') = 
    \left\{ \begin{aligned}
        \;& \varphi(x') \big( e^{-\mu x} - e^{\mu x} \big), \; 0 \leq x \leq x' \\
          & \psi(x') e^{-\mu x}, \; x' \leq x \leq \infty,
    \end{aligned} \right.
\end{equation}
де $\varphi(x)$ та $\psi(x)$ -- гарні функції, та задовольняє умовам
\begin{equation} \label{green-cond}
    \left\{ \begin{aligned}
        \;& G(x,x')\bigg|_{x = x'+0} = G(x,x')\bigg|_{x = x'-0},\\
          & \frac{\partial \;}{\partial x}G(x,x')\bigg|_{x = x'+0} - \frac{\partial \;}{\partial x} G(x,x')\bigg|_{x = x'-0} = 1.
    \end{aligned} \right.
\end{equation}

Підставимо (\ref{green-draft}) в (\ref{green-cond}) та розв'яжемо отриману ситему рівнянь відносно невідомих функцій.
\begin{equation} 
    \left\{ \begin{aligned}
        \;& \psi(x') e^{-\mu x'} - \varphi(x') (e^{-\mu x'} - e^{\mu x'}) = 0,\\
          & -\mu \psi(x') e^{-\mu x'} + \mu \varphi(x') (e^{-\mu x'} + e^{\mu x'}) = 1.
    \end{aligned} \right.
\end{equation}
Поділимо на $\mu$ друге рівняння та додамо до першого
\begin{equation*} 
    \left\{ \begin{aligned}
        \;& \varphi(x') (e^{-\mu x'} + e^{\mu x'} - e^{-\mu x'} + e^{\mu x'}) = \frac{1}{\mu},\\
          & \psi(x') e^{-\mu x'} = \varphi(x') (e^{-\mu x'} - e^{\mu x'});
    \end{aligned} \right.
    \quad\Rightarrow\;
    \left\{ \begin{aligned}
        \;& 2\varphi(x') e^{\mu x'} = \frac{1}{\mu},\\
          & \psi(x')  = \varphi(x') (1 - e^{2\mu x'});
    \end{aligned} \right.
    \;\Rightarrow
\end{equation*}
\begin{equation*} 
    \Rightarrow\quad
    \left\{ \begin{aligned}
        \;& \varphi(x') = \frac{1}{2\mu}e^{-\mu x'},\\
          & \psi(x')  = \frac{1}{2\mu}e^{-\mu x'} (1 - e^{2\mu x'}) = \frac{1}{2\mu} (e^{-\mu x'} - e^{\mu x'}).
    \end{aligned} \right.
\end{equation*}
Отже, функція Гріна (\ref{green-draft}) має вигляд
\begin{equation} 
    G(x,x') = 
    \left\{ \begin{aligned}
        \;& \frac{1}{2\mu}e^{-\mu x'} \big( e^{-\mu x} - e^{\mu x} \big), \; 0 \leq x \leq x' \\
          & \frac{1}{2\mu} \big(e^{-\mu x'} - e^{\mu x'}\big) e^{-\mu x}, \; x' \leq x \leq \infty,
    \end{aligned} \right.
\end{equation}
або якщо розкрити дужки та врахувати невід'ємність різниці $(x - x')$ в аргументі другої експоненти модулем
\begin{equation} 
    G(x,x') = \frac{1}{2\mu} \big( e^{-\mu (x + x')} - e^{-\mu |x - x'|} \big)
\end{equation}

Розв'язок для задачі на нескінченному інтервалі (випадок б) ) знайдемо граничним переходом $x, x' \to \infty$ при сталій різниці $x - x'$
\begin{equation}
    \lim_{\substack{|x-x'| = \mathrm{c}\\ x,x'\to\infty}} G(x,x') = \lim_{\substack{|x-x'| = \mathrm{c}\\ x,x'\to\infty}} \frac{1}{2\mu} \big( e^{-\mu (x + x')} - e^{-\mu |x - x'|} \big) = -\frac{1}{2\mu} e^{-\mu |x - x'|}
\end{equation}  

%\end{document}