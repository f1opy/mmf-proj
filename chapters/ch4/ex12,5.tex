\documentclass[a4paper, 14pt]{extreport}

\usepackage{../../main/StyleMMF}

\setcounter{chapter}{11}

\begin{document}

\chapter{Функції Гріна звичайних диференціальних задач}

\section[Задача №12.5]{12.5}

\textit{Функція Гріна $G(x,x')$ крайової задачі для одновимірного рівняння Гельмгольца $u'' - \mu^2u = -f(x), \, u(0) = 0, \, |u| < \infty $ при $x \to \infty$ за означенням є неперервним розв’язком цієї задачі для $f(x) = \delta(x-x'), \, 0 < x' < \infty$.\\
а) Знайти функцію Гріна цієї задачі шляхом зшивання розв’язків однорідного рівняння і подальшого нормування (для даної задачі можливі принаймні три різні способи нормування розв’язку, які?).\\
б) Знайти функцію Гріна $G(x,x')$ крайової задачі для одновимірного рівняння Гельмгольца $u'' - \mu^2u = -f(x), \, x \in \mathbb{R}, \, |u| < \pm\infty $ при $x \to \pm\infty$ – шляхом граничного переходу $x, x' \to \infty$ при сталому $x-x'$ у $G(x,x')$, одержаній у пункті а) цієї задачі.\\
Дайте фізичну інтерпретацію знайдених функцій Гріна у термінах стаціонарної дифузії частинок зі скінченним часом життя. Якою є залежність від кожного з аргументів функції Гріна та симетрія відносно їх перестановки? Чому в одних випадках функція Гріна залежить від кожного з аргументів окремо, а в інших – тільки від їх різниці?}




\end{document}