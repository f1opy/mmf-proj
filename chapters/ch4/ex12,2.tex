%\documentclass[a4paper, 14pt]{extreport}
%
%\usepackage{../../main/StyleMMF}
%
%\setcounter{chapter}{11}
%
%\begin{document}
%
%\chapter{Функції Гріна звичайних диференціальних задач}

\section[Задача №12.2]{12.2}

\textit{Користуючись означенням функції Гріна $G(t)$, але не використовуючи її явного вигляду, показати безпосередньою підстановкою в умови задачі, що функція \[y(t) = \int\limits_0^t G(t - t') f(t') \;\mathrm{d}t' + y'(0) G(t) + y(0) G'(t)\] є розв’язком задачі про вимушені коливання гармонічного осцилятора при $t>0$ під дією узагальненої сили $f(t)$ з початковими умовами $y(0)=y_0, \, y'(0)=\nu_0$. Розв’язками яких частинних задач є окремі доданки цього виразу?}

\begin{center}
    \large{\textbf{Розв'язок}}
\end{center}

Закон руху  
\begin{equation} \label{sol12,2}
    y(t) = \int\limits_0^t G(t - t') f(t') \;\mathrm{d}t' + y'(0) G(t) + y(0) G'(t)
\end{equation}
є розв'язком задачі:
\begin{equation} \label{cond12,2}
    \left\{ \begin{aligned}
        \;&y'' + \omega^2y = 0,\, t \geq 0,\\
          &y(0) = y_0,\, y'(0) = v_0.
    \end{aligned} \right.
\end{equation}

Обчислимо першу похідну по часу від розв'язку (\ref{sol12,2})
\begin{equation*} 
    y'(t) = G(0)f(t) + \int\limits_0^t G'(t - t') f(t') \;\mathrm{d}t' + y'(0) G'(t) + y(0) G''(t) \textcolor{red}{=}
\end{equation*}
За означенням функції Гріна $G(t)$ (\ref{cond12,1})
\begin{equation*}
    G''(t) = -\omega^2 G(t), \quad G(0) = 0, \quad G'(0) = 1
\end{equation*}
Підставимо $G''(t)$ та $G(0)$
\begin{equation*} 
    \textcolor{red}{=} \int\limits_0^t G'(t - t') f(t') \;\mathrm{d}t' + y'(0) G'(t) - \omega^2 y(0)G(t) 
\end{equation*}

Аналогічно друга похідна
\begin{equation*} 
    \begin{gathered}
        y''(t) = G'(0) f(t) + \int\limits_0^t G''(t - t') f(t') \;\mathrm{d}t' + y'(0) G''(t) - \omega^2 y(0) G'(t) =\\
        = f(t) - \omega^2 \left(\int\limits_0^t G(t - t') f(t') \;\mathrm{d}t' + y'(0) G(t) + y(0) G'(t) \right) = f(t) - \omega^2 y(t)
    \end{gathered}
\end{equation*}

Підставимо другу похідну в рівняння
\begin{equation*}
    y'' + \omega^2 y = f(t) - \omega^2 y + \omega^2 y \equiv f(t)
\end{equation*}
Таким чином (\ref{sol12,2}) задовільняє рівняння (\ref{cond12,2}) 

Визначимо для яких задач є розв'язками кожен з доданків (\ref{sol12,2}). Для цього треба покласти 2 з 3 параметрів (зовнішня сила та початкові умови) рівними нулю.
\begin{enumerate}
    \item $y(0) = 0, y'(0) = 0$
    \[y(t) = \int\limits_0^t G(t - t') f(t') \;\mathrm{d}t' \quad \text{є розв'язком задачі:}\] 
    \begin{equation}
        \left\{ \begin{aligned}
            \;&y'' + \omega^2y = f(t),\, t \geq 0,\\
              &y(0) = 0,\, y'(0) = 0.
        \end{aligned} \right.
    \end{equation}
    \item $f(t) = 0, y(0) = 0$
    \[y(t) = y'(0) G(t) \quad \text{є розв'язком задачі:}\] 
    \begin{equation}
        \left\{ \begin{aligned}
            \;&y'' + \omega^2y = 0,\, t \geq 0,\\
              &y(0) = 0,\, y'(0) = 1.
        \end{aligned} \right.
    \end{equation}
    \item $f(t) = 0, y'(0) = 0$
    \[y(t) = y(0) G'(t) \quad \text{є розв'язком задачі:}\] 
    \begin{equation}
        \left\{ \begin{aligned}
            \;&y'' + \omega^2y = 0,\, t \geq 0,\\
              &y(0) = 1,\, y'(0) = 0.
        \end{aligned} \right.
    \end{equation}
\end{enumerate}

%\end{document}