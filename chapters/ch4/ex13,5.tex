%\documentclass[a4paper, 14pt]{extreport}
%
%\usepackage{../../main/StyleMMF}
%
%\setcounter{chapter}{12}
%
%\begin{document}
%
%\chapter{Функції Гріна і розв’язки задач для рівнянь у частинних похідних з однорідними межовими умовами}

\section[Задача №13.5]{13.5}

\textit{Обчислити Фур’є-образ і знайти формальне представлення у вигляді інтеграла Фур’є:\\ 
б) просторової дельта-функції $\delta(\vec{r} - \vec{r}')$ у необмеженому тривимірному просторі.}

\begin{center}
    \large{\textbf{Розв'язок}}
\end{center}

За означенням перетворень Фур'є мають вигляд
\begin{equation}
    \begin{gathered}
        \text{Пряме: } \hat{f}(k) = \int_{-\infty}^{+\infty} f(x) e^{-ikx} \, \mathrm{d}x, \\
        \text{Обернене: } f(x) = \frac{1}{2\pi} \int_{-\infty}^{+\infty} \hat{f}(k) e^{ikx} \, \mathrm{d}k.
    \end{gathered}
\end{equation}

Також згадаємо властивість дельта-функції 
\begin{equation}
    \int_{-\infty}^{+\infty} f(x) \delta(x - x') \, \mathrm{d}x = f(x'), 
\end{equation}

Запишемо формулу Фур'є-образу просторової дельта-функції 
\begin{equation}
    \hat{f}(\vec{k}) = \int_{\mathbb{R}^3} \delta(\vec{r} - \vec{r}^{\,\prime}) e^{-i(\vec{k}\cdot\vec{r})} \; \mathrm{d}\vec{r} \; \textcolor{red}{=} 
\end{equation}
Скористаємося властивістю експоненти та представленням просторової дельта-функції у вигляді добутку одновимірних щоб отримати три окремих перетворення Фур'є
\begin{equation*}
    \begin{gathered}
        \textcolor{red}{=} \; \int_{\mathbb{R}^3} \delta(x - x') \delta(y - y') \delta(z - z') \exp[-i (k_x x + k_y y + k_z z)] \; \mathrm{d}\vec{r} = \\
        = \int_{-\infty}^{+\infty} e^{-ik_x x} \delta(x - x') \, \mathrm{d}x \cdot \int_{-\infty}^{+\infty} e^{-ik_y y} \delta(y - y') \, \mathrm{d}y \cdot \int_{-\infty}^{+\infty} e^{-ik_z z} \delta(z - z') \, \mathrm{d}z =\\
        = e^{-ik_x x'} \cdot e^{-ik_y y'} \cdot e^{-ik_z z'} = e^{-i(\vec{k}\cdot\vec{r}^{\,\prime})}
    \end{gathered}
\end{equation*}

Тепер ми можемо знайти представлення дельта-функції у вигляді інтеграла Фур'є використовуючи обернене перетворення
\begin{equation}
    \delta(\vec{r} - \vec{r}^{\,\prime}) = \int_{\mathbb{R}^3} e^{-i(\vec{k}\cdot\vec{r}^{\,\prime})} \cdot e^{-i(\vec{k}\cdot\vec{r})} \; \mathrm{d}\vec{k} = \int_{\mathbb{R}^3} e^{i(\vec{k}\cdot(\vec{r} - \vec{r}^{\,\prime}))} \; \mathrm{d}\vec{k}
\end{equation}

%\end{document}