\documentclass[a4paper, 14pt]{extreport}

\usepackage{StyleMMF}

\begin{document}

\tableofcontents
\setcounter{page}{2}

\chapter{ЗАСТОСУВАННЯ ПРОЦЕДУРИ ФУР’Є БЕЗПОСЕРЕДНЬОГО ВІДОКРЕМЛЕННЯ ЗМІННИХ}

\section{Відокремлення змінних, задача Штурма-Ліувілля і власні моди коливань струни для різних межових умов}
%\documentclass[a4paper, 14pt]{extreport}
\usepackage[top=2cm, bottom=2cm, left=2.5cm, right=1.5cm]{geometry}

\usepackage[utf8]{inputenc}
\usepackage[english, russian, ukrainian]{babel}
\usepackage{amssymb,amsfonts,amsmath,amsthm}

\usepackage[pdftex, unicode, colorlinks=true, linkcolor=black]{hyperref}

%\usepackage{relsize} %%позволяет пользоваться функцией \mathlarger{}
\usepackage{xcolor}
\usepackage[pdf]{xy}

\usepackage{wrapfig}
\usepackage{tikz} 
\usetikzlibrary{math}

\usepackage{pgfplots}
\pgfplotsset{compat=1.18}


\usepackage{titlesec}
\titleformat{\chapter}[display]{\normalfont\Large\bfseries}{\chaptertitlename\ \thechapter}{24pt}{\large\bfseries}
\titleformat{\section}{\normalfont\Large\bfseries}{\thesection}{20pt}{\large\bfseries}

\renewcommand{\labelenumii}{\theenumii)} %% заменяем счёчтик 2 уровня вида (a), (b), (c) и т.д. на русский алфавит а), б), в), и т.д. 


\begin{document}

\tableofcontents
\setcounter{page}{2}

\chapter{ЗАСТОСУВАННЯ ПРОЦЕДУРИ ФУР’Є БЕЗПОСЕРЕДНЬОГО ВІДОКРЕМЛЕННЯ ЗМІННИХ}

\section{Відокремлення змінних, задача Штурма-Ліувілля і власні моди коливань струни для різних межових умов}

\subsection*{Задача №1.1}

\textit{\textbf{Знайти власні моди коливань струни завдовжки $l$ із закріпленими кінцями (знайти функції вигляду $u(x,t) = X(x) \cdot T(t)$, визначені і достатньо гладкі в області $0 \leq x \leq l, -\infty \leq t \leq \infty$, не рівні тотожно нулю, які задовольняють одновимірне хвильове рівняння $u_{tt} = v^2 u_{xx}$ на проміжку $0 \leq x \leq l$ і межові умови $u(0,t) = 0, u(l,t) = 0$ на його кінцях).} Результат перевірити аналітично й графічно (див. текст до модульної контрольної роботи №1, с. 25) та проаналізувати його фізичний смисл. Знайти початкові умови (початкове відхилення і початкову швидкість) для кожної з мод.}

\begin{center}
    \large{\textbf{Розв'язок}}
\end{center}

\noindent Постановка задачі:
\begin{equation}
    \left\{ \begin{aligned} %%
        \;&u = u(x,t), \\  &u_{tt} = v^2 u_{xx}, \\ &0 \leq x \leq l, t \in \mathbb{R}, \\  &u(0,t) = 0, \\ &u(l,t) = 0. 
    \end{aligned} \right.
\end{equation}
Шукаємо нетривіальні розв'язки рівняння у виді:
\begin{equation} \label{subst}
    u(x,t) = X(x) \cdot T(t) \neq 0 
\end{equation}

Тепер можливе відокремлення змінних в задачі. Почнемо з межових умов:
\begin{equation*}
    \begin{aligned}
        \;u(0,t) = X(0) \cdot T(t) = 0
        \;\Rightarrow\;
        \left\{ \begin{aligned}
            &T(t) \neq 0, \forall t, \\  &X(0) = 0; 
        \end{aligned} \right.\\
        u(l,t) = X(l) \cdot T(t) = 0
        \;\Rightarrow\;
        \left\{ \begin{aligned}
            &T(t) \neq 0, \forall t, \\  &X(l) = 0; 
        \end{aligned} \right.\\
    \end{aligned}
\end{equation*}
Далі підставимо (\ref{subst}) в рівняння та виконаємо ряд перетворень:
\begin{equation*}
    \frac{\partial^2}{\partial t^2}\left[X(x)T(t)\right] = v^2 \frac{\partial^2}{\partial x^2}\left[X(x)T(t)\right]
    \;\to\; 
    X T^{\prime\prime} = v^2 X^{\prime\prime} T 
    \;\to\; 
    \frac{T^{\prime\prime}}{v^2T} = \frac{X^{\prime\prime}}{X} = - \lambda,
\end{equation*}
де $\lambda$ -- стала відокремлення.\\
Виписуємо результат відокремлення змінних:
\begin{equation} \label{sepvar}
    \left\{ \begin{aligned}
        \;&X = X(x), \\  &X^{\prime\prime} = -\lambda X, \\ &0 \leq x \leq l, \\  &X(0) = 0, \\ &X(l) = 0. 
    \end{aligned} \right.
    \qquad\qquad
    \begin{aligned}
        T^{\prime\prime} + \lambda v^2 T = 0\\
        \lambda \text{ -- невідома}
    \end{aligned}
\end{equation}

Для $X = X(x)$ отримуємо задачу Штурма-Ліувілля. Розв'яжемо її:
\begin{enumerate}
    \item[] \begin{enumerate}
        \item Розглянемо випадок $\lambda = 0$:
        \begin{equation*}
            X^{\prime\prime} = -\lambda X
            \;\to\;
            X^{\prime\prime} = 0
            \;\to\;
            X(x) = C_1 + C_2 x
        \end{equation*}
        Знаходимо константи з межових умов:
        \begin{equation*}
            \begin{aligned}
                &\left\{ \begin{aligned}
                    &X(0) = C_1 = 0, \\ 
                    &X(l) = C_1 + C_2 l = 0;
                \end{aligned} \right.
                \\   
                &\left\{ \begin{aligned}
                    C_1 = 0, \\ 
                    C_2 = 0;
                \end{aligned} \right.
            \end{aligned}
            \quad\Rightarrow\;
            X(x) = 0 \text{ -- розв'язок тривівльний}
        \end{equation*}
    
        \item Розглянемо випадок $\lambda < 0$. Розв'язок рівняння шукаємо у виді $X(x) = e^{\alpha x}$, підставимо це в рівняння: 
        \begin{equation*}
            \begin{aligned}
                &X^{\prime\prime} = -\lambda X
                \quad\to\quad
                \alpha^2 \textcolor{red}{\begin{xy}*{\textcolor{black}{e^{\alpha x}}};p+LU;+RD**h@{}+/\jot/**h@{-}\end{xy}} = +|\lambda| \textcolor{red}{\begin{xy}*{\textcolor{black}{e^{\alpha x}}};p+LU;+RD**h@{}+/\jot/**h@{-}\end{xy}}
                \quad\to\quad
                \alpha = \pm \sqrt{|\lambda|}
                \;\Rightarrow\\
                \Rightarrow\;
                &X(x) = \widetilde{C}_1 e^{\sqrt{|\lambda|}x} + \widetilde{C}_2 e^{-\sqrt{|\lambda|}x} \equiv C_1 sh(\sqrt{|\lambda|}x) + C_2 ch({\sqrt{|\lambda|}x})
            \end{aligned}
        \end{equation*}
        Знаходимо константи з межових умов:
        \begin{equation*}
            \begin{aligned}
                &\left\{ \begin{aligned}
                    &X(0) = C_2 = 0, \\ 
                    &X(l) = C_1 sh(\sqrt{|\lambda|}l) + C_2 ch({\sqrt{|\lambda|}l}) = 0;
                \end{aligned} \right.
                \;\to\\
                \to\;
                &\left\{ \begin{aligned}
                    &C_2 = 0, \\ 
                    &C_1 sh(\sqrt{|\lambda|}l) = 0, \\
                    &sh(\sqrt{|\lambda|}l) \neq 0;
                \end{aligned} \right.
                \;\to\;
                \left\{ \begin{aligned}
                    C_1 = 0, \\ 
                    C_2 = 0;
                \end{aligned} \right.
                \quad\Rightarrow\;
                \text{розв'язок тривівльний}
            \end{aligned}
        \end{equation*}
    
        \item Розглянемо випадок $\lambda > 0$. Розв'язок рівняння шукаємо у виді $X(x) = e^{\alpha x}$, підставимо це в рівняння: 
        \begin{equation*}
            \begin{aligned}
                &X^{\prime\prime} = -\lambda X
                \quad\to\quad
                \alpha^2 \textcolor{red}{\begin{xy}*{\textcolor{black}{e^{\alpha x}}};p+LD;+RU**h@{}+/\jot/**h@{-}\end{xy}} = -\lambda \textcolor{red}{\begin{xy}*{\textcolor{black}{e^{\alpha x}}};p+LD;+RU**h@{}+/\jot/**h@{-}\end{xy}}
                \quad\to\quad
                \alpha = \pm i\sqrt{\lambda}
                \;\Rightarrow\\
                \Rightarrow\;
                &X(x) = \widetilde{C}_1 e^{i\sqrt{\lambda}x} + \widetilde{C}_2 e^{-\sqrt{\lambda}x} \equiv C_1 \sin(\sqrt{\lambda}x) + C_2 \cos({\sqrt{\lambda}x})
            \end{aligned}
        \end{equation*}
        Знаходимо константи з межових умов:
        \begin{equation*}
            \begin{aligned}
                \left\{ \begin{aligned}
                    &X(0) = C_2 = 0, \\ 
                    &X(l) = C_1 \sin(\sqrt{\lambda}l) + \textcolor{red}{\begin{xy}*{\textcolor{black}{C_2}};p+LU;+RD**h@{}+//**h@{-}*h@{>}*h!LD{\scriptstyle 0}\end{xy}} \cos({\sqrt{\lambda}l}) = 0;
                \end{aligned} \right.
                \;\to\;
                \left\{ \begin{aligned}
                    &C_1 \neq 0, \\ 
                    &\sin(\sqrt{\lambda}l) = 0;
                \end{aligned} \right.
            \end{aligned}
        \end{equation*}
        Маємо нетривіальний розв'язок. Визначимо з характерисичного рівняння при яких значеннях $\lambda$ він можливий:
        \begin{equation*}
            \sin(\sqrt{\lambda}l) = 0
            \;\to\;
            \sqrt{\lambda}l = \pi n, \, n \in \mathbb{Z}
            \;\to\;
            \lambda_n = \frac{\pi^2 n^2}{l^2}, \, n \in \mathbb{N}.
        \end{equation*}
    \end{enumerate}
\end{enumerate} 
Отже, ми визначили всі власні значення та відповідні їм власні функції.
    \begin{equation} \label{ShLsol}
        \left\{ \begin{aligned}
            \;&\lambda_n = \frac{\pi^2 n^2}{l^2},\\ 
            &X_n(x) = C_n \sin\left(\frac{\pi n x}{l}\right),
        \end{aligned} \right.
        \quad \text{де } n \in \mathbb{N}.
    \end{equation}
    
Повертаємося до рівняння для $T(t)$ - (\ref{sepvar}). Підставляємо знайдені значення та знаходимо $T_n(t)$:
\begin{equation*}
    \left. \begin{aligned}
        \lambda_n = \frac{\pi^2 n^2}{l^2},&\;\\ 
        T^{\prime\prime} + \lambda v^2T = 0,&
    \end{aligned} \right\}
    \;\Rightarrow\;
    T_n(t) = A\cos(\omega_n t) + B\sin(\omega_n t),
\end{equation*}
де $\omega_n^2 = \lambda_n v^2, \, n \in \mathbb{N}.$\\
Власними модами коливань струни будуть всі розв'язки вигляду:
\begin{equation*}
    u_n(x,t) = X_n(x) \cdot T_n(t)
\end{equation*}
Виконаємо перепозначення і запишемо остаточний розв'язок:
\begin{equation}
    \left\{ \begin{aligned}
        \;&u_n(x,t) = \left(A_n\cos(\omega_n t) + B_n\sin(\omega_n t)\right) \sin(k_n x), \\
        &k_n = \frac{\pi n}{l} - \text{ хвильові вектори}, \\
        &\omega_n = vk_n = \frac{v \pi n}{l} - \text{ власні частоти}, \\
        &n = 1, 2,\ldots
    \end{aligned}\right.
\end{equation}
\newpage

\begin{center}
    \large{\textbf{Перевірка розв'язку задачі Штурма-Ліувілля}}
\end{center}

\noindent Щоб перевірити правильність отриманого результату, (\ref{ShLsol}), треба використовувати постановки самої задачі -- (\ref{sepvar}). 
\begin{enumerate}
    \item Аналітична перевірка (пряма підстановка результату в рівняння).
    \begin{enumerate}
        \item[1)] Перевіряємо рівняння, для цього обчислимо другу похідну власної функції. 
        \begin{equation*}
            X_n^{''} = \frac{\pi n}{l} \left(C_n\cos\left(\frac{\pi n x}{l}\right)\right)^{'} = -\left(\frac{\pi n}{l}\right)^2 C_n\sin\left(\frac{\pi n x}{l}\right) = -\left(\frac{\pi n}{l}\right)^2 X_n
        \end{equation*}
        Порівнюємо з вихідним рівнянням і робимо перший висновок: кожна з функцій $X_n(x)$ дійсно є розв’язком рівняння (\ref{sepvar}) для значення спектрального параметра $\lambda = \left(\frac{\pi n}{l}\right)^2$. Далі, ці значення $\lambda$ збігаються зі знайденими раніше власними значеннями (\ref{ShLsol}), отже робимо другий висновок: знайдені власні значення дійсно відповідають знайденим власним функціям.
        \item[2)] Перевіряємо виконання межових умов, підставляємо власні функції в межові умови.
        \begin{equation*}
            \begin{aligned}
                X_n(0) &=  0:\\
                &\begin{aligned}
                    X_n(0) = C_n \sin(\sqrt{\lambda_n} \cdot 0) = 0 &\text{ -- виконується,}\\
                    &\text{ причому незалежно від }\lambda_n
                \end{aligned}\\
                \\
                X_n(l) &= 0:\\
                &\begin{aligned}
                    X_n(l) = C_n \sin\left(\frac{\pi n}{l} \cdot l\right) = C_n &\sin(\pi n) = 0 \text{ -- виконується,  але}\\
                    &\text{ саме для знайдених значень }\lambda_n
                \end{aligned}\\
            \end{aligned}
        \end{equation*}
    \end{enumerate}
    \item Графічна перевірка:
    \begin{center}
        \begin{tikzpicture}
            \begin{axis}
                [width = 0.85\textwidth, height = 0.4\textwidth,
                 axis x line = center, axis y line = center,
                 ylabel = $X(x)$, xlabel = $x$,
                 xmin = 0.7, xmax = 6.7, ymin = -3.3, ymax = 5.3,
                 axis line style = thin, xtick = {0}, ytick = {0}]   
                
                \tikzmath{\A1 = 5; \l = 5; \k = pi/\l;}
                
                \addplot [black, domain = 1:(\l+1), samples = 1000] {\A1 * sin(deg(\k*(x-1)))}
                node[anchor=north, pos=0] {0} 
                node[pos=0.75, fill=white] {$X_0(x)$} 
                node[anchor=north, pos=1] {$l$};
                
                \addplot [black, domain = 1:(\l+1), samples = 1000] {\A1/2.5 * sin(deg(2*\k*(x-1)))}
                node[pos=0.83, fill=white] {$X_1(x)$};
                
                \addplot [black, domain = 1:(\l+1), samples = 1000] {\A1/3.7 * sin(deg(3*\k*(x-1)))}
                node[pos=0.74, fill=white] {$X_2(x)$};
                
            \end{axis}
        \end{tikzpicture}
    \end{center}
\end{enumerate}




\end{document}

\section{Власні моди інших систем. Вільні коливання для заданих початкових умов.}
%\documentclass[a4paper, 14pt]{extreport}

\usepackage{StyleMMF}

\begin{document}

\subsection{Стержень з вільними та пружно закріпленими кінцями; системи, описувані іншими рівняннями.}

\subsubsection{Задача №1}

\textit{Знайти власні моди повздовжніх рухів тонкого стержня $0 \leq x \leq l$ із вільними кінцями  (задача для хвильового рівняння з межовими умовами $u_x(0,t) = 0, u_x(l,t) = 0$).\\
Результат перевірити аналітично й графічно (див. заняття №6, зразок модульної контрольної роботи №1) та проаналізувати його фізичний смисл. Чим відрізняється від інших основна (нульова) мода? Якому рухові стержня вона відповідає?}

\begin{center}
    \large{\textbf{Розв'язок}}
\end{center}

\noindent Формальна постановка задачі:
\begin{equation} \label{probcond2}
    \left\{ \begin{aligned} %%
            \;&u = u(x,t), \\
            &u_{tt} = v^2 u_{xx}, \\
            &0 \leq x \leq l, t \in \mathbb{R} \\
            &u_x(0,t) = 0, \\
            &u_x(l,t) = 0. 
    \end{aligned} \right.
\end{equation}
Необхідно знайти розв'язки (\ref{probcond2}) вигляду:
\begin{equation} \label{subst2}
    u(x,t) = X(x) \cdot T(t) \neq 0 
\end{equation}

Від задачі №1 попереднього заняття задача відрізняється тільки межовою умовою, тому підставляємо розв'язок у вигляді добутку (\ref{subst2}) тільки у межові умови (\ref{probcond2}):
\begin{equation*}
    \begin{aligned}
        \;u_x(0,t) = X'(0) \cdot T(t) = 0
        \;\Rightarrow\;
        \left\{ \begin{aligned}
            &T(t) \neq 0, \forall t, \\  &X'(0) = 0; 
        \end{aligned} \right.\\
        u_x(l,t) = X'(l) \cdot T(t) = 0
        \;\Rightarrow\;
        \left\{ \begin{aligned}
            &T(t) \neq 0, \forall t, \\  &X'(l) = 0; 
        \end{aligned} \right.\\
    \end{aligned}
\end{equation*}
Тут ми врахували, що умови на кінцях струни виконуються при всіх $t$, тому $T(t)$ не може бути рівним нулю.\\

Виписуємо результат відокремлення змінних:
\begin{equation} \label{sepvar2}
    \left\{ \begin{aligned}
        \;&X = X(x), \\
          &X^{\prime\prime} = -\lambda X, \\
          &0 \leq x \leq l, \\
          &X'(0) = 0, \\ 
          &X'(l) = 0. 
    \end{aligned} \right.
    \qquad\qquad
    T^{\prime\prime} + \lambda v^2 T = 0
\end{equation}

\begin{enumerate}
    \item[] Розв'язуємо задачу Штурма-Ліувілля (\ref{sepvar2}). Знову скористаємося результатами попередньої задачі та одразу запишемо якого типу отримуємо розв'язки для різних $\lambda$.
    \begin{enumerate}[wide, labelindent=0pt]
        
        \item Випадок $\lambda < 0$. 
        \begin{equation*}
            X(x) = C_1 sh(\sqrt{|\lambda|}x) + C_2 ch({\sqrt{|\lambda|}x})
        \end{equation*}
        Знаходимо константи з межових умов:
        \begin{equation*}
            \begin{aligned}
                &X'(0) = C_1\sqrt{|\lambda|}
                \;\Rightarrow\;
                X(x) = C_2 &ch(\sqrt{|\lambda|}x)\\
                &\left\{ \begin{aligned}
                    &X'(l) = C_2\sqrt{|\lambda|} sh(\sqrt{|\lambda|}l) = 0, \\
                    &sh(\sqrt{|\lambda|}l) \neq 0;
                \end{aligned} \right.&\\
                &\left\{ \begin{aligned}
                    C_1 = 0, \\ 
                    C_2 = 0;
                \end{aligned} \right. \qquad\qquad\qquad\qquad&
            \end{aligned}
            \;\Rightarrow\;
            \begin{aligned}
                \text{розв'язок тривівльний,}\\
                \text{немає від'ємних}\\
                \text{власних значень.}
            \end{aligned}
        \end{equation*}

        \item Випадок $\lambda = 0$:
        \begin{equation*}
            X(x) = C_1 + C_2 x
        \end{equation*}
        Знаходимо константи з межових умов:
        \begin{equation*}
            \begin{aligned}
                &\left\{ \begin{aligned}
                    &X'(0) = C_2 = 0, \\ 
                    &X'(l) = C_2 = 0;
                \end{aligned} \right.
                \\   
                &\left\{ \begin{aligned}
                    C_1 \in \mathbb{R}, \\ 
                    C_2 = 0;
                \end{aligned} \right.
            \end{aligned}
            \quad\Rightarrow\;
            \begin{aligned}
                X(x) = 0 \text{ -- розв'язок нетривівльний,}\\
                \lambda = 0 \text{ є власним значенням.}
            \end{aligned}
        \end{equation*}

        \item Випадок $\lambda > 0$
        \begin{equation*}
            X(x) = C_1 \sin(\sqrt{\lambda}x) + C_2 \cos({\sqrt{\lambda}x})
        \end{equation*}
        Знаходимо константи з межових умов:
        \begin{equation*}
            \begin{aligned}
                \left\{ \begin{aligned}
                    &X'(0) = C_1\sqrt{\lambda} = 0, \\ 
                    &X'(l) = \sqrt{\lambda}\left(\textcolor{red}{\begin{xy}*{\textcolor{black}{C_1}};p+LU;+RD**h@{}+//**h@{-}*h@{>}*h!LD{\scriptstyle 0}\end{xy}} \cos({\sqrt{\lambda}l}) - C_2 \sin(\sqrt{\lambda}l)\right) = 0;
                \end{aligned} \right.
                \;\Rightarrow\;
                \left\{ \begin{aligned}
                    &C_2 \neq 0, \\ 
                    &\sin(\sqrt{\lambda}l) = 0;
                \end{aligned} \right.
            \end{aligned}
        \end{equation*}
        Отже, нетривіальні розв'язки існують при значеннях параметра $\lambda$, які задовольняють характеристичне рівняння :
        \begin{equation*}
            \sin(\sqrt{\lambda}l) = 0
            \;\Rightarrow\;
            \sqrt{\lambda_n}l = \pi n, \, n \in \mathbb{Z}
            \;\Rightarrow\;
            \lambda_n = \frac{\pi^2 n^2}{l^2}.
        \end{equation*}
    \end{enumerate}
\end{enumerate} 
Випишемо тепер розв'язки для всіх $n$ і визначимо, які з них необхідно залишити:
    \begin{equation*}
        \left\{ \begin{aligned}
            &X_0(x) = C_0,\\
            &\lambda_0 = 0;
        \end{aligned} \right.
        \qquad
        \left\{ \begin{aligned}
            &X_n(x) = C_n \sin\left(\frac{\pi n x}{l}\right),\\
            &\lambda_n = \frac{\pi^2 n^2}{l^2}, n \in \mathbb{N}.
        \end{aligned} \right.
    \end{equation*}

    \textbf{\Large Необхідно відредагувати наступний текст}
Видно, що $n = 0$ відповідає тривіальному розв'язку. Видно також, що всі інші розв'язки визначені з точністю до довільного множника.\\
Тому власні функції, які співпадають з точністю до множника, вважають однаковими. У загальному випадку різними вважають лише лінійно незалежні власні функції, а розвя'зати задачу Штурма-Ліувілля означає знайти всі різні власні функції і відповідні власні значення. Отже, різним власним функціям відповідають лише натуральні $n$, а коефіцієнти $C_n$ можна покласти рівними одиниці.\\
    Власними значеннями і власними функціями є
    \begin{equation} %\label{ShLsol}
        \left\{ \begin{aligned}
            \;&\lambda_n = \frac{\pi^2 n^2}{l^2},\\ 
            &X_n(x) = \sin\left(\frac{\pi n x}{l}\right),
        \end{aligned} \right.
        \quad \text{де } n \in \mathbb{N}.
    \end{equation}

Повертаємося до рівняння для $T(t)$ (\ref{sepvar2}). Підставляємо знайдені власні значення та знаходимо $T_n(t)$:
\begin{equation*}
    \left. \begin{aligned}
        \lambda_n = \frac{\pi^2 n^2}{l^2},&\;\\ 
        T^{\prime\prime} + \lambda v^2T = 0,&
    \end{aligned} \right\}
    \;\Rightarrow\;
    T_n(t) = A\cos(\omega_n t) + B\sin(\omega_n t),
\end{equation*}
де $\omega_n^2 = \lambda_n v^2, \, n \in \mathbb{N}.$\\
\begin{equation*}
    \left. \begin{aligned}
        \lambda_0 = 0,&\;\\ 
        T^{\prime\prime} = 0,&
    \end{aligned} \right\}
    \;\Rightarrow\;
    T_0(t) = A_0 + B_0 t,
\end{equation*}
Власними модами коливань струни будуть всі розв'язки вигляду:
\begin{equation*}
    u_n(x,t) = X_n(x) \cdot T_n(t)
\end{equation*}
Виконаємо перепозначення і запишемо остаточний розв'язок:
\begin{equation}
    \left\{ \begin{aligned} \label{sol2}
        \;&u_0(x,t) = A_0 + B_0 t, \\
        &u_n(x,t) = \left[A_n\cos(\omega_n t) + B_n\sin(\omega_n t)\right] \sin(k_n x), \\
        &k_n = \frac{\pi n}{l} - \text{ хвильові вектори}, \\
        &\omega_n = vk_n = \frac{v \pi n}{l} - \text{ власні частоти}, \\
        &n = 1, 2,\ldots
    \end{aligned}\right.
\end{equation}

\end{document}
%\subsection{Вільні коливання поля в резонаторі для заданих початкових умов. Ряд Фур'є по системі ортогональних функцій.}

\subsubsection{Задача №3}

\textit{Знайти коливання струни завдовжки $0 \leq x \leq l$ із закріпленими кінцями, якщо початкове відхил є $\varphi(x) = hx/l$, а початкова швидкість $\psi(x) = \nu_0$. Обчислити інтеграл ортогональності власних функцій і знайти квадрат норми. Чи є рух струни періодичним (тобто повторюється початковий стан струни через деякий проміжок часу?) Чи буде рух періодичним, якщо він описується рівнянням $u_{tt} = v^2 u_{xx} - \omega_0^2 u$}

\begin{center}
    \textbf{Розв'язок}
\end{center}
Формальна постановка задачі:
\begin{equation} \label{probcond3}
    \left\{ \begin{aligned} %%
        &\;u = u(x,t), \\
        &\;u_{tt} = v^2 u_{xx}, \\
        &\;0 \leq x \leq l, t \in \mathbb{R}, \\
        &\;u(0,t) = u(l,t) = 0,\\
        &\left.\begin{aligned}
            &u(x,0) = \varphi(x) = \frac{hx}{l}, \\ 
            &u_t(x,0) = \psi(x) = \nu_0.
        \end{aligned}\right\} \; 
        \begin{aligned}
            &\text{ початкові умови задають} \\
          - &\text{ механічний стан} \\
            &\text{ системи при } t = 0
        \end{aligned}
    \end{aligned} \right.
\end{equation}

Задача з заданими початковими умовами має єдиний розв'язок. Скористаємося результатами задачі 1 попередньго заняття (\ref{sol1}).
\begin{equation*}
    \left\{ \begin{aligned} \label{fullsol}
        \;&u_n(x,t) = \left[A_n\cos(\omega_n t) + B_n\sin(\omega_n t)\right] \sin(k_n x), \\
        &k_n = \frac{\pi n}{l}, \, n = 1, 2,\ldots\\
        &\omega_n = vk_n = \frac{v \pi n}{l} - \text{ власні частоти}.
    \end{aligned}\right.
\end{equation*}

Запишемо загальний розв'язок задачі:
\begin{equation} \label{gensol}
    u(x,t) = \sum^{\infty}_{n=1} u_n(x,t) = \sum^{\infty}_{n=1} \left[A_n\cos(\omega_n t) + B_n\sin(\omega_n t)\right] \sin(k_n x)
\end{equation}
Коефіцієнти $A_n$ та $B_n$ визначаємо із початкових умов. Підставляємо (\ref{gensol}) в початкові умови (\ref{probcond3}):
\begin{equation} \label{sol-init-cond}
    \begin{aligned}
        &u(x,0) = \varphi(x)
        \;\Rightarrow\;
        \sum^{\infty}_{n=1} A_n\sin(k_n x) = \varphi(x)\\
        &\begin{aligned}
            u_t(x,0) = \psi(x)
            \;&\Rightarrow\\
            \Rightarrow \left(\sum^{\infty}_{n=1}\right.&\left.\left. \left[-A_n\omega_n\sin(\omega_n t) + B_n\omega_n\cos(\omega_n t)\right] \sin(k_n x)\right)\right|_{t=0} =\\
            &= \sum^{\infty}_{n=1} B_n\omega_n\sin(k_n x) = \psi(x)
        \end{aligned}
    \end{aligned}
\end{equation}
Отже, ми отримали дві умови для визначення $A_n$, $B_n$.\\
Далі скористаємося ортогональністю власних функцій задачі Штурма-Ліувілля.
\begin{equation} \label{orth}
    \int_0^l X_n(x) \cdot X_m(x) \,\mathrm{d}x = ||X_n||^2\delta_{n,m},
\end{equation}
де $||X_n||$ -- норма власної функції.

Доможуємо отримані вирази в (\ref{sol-init-cond}) на $m$-ту власну функції $\sin(k_m x)$ та інтегруємо від $0$ до $l$. 
\begin{equation*}
    \begin{aligned}
        \begin{aligned}
            &\int\limits_0^l \varphi(x) \sin(k_m x) \,\mathrm{d}x = \sum^{\infty}_{n=1} A_n \int\limits_0^l \sin(k_n x) \sin(k_m x) \,\mathrm{d}x =\\
            &= \sum^{\infty}_{n=1} A_n \cdot \frac{l}{2} \delta_{n,m} = \frac{A_m l}{2}
            \;\Rightarrow\;
            A_n = \frac{2}{l} \int\limits_0^l \varphi(x) \sin(k_n x) \,\mathrm{d}x =\\
            &= \frac{2}{l} \int\limits_0^l \frac{hx}{l} \sin(k_n x) \,\mathrm{d}x = \frac{2h}{l^2} \left(\left.-\frac{1}{k_n} x \cos(k_n x)\right|_0^l + \int\limits_0^l \frac{\cos(k_n x)}{k_n} \,\mathrm{d}t\right) =\\
            &= \left| k_n l = \frac{\pi n}{l} l = \pi n \Rightarrow \sin(k_n l) = 0,\, \cos(k_n l) = (-1)^n \right| =\\
            &= \frac{2h}{l^2} \left(-\frac{l}{k_n}(-1)^n + \left.\frac{\sin(k_n x)}{k_n^2}\right|_0^l \right) = \frac{2h}{l} \frac{(-1)^{n+1}}{k_n} \equiv A_n
        \end{aligned}\\
        \begin{aligned}
            \int\limits_0^l \psi(x) \sin(k_m x) \,\mathrm{d}x &= \sum^{\infty}_{n=1} B_n\omega_n \cdot \frac{l}{2} \delta_{n,m} = \frac{B_m \omega_m l}{2}
            \;\Rightarrow\\
            \Rightarrow\;
            B_n =&\ \frac{2}{\omega_n l} \int\limits_0^l \psi(x) \sin(k_n x) \,\mathrm{d}x = \frac{2\nu_0}{\omega_n l} \int\limits_0^l \sin(k_n x) \,\mathrm{d}x =\\
            =&\ \left.\frac{2\nu_0}{k_n\omega_n l} \cos(k_n x)\right|_l^0 = \frac{2\nu_0}{l} \frac{1 - (-1)^n}{k_n\omega_n} \equiv B_n
        \end{aligned}
    \end{aligned}
\end{equation*} 
Підставляємо визначені константи в (\ref{gensol})
\begin{equation} \label{sol3}
    u(x,t) = \frac{2}{l}\sum^{\infty}_{n=1} \left[\nu_0 (1 - (-1)^n)\frac{\sin(\omega_n t)}{\omega_n} - h(-1)^n\cos(\omega_n t)\right] \frac{\sin(k_n x)}{k_n}
\end{equation}

Перевіримо періодичність розв'язку. Період коливання визначається за відомою формулою \[T_n = \frac{2\pi}{\omega_n},\] де $n$ - номер власної моди. Підставимо в (\ref{sol3}) $t = t + T_n$
\begin{equation*}
    \begin{aligned}
        u(x,t+T_n) & = \frac{2}{l}\sum^{\infty}_{n=1} \left[\nu_0 (1 - (-1)^n)\frac{\sin(\omega_n t + \omega_n \cdot \frac{2\pi}{\omega_n})}{\omega_n} -\right.\\
        &\qquad\qquad\qquad\qquad\quad\left.- h(-1)^n\cos(\omega_n t + \omega_n \cdot \frac{2\pi}{\omega_n})\right] \frac{\sin(k_n x)}{k_n} =\\
        = \frac{2}{l}\sum^{\infty}_{n=1} & \left[\nu_0 (1 - (-1)^n)\frac{\sin(\omega_n t + 2\pi)}{\omega_n} - h(-1)^n\cos(\omega_n t + 2\pi)\right] \frac{\sin(k_n x)}{k_n} =\\ 
        = \frac{2}{l}\sum^{\infty}_{n=1} & \left[\nu_0 (1 - (-1)^n)\frac{\sin(\omega_n t)}{\omega_n} - h(-1)^n\cos(\omega_n t)\right] \frac{\sin(k_n x)}{k_n} = u(x,t)
    \end{aligned}
\end{equation*} 
Тобто коливання струни буде періодичним.

\section{Другий спосіб знаходження коефіцієнтів. Коливання стержня з вільними кінцями, неповнота базису.}
%%\documentclass[a4paper, 14pt]{extreport}

%\usepackage{StyleMMF}

%\begin{document}

%\setcounter{chapter} {2}
%\chapter{Другий спосіб знаходження коефіцієнтів. Коливання стержня з вільними кінцями, неповнота базису.}

\section[Задача №3.1]{3.1}

\textit{Знайти коливання пружного стержня $0 \leq x \leq l$, лівий кінець якого закріплений, а правий вільний, якщо початкове відхилення $\varphi(x) = h \sin(3\pi x/2l)$, а початкова швидкість $\psi(x) = \nu_0 \sin(\pi x/2l)$.}

\begin{center}
    \textbf{Розв'язок}
\end{center}
Формальна постановка задачі:
\begin{equation} \label{cond3,1}
    \left\{ \begin{aligned} %%
        &\;u = u(x,t), \\
        &\;u_{tt} = v^2 u_{xx}, \\
        &\;0 \leq x \leq l, t \geq 0, \\
        &\;u(0,t) = 0,\\
        &\;u_x(0,t) = 0,\\
        &\left.\begin{aligned}
            &u(x,0) = \varphi(x) = h \sin \left(\frac{3 \pi x}{2 l} \right), \\ 
            &u_t(x,0) = \psi(x) = v_0 \sin \left(\frac{\pi x}{2 l}\right).
        \end{aligned}\right\} \; 
        \begin{aligned}
            &\text{ специфіка задачі} \\
          - &\text{ у вигляді } \\
            &\text{ початкових умов } 
        \end{aligned}
    \end{aligned} \right.
\end{equation}

Це задача із заданими початковими умовами (а саме - початковим розподілом зміщення та швидкостей), яка має єдиний розв'язок.

Для початку скористаємося розв'язком задачі 1.2:

\begin{equation}
    \left\{ \begin{aligned} \label{mode3.1}
        \;&u_n(x,t) = \left[A_n\cos(\omega_n t) + B_n\sin(\omega_n t)\right] \sin(k_n x), \\
        &k_n = (n + \frac{1}{2})\frac{\pi}{l}, \, n = 1, 2,\ldots\\
        &\omega_n = vk_n = (n + \frac{1}{2})\frac{\pi v}{l} - \text{ власні частоти}.
    \end{aligned}\right.
\end{equation}

І запишемо загальний розв'язок:

\begin{equation} \label{gensol3,1}
    u(x,t) = \sum^{\infty}_{n=0} \left[A_n\cos(\omega_n t) + B_n\sin(\omega_n t)\right] \sin(k_n x)
\end{equation}

\begin{equation} \label{(sol3,1)_t}
    \begin{aligned}
        u_t(x,t) &= 
   \sum^{\infty}_{n=0}\left[-A_n\omega_n\sin(\omega_n t) + B_n\omega_n\cos(\omega_n t)\right] \sin(k_n x)  
    \end{aligned}
\end{equation}

Підставляємо (\ref{gensol3,1}) у початкові умови (\ref{cond3,1}):

\begin{equation}
    u(x,0) = \varphi(x) \;\Rightarrow\; \sum^{\infty}_{n=0} A_n\sin\left((n + \frac{1}{2}) \frac{\pi x}{l} \right) = h \sin \left( \frac{3 \pi x}{2l} \right)
\end{equation}


Підставляємо (\ref{(sol3,1)_t}) у початкові умови (\ref{cond3,1}):

\begin{equation}
    u_t(x,0) = \psi(x) \;\Rightarrow\; \sum^{\infty}_{n=0} B_n \omega_n \sin\left((n + \frac{1}{2}) \frac{\pi x}{l} \right) = v_0 \sin \left( \frac{ \pi x}{2l} \right)
\end{equation}

Особливі ситуації: функції у правій частині є однією з власних функцій задачі Штурма-Ліувіля. Це дозволяє знайти коефіцієнти $A_n, B_n$ простіше, порівнюючи з загальнім знаходженням з вихідних функцій $\varphi(x)$ і  $\psi(x)$ загального вигляду! Тобто брати інтеграл у цій особливій ситуації не потрібно.


Якщо 2 ряда Фур'є по одній системі функцій рівні, то і відповідні коефіцієнти цих рядів рівні.

\begin{equation}
    \begin{aligned}
        \sum_{n=0} A_n \sin \left( (n + \frac{1}{2}) \frac{\pi x}{l} \right) = A_0 \sin \left(\frac{\pi x}{2 l} \right) &+\\
        + A_1 \sin \left(\frac{3 \pi x}{2 l} \right) + A_2 \sin \left(\frac{5 \pi x}{2 l} \right) + &... = h \sin \left( \frac{3 \pi x}{2 l} \right)
    \end{aligned}
\end{equation}

Результат $A_1 = h, A_0 = A_2 = A_3 = ... = 0$

Аналогічно робимо з \ref{(sol3,1)_t}: 

\begin{equation}
    \begin{aligned}
        \sum_{n=0} \omega_n B_n \sin \left( (n + \frac{1}{2}) \frac{\pi x}{l} \right) = \omega_0 B_0 \sin \left(\frac{\pi x}{2 l} \right) &+\\
        + \omega_1 B_1 \sin \left(\frac{3 \pi x}{2 l} \right) + \omega_2 B_2 \sin \left(\frac{5 \pi x}{2 l} \right) + &... = v_0 \sin \left( \frac{\pi x}{2 l} \right)
    \end{aligned}
\end{equation}

Результат $B_0 = \frac{v_0}{\omega_0}, B_1 = B_2 = B_3 = ... = 0$

Тепер треба правильно написати відповідь через знайдені коефіцієнти $A_n, B_n$! Підставляємо знайдені коефіцієнти у загальний розв'язок (тільки два коефіцієнти - $A_1$ і $B_0$ не дорівнюють нулю, тож членів у розв'язку всего два!)

Фінальна відповідь:

\begin{equation}
    u (x,t) = h \cos (\omega_1 t) \sin (k_1 x) + \frac{v_0}{\omega_0} \sin (\omega_0 t) \sin (k_0 x)
\end{equation}

де $k_0 = \frac{\pi x}{2 l}, k_1 = \frac{3 \pi x}{2 l}, \omega_0 = v k_0, \omega_1 = v k_1 $. Можемо помітити, що у кожної моди своя частота.

Перевіряємо відповідь

\begin{itemize}
    \item Власні функції перевірені в задачі 1.2
    \item Постановка задачі містить два неоднорідних члени у початкових умовах. Один пропорційний $\sim h$, інший пропорційний $\sim v_0$. Перевірити наявність цих множників у загальному розв'язку.
    \item Перевіряємо початкові умови - виконуються?

\end{itemize}

Альтернативний шлях -- знайти за означенням коефіцієнти розкладу у ряд Фур'є.

\begin{equation}
A_n = \frac{2}{l} \int_{0}^{l} \varphi (x)  \sin \left( (\frac{1}{2} + n) \frac{\pi x}{l} \right) dx    
\end{equation}


Одержали інтеграл ортогональності 

\begin{equation}
    \int^{l}_0 \sin \left( \frac{3 \pi x}{2 l} \right) \sin \left( (\frac{1}{2} + n) \frac{\pi x}{l} \right) dx = \int^{l}_0 \chi_1 (x) \chi_n (x) dx = \delta_{1n}
\end{equation}

Якщо ви не побачите що інтеграл є інтегралом ортогональності, і будете його обчилювати, то втратите час і можете помилитися і одержати неправильну відповідь (що часто і буває).

Результат $A_1 = h, A_0 = A_2 = A_3 = ... = 0$ та для швидкостей $B_0 = \frac{v_0}{\omega_0}, B_1 = B_2 = B_3 = ... = 0$

Отримали теж саме, але складнішим шляхом!

%\end{document}
%%\documentclass[a4paper, 14pt]{extreport}

%\usepackage{StyleMMF}

%\begin{document}

%\chapter{Другий спосіб знаходження коефіцієнтів. Коливання стержня з вільними кінцями, неповнота базису.}

\section[Задача №3.3]{3.3}

\textit{Знайти коливання пружного стержня довжиною $l$ з вільними кінцями, якщо початкове відхилення дорівнює нулю, а початкова швидкість $\psi(x) = \nu_0$. Якщо всі знайдені вами коефіцієнти Фур'є (коефіцієнти загального\\ розв’язку) дорівнюють нулю, поясніть, що це означає, і знайдіть, де була допущена помилка.}

%\end{document}

\section{Рівняння теплопровідності з однорідними межовими умовами}
%\documentclass[a4paper, 14pt]{extreport}

\usepackage{StyleMMF}

\begin{document}

\section{Рівняння теплопровідності з однорідними межовими умовами}

\subsubsection{Задача №1}

\textit{Одну і ту ж функцію, наприклад $f(x) = \alpha x$, можна представити на проміжку $0 \leq x \leq l$ узагальненим рядом Фур’є по кожній із систем власних функцій чотирьох задач Штурма-Ліувілля, одержаних у задачах 1.1, 1.2, 1.3, 2.1. Користуючись явним виглядом власних функцій і не обчислюючи коєфіцієнтів рядів, дайте відповіді на такі запитання.
\begin{enumerate}
    \item Який вигляд матиме графік суми кожного з таких рядів на всій числовій осі? Якою є парність суми ряду відносно точок $x = nl$, де $n$ – ціле число, і як це пов’язано з виглядом крайових умов задачі Штурма-Ліувілля?
    \item Покажіть, що кожний з рядів є частинним випадком класичного тригонометричного ряду Фур’є, сума якого є періодичною функцією. Які саме періоди відповідають кожному з рядів? Яка саме частина повного тригонометричного базису використовується в кожному з розкладань, а які коефіцієнти Фур’є дорівнюють нулю і чому?
    \item Як пов’язаний характер збіжності вказаних рядів з крайовими умовами, які задовольняє функція $f(x)$ у точках $x = 0, l$ ? Чи дорівнює сума ряду Фур’є функції $f(x)$ на відкритому проміжку $0 < x < l$? на закритому проміжку
    $0 \leq x \leq l$
\end{enumerate}}


\end{document}
%\documentclass[a4paper, 14pt]{extreport}

\usepackage{StyleMMF}

\begin{document}

\section{Рівняння теплопровідності з однорідними межовими умовами}

\subsubsection{Задача №2}

\textit{У початковий момент часу ліва половина стержня з теплоізольованою бічною поверхнею має температуру $T_1$ , а права -- температуру $T_2$ . Знайти розподіл температури при $t> 0$, якщо кінці стержня підтримуються при температурі $T_0$. Указівка: подумайте, що означає «температура дорівнює нулю», що це за нуль? Покладіть у кінцевому результаті $T_0 = 0$ і розгляньте частинні випадки: $T_1 = T_2$ та $T_1 = -T_2$. Які члени ряду при цьому обертаються в нуль? Чому? Нарисуйте графіки та порівняйте часову залежність температури для     різних мод. Нарисуйте (якісно) графіки розподілу     температури вдовж стержня у різні характерні послідовні моменти часу. Що таке «малий» і «великий» проміжок часу для цієї задачі? Як характерні часи залежать від розмірів системи?}

\end{document}
%%\documentclass[a4paper, 14pt]{extreport}

%\usepackage{StyleMMF}

%\begin{document}

%\chapter{Рівняння теплопровідності з однорідними межовими умовами}

\section[Задача №4.4]{4.4}

\textit{Початкова температура повністю теплоізольованого тонкого стержня\\ $0 \leq x \leq l$ дорівнює $T_1 \cos(\pi x/2l) + T_2 \cos(2\pi x/l)$ . Знайти поле температур при $t > 0$. Перевірити виконання початкових умови при $T_1 = 0$ і $T_2 = 0$.}

\begin{center}
    \large{\textbf{Розв'язок}}
\end{center}

\noindent Формальна постановка задачі:
\begin{equation} \label{cond4,4}
    \left\{ \begin{aligned} %%
            \;&u = u(x,t), \\
            &u_t = D u_{xx}, \\
            &0 \leq x \leq l, t \geq 0, \\
            &u_x(0,t) = 0, \, u_x(l,t) = 0,\\ 
            &u(x,0) = T_1 \cos(\pi x/2l) + T_2 \cos(2\pi x/l).
    \end{aligned} \right.
\end{equation}

Виконуючи розділення змінних ми отримаємо дві вже розв'язані задачі. Задачу Штурма-Ліувілля (\ref{sepvar2,1}) з задачі №2.1 та часове диференціальне рівняня (\ref{time-eq4,2}) з задачі №4.2. Отже, загальний розв'язок можна одразу записати комбінуюці відомі.

\begin{equation} \label{gensol4,4}
    u(x,t) = C_0 + \sum_{n=1}^{\infty}C_n e^{-t/\tau_n} \cos k_n x,
\end{equation}
\begin{equation*}
    \begin{aligned}
        &k_n = \frac{\pi n}{l} - \text{ хвильові вектори}, \\
        &\tau_n = \frac{1}{D k_n^2} - \text{ характерний час зміни температури}, \\
        &n = 0, 1, 2,\ldots
    \end{aligned}
\end{equation*}

З початковї умови визначимо невіомі коефіцієнти. Для цього треба розкласти $\cos(\pi x/2l)$ по набору власних функцій задачі Ш.-Л. 
\begin{equation*}
    \begin{gathered}
        \cos(\pi x/2l) = a_0 + \sum_{n=1}^{\infty} a_n \cos k_nx \\
        a_0 = \frac{1}{l}\int\limits_0^l \cos(\pi x/2l) \;\mathrm{d}x = \frac{2}{\pi} \sin(\pi x/2l) \bigg|_0^l = \frac{2}{\pi}\\
        a_n = \frac{2}{l}\int\limits_0^l \cos(\pi x/2l)\cos k_nx \;\mathrm{d}x = \frac{1}{l}\bigg(\int\limits_0^l \cos((k_n + \pi/2l)x) \;\mathrm{d}x +\\
        + \int\limits_0^l \cos((k_n - \pi/2l)x) \;\mathrm{d}x\bigg) = \frac{1}{l}\left(\frac{\sin((k_n + \pi/2l)x)}{k_n + \pi/2l}\bigg|_0^l + \frac{\sin((k_n - \pi/2l)x)}{k_n - \pi/2l}\bigg|_0^l\right) =\\
        = \left(\frac{\sin(k_nl + \pi/2)}{k_nl + \pi/2} + \frac{\sin(k_nl - \pi/2)}{k_nl - \pi/2}\right) = \left(\frac{1}{k_nl + \pi/2} - \frac{1}{k_nl - \pi/2}\right)\cos k_nl =\\
        = \big|\cos k_nl = (-1)^n\big| = \frac{(-1)^{n+1} \pi}{(k_nl + \pi/2)(k_nl - \pi/2)} =\\
        = (-1)^{n+1} \cdot \frac{4\pi}{4k_n^2l^2 - \pi^2} = \frac{4}{\pi} \cdot \frac{(-1)^{n+1}}{4n^2 - 1}
    \end{gathered}
\end{equation*}
Тепер підставимо (\ref{gensol4,4}) в початкову умову (\ref{cond4,4}) і отримаємо:
\begin{equation}
    \begin{gathered}
        u(x,0) = C_0 + \sum_{n=1}^{\infty}C_n \cos k_n x = T_1 \cos(\pi x/2l) + T_2 \cos(2\pi x/l) =\\
        = \frac{2T_1}{\pi} + T_2 \cos k_2x + \frac{4T_1}{\pi} \sum_{n=1}^{\infty} \frac{(-1)^{n+1} \cos k_nx}{4n^2 - 1}
    \end{gathered}
\end{equation} 
З чого слідує 
\begin{equation*}
    C_0 = \frac{2T_1}{\pi},\, C_2 = T_2 - \frac{4T_1}{15\pi},\, C_n = \frac{4T_1}{\pi} \cdot \frac{(-1)^{n+1}}{4n^2 - 1}, \text{ де } n \neq 2
\end{equation*}
Отже, остаточним розв'язком буде 
\begin{equation}
    u(x,t) = \frac{2T_1}{\pi} + T_2 e^{-t/\tau_2}\cos k_2x + \frac{4T_1}{\pi} \sum_{n=1}^{\infty} \frac{(-1)^{n+1} \cos k_nx}{4n^2 - 1}
\end{equation}

Прямою підстановкою можна переконатися, що при $T_1 = 0$ та $T_2 = 0$ початкові умови виконуються.


%\end{document}

\chapter{МЕТОД ЧАСТИННИХ РОЗВ’ЯЗКІВ ТА МЕТОД РОЗКЛАДАННЯ ЗА ВЛАСНИМИ ФУНКЦІЯМИ.}

\section{Еволюційні задачі з неоднорідним рівнянням або неоднорідними межовими умовами: стаціонарні неоднорідності}
%\documentclass[a4paper, 14pt]{extreport}

\usepackage{StyleMMF}

\begin{document}

\section{Еволюційні задачі з неоднорідним рівнянням або неоднорідними межовими умовами: стаціонарні неоднорідності}

\subsubsection{Задача №1}

\textit{Знайти коливання вертикально розташованого пружного стержня під дією сили тяжіння для $t > 0$. Верхній кінець стержня закріплений, а нижній вільний. При $t < 0$ стержень був нерухомим і деформацій не було. Знайти спочатку стаціонарний розв’язок, що відповідає положенню рівноваги стержня в полі тяжіння, а потім знайти відхилення від нього, що відповідає коливанням навколо нового положення рівноваги. Намалювати графіки розподілу поля зміщень та поля напружень у положенні рівноваги.}

\end{document}
%\documentclass[a4paper, 14pt]{extreport}

\usepackage{StyleMMF}

\begin{document}

\section{Еволюційні задачі з неоднорідним рівнянням або неоднорідними межовими умовами: стаціонарні неоднорідності}

\subsubsection{Задача №3}

\textit{У стержні довжиною $l$ з непроникною бічною поверхнею відбувається дифузія частинок (коефіцієнт дифузії $D$), що мають час життя $\tau$. Через правий кінець всередину стержня подається постійний потік частинок $I_0$. Знайти стаціонарний розподіл концентрації та розв’язок, що задовольняє нульову початкову умову, якщо через лівий кінець частинки вільно виходять назовні й назад не вертаються. Знайти вигляд стаціонарного розв’язку в граничних випадках великих і малих $\tau$ та нарисувати графіки.\\
Указівка. Рівняння дифузії частинок зі скінченним часом життя має вигляд:
$u_t = D u_{xx} - u/\tau$. Його зручно переписати через так звану
довжину дифузійного зміщення $L = \sqrt{D\tau}$: \[\tau u_t = L^2 u_{xx} - u\] Величина $L$ має смисл характерної відстані, на яку частинки встигають зміститися (в середньому) за час свого життя. «Великі» й «малі» $\tau$ означають у дійсності $L \gg l$ і $L \ll l$ відповідно. Останній випадок фактично означає перехід до наближення півнескінченного стержня $-\infty < x \leq l$}

\end{document}

\section{Задачі з неоднорідним рівнянням або неоднорідними межовими умовами}
%\documentclass[a4paper, 14pt]{extreport}

\usepackage{StyleMMF}

\begin{document}

\section{Задачі з неоднорідним рівнянням або неоднорідними межовими умовами}

\subsection{Джерела з гармонічною залежністю від часу.}

\subsubsection{Задача №1}

\textit{Знайти коливання струни $0 \leq x \leq l$, лівий кінець якої закріплений, а правий вільний, при $t > 0$ під дією розподіленої сили $f(x,t) = f(x)\cos\omega t$. При $t < 0$ струна перебувала в положенні рівноваги. Розглянути окремий випадок $f(x) = f_0$. Виділити складову розв’язку, яка відповідає усталеним вимушеним коливанням і проаналізувати картину резонансу. Перевірити, чи переходить одержаний розв’язок у розв’язок задачі 5.1 за відповідних умов.}

\end{document}
%\documentclass[a4paper, 14pt]{extreport}

\usepackage{StyleMMF}

\begin{document}

\section{Задачі з неоднорідним рівнянням або неоднорідними межовими умовами}

\subsection{Метод розкладання по власних функціях в задачах з неоднорідним рівнянням}

\subsubsection{Задача №3}

\textit{Знайти коливання струни із закріпленими кінцями під дією сили $f(x,t) = f_0 t^N, \, N > 0$ однорідно розподіленої по довжині струни. У початковий момент струна нерухома, і зміщення дорівнює нулю. Остаточні обчислення виконати
для $N=2$.}

\end{document}

%\section{Задачі з неоднорідними межовими умовами загального вигляду}



%\chapter{??}

%\chapter{РІВНЯННЯ ЛАПЛАСА І ПУАССОНА.}

\end{document}