\documentclass[a4paper, 14pt]{extreport}

\usepackage{StyleMMF}

\begin{document}

\tableofcontents
\setcounter{page}{2}

\part{ЗАСТОСУВАННЯ ПРОЦЕДУРИ ФУР’Є БЕЗПОСЕРЕДНЬОГО ВІДОКРЕМЛЕННЯ ЗМІННИХ}

\chapter{Відокремлення змінних, задача Штурма-Ліувілля і власні моди коливань струни для різних межових умов}
%\documentclass[a4paper, 14pt]{extreport}

%\usepackage{StyleMMF}

%\begin{document}

\section[Задача №1.1]{1.1}

\textit{\textbf{Знайти власні моди коливань струни завдовжки $l$ із закріпленими кінцями (знайти функції вигляду $u(x,t) = X(x) \cdot T(t)$, визначені і достатньо гладкі в області $0 \leq x \leq l, -\infty \leq t \leq \infty$, не рівні тотожно нулю, які задовольняють одновимірне хвильове рівняння $u_{tt} = v^2 u_{xx}$ на проміжку $0 \leq x \leq l$ і межові умови $u(0,t) = 0, u(l,t) = 0$ на його кінцях).} Результат перевірити аналітично й графічно (див. текст до модульної контрольної роботи №1, с. 25) та проаналізувати його фізичний смисл. Знайти початкові умови (початкове відхилення і початкову швидкість) для кожної з мод.}

\begin{center}
    \large{\textbf{Розв'язок}}
\end{center}

\noindent Формальна постановка задачі:
\begin{equation} \label{probcond}
    \left\{ \begin{aligned} %%
        \;&u = u(x,t), \\
          &u_{tt} = v^2 u_{xx}, \\
          &0 \leq x \leq l, t \in \mathbb{R}, \\
          &u(0,t) = 0, \\
          &u(l,t) = 0. 
    \end{aligned} \right.
\end{equation}
Необхідно знайти нетривіальні (тобто не рівні тотожно нулю) розв'язки (\ref{probcond}) вигляду:
\begin{equation} \label{subst}
    u(x,t) = X(x) \cdot T(t) \neq 0 
\end{equation}

Хвильове рівняння з двома межовими умовами (\ref{probcond}) на кінцях проміжку по координаті $x$ описує малі поперечні коливання струни із закріпленими кінцями, її довільний вільний рух. Струна має попередній натяг, і у положенні рівноваги  всі її точки знаходяться на осі $x$, а при коливаннях відхиляться у напрямку осі $y$; $u(x,t)$ - це відповідне зміщення точки струни з координатою $x$ в напрямку $y$ відносно її рівноважного положення у даний момент часу $t$. Власні моди струни - це особливі рухи струни, які описуються розв'язками у вигляді добутків (\ref{subst}). 

Підставляємо розв'язок у вигляді добутку (\ref{subst})  у рівняння й умови (\ref{probcond}) Почнемо з межових умов:
\begin{equation*}
    \begin{aligned}
        \;u(0,t) = X(0) \cdot T(t) = 0
        \;\Rightarrow\;
        \left\{ \begin{aligned}
            &T(t) \neq 0, \forall t, \\  &X(0) = 0; 
        \end{aligned} \right.\\
        u(l,t) = X(l) \cdot T(t) = 0
        \;\Rightarrow\;
        \left\{ \begin{aligned}
            &T(t) \neq 0, \forall t, \\  &X(l) = 0; 
        \end{aligned} \right.\\
    \end{aligned}
\end{equation*}
Тут ми врахували, що умови на кінцях струни виконуються при всіх $t$, тому $T(t)$ не може бути рівним нулю.\\
Далі підставимо (\ref{subst}) у рівняння:
\begin{equation*}
    \frac{\partial^2}{\partial t^2}\left[X(x)T(t)\right] = v^2 \frac{\partial^2}{\partial x^2}\left[X(x)T(t)\right]
    \;\Rightarrow\; 
    X T^{\prime\prime} = v^2 X^{\prime\prime} T 
\end{equation*}
Звідси переходимо до рівності двох функцій від різних змінних:
\begin{equation}
    \frac{T^{\prime\prime}}{v^2T} = \frac{X^{\prime\prime}}{X}
\end{equation}
Це і є ситуація відокремлення змінних: функція від $x$ має дорівнювати функції від $t$ при всіх $x$ і $t$. Це можливо тільки у випадку, якщо обидві ці функції є сталими. Тому маємо
\begin{equation} \label{managedvar}
    \frac{T^{\prime\prime}}{v^2T} = \frac{X^{\prime\prime}}{X} = - \lambda,
\end{equation}
де $\lambda$ -- стала відокремлення. Її можливі значення необхідно буде знайти.

Виписуємо результат відокремлення змінних:
\begin{equation} \label{sepvar}
    \left\{ \begin{aligned}
        \;&X = X(x), \\  &X^{\prime\prime} = -\lambda X, \\ &0 \leq x \leq l, \\  &X(0) = 0, \\ &X(l) = 0. 
    \end{aligned} \right.
    \qquad\qquad
    T^{\prime\prime} + \lambda v^2 T = 0
\end{equation}

Задача для $X = X(x)$ є так званою Штурма-Ліувілля. Необхідно знайти нетривіальні розв'язки цієї задачі і значення параметра   $\lambda$, при яких вони існують; їх називають, відповідно, власними функціями і власними значеннями задачі. З умов задачі можна показати (див. лекції), що її власні значення є дійсними.\\
\begin{enumerate}
    \item[] Розв'язуємо задачу Штурма-Ліувілля (\ref{sepvar}):
    \begin{enumerate}[wide, labelindent=0pt]
        \item Розглянемо випадок $\lambda = 0$:
        \begin{equation*}
            X^{\prime\prime} = -\lambda X
            \;\Rightarrow\;
            X^{\prime\prime} = 0
            \;\Rightarrow\;
            X(x) = C_1 + C_2 x
        \end{equation*}
        Знаходимо константи з межових умов:
        \begin{equation*}
            \begin{aligned}
                &\left\{ \begin{aligned}
                    &X(0) = C_1 = 0, \\ 
                    &X(l) = C_1 + C_2 l = 0;
                \end{aligned} \right.
                \\   
                &\left\{ \begin{aligned}
                    C_1 = 0, \\ 
                    C_2 = 0;
                \end{aligned} \right.
            \end{aligned}
            \quad\Rightarrow\;
            \begin{aligned}
                X(x) = 0 \text{ -- розв'язок тривівльний,}\\
                \lambda = 0 \text{ не є власним значенням.}
            \end{aligned}
        \end{equation*}
    
        \item Розглянемо випадок $\lambda < 0$. Розв'язок рівняння шукаємо у вигляді $X(x) = e^{\alpha x}$: 
        \begin{equation*}
            \begin{aligned}
                &X^{\prime\prime} = -\lambda X
                \quad\Rightarrow\quad
                \alpha^2 \textcolor{red}{\begin{xy}*{\textcolor{black}{e^{\alpha x}}};p+LU;+RD**h@{}+/\jot/**h@{-}\end{xy}} = +|\lambda| \textcolor{red}{\begin{xy}*{\textcolor{black}{e^{\alpha x}}};p+LU;+RD**h@{}+/\jot/**h@{-}\end{xy}}
                \quad\Rightarrow\quad
                \alpha = \pm \sqrt{|\lambda|}
                \;\Rightarrow\\
                \Rightarrow\;
                &X(x) = \widetilde{C}_1 e^{\sqrt{|\lambda|}x} + \widetilde{C}_2 e^{-\sqrt{|\lambda|}x} = C_1 sh(\sqrt{|\lambda|}x) + C_2 ch({\sqrt{|\lambda|}x})
            \end{aligned}
        \end{equation*}
        Знаходимо константи з межових умов:
        \begin{equation*}
            \begin{aligned}
                X(0) = C_2
                \;\Rightarrow\;
                X(x) = C_1 &sh(\sqrt{|\lambda|}x)\\
                \left\{ \begin{aligned}
                    &X(l) = C_1 sh(\sqrt{|\lambda|}l) = 0, \\
                    &sh(\sqrt{|\lambda|}l) \neq 0;
                \end{aligned} \right.&\\
                \left\{ \begin{aligned}
                    C_1 = 0, \\ 
                    C_2 = 0;
                \end{aligned} \right. \qquad\qquad\qquad\qquad&
            \end{aligned}
            \;\Rightarrow\;
            \begin{aligned}
                \text{розв'язок тривівльний,}\\
                \text{немає від'ємних}\\
                \text{власних значень.}
            \end{aligned}
        \end{equation*}

        \item Розглянемо випадок $\lambda > 0$. Розв'язок рівняння шукаємо у вигляді $X(x) = e^{\alpha x}$: 
        \begin{equation*}
            \begin{aligned}
                &X^{\prime\prime} = -\lambda X
                \quad\Rightarrow\quad
                \alpha^2 \textcolor{red}{\begin{xy}*{\textcolor{black}{e^{\alpha x}}};p+LD;+RU**h@{}+/\jot/**h@{-}\end{xy}} = -\lambda \textcolor{red}{\begin{xy}*{\textcolor{black}{e^{\alpha x}}};p+LD;+RU**h@{}+/\jot/**h@{-}\end{xy}}
                \quad\Rightarrow\quad
                \alpha = \pm i\sqrt{\lambda}
                \;\Rightarrow\\
                \Rightarrow\;
                &X(x) = \widetilde{C}_1 e^{i\sqrt{\lambda}x} + \widetilde{C}_2 e^{-\sqrt{\lambda}x} = C_1 \sin(\sqrt{\lambda}x) + C_2 \cos({\sqrt{\lambda}x})
            \end{aligned}
        \end{equation*}
        Знаходимо константи з межових умов:
        \begin{equation*}
            \begin{aligned}
                \left\{ \begin{aligned}
                    &X(0) = C_2 = 0, \\ 
                    &X(l) = C_1 \sin(\sqrt{\lambda}l) + \textcolor{red}{\begin{xy}*{\textcolor{black}{C_2}};p+LU;+RD**h@{}+//**h@{-}*h@{>}*h!LD{\scriptstyle 0}\end{xy}} \cos({\sqrt{\lambda}l}) = 0;
                \end{aligned} \right.
                \;\Rightarrow\;
                \left\{ \begin{aligned}
                    &C_1 \neq 0, \\ 
                    &\sin(\sqrt{\lambda}l) = 0;
                \end{aligned} \right.
            \end{aligned}
        \end{equation*}
        Отже, нетривіальні розв'язки існують при значеннях параметра $\lambda$, які задовольняють характеристичне рівняння :
        \begin{equation*}
            \sin(\sqrt{\lambda}l) = 0
            \;\Rightarrow\;
            \sqrt{\lambda_n}l = \pi n, \, n \in \mathbb{Z}
            \;\Rightarrow\;
            \lambda_n = \frac{\pi^2 n^2}{l^2}.
        \end{equation*}
    \end{enumerate}
\end{enumerate} 
Випишемо тепер розв'язки для всіх $n$ і визначимо, які з них необхідно залишити:
    \begin{equation*}
        X_n(x) = C_n \sin\left(\frac{\pi n x}{l}\right)
    \end{equation*}
    Видно, що $n = 0$ відповідає тривіальному розв'язку. Видно також, що всі інші розв'язки визначені з точністю до довільного множника.\\
    Тому власні функції, які співпадають з точністю до множника, вважають однаковими. У загальному випадку різними вважають лише лінійно незалежні власні функції, а розвя'зати задачу Штурма-Ліувілля означає знайти всі різні власні функції і відповідні власні значення. Отже, різним власним функціям відповідають лише натуральні $n$, а коефіцієнти $C_n$ можна покласти рівними одиниці.\\
    Власними значеннями і власними функціями є
    \begin{equation} \label{ShLsol}
        \left\{ \begin{aligned}
            \;&\lambda_n = \frac{\pi^2 n^2}{l^2},\\ 
            &X_n(x) = \sin\left(\frac{\pi n x}{l}\right),
        \end{aligned} \right.
        \quad \text{де } n \in \mathbb{N}.
    \end{equation}

Повертаємося до рівняння для $T(t)$ (\ref{sepvar}). Підставляємо знайдені власні значення та знаходимо $T_n(t)$:
\begin{equation*}
    \left. \begin{aligned}
        \lambda_n = \frac{\pi^2 n^2}{l^2},&\;\\ 
        T^{\prime\prime} + \lambda v^2T = 0,&
    \end{aligned} \right\}
    \;\Rightarrow\;
    T_n(t) = A\cos(\omega_n t) + B\sin(\omega_n t),
\end{equation*}
де $\omega_n^2 = \lambda_n v^2, \, n \in \mathbb{N}.$\\
Власними модами коливань струни будуть всі розв'язки вигляду:
\begin{equation*}
    u_n(x,t) = X_n(x) \cdot T_n(t)
\end{equation*}
Виконаємо перепозначення і запишемо остаточний розв'язок:
\begin{equation}
    \left\{ \begin{aligned} \label{sol1}
        \;&u_n(x,t) = \left[A_n\cos(\omega_n t) + B_n\sin(\omega_n t)\right] \sin(k_n x), \\
        &k_n = \frac{\pi n}{l} - \text{ хвильові вектори}, \\
        &\omega_n = vk_n = \frac{v \pi n}{l} - \text{ власні частоти}, \\
        &n = 1, 2,\ldots
    \end{aligned}\right.
\end{equation}

\begin{center}
    \large{\textbf{Перевірка розв'язку задачі Штурма-Ліувілля}}
\end{center}

\noindent Перевірка результату (\ref{ShLsol}) включає аналітичну і графічну перевірку. Необхідно перевірити, що знайдені розв'язки і числа (\ref{ShLsol}) дійсно є власними функціями і власними значеннями задачі (\ref{sepvar}), а також, що знайдені всі її власні функції і власні значення. Перш за все, перевіряємо виконання всіх умов задачі Штурма-Ліувілля (\ref{sepvar}).
\begin{enumerate}[wide, labelindent=0pt]
    \item Аналітична перевірка 
    Підставляємо знайдені функції у крайові умови і рівняння задачі Штурма-Ліувілля. Відмінні від нуля функції, які задовольняють крайові умови і рівняння, за означенням є власними функціями задачі. Одночасно знаходимо з рівняння відповідне власне значення і звіряємо його з указаним у відповіді.
    \begin{enumerate}
        \item[1)] Перевіряємо виконання крайових умов, підставляємо власні функції в умови (\ref{sepvar}) 
        \begin{equation*}
            \begin{aligned}
                X(0) &=  0:\\
                &\begin{aligned}
                    &X_n(0) = C_n \sin(\sqrt{\lambda_n} \cdot 0) = 0 \text{ -- виконується,}\\
                    &\text{ причому незалежно від }\lambda_n
                \end{aligned}\\
                \\
                X(l) &= 0:\\
                &\begin{aligned}
                    &X_n(l) = C_n \sin\left(\frac{\pi n}{l} \cdot l\right) = C_n \sin(\pi n) = 0 \text{ -- виконується}\\
                    &\text{ причому саме для знайдених значень }\lambda_n.
                \end{aligned}\\
            \end{aligned}
        \end{equation*}
        \item[2)] Перевіряємо рівняння і власні значення. Виконання крайових умов уже перевірено; якщо ми підставимо функцію у рівняння, то одночасно знайдемо і відповідне їй власне значення (якщо рівняння виконується). Це власне значення має співпасти з указаним у відповіді. 
        
        Почергово перевіряємо всі функції, вказані у відповіді. Обчислимо другу похідну  
        \begin{equation*}
            X_n^{''} = \frac{\pi n}{l} \left(C_n\cos\left(\frac{\pi n x}{l}\right)\right)^{'} = -\left(\frac{\pi n}{l}\right)^2 C_n\sin\left(\frac{\pi n x}{l}\right) = -\left(\frac{\pi n}{l}\right)^2 X_n
        \end{equation*}
        Порівнюємо з вихідним рівнянням і робимо перший висновок: кожна з функцій $X_n(x)$ дійсно є розв’язком рівняння (\ref{sepvar}). Одночасно, знаходимо з рівняння відповідне даній функції значення спектрального параметра, - це $\lambda = \left(\frac{\pi n}{l}\right)^2$. Порівняємо це значення з тим, яке вказане у відповіді (\ref{ShLsol}) і робимо другий висновок: знайдені власні значення дійсно відповідають знайденим власним функціям.
    \end{enumerate}
    \item Графічна перевірка.\\
    Будуємо графіки кількох перших власних функцій. Масштаб по вертикалі може бути довільним і різним для різних функцій, оскільки значення він не має.
    \begin{figure}[h]
        \centering
        %\large \textbf{Graph under comment}%
        \begin{tikzpicture}
    \begin{axis}
        [width = 0.85\textwidth, height = 0.4\textwidth,
         axis x line = center, axis y line = center,
         ylabel = $X(x)$, xlabel = $x$,
         xmin = -0.3, xmax = 5.7, ymin = -3.3, ymax = 5.3,
         axis line style = thin, xtick = {0}, ytick = {0}]   
        
        \tikzmath{\A1 = 5; \l = 5; \k = pi/\l;}
        
        \addplot [black, domain=0:\l, samples = 1000] {\A1 * sin(deg(\k*x))}
        node[anchor=130, pos=0] {0} 
        node[pos=0.75, fill=white] {$X_1(x)$} 
        node[anchor=130, pos=1] {$l$};
        
        \addplot [black, domain=0:\l, samples = 1000] {\A1/2.5 * sin(deg(2*\k*x))}
        node[pos=0.83, fill=white] {$X_2(x)$};
        
        \addplot [black, domain=0:\l, samples = 1000] {\A1/3.7 * sin(deg(3*\k*x))}
        node[pos=0.76, fill=white] {$X_3(x)$};
        
    \end{axis}
\end{tikzpicture} %%for compiling main 
        %\begin{tikzpicture}
    \begin{axis}
        [width = 0.85\textwidth, height = 0.4\textwidth,
         axis x line = center, axis y line = center,
         ylabel = $X(x)$, xlabel = $x$,
         xmin = -0.3, xmax = 5.7, ymin = -3.3, ymax = 5.3,
         axis line style = thin, xtick = {0}, ytick = {0}]   
        
        \tikzmath{\A1 = 5; \l = 5; \k = pi/\l;}
        
        \addplot [black, domain=0:\l, samples = 1000] {\A1 * sin(deg(\k*x))}
        node[anchor=130, pos=0] {0} 
        node[pos=0.75, fill=white] {$X_1(x)$} 
        node[anchor=130, pos=1] {$l$};
        
        \addplot [black, domain=0:\l, samples = 1000] {\A1/2.5 * sin(deg(2*\k*x))}
        node[pos=0.83, fill=white] {$X_2(x)$};
        
        \addplot [black, domain=0:\l, samples = 1000] {\A1/3.7 * sin(deg(3*\k*x))}
        node[pos=0.76, fill=white] {$X_3(x)$};
        
    \end{axis}
\end{tikzpicture} %%for compiling this file
    \caption{Графічний розв'язок, наведені три перші власні функції}
    \end{figure}\\
    Функція, що відповідає найменшому власному значенню (у даному випадку це $X_1(x) = \sin\left(\frac{\pi x}{l}\right)$), відповідає так званій \textit{основній моді} резонатора (струна є частинним випадком одномірного резонатора). \textit{Для стаціонарних станів у квантовій механіці це основний стан системи, стан з найменшою можливою енергією.}\\
    Графіки власних функцій відображають їх основні властивості, які і треба перевірити. З рисунку видно, що в точках $x = 0$ та $x = l$ всі графіки проходять через нуль, отже крайова умова (\ref{sepvar}) на обох кінцях виконується. \textit{Графічна перевірка підсилює надійність аналітичної, в якій також іноді припускаються помилок, приймаючи бажане за дійсне.}
\end{enumerate}

Проведені перевірки допомагають позбавитись більшості типових помилок, проте є надзвичайно підступний тип помилки, який проведені перевірки виявити не взмозі. Це випадок, коли ви дійсно знайшли власні функції та власні значення, але \textbf{не всі}, а отже задача розв'язана неправильно.\\
Помітити таку помилку допомагає так звана \textbf{осциляційна теорема}. З рисунку, наведеного в графічному методі, видно, що всі власні функції на проміжку $0 \leq x \leq l$ осцилюють, і при цьому всі вони мають \textbf{різне} число нулів. Число нулів (або "вузлів") всередені проміжку, на якому ророзв’язується задача, є своєрідною унікальною міткою власної функції. Власні значення необхідно розташувати у порядку зростання. Тоді за осциляційною теоремою основна мода одновимірної задачі Штурма-Ліувілля не має нулів у внутрішніх точках проміжку $[0, l]$. Тобто основна мода завжди є безвузловою. Далі, наступному за величиною власному значенню відповідає власна функція, що має один нуль, наступному - два нулі, і так далі. Для кожної наступної моди число вузлів збільшується на одиницю. Іншими словами, число нулів власної функції збігається з порядковим номером відповідного власного значення, якщо нумерувати їх у порядку зростання, починаючи з нуля.

\begin{center}
    \large{\textbf{Аналіз результату}}
\end{center}
З'ясуємо фізичний смисл одержаних розв'язків: яким саме рухам відповідають її власні моди. Розв’язки (\ref{sol1}) є частинними розв'язками однорідного хвильового рівняння з однорідними межовими умовами (\ref{probcond}). Це означає (див. лекції), що зовнішні сили на систему не діють, тому знайдені розв’язки відповідають вільним коливанням (рухам) струни. Це коливання (рухи) спеціального вигляду, оскільки відповідні розв’язки мають вигляд добутків $X_n(x) \cdot T_n(t)$.\\
Усі розв’язки $u_n(x, t)$ -- дійсні. Візьмемо один із них. Зафіксуємо певний довільний момент часу $t = t_1$, це буде миттєве фото $n$–ї моди струни. Просторовий розподіл зміщень описується формулою \[u_n(x, t_1) = X_n(x) \cdot T_n(t_1).\] Видно, що в будь-який момент часу форма просторового розподілу зміщень (тобто форма струни) залишається однаковою і описується відповідною власною функцією $X_n(x)$; за рахунок множника $T_n(t)$ змінюється лише спільна амплітуда просторового розподілу і його знак. Отже, $X_n(x)$ задає просторовий «профіль» моди, це її унікальне просторове «обличчя», - найперша визначальна характеристика певної моди, за якою можна ідентифікувати відповідний рух струни.\\
Тепер прослідкуємо за рухом певної точки струни $x = x_1$: \[u_n(x_1, t) = X_n(x_1) \cdot T_n(t).\] Видно, що всі точки струни здійснюють один і той же рух, одне і те ж гармонічне коливання з частотою $\omega_n$, але для різних мод частоти коливань різні. Коливання будь-якої точки задається однією функцією $T_n(t)$, але амплітуда визначається величиною $\left|X_n(x)\right|$. У точках, де $X_n(x)$ має нулі, амплітуда коливань дорівнює нулю, це \textbf{вузли} моди. Якщо ж $X_n(x)$ змінює знак, фаза коливання змінюється на $\pi$: частини струни, розділені вузлами, коливаються у протифазі.

Рух струни - це єдиний часово-просторовий процес. Для мод $n= 2$ і $n= 3$ він зображений на рисунку нижче. Усі вузли залишаються нерухомими, тільки якщо рух струни відповідає певній моді. У точках, де $X_n(x)$ максимальне (за модулем), максимальна й амплітуда коливань, -- це пучності.
\begin{figure}[h]
    \centering
    %\large \textbf{Graph under comment}%
    \begin{tikzpicture}
    \begin{axis}
        [width = 0.85\textwidth, height = 0.4\textwidth,
         axis x line = center, axis y line = center,
         ylabel = $X(x)$, xlabel = $x$,
         xmin = -0.3, xmax = 5.7, ymin = -5.3, ymax = 5.3,
         axis line style = thin, xtick = {0}, ytick = {0}]   
        
        \tikzmath{\A = 4; \l = 5; \k = pi/\l;}
        
        \addplot [black, domain=0:\l, samples = 1000] {\A * sin(deg(2*\k*x))}
        node[anchor=50, pos=0] {0} 
        node[anchor=west, pos=0.26, fill=white] {$u_2(x, t)$} 
        node[anchor=130, pos=1] {$l$};

        \addplot [red, domain=0:\l, samples = 1000] {-\A * sin(deg(2*\k*x))}
        node[anchor=north, pos=0.19] {$\textcolor{black}{\omega_2}$};
        
        \tikzmath{\p1 = \l/8; \p2 = \l/4; \p3 = 3*\l/8;
        \y1 = \A * sin(deg(2*\k*\p1)); \y2 = \A * sin(deg(2*\k*\p2)); \y3 = \A * sin(deg(2*\k*\p3));}

        \draw [arrows = {-Stealth[scale=1.5]}] (\p1,-\y1) -- (\p1,\y1);
        \draw [arrows = {-Stealth[scale=1.5]}] (\p2,-\y2) -- (\p2,\y2);
        \draw [arrows = {-Stealth[scale=1.5]}] (\p3,-\y3) -- (\p3,\y3);

        \tikzmath{\p4 = 5*\l/8; \p5 = 3*\l/4; \p6 = 7*\l/8;
        \y4 = \A * sin(deg(2*\k*\p4)); \y5 = \A * sin(deg(2*\k*\p5)); \y6 = \A * sin(deg(2*\k*\p6));}

        \draw [arrows = {-Stealth[scale=1.5]}] (\p4,-\y4) -- (\p4,\y4);
        \draw [arrows = {-Stealth[scale=1.5]}] (\p5,-\y5) -- (\p5,\y5);
        \draw [arrows = {-Stealth[scale=1.5]}] (\p6,-\y6) -- (\p6,\y6);
        
    \end{axis}
\end{tikzpicture}

\vspace{1cm}

\begin{tikzpicture}
    \begin{axis}
        [width = 0.85\textwidth, height = 0.4\textwidth,
         axis x line = center, axis y line = center,
         ylabel = $X(x)$, xlabel = $x$,
         xmin = -0.3, xmax = 5.7, ymin = -5.3, ymax = 5.3,
         axis line style = thin, xtick = {0}, ytick = {0}]   
        
        \tikzmath{\A = 4; \l = 5; \k = pi/\l;}
        
        \addplot [black, domain=0:\l, samples = 1000] {\A * sin(deg(3*\k*x))}
        node[anchor=50, pos=0] {0} 
        node[anchor=south, pos=0.19, fill=white] {$u_3(x, t)$} 
        node[anchor=130, pos=1] {$l$};

        \addplot [red, domain=0:\l, samples = 1000] {-\A * sin(deg(3*\k*x))}
        node[anchor=west, pos=0.23] {$\textcolor{black}{\omega_3}$};
        
        \tikzmath{\p1 = \l/9; \p2 = 2*\l/9; \p3 = 4*\l/9;
        \y1 = \A * sin(deg(3*\k*\p1)); \y2 = \A * sin(deg(3*\k*\p2)); \y3 = \A * sin(deg(3*\k*\p3));}

        \draw [arrows = {-Stealth[scale=1.5]}] (\p1,-\y1) -- (\p1,\y1);
        \draw [arrows = {-Stealth[scale=1.5]}] (\p2,-\y2) -- (\p2,\y2);
        \draw [arrows = {-Stealth[scale=1.5]}] (\p3,-\y3) -- (\p3,\y3);

        \tikzmath{\p4 = 5*\l/9; \p5 = 7*\l/9; \p6 = 8*\l/9;
        \y4 = \A * sin(deg(3*\k*\p4)); \y5 = \A * sin(deg(3*\k*\p5)); \y6 = \A * sin(deg(3*\k*\p6));}

        \draw [arrows = {-Stealth[scale=1.5]}] (\p4,-\y4) -- (\p4,\y4);
        \draw [arrows = {-Stealth[scale=1.5]}] (\p5,-\y5) -- (\p5,\y5);
        \draw [arrows = {-Stealth[scale=1.5]}] (\p6,-\y6) -- (\p6,\y6);
        
    \end{axis}
\end{tikzpicture} %%for compiling main 
    %\begin{tikzpicture}
    \begin{axis}
        [width = 0.85\textwidth, height = 0.4\textwidth,
         axis x line = center, axis y line = center,
         ylabel = $X(x)$, xlabel = $x$,
         xmin = -0.3, xmax = 5.7, ymin = -5.3, ymax = 5.3,
         axis line style = thin, xtick = {0}, ytick = {0}]   
        
        \tikzmath{\A = 4; \l = 5; \k = pi/\l;}
        
        \addplot [black, domain=0:\l, samples = 1000] {\A * sin(deg(2*\k*x))}
        node[anchor=50, pos=0] {0} 
        node[anchor=west, pos=0.26, fill=white] {$u_2(x, t)$} 
        node[anchor=130, pos=1] {$l$};

        \addplot [red, domain=0:\l, samples = 1000] {-\A * sin(deg(2*\k*x))}
        node[anchor=north, pos=0.19] {$\textcolor{black}{\omega_2}$};
        
        \tikzmath{\p1 = \l/8; \p2 = \l/4; \p3 = 3*\l/8;
        \y1 = \A * sin(deg(2*\k*\p1)); \y2 = \A * sin(deg(2*\k*\p2)); \y3 = \A * sin(deg(2*\k*\p3));}

        \draw [arrows = {-Stealth[scale=1.5]}] (\p1,-\y1) -- (\p1,\y1);
        \draw [arrows = {-Stealth[scale=1.5]}] (\p2,-\y2) -- (\p2,\y2);
        \draw [arrows = {-Stealth[scale=1.5]}] (\p3,-\y3) -- (\p3,\y3);

        \tikzmath{\p4 = 5*\l/8; \p5 = 3*\l/4; \p6 = 7*\l/8;
        \y4 = \A * sin(deg(2*\k*\p4)); \y5 = \A * sin(deg(2*\k*\p5)); \y6 = \A * sin(deg(2*\k*\p6));}

        \draw [arrows = {-Stealth[scale=1.5]}] (\p4,-\y4) -- (\p4,\y4);
        \draw [arrows = {-Stealth[scale=1.5]}] (\p5,-\y5) -- (\p5,\y5);
        \draw [arrows = {-Stealth[scale=1.5]}] (\p6,-\y6) -- (\p6,\y6);
        
    \end{axis}
\end{tikzpicture}

\vspace{1cm}

\begin{tikzpicture}
    \begin{axis}
        [width = 0.85\textwidth, height = 0.4\textwidth,
         axis x line = center, axis y line = center,
         ylabel = $X(x)$, xlabel = $x$,
         xmin = -0.3, xmax = 5.7, ymin = -5.3, ymax = 5.3,
         axis line style = thin, xtick = {0}, ytick = {0}]   
        
        \tikzmath{\A = 4; \l = 5; \k = pi/\l;}
        
        \addplot [black, domain=0:\l, samples = 1000] {\A * sin(deg(3*\k*x))}
        node[anchor=50, pos=0] {0} 
        node[anchor=south, pos=0.19, fill=white] {$u_3(x, t)$} 
        node[anchor=130, pos=1] {$l$};

        \addplot [red, domain=0:\l, samples = 1000] {-\A * sin(deg(3*\k*x))}
        node[anchor=west, pos=0.23] {$\textcolor{black}{\omega_3}$};
        
        \tikzmath{\p1 = \l/9; \p2 = 2*\l/9; \p3 = 4*\l/9;
        \y1 = \A * sin(deg(3*\k*\p1)); \y2 = \A * sin(deg(3*\k*\p2)); \y3 = \A * sin(deg(3*\k*\p3));}

        \draw [arrows = {-Stealth[scale=1.5]}] (\p1,-\y1) -- (\p1,\y1);
        \draw [arrows = {-Stealth[scale=1.5]}] (\p2,-\y2) -- (\p2,\y2);
        \draw [arrows = {-Stealth[scale=1.5]}] (\p3,-\y3) -- (\p3,\y3);

        \tikzmath{\p4 = 5*\l/9; \p5 = 7*\l/9; \p6 = 8*\l/9;
        \y4 = \A * sin(deg(3*\k*\p4)); \y5 = \A * sin(deg(3*\k*\p5)); \y6 = \A * sin(deg(3*\k*\p6));}

        \draw [arrows = {-Stealth[scale=1.5]}] (\p4,-\y4) -- (\p4,\y4);
        \draw [arrows = {-Stealth[scale=1.5]}] (\p5,-\y5) -- (\p5,\y5);
        \draw [arrows = {-Stealth[scale=1.5]}] (\p6,-\y6) -- (\p6,\y6);
        
    \end{axis}
\end{tikzpicture} %%for compiling this file
    \caption{Графіки другої та третьої моди}
\end{figure}
У нашому випадку моди занумеровані так, що число вузлів всередині струни на одиницю менше номера моди. При цьому кожна мода має свою частоту коливань, за якою також можна розрізнити різні моди. Для струни з обома закріпленими кінцями частоти всіх мод кратні частоті основної моди (див. (\ref{sol1})).\\
У цілому рух у часі і розподіл зміщень у просторі залишаються ніби незалежними. Струна коливається, а просторовий розподіл «стоїть». Подібні рухи прийнято називати \textbf{стоячими хвилями}. Стоячі хвилі описуються добутком вигляду (\ref{subst}), в яких обидва множники є дійсними.

%\end{document}

\chapter{Власні моди інших систем. Вільні коливання для заданих початкових умов.}
%\documentclass[a4paper, 14pt]{extreport}

%\usepackage{StyleMMF}

%\begin{document}

%\setcounter{chapter}{1}
%\chapter{Власні моди інших систем. Вільні коливання для заданих початкових умов.}

\textbf{\large Стержень з вільними та пружно закріпленими кінцями; системи, описувані іншими рівняннями.}

\section[Задача №2.1]{2.1}

\textit{Знайти власні моди повздовжніх рухів тонкого стержня $0 \leq x \leq l$ із вільними кінцями  (задача для хвильового рівняння з межовими умовами $u_x(0,t) = 0, u_x(l,t) = 0$).\\
Результат перевірити аналітично й графічно (див. заняття №6, зразок модульної контрольної роботи №1) та проаналізувати його фізичний смисл. Чим відрізняється від інших основна (нульова) мода? Якому рухові стержня вона відповідає?}

\begin{center}
    \large{\textbf{Розв'язок}}
\end{center}

\noindent Формальна постановка задачі:
\begin{equation} \label{cond2,1}
    \left\{ \begin{aligned} %%
            \;&u = u(x,t), \\
            &u_{tt} = v^2 u_{xx}, \\
            &0 \leq x \leq l, t \in \mathbb{R} \\
            &u_x(0,t) = 0, \\
            &u_x(l,t) = 0. 
    \end{aligned} \right.
\end{equation}
Необхідно знайти розв'язки (\ref{cond2,1}) вигляду:
\begin{equation} \label{subst2,1}
    u(x,t) = X(x) \cdot T(t) \neq 0 
\end{equation}

Від задачі №1.1 попереднього заняття задача відрізняється тільки межовою умовою, тому підставляємо розв'язок у вигляді добутку (\ref{subst2,1}) тільки у межові умови (\ref{cond2,1}):
\begin{equation*}
    \begin{aligned}
        \;u_x(0,t) = X'(0) \cdot T(t) = 0
        \;\Rightarrow\;
        \left\{ \begin{aligned}
            &T(t) \neq 0, \forall t, \\  &X'(0) = 0; 
        \end{aligned} \right.\\
        u_x(l,t) = X'(l) \cdot T(t) = 0
        \;\Rightarrow\;
        \left\{ \begin{aligned}
            &T(t) \neq 0, \forall t, \\  &X'(l) = 0; 
        \end{aligned} \right.\\
    \end{aligned}
\end{equation*}
Тут ми врахували, що умови на кінцях струни виконуються при всіх $t$, тому $T(t)$ не може бути рівним нулю.\\

Виписуємо результат відокремлення змінних:
\begin{equation} \label{sepvar2,1}
    \left\{ \begin{aligned}
        \;&X = X(x), \\
          &X^{\prime\prime} = -\lambda X, \\
          &0 \leq x \leq l, \\
          &X'(0) = 0, \\ 
          &X'(l) = 0. 
    \end{aligned} \right.
    \qquad\qquad
    T^{\prime\prime} + \lambda v^2 T = 0
\end{equation}

\begin{enumerate}
    \item[] Розв'язуємо задачу Штурма-Ліувілля (\ref{sepvar2,1}). Розв'язки рівняння задачі для різних $\lambda$ є такими ж, як у задачі 1.1, відмінність полягає у крайових умовах.
    \begin{enumerate}[wide, labelindent=0pt]
        
        \item Випадок $\lambda < 0$. 
        \begin{equation*}
            X(x) = C_1 sh(\sqrt{|\lambda|}x) + C_2 ch({\sqrt{|\lambda|}x})
        \end{equation*}
        Знаходимо константи з межових умов:
        \begin{equation*}
            \begin{aligned}
                &X'(0) = C_1\sqrt{|\lambda|}
                \;\Rightarrow\;
                X(x) = C_2 &ch(\sqrt{|\lambda|}x)\\
                &\left\{ \begin{aligned}
                    &X'(l) = C_2\sqrt{|\lambda|} sh(\sqrt{|\lambda|}l) = 0, \\
                    &sh(\sqrt{|\lambda|}l) \neq 0;
                \end{aligned} \right.&\\
                &\left\{ \begin{aligned}
                    C_1 = 0, \\ 
                    C_2 = 0;
                \end{aligned} \right. \qquad\qquad\qquad\qquad&
            \end{aligned}
            \;\Rightarrow\;
            \begin{aligned}
                \text{розв'язок тривівльний,}\\
                \text{немає від'ємних}\\
                \text{власних значень.}
            \end{aligned}
        \end{equation*}

        \item Випадок $\lambda = 0$:
        \begin{equation*}
            X(x) = C_1 + C_2 x
        \end{equation*}
        Знаходимо константи з межових умов:
        \begin{equation*}
            \begin{aligned}
                &\left\{ \begin{aligned}
                    &X'(0) = C_2 = 0, \\ 
                    &X'(l) = C_2 = 0;
                \end{aligned} \right.
                \\   
                &\left\{ \begin{aligned}
                    C_1 \in \mathbb{R}, \\ 
                    C_2 = 0;
                \end{aligned} \right.
            \end{aligned}
            \quad\Rightarrow\;
            \begin{aligned}
                X(x) = C \text{ -- розв'язок нетривівльний,}\\
                \lambda = 0 \text{ є власним значенням.}
            \end{aligned}
        \end{equation*}

        \item Випадок $\lambda > 0$
        \begin{equation*}
            X(x) = C_1 \sin(\sqrt{\lambda}x) + C_2 \cos({\sqrt{\lambda}x})
        \end{equation*}
        Знаходимо константи з межових умов:
        \begin{equation*}
            \begin{aligned}
                \left\{ \begin{aligned}
                    &X'(0) = C_1\sqrt{\lambda} = 0, \\ 
                    &X(x) = C_2 \cos({\sqrt{\lambda}x}), \\
                    &X'(l) = - C_2 \sin(\sqrt{\lambda}l) = 0;
                \end{aligned} \right.
                \;\Rightarrow\;
                \left\{ \begin{aligned}
                    &C_2 \neq 0, \\ 
                    &\sin(\sqrt{\lambda}l) = 0;
                \end{aligned} \right.
            \end{aligned}
        \end{equation*}
        Отже, нетривіальні розв'язки існують при значеннях параметра $\lambda$, які задовольняють характеристичне рівняння :
        \begin{equation*}
            \sin(\sqrt{\lambda}l) = 0
            \;\Rightarrow\;
            \sqrt{\lambda_n}l = \pi n, \, n \in \mathbb{Z}
            \;\Rightarrow\;
            \lambda_n = \frac{\pi^2 n^2}{l^2}.
        \end{equation*}
    \end{enumerate}
\end{enumerate} 
Випишемо тепер розв'язки для всіх $n$, поклавши всі довільні сталі рівними $1$. Залишаємо з них лише нетривіальні розв'язки для тих $n$, які відповідають різним власним функціям:
    \begin{equation*}
        \left\{ \begin{aligned}
            &X_0(x) = 1,\\
            &\lambda_0 = 0;
        \end{aligned} \right.
        \qquad
        \left\{ \begin{aligned}
            &X_n(x) = \cos\left(\frac{\pi n x}{l}\right),\\
            &\lambda_n = \frac{\pi^2 n^2}{l^2}, n \in \mathbb{N}.
        \end{aligned} \right.
    \end{equation*}


На відміну від задачі 1 з попереднього заняття тут $n = 0$ відповідає нетривіальному розв'язку. Випадок $\lambda > 0$ знову приводить до набору власних функцій занумерованих натуральними числами.\\
Отже, різним власним функціям відповідають натуральні $n$ та $0$.\\
    Власними значеннями і власними функціями є
    \begin{equation} \label{ShLsol2,1}
        \left\{ \begin{aligned}
            &\lambda_0 = 0,\\
            &X_0(x) = 1;\\
            \;&\lambda_n = \frac{\pi^2 n^2}{l^2}, \text{ де } n \in \mathbb{N}\\ 
            &X_n(x) = \cos\left(\frac{\pi n x}{l}\right),
        \end{aligned} \right.
    \end{equation}

Повертаємося до рівняння для $T(t)$ (\ref{sepvar2,1}). Підставляємо знайдені власні значення та знаходимо $T_n(t)$:
\begin{equation*}
    \left. \begin{aligned}
        \lambda_n = \frac{\pi^2 n^2}{l^2},&\;\\ 
        T^{\prime\prime} + \lambda v^2T = 0,&
    \end{aligned} \right\}
    \;\Rightarrow\;
    T_n(t) = A\cos(\omega_n t) + B\sin(\omega_n t),
\end{equation*}
де $\omega_n^2 = \lambda_n v^2, \, n \in \mathbb{N}.$\\
\begin{equation*}
    \left. \begin{aligned}
        \lambda_0 = 0,&\;\\ 
        T^{\prime\prime} = 0,&
    \end{aligned} \right\}
    \;\Rightarrow\;
    T_0(t) = A_0 + B_0 t,
\end{equation*}
Власними модами коливань струни будуть всі розв'язки вигляду:
\begin{equation*}
    u_n(x,t) = X_n(x) \cdot T_n(t)
\end{equation*}
Виконаємо перепозначення і запишемо остаточний розв'язок:
\begin{equation}
    \left\{ \begin{aligned} \label{mode2,1}
        \;&u_0(x,t) = A_0 + B_0 t, \\
        &u_n(x,t) = \left[A_n\cos(\omega_n t) + B_n\sin(\omega_n t)\right] \sin(k_n x), \\
        &k_n = \frac{\pi n}{l} - \text{ хвильові вектори}, \\
        &\omega_n = vk_n = \frac{v \pi n}{l} - \text{ власні частоти}, \\
        &n = 1, 2,\ldots
    \end{aligned}\right.
\end{equation}

\begin{center}
    \large{\textbf{Перевірка розв'язку задачі Штурма-Ліувілля}}
\end{center}

\begin{enumerate}[wide, labelindent=0pt]
    \item Аналітична перевірка
    \begin{enumerate}
        \item[1)] Перевіряємо, чи виконуються крайові умови задачі:
        \begin{equation*}
            \begin{aligned}
                X'(0) &=  0:\\
                &\begin{aligned}
                    &X_0'(0) = 0,\\
                    &X_n'(0) = \sin(\sqrt{\lambda_n} \cdot 0) = 0 \text{ -- виконується,}\\
                    &\text{ причому незалежно від }\lambda_n
                \end{aligned}\\
                \\
                X'(l) &= 0:\\
                &\begin{aligned}
                    &X_0'(l) = 0,\\
                    &X_n'(l) = \sin\left(\frac{\pi n}{l} \cdot l\right) = \sin(\pi n) = 0 \text{ -- виконується}\\
                    &\text{ причому саме для знайдених значень }\lambda_n.
                \end{aligned}\\
            \end{aligned}
        \end{equation*}
        \item[2)] Перевіряємо, чи задовольняють знайдені функції рівняння на власні значення $X^{\prime\prime} = -\lambda X$, і якщо так, то знаходимо відповідне значення спектрального параметра $\lambda$:
        \begin{equation*}
            \begin{aligned}
                &X_0^{''} = \left(1\right)^{''} = 0 \cdot 1 = 0 \cdot X_0 \\
                &X_n^{''} = -\frac{\pi n}{l} \left(\sin\left(\frac{\pi n x}{l}\right)\right)^{'} = -\left(\frac{\pi n}{l}\right)^2 \cos\left(\frac{\pi n x}{l}\right) = -\left(\frac{\pi n}{l}\right)^2 X_n
            \end{aligned} 
        \end{equation*}
        Отже знайдені функції задовольняють і крайові умови, і рівняння задачі Штурма-Ліувілля, причому для значень спектрального параметра \begin{equation}
        \lambda_0 = 0 \text{ i }\lambda_n = \frac{\pi^2 n^2}{l^2}, \text{ де } n \in \mathbb{N}
        \end{equation}які співпадають з раніше знайденими. Звідси робимо висновок, що вказані у відповіді (\ref{ShLsol2,1}) функції та значення спектрального параметра дійсно є власними функціями і відповідними їм власними значеннями задачі Штурма-Ліувілля.
    \end{enumerate}
    \item Графічна перевірка.\\
    Будуємо графіки кількох перших власних функцій. Масштаб по вертикалі може бути довільним і різним для різних функцій, оскільки значення він не має.
    \begin{figure}[h]
        \centering
        %\large \textbf{Graph under comment}%
        \begin{tikzpicture}
    \begin{axis}
        [width = 0.85\textwidth, height = 0.4\textwidth,
         axis x line = center, axis y line = center,
         ylabel = $X(x)$, xlabel = $x$,
         xmin = -0.3, xmax = 5.7, ymin = -5.3, ymax = 7.3,
         axis line style = thin, xtick = {0}, ytick = {0}]   
        
        \tikzmath{\A1 = 5; \l = 5; \k = pi/\l;}
        
        \addplot [black, domain=0:\l, samples = 1000] {0}
        node[anchor=130, pos=0] {0} 
        node[anchor=130, pos=1] {$l$};
        
        \addplot [black, domain=0:\l, samples = 1000] {\A1}
        node[anchor=south, pos=0.55, fill=white] {$X_0(x)$}; 


        \addplot [black, domain=0:\l, samples = 1000] {\A1 * cos(deg(\k*x))}
        node[anchor=south, pos=0.31, fill=white] {$X_1(x)$}; 

        \addplot [black, domain=0:\l, samples = 1000] {\A1/2 * cos(deg(2*\k*x))}
        node[anchor=south, pos=0.15, fill=white] {$X_2(x)$};
        
    \end{axis}
\end{tikzpicture} %%for compiling main 
        %\begin{tikzpicture}
    \begin{axis}
        [width = 0.85\textwidth, height = 0.4\textwidth,
         axis x line = center, axis y line = center,
         ylabel = $X(x)$, xlabel = $x$,
         xmin = -0.3, xmax = 5.7, ymin = -5.3, ymax = 7.3,
         axis line style = thin, xtick = {0}, ytick = {0}]   
        
        \tikzmath{\A1 = 5; \l = 5; \k = pi/\l;}
        
        \addplot [black, domain=0:\l, samples = 1000] {0}
        node[anchor=130, pos=0] {0} 
        node[anchor=130, pos=1] {$l$};
        
        \addplot [black, domain=0:\l, samples = 1000] {\A1}
        node[anchor=south, pos=0.55, fill=white] {$X_0(x)$}; 


        \addplot [black, domain=0:\l, samples = 1000] {\A1 * cos(deg(\k*x))}
        node[anchor=south, pos=0.31, fill=white] {$X_1(x)$}; 

        \addplot [black, domain=0:\l, samples = 1000] {\A1/2 * cos(deg(2*\k*x))}
        node[anchor=south, pos=0.15, fill=white] {$X_2(x)$};
        
    \end{axis}
\end{tikzpicture} %%for compiling only this file
        \caption{Графічний розв'язок задачі, наведені три власні функції}        
    \end{figure}\\
     З рисунку бачимо, що дотичні до всіх графіків у точках $x = 0$ та $x = l$ горизонтальні, тобто крайові умови (\ref{sepvar2,1}) виконуються на обох кінцях проміжку. Далі перевіряємо, чи виконується осциляційна теорема. Власні функції занумеровані у нас у порядку зростання власних значень, а мінімальному власному значенню відповідає функція $X_0(x) = 1$. Як видно з рисунка, вона не має нулів всередині проміжку, як і має бути для основної моди; кожна наступна власна функція має рівно на один нуль більше. Тобто осциляційна теорема виконується. Звідси робимо висновок, що ми знайшли всі власні функції і власні значення задачі.  \\
   
\end{enumerate}

\begin{center}
    \large{\textbf{Аналіз результату}}
\end{center}
Моди $n=1,2,\ldots$ відповідають стоячим хвилям. Коливання стержня повздовжні, і тому показані на рисунку графіки власних функцій пов'язані з реальним рухом стержня не настульки очевидним чином, як для поперечних коливань струни. У процесі коливань певні частини стердня зміщуються поперемінно праворуч і ліворуч, а між ним виникають області розтягу і стиснення.  Області максимального відносного стиснення і розтягу (екстремуми похідної по координаті) припадають на вузли поля зміщень. Оскільки стежень з обома вільними кінцями симетричний відносно сердини, то власні функції почергово є або симетричними (парними), або антисиметричними (непарними) відносно середини проміжку (див. рисунок). Аналогічну картину ми спостерігали і у задачі №1.1, для поперечних коливань струни з обома закріпленими кінцями, яка теж є симетричною відносно середини. Проте, якщо врахувати векторний характер поля зміщень, то для поперечних коливань (струна) симетричній власній функції (див. рисунок вище) відповідає симетричне поле зміщень, а для повздовжніх коливань (стержнь) - антисеместричне! Нарисуйте самостійно векторне поле зміщень, яке відповідає моді $n = 1$, наприклад. \\
Основна мода у даній задачі одночасно є нульовою модою, оскільки вона відповідає нульовому власному значенню. Нульові моди, як правило, є особливими і відрізняються від інших. Так, у даній задачі нульова мода відповідає не коливанню, а стану спокою або рівномірного прямолінійного руху стержня як цілого, залежно від початкових умов, які у задачі на власні моди не задається. Кожній власній можі можна поставити у відповідність окремий ступінь вільності. Нульова мода відповідає рухові центра мас стержня, а інші - коливанням різних типів відносно нерухомого центра мас. 

%\end{document}
\vspace{2cm}
%\documentclass[a4paper, 14pt]{extreport}

%\usepackage{StyleMMF}

%\begin{document}

%\setcounter{chapter}{1}

%\chapter{Власні моди інших систем. Вільні коливання для заданих початкових умов.}

\textbf{\large Вільні коливання поля в резонаторі для заданих початкових умов. Ряд Фур'є по системі ортогональних функцій.}

\section[Задача №2.3]{2.3}

\textit{Знайти коливання струни завдовжки $0 \leq x \leq l$ із закріпленими кінцями, якщо початкове відхил є $\varphi(x) = hx/l$, а початкова швидкість $\psi(x) = \nu_0$. Обчислити інтеграл ортогональності власних функцій і знайти квадрат норми. Чи є рух струни періодичним (тобто чи буде повторюватись початковий стан струни через деякий проміжок часу?) Чи буде рух періодичним, якщо він описується рівнянням $u_{tt} = v^2 u_{xx} - \omega_0^2 u$}?

\begin{center}
    \textbf{Розв'язок}
\end{center}
Формальна постановка задачі:
\begin{equation} \label{probcond3}
    \left\{ \begin{aligned} %%
        &\;u = u(x,t), \\
        &\;u_{tt} = v^2 u_{xx}, \\
        &\;0 \leq x \leq l, t \geq 0, \\
        &\;u(0,t) = u(l,t) = 0,\\
        &\left.\begin{aligned}
            &u(x,0) = \varphi(x) = \frac{hx}{l}, \\ 
            &u_t(x,0) = \psi(x) = \nu_0.
        \end{aligned}\right\} \; 
        \begin{aligned}
            &\text{ початкові умови задають} \\
          - &\text{ механічний стан} \\
            &\text{ системи при } t = 0
        \end{aligned}
    \end{aligned} \right.
\end{equation}

Це задача із заданими початковими умовами, яка має єдиний розв'язок. Щоб розв’язати її, необiдно роздiлити змiннi, знайти власнi функцiї i власнi значення задачi Штурма-Лiувiлля i знайти власнi моди. Це було зроблено у  задачі №1.1 попередньго заняття (\ref{sol1}). Результатом є нескiнченний набiр власних мод: 
\begin{equation}
    \left\{ \begin{aligned} \label{fullsol}
        \;&u_n(x,t) = \left[A_n\cos(\omega_n t) + B_n\sin(\omega_n t)\right] \sin(k_n x), \\
        &k_n = \frac{\pi n}{l}, \, n = 1, 2,\ldots\\
        &\omega_n = vk_n = \frac{v \pi n}{l} - \text{ власні частоти}.
    \end{aligned}\right.
\end{equation}

Легко переконатися, що жодна окрема власна мода не може задовольнити початковi умови задачi (чому?). Щоб задовольнити початковi умови, необхiдно записати так званий загальний або формальний розв’язок задачi, який є суперпозицiєю всiх мод:
\begin{equation} \label{gensol}
    u(x,t) = \sum^{\infty}_{n=1} u_n(x,t) = \sum^{\infty}_{n=1} \left[A_n\cos(\omega_n t) + B_n\sin(\omega_n t)\right] \sin(k_n x)
\end{equation}
Поява знака суми у цьому виразi означає, що його права частина бiльше не залежить вiд $n$. Коли коефiцiєнти $A_n$ i $B_n$ будуть знайденi, вiн стане розв’язком (єдиним!) вихiдної задачi. Коефiцiєнти загального розв’язку знаходимо iз початкових умов. Підставляємо (\ref{gensol}) у початкові умови (\ref{probcond3}):
\begin{equation} \label{sol-init-pos}
    u(x,0) = \varphi(x) \;\Rightarrow\; \sum^{\infty}_{n=1} A_n\sin(k_n x) = \varphi(x)
\end{equation}

\begin{equation} \label{sol-init-vel}
    \begin{aligned}
        u_t(x,0) &= \psi(x)
        \;\Rightarrow\\
        \Rightarrow& \left.\left(\sum^{\infty}_{n=1}\left[-A_n\omega_n\sin(\omega_n t) + B_n\omega_n\cos(\omega_n t)\right] \sin(k_n x)\right)\right|_{t=0} =\\
        &= \sum^{\infty}_{n=1} B_n\omega_n\sin(k_n x) = \psi(x)
    \end{aligned}
\end{equation}
Отже, ми одержали дві умови, для визначення $A_n$ і $B_n$, вiдповiдно. Далi необхiдно скористатися ортогональнiстю власних функцiй задачi Штурма-Лiувiлля (див. Конспект лекцiй, §4). У загальному виглядi iнтеграл ортогональностi власних функцiй має вигляд:
\begin{equation} \label{orth}
    \int_0^l X_n(x) \cdot X_m(x) \,\mathrm{d}x = ||X_n||^2\delta_{n,m},
\end{equation}
де $||X_n||$ -- норма власної функції. Переконаємося, що власнi функцiї дiйсно ортогональнi, i обчислимо квадрат норми.

\begin{enumerate}[wide, labelindent=0pt]
    \item Розглянемо випадок $n=m$:
    \begin{equation*}
        \begin{aligned}
            \int_0^l X_n(x)^2 \,\mathrm{d}x = \int_0^l \sin^2(k_n x) \,\mathrm{d}x =&\\
            =\frac{1}{2k_n} \int_0^l (1 - \cos(k_n x)) \,\mathrm{d}(k_n x) =&\ \frac{1}{2}\left.\left(x - \frac{\sin(k_n x)}{k_n}\right)\right|_0^l = \frac{l}{2}
        \end{aligned}
    \end{equation*}
    \item Випадок $n \neq m$:
    \begin{equation*}
        \begin{aligned}
            \int_0^l X_n(x) \cdot X_m(x) \,\mathrm{d}x = \int_0^l \sin(k_n x)\sin(k_m x) \,\mathrm{d}x =\\
            = \frac{1}{2} \int_0^l (\cos(k_n - k_m)x - \cos(k_n + k_m)x) \,\mathrm{d}x =\\
            = \frac{1}{2}\left.\left(\frac{\sin(k_n - k_m)x}{k_n - k_m} - \frac{\sin(k_n + k_m)x}{k_n + k_m}\right)\right|_0^l = 0
        \end{aligned}
    \end{equation*}
\end{enumerate}

Щоб одержати правильні вирази для коефіцієнтів загального розв'язку, застосовуємо формальну процедуру, описану у §4 Конспекту лекцій. Доможуємо кожну з одержаних рівностей (\ref{sol-init-pos}) та (\ref{sol-init-vel}) на $m$-ту власну функції $\sin(k_m x)$ та інтегруємо від $0$ до $l$. 
\begin{subequations} \label{fourier-coefs}
    \begin{gather}
        \begin{aligned}
            &\int\limits_0^l \varphi(x) \sin(k_m x) \,\mathrm{d}x = \sum^{\infty}_{n=1} A_n \int\limits_0^l \sin(k_n x) \sin(k_m x) \,\mathrm{d}x =\\
            &= \sum^{\infty}_{n=1} A_n \cdot \frac{l}{2} \delta_{n,m} = \frac{A_m l}{2}
            \;\Rightarrow\;
            A_n = \frac{2}{l} \int\limits_0^l \varphi(x) \sin(k_n x) \,\mathrm{d}x 
        \end{aligned}\\
        \begin{aligned}
            &\int\limits_0^l \psi(x) \sin(k_m x) \,\mathrm{d}x = \sum^{\infty}_{n=1} B_n\omega_n \int\limits_0^l \sin(k_n x) \sin(k_m x) \,\mathrm{d}x =\\
            &= \sum^{\infty}_{n=1} B_n\omega_n \cdot \frac{l}{2} \delta_{n,m} = \frac{B_m \omega_m l}{2}
            \;\Rightarrow\;
            B_n = \frac{2}{\omega_n l} \int\limits_0^l \psi(x) \sin(k_n x) \,\mathrm{d}x
        \end{aligned}
    \end{gather}
\end{subequations} 
Обчислюємо отримані інтеграли (\ref{fourier-coefs}).
\begin{equation*}
    \begin{aligned}
        &A_n = \frac{2}{l} \int\limits_0^l \varphi(x) \sin(k_n x) \,\mathrm{d}x = \frac{2}{l} \int\limits_0^l \frac{hx}{l} \sin(k_n x) \,\mathrm{d}x =\\
        &= \frac{2h}{l^2} \left(\left.-\frac{1}{k_n} x \cos(k_n x)\right|_0^l + \int\limits_0^l \frac{\cos(k_n x)}{k_n} \,\mathrm{d}t\right) =\\
        &= \left| k_n l = \frac{\pi n}{l} l = \pi n \Rightarrow \sin(k_n l) = 0,\, \cos(k_n l) = (-1)^n \right| =\\
        &= \frac{2h}{l^2} \left(-\frac{l}{k_n}(-1)^n + \left.\frac{\sin(k_n x)}{k_n^2}\right|_0^l \right) = \frac{2h}{l} \frac{(-1)^{n+1}}{k_n}
    \end{aligned}
\end{equation*}
\begin{equation*}
    \begin{aligned}
        B_n =&\ \frac{2}{\omega_n l} \int\limits_0^l \psi(x) \sin(k_n x) \,\mathrm{d}x = \frac{2\nu_0}{\omega_n l} \int\limits_0^l \sin(k_n x) \,\mathrm{d}x =\\
        =&\ \left.\frac{2\nu_0}{k_n\omega_n l} \cos(k_n x)\right|_l^0 = \frac{2\nu_0}{l} \frac{1 - (-1)^n}{k_n\omega_n}
    \end{aligned}
\end{equation*}
Підставляємо визначені константи в (\ref{gensol}) і одержуємо відповідь:
\begin{equation} \label{sol3}
    u(x,t) = \frac{2}{l}\sum^{\infty}_{n=1} \left[\nu_0 (1 - (-1)^n)\frac{\sin(\omega_n t)}{\omega_n} - h(-1)^n\cos(\omega_n t)\right] \frac{\sin(k_n x)}{k_n}
\end{equation}

Процедура, за якою ми визначали константи $A_n$ та $B_n$, фактично зводиться до розкладання даних початкових умов в узагальнений ряд Фур'є по системі власних функцій задачі Штурма-Ліувілля. У даному випадку цей ряд є частинним випадком тригонометричного ряду Фур'є. 

З'ясуємо, чи є розв'язок періодичною функцією часу. Розв'язок є суперпозицією всіх мод, кожна з них має іншу частоту коливань $\omega_n$. У розглянутій задачі  частоти всіх мод кратні частоті основної моди $\omega_1$. Тому період (найменший) коливань основної моди \[T = \frac{2l}{v},\] є спільним періодом для всіх мод. Отже, рух струни буде періодичним з періодом коливань основної моди. 

Нехай тепер замість хвильового рух системи описується рівнянням \[u_{tt} = v^2 u_{xx} - \omega_0^2 u\] з тими ж межовими умовами. Легко бачити, що після розділення змінних вигляд задачі Штурма-Ліувілля не зміниться, а зміниться лише рівняння для часової частини розв'язку $T(t)$:
\begin{equation*}
    T'' + (\lambda_n*v^2 + \omega_0^2) T = 0
    \;\Rightarrow\; 
    T'' + \widetilde{\omega}_n^2 T = 0,
\end{equation*}
де $\widetilde{\omega}_n = \sqrt{\omega_n^2 + \omega_0^2}$, а $\omega_n$ - частоти коливань розв'язаної вище задачі. Тобто частоти коливань зміняться і вже не будуть цілими кратними частоти основної моди. Тобто рух такої системи не буде періодимчним.\\
Якщо рівняння міститиме доданок пропорційний $u_t$, то буде спостерігатися затухання чи підсилення коливань (залежно від знаку коефіцієнта при $u_t$).

%\end{document}

\chapter{Другий спосіб знаходження коефіцієнтів. Коливання стержня з вільними кінцями, неповнота базису.}
%\documentclass[a4paper, 14pt]{extreport}

%\usepackage{StyleMMF}

%\begin{document}

%\setcounter{chapter} {2}
%\chapter{Другий спосіб знаходження коефіцієнтів. Коливання стержня з вільними кінцями, неповнота базису.}

\section[Задача №3.1]{3.1}

\textit{Знайти коливання пружного стержня $0 \leq x \leq l$, лівий кінець якого закріплений, а правий вільний, якщо початкове відхилення $\varphi(x) = h \sin(3\pi x/2l)$, а початкова швидкість $\psi(x) = \nu_0 \sin(\pi x/2l)$.}

\begin{center}
    \textbf{Розв'язок}
\end{center}
Формальна постановка задачі:
\begin{equation} \label{cond3,1}
    \left\{ \begin{aligned} %%
        &\;u = u(x,t), \\
        &\;u_{tt} = v^2 u_{xx}, \\
        &\;0 \leq x \leq l, t \geq 0, \\
        &\;u(0,t) = 0,\\
        &\;u_x(0,t) = 0,\\
        &\left.\begin{aligned}
            &u(x,0) = \varphi(x) = h \sin \left(\frac{3 \pi x}{2 l} \right), \\ 
            &u_t(x,0) = \psi(x) = v_0 \sin \left(\frac{\pi x}{2 l}\right).
        \end{aligned}\right\} \; 
        \begin{aligned}
            &\text{ специфіка задачі} \\
          - &\text{ у вигляді } \\
            &\text{ початкових умов } 
        \end{aligned}
    \end{aligned} \right.
\end{equation}

Це задача із заданими початковими умовами (а саме - початковим розподілом зміщення та швидкостей), яка має єдиний розв'язок.

Для початку скористаємося розв'язком задачі 1.2:

\begin{equation}
    \left\{ \begin{aligned} \label{fullsol}
        \;&u_n(x,t) = \left[A_n\cos(\omega_n t) + B_n\sin(\omega_n t)\right] \sin(k_n x), \\
        &k_n = (n + \frac{1}{2})\frac{\pi}{l}, \, n = 1, 2,\ldots\\
        &\omega_n = vk_n = (n + \frac{1}{2})\frac{\pi v}{l} - \text{ власні частоти}.
    \end{aligned}\right.
\end{equation}

І запишемо загальний розв'язок:

\begin{equation} \label{gensol3,1}
    u(x,t) = \sum^{\infty}_{n=0} \left[A_n\cos(\omega_n t) + B_n\sin(\omega_n t)\right] \sin(k_n x)
\end{equation}

\begin{equation} \label{(sol3,1)_t}
    \begin{aligned}
        u_t(x,t) &= 
   \sum^{\infty}_{n=0}\left[-A_n\omega_n\sin(\omega_n t) + B_n\omega_n\cos(\omega_n t)\right] \sin(k_n x)  
    \end{aligned}
\end{equation}

Підставляємо (\ref{gensol3,1}) у початкові умови (\ref{cond3,1}):

\begin{equation}
    u(x,0) = \varphi(x) \;\Rightarrow\; \sum^{\infty}_{n=0} A_n\sin\left((n + \frac{1}{2}) \frac{\pi x}{l} \right) = h \sin \left( \frac{3 \pi x}{2l} \right)
\end{equation}


Підставляємо (\ref{(sol3,1)_t}) у початкові умови (\ref{cond3,1}):

\begin{equation}
    u_t(x,0) = \psi(x) \;\Rightarrow\; \sum^{\infty}_{n=0} B_n \omega_n \sin\left((n + \frac{1}{2}) \frac{\pi x}{l} \right) = v_0 \sin \left( \frac{ \pi x}{2l} \right)
\end{equation}

Особливі ситуації: функції у правій частині є однією з власних функцій задачі Штурма-Ліувіля. Це дозволяє знайти коефіцієнти $A_n, B_n$ простіше, порівнюючи з загальнім знаходженням з вихідних функцій $\varphi(x)$ і  $\psi(x)$ загального вигляду! Тобто брати інтеграл у цій особливій ситуації не потрібно.


Якщо 2 ряда Фур'є по одній системі функцій рівні, то і відповідні коефіцієнти цих рядів рівні.

\begin{equation}
    \begin{aligned}
        \sum_{n=0} A_n \sin \left( (n + \frac{1}{2}) \frac{\pi x}{l} \right) = A_0 \sin \left(\frac{\pi x}{2 l} \right) &+\\
        + A_1 \sin \left(\frac{3 \pi x}{2 l} \right) + A_2 \sin \left(\frac{5 \pi x}{2 l} \right) + &... = h \sin \left( \frac{3 \pi x}{2 l} \right)
    \end{aligned}
\end{equation}

Результат $A_1 = h, A_0 = A_2 = A_3 = ... = 0$

Аналогічно робимо з \ref{(sol3,1)_t}: 

\begin{equation}
    \begin{aligned}
        \sum_{n=0} \omega_n B_n \sin \left( (n + \frac{1}{2}) \frac{\pi x}{l} \right) = \omega_0 B_0 \sin \left(\frac{\pi x}{2 l} \right) &+\\
        + \omega_1 B_1 \sin \left(\frac{3 \pi x}{2 l} \right) + \omega_2 B_2 \sin \left(\frac{5 \pi x}{2 l} \right) + &... = v_0 \sin \left( \frac{\pi x}{2 l} \right)
    \end{aligned}
\end{equation}

Результат $B_0 = \frac{v_0}{\omega_0}, B_1 = B_2 = B_3 = ... = 0$

Тепер треба правильно написати відповідь через знайдені коефіцієнти $A_n, B_n$! Підставляємо знайдені коефіцієнти у загальний розв'язок (тільки два коефіцієнти - $A_1$ і $B_0$ не дорівнюють нулю, тож членів у розв'язку всего два!)

Фінальна відповідь:

\begin{equation}
    u (x,t) = h \cos (\omega_1 t) \sin (k_1 x) + \frac{v_0}{\omega_0} \sin (\omega_0 t) \sin (k_0 x)
\end{equation}

де $k_0 = \frac{\pi x}{2 l}, k_1 = \frac{3 \pi x}{2 l}, \omega_0 = v k_0, \omega_1 = v k_1 $. Можемо помітити, що у кожної моди своя частота.

Перевіряємо відповідь

\begin{itemize}
    \item Власні функції перевірені в задачі 1.2
    \item Постановка задачі містить два неоднорідних члени у початкових умовах. Один пропорційний $\sim h$, інший пропорційний $\sim v_0$. Перевірити наявність цих множників у загальному розв'язку.
    \item Перевіряємо початкові умови - виконуються?

\end{itemize}

Альтернативний шлях -- знайти за означенням коефіцієнти розкладу у ряд Фур'є.

\begin{equation}
A_n = \frac{2}{l} \int_{0}^{l} \varphi (x)  \sin \left( (\frac{1}{2} + n) \frac{\pi x}{l} \right) dx    
\end{equation}


Одержали інтеграл ортогональності 

\begin{equation}
    \int^{l}_0 \sin \left( \frac{3 \pi x}{2 l} \right) \sin \left( (\frac{1}{2} + n) \frac{\pi x}{l} \right) dx = \int^{l}_0 \chi_1 (x) \chi_n (x) dx = \delta_{1n}
\end{equation}

Якщо ви не побачите що інтеграл є інтегралом ортогональності, і будете його обчилювати, то втратите час і можете помилитися і одержати неправильну відповідь (що часто і буває).

Результат $A_1 = h, A_0 = A_2 = A_3 = ... = 0$ та для швидкостей $B_0 = \frac{v_0}{\omega_0}, B_1 = B_2 = B_3 = ... = 0$

Отримали теж саме, але складнішим шляхом!

%\end{document}
%\documentclass[a4paper, 14pt]{extreport}

%\usepackage{StyleMMF}

%\begin{document}

%\chapter{Другий спосіб знаходження коефіцієнтів. Коливання стержня з вільними кінцями, неповнота базису.}

\section[Задача №3.3]{3.3}

\textit{Знайти коливання пружного стержня довжиною $l$ з вільними кінцями, якщо початкове відхилення дорівнює нулю, а початкова швидкість $\psi(x) = \nu_0$. Якщо всі знайдені вами коефіцієнти Фур'є (коефіцієнти загального\\ розв’язку) дорівнюють нулю, поясніть, що це означає, і знайдіть, де була допущена помилка.}

%\end{document}

\chapter{Рівняння теплопровідності з однорідними межовими умовами}
%\documentclass[a4paper, 14pt]{extreport}

%\usepackage{../StyleMMF}

%\begin{document}

%\setcounter{chapter}{3}
%\chapter{Рівняння теплопровідності з однорідними межовими умовами}

\section[Задача №4.1]{4.1}

\textit{Одну і ту ж функцію, наприклад $f(x) = \alpha x$, можна представити на проміжку $0 \leq x \leq l$ узагальненим рядом Фур’є по кожній із систем власних функцій чотирьох задач Штурма-Ліувілля, одержаних у задачах 1.1, 1.2, 1.3, 2.1. Користуючись явним виглядом власних функцій і не обчислюючи коєфіцієнтів рядів, дайте відповіді на такі запитання.
\begin{enumerate}
    \item Який вигляд матиме графік суми кожного з таких рядів на всій числовій осі? Якою є парність суми ряду відносно точок $x = nl$, де $n$ – ціле число, і як це пов’язано з виглядом крайових умов задачі Штурма-Ліувілля?
    \item Покажіть, що кожний з рядів є частинним випадком класичного тригонометричного ряду Фур’є, сума якого є періодичною функцією. Які саме періоди відповідають кожному з рядів? Яка саме частина повного тригонометричного базису використовується в кожному з розкладань, а які коефіцієнти Фур’є дорівнюють нулю і чому?
    \item Як пов’язаний характер збіжності вказаних рядів з крайовими умовами, які задовольняє функція $f(x)$ у точках $x = 0, l$ ? Чи дорівнює сума ряду Фур’є функції $f(x)$ на відкритому проміжку $0 < x < l$? на закритому проміжку
    $0 \leq x \leq l$
\end{enumerate}}

\begin{center}
    \large \textbf{Розв'язок}
\end{center}

\subsection*{Запитання №1} 
Графік суми ряду буде періодично повторювати розкладену в ряд Фур'є функцію з періодом $l$. Відносно точок $x = nl$ сума ряду буде або парною, або непарною, функцією. Для задачі, де кінці струни закріплені, відносто всіх цих точок буде непарною. Це випливає із умов при яких можливий розклад в ряд Фур'є -- кусково-гладкість та періодичність функції, тобто щоб задовольняти всім цим умовам і отримували в точках $x = nl$ значення функції рівним нулю, необхідно щоб функція була непарна відносно ціє точки. Аналогічна, ситуація для межової умови, де кінець струни вільний ($X'(nl) = 0$), тільки треба згадати, що похідна змінює парність функції (звісно якщо функція має первну парність), тому відносно вільних кінців сума ряду буде парною функцією.     

\subsection*{Запитання №2}


\subsection*{Запитання №3} 
Графіки суми рядів Фур'є буде періодичною функцією, що на відкритому проміжку $(0,l)$ співпадає з нашою функцією, а на в точках може бути розрив першого роду.


%% малюнки треба якось розподілити по тексту задачі 
\begin{figure} \label{fourier1}
    %\begin{tikzpicture}
    \begin{axis}
        [width = \textwidth, height = 0.7\textwidth,
         axis x line = center, axis y line = center,
         ylabel = $X(x)$, xlabel = $x$,
         xmin = -19, xmax = 19, ymin = -3, ymax = 3,
         axis line style = thin, xtick = {0}, ytick = {0}]   
        
        \tikzmath{\l = 5; \alp = 2/5;}
        
        \addplot[gray, dashed, samples=50, domain=-20:20, name path=three] coordinates {(0,-3)(0,3)}
        node[anchor=130, pos=0.5] {\footnotesize\textcolor{black}{0}};

        % вертикальні пунктирні лінії для x > 0
        \addplot[gray, dashed, samples=50, domain=-20:20, name path=three] coordinates {(\l,-3)(\l,3)}
        node[anchor=130, pos=0.5] {\footnotesize$\textcolor{black}{l}$};
        \addplot[gray, dashed, samples=50, domain=-20:20, name path=three] coordinates {(2*\l,-3)(2*\l,3)}
        node[anchor=130, pos=0.5] {\footnotesize$\textcolor{black}{2l}$};
        \addplot[gray, dashed, samples=50, domain=-20:20, name path=three] coordinates {(3*\l,-3)(3*\l,3)}
        node[anchor=130, pos=0.5] {\footnotesize$\textcolor{black}{3l}$};
        
        % вертикальні пунктирні лінії для x < 0
        \addplot[gray, dashed, samples=50, domain=-20:20, name path=three] coordinates {(-\l,-3)(-\l,3)}
        node[anchor=100, pos=0.5] {\footnotesize$\textcolor{black}{-l}$};
        \addplot[gray, dashed, samples=50, domain=-20:20, name path=three] coordinates {(-2*\l,-3)(-2*\l,3)}
        node[anchor=100, pos=0.5] {\footnotesize$\textcolor{black}{-2l}$};
        \addplot[gray, dashed, samples=50, domain=-20:20, name path=three] coordinates {(-3*\l,-3)(-3*\l,3)}
        node[anchor=100, pos=0.5] {\footnotesize$\textcolor{black}{-3l}$};
        
        % функція, яку розкладаємо
        \addplot [red, thick, domain=0:0.99*\l, samples=150] {\alp*x};
        % точки закріплення струни
        \node (mark) [draw, red, fill=red, circle, minimum size = 2pt, inner sep=0.5pt] at (axis cs: 0, 0) {};
        \node (mark) [draw, red, fill=red, circle, minimum size = 2pt, inner sep=0.5pt] at (axis cs: \l, 0) {};
        % виколата точка (l,αl)
        \node (mark) [draw, red, circle, minimum size = 2pt, inner sep=0.5pt] at (axis cs: \l, \alp*\l) {};

        % непарне продовження відносно нуля
        \addplot [red, dashed, thick, domain=-0.99*\l:0, samples=150] {\alp*x};
        % виколата точка (-l,-αl)
        \node (mark) [draw, red, circle, minimum size = 2pt, inner sep=0.5pt] at (axis cs: -\l, -\alp*\l) {};
        % ізольована точка 
        \node (mark) [draw, black, fill=black, circle, minimum size = 2pt, inner sep=0.5pt] at (axis cs: -\l, 0) {};

        % непарне продовження відносно точки x = l 
        \addplot [black, thick, dashed, domain=1.01*\l:3*0.998*\l, samples=150] {\alp*(x - 2*\l)};
        % виколоті точки
        \node (mark) [draw, black, circle, minimum size = 2pt, inner sep=0.5pt] at (axis cs: \l, -\alp*\l) {};
        \node (mark) [draw, black, circle, minimum size = 2pt, inner sep=0.5pt] at (axis cs: 3*\l, \alp*\l) {};
        % ізольована точка
        \node (mark) [draw, black, fill=black, circle, minimum size = 2pt, inner sep=0.5pt] at (axis cs: 3*\l, 0) {};
        
        % непарне продовження відносно точки x = -l 
        \addplot [black, thick, dashed, domain=-3*0.998*\l:-0.99*\l, samples=150] {\alp*(x + 2*\l)};
        \node (mark) [draw, black, circle, minimum size = 2pt, inner sep=0.5pt] at (axis cs: -3*\l, -\alp*\l) {};
        \node (mark) [draw, black, fill=black, circle, minimum size = 2pt, inner sep=0.5pt] at (axis cs: -3*\l, 0) {};
        \node (mark) [draw, black, circle, minimum size = 2pt, inner sep=0.5pt] at (axis cs: -\l, \alp*\l) {};
        
        \tikzmath{\k = pi/\l; \A = 2*\alp/\k;}

        \addplot [black, domain=-19:19, samples = 1500] {\A*sin(deg(\k*x))};

    \end{axis}
\end{tikzpicture} %% this file compilation
    \begin{tikzpicture}
    \begin{axis}
        [width = \textwidth, height = 0.7\textwidth,
         axis x line = center, axis y line = center,
         ylabel = $X(x)$, xlabel = $x$,
         xmin = -19, xmax = 19, ymin = -3, ymax = 3,
         axis line style = thin, xtick = {0}, ytick = {0}]   
        
        \tikzmath{\l = 5; \alp = 2/5;}
        
        \addplot[gray, dashed, samples=50, domain=-20:20, name path=three] coordinates {(0,-3)(0,3)}
        node[anchor=130, pos=0.5] {\footnotesize\textcolor{black}{0}};

        % вертикальні пунктирні лінії для x > 0
        \addplot[gray, dashed, samples=50, domain=-20:20, name path=three] coordinates {(\l,-3)(\l,3)}
        node[anchor=130, pos=0.5] {\footnotesize$\textcolor{black}{l}$};
        \addplot[gray, dashed, samples=50, domain=-20:20, name path=three] coordinates {(2*\l,-3)(2*\l,3)}
        node[anchor=130, pos=0.5] {\footnotesize$\textcolor{black}{2l}$};
        \addplot[gray, dashed, samples=50, domain=-20:20, name path=three] coordinates {(3*\l,-3)(3*\l,3)}
        node[anchor=130, pos=0.5] {\footnotesize$\textcolor{black}{3l}$};
        
        % вертикальні пунктирні лінії для x < 0
        \addplot[gray, dashed, samples=50, domain=-20:20, name path=three] coordinates {(-\l,-3)(-\l,3)}
        node[anchor=100, pos=0.5] {\footnotesize$\textcolor{black}{-l}$};
        \addplot[gray, dashed, samples=50, domain=-20:20, name path=three] coordinates {(-2*\l,-3)(-2*\l,3)}
        node[anchor=100, pos=0.5] {\footnotesize$\textcolor{black}{-2l}$};
        \addplot[gray, dashed, samples=50, domain=-20:20, name path=three] coordinates {(-3*\l,-3)(-3*\l,3)}
        node[anchor=100, pos=0.5] {\footnotesize$\textcolor{black}{-3l}$};
        
        % функція, яку розкладаємо
        \addplot [red, thick, domain=0:0.99*\l, samples=150] {\alp*x};
        % точки закріплення струни
        \node (mark) [draw, red, fill=red, circle, minimum size = 2pt, inner sep=0.5pt] at (axis cs: 0, 0) {};
        \node (mark) [draw, red, fill=red, circle, minimum size = 2pt, inner sep=0.5pt] at (axis cs: \l, 0) {};
        % виколата точка (l,αl)
        \node (mark) [draw, red, circle, minimum size = 2pt, inner sep=0.5pt] at (axis cs: \l, \alp*\l) {};

        % непарне продовження відносно нуля
        \addplot [red, dashed, thick, domain=-0.99*\l:0, samples=150] {\alp*x};
        % виколата точка (-l,-αl)
        \node (mark) [draw, red, circle, minimum size = 2pt, inner sep=0.5pt] at (axis cs: -\l, -\alp*\l) {};
        % ізольована точка 
        \node (mark) [draw, black, fill=black, circle, minimum size = 2pt, inner sep=0.5pt] at (axis cs: -\l, 0) {};

        % непарне продовження відносно точки x = l 
        \addplot [black, thick, dashed, domain=1.01*\l:3*0.998*\l, samples=150] {\alp*(x - 2*\l)};
        % виколоті точки
        \node (mark) [draw, black, circle, minimum size = 2pt, inner sep=0.5pt] at (axis cs: \l, -\alp*\l) {};
        \node (mark) [draw, black, circle, minimum size = 2pt, inner sep=0.5pt] at (axis cs: 3*\l, \alp*\l) {};
        % ізольована точка
        \node (mark) [draw, black, fill=black, circle, minimum size = 2pt, inner sep=0.5pt] at (axis cs: 3*\l, 0) {};
        
        % непарне продовження відносно точки x = -l 
        \addplot [black, thick, dashed, domain=-3*0.998*\l:-0.99*\l, samples=150] {\alp*(x + 2*\l)};
        \node (mark) [draw, black, circle, minimum size = 2pt, inner sep=0.5pt] at (axis cs: -3*\l, -\alp*\l) {};
        \node (mark) [draw, black, fill=black, circle, minimum size = 2pt, inner sep=0.5pt] at (axis cs: -3*\l, 0) {};
        \node (mark) [draw, black, circle, minimum size = 2pt, inner sep=0.5pt] at (axis cs: -\l, \alp*\l) {};
        
        \tikzmath{\k = pi/\l; \A = 2*\alp/\k;}

        \addplot [black, domain=-19:19, samples = 1500] {\A*sin(deg(\k*x))};

    \end{axis}
\end{tikzpicture} %% main compilation
    \caption{Аналітичне продовження функції $\alpha x$ для розкладу системі власних функцій із задачі 1.1}
\end{figure}

Червоним позначена функція, яку ми розкладуємо в ряд Фур'є. Пунктина лінія, знову червона, показує відповідне симетрійне продовження відносно нуля. Чорною пунктирною лінією позначено симетрійні відображення відносно інших точкок.

\begin{figure} \label{fourier2}
    %\begin{tikzpicture}
    \begin{axis}
        [width = \textwidth, height = 0.7\textwidth,
         axis x line = center, axis y line = center,
         ylabel = $X(x)$, xlabel = $x$,
         xmin = -19, xmax = 19, ymin = -3, ymax = 3,
         axis line style = thin, xtick = {0}, ytick = {0}]   
        
        \tikzmath{\l = 5; \alp = 2/5;}
        
        \addplot[gray, dashed, samples=50, domain=-20:20, name path=three] coordinates {(0,-3)(0,3)}
        node[anchor=130, pos=0.5] {\footnotesize\textcolor{black}{0}};

        % вертикальні пунктирні лінії для x > 0
        \addplot[gray, dashed, samples=50, domain=-20:20, name path=three] coordinates {(\l,-3)(\l,3)}
        node[anchor=130, pos=0.5] {\footnotesize$\textcolor{black}{l}$};
        \addplot[gray, dashed, samples=50, domain=-20:20, name path=three] coordinates {(2*\l,-3)(2*\l,3)}
        node[anchor=130, pos=0.5] {\footnotesize$\textcolor{black}{2l}$};
        \addplot[gray, dashed, samples=50, domain=-20:20, name path=three] coordinates {(3*\l,-3)(3*\l,3)}
        node[anchor=130, pos=0.5] {\footnotesize$\textcolor{black}{3l}$};
        
        % вертикальні пунктирні лінії для x < 0
        \addplot[gray, dashed, samples=50, domain=-20:20, name path=three] coordinates {(-\l,-3)(-\l,3)}
        node[anchor=100, pos=0.5] {\footnotesize$\textcolor{black}{-l}$};
        \addplot[gray, dashed, samples=50, domain=-20:20, name path=three] coordinates {(-2*\l,-3)(-2*\l,3)}
        node[anchor=100, pos=0.5] {\footnotesize$\textcolor{black}{-2l}$};
        \addplot[gray, dashed, samples=50, domain=-20:20, name path=three] coordinates {(-3*\l,-3)(-3*\l,3)}
        node[anchor=100, pos=0.5] {\footnotesize$\textcolor{black}{-3l}$};
        
        % функція, яку розкладаємо
        \addplot [red, thick, domain=0:0.99*\l, samples=150] {\alp*x};
        
        % парне продовження відносно нуля
        \addplot [red, dashed, thick, domain=-0.99*\l:0, samples=150] {-\alp*x};
        
        % парне продовження відносно точки x = l 
        \addplot [black, thick, dashed, domain=\l:2*\l, samples=150] {-\alp*(x - 2*\l)};
        \addplot [black, thick, dashed, domain=2*\l:3*\l, samples=150] {\alp*(x - 2*\l)};
        
        \node (mark) [draw, black, fill=black, circle, minimum size = 2pt, inner sep=0.5pt] at (axis cs: 3*\l, 0) {};
        
        % парне продовження відносно точки x = -l 
        \addplot [black, thick, dashed, domain=-2*\l:-1*\l, samples=150] {\alp*(x + 2*\l)};
        \addplot [black, thick, dashed, domain=-3*\l:-2*\l, samples=150] {-\alp*(x + 2*\l)};
        
        \tikzmath{\k = pi/\l; \A = 4*\alp*\l/pi^2;}

        \addplot [black, domain=-19:19, samples = 1000] {\alp*\l/2};
        \addplot [black, domain=-19:19, samples = 1500] {\A*cos(deg(\k*x))};
        
    \end{axis}
\end{tikzpicture}
 %% this file compilation
    \begin{tikzpicture}
    \begin{axis}
        [width = \textwidth, height = 0.7\textwidth,
         axis x line = center, axis y line = center,
         ylabel = $X(x)$, xlabel = $x$,
         xmin = -19, xmax = 19, ymin = -3, ymax = 3,
         axis line style = thin, xtick = {0}, ytick = {0}]   
        
        \tikzmath{\l = 5; \alp = 2/5;}
        
        \addplot[gray, dashed, samples=50, domain=-20:20, name path=three] coordinates {(0,-3)(0,3)}
        node[anchor=130, pos=0.5] {\footnotesize\textcolor{black}{0}};

        % вертикальні пунктирні лінії для x > 0
        \addplot[gray, dashed, samples=50, domain=-20:20, name path=three] coordinates {(\l,-3)(\l,3)}
        node[anchor=130, pos=0.5] {\footnotesize$\textcolor{black}{l}$};
        \addplot[gray, dashed, samples=50, domain=-20:20, name path=three] coordinates {(2*\l,-3)(2*\l,3)}
        node[anchor=130, pos=0.5] {\footnotesize$\textcolor{black}{2l}$};
        \addplot[gray, dashed, samples=50, domain=-20:20, name path=three] coordinates {(3*\l,-3)(3*\l,3)}
        node[anchor=130, pos=0.5] {\footnotesize$\textcolor{black}{3l}$};
        
        % вертикальні пунктирні лінії для x < 0
        \addplot[gray, dashed, samples=50, domain=-20:20, name path=three] coordinates {(-\l,-3)(-\l,3)}
        node[anchor=100, pos=0.5] {\footnotesize$\textcolor{black}{-l}$};
        \addplot[gray, dashed, samples=50, domain=-20:20, name path=three] coordinates {(-2*\l,-3)(-2*\l,3)}
        node[anchor=100, pos=0.5] {\footnotesize$\textcolor{black}{-2l}$};
        \addplot[gray, dashed, samples=50, domain=-20:20, name path=three] coordinates {(-3*\l,-3)(-3*\l,3)}
        node[anchor=100, pos=0.5] {\footnotesize$\textcolor{black}{-3l}$};
        
        % функція, яку розкладаємо
        \addplot [red, thick, domain=0:0.99*\l, samples=150] {\alp*x};
        
        % парне продовження відносно нуля
        \addplot [red, dashed, thick, domain=-0.99*\l:0, samples=150] {-\alp*x};
        
        % парне продовження відносно точки x = l 
        \addplot [black, thick, dashed, domain=\l:2*\l, samples=150] {-\alp*(x - 2*\l)};
        \addplot [black, thick, dashed, domain=2*\l:3*\l, samples=150] {\alp*(x - 2*\l)};
        
        \node (mark) [draw, black, fill=black, circle, minimum size = 2pt, inner sep=0.5pt] at (axis cs: 3*\l, 0) {};
        
        % парне продовження відносно точки x = -l 
        \addplot [black, thick, dashed, domain=-2*\l:-1*\l, samples=150] {\alp*(x + 2*\l)};
        \addplot [black, thick, dashed, domain=-3*\l:-2*\l, samples=150] {-\alp*(x + 2*\l)};
        
        \tikzmath{\k = pi/\l; \A = 4*\alp*\l/pi^2;}

        \addplot [black, domain=-19:19, samples = 1000] {\alp*\l/2};
        \addplot [black, domain=-19:19, samples = 1500] {\A*cos(deg(\k*x))};
        
    \end{axis}
\end{tikzpicture}
 %% main compilation
    \caption{Аналітичне продовження функції $\alpha x$ для розкладу системі власних функцій із задачі 2.1}
\end{figure}

%\end{document}
\vspace{2cm}
%\documentclass[a4paper, 14pt]{extreport}

%\usepackage{StyleMMF}

%\begin{document}

%\setcounter{chapter}{3}
%\chapter{Рівняння теплопровідності з однорідними межовими умовами}

\clearpage

\section[Задача №4.2]{4.2}

\textit{У початковий момент часу ліва половина стержня з теплоізольованою бічною поверхнею має температуру $T_1$ , а права -- температуру $T_2$ . Знайти розподіл температури при $t> 0$, якщо кінці стержня підтримуються при температурі $T_0$. Указівка: подумайте, що означає «температура дорівнює нулю», що це за нуль? Покладіть у кінцевому результаті $T_0 = 0$ і розгляньте частинні випадки: $T_1 = T_2$ та $T_1 = -T_2$. Які члени ряду при цьому обертаються в нуль? Чому? Нарисуйте графіки та порівняйте часову залежність температури для     різних мод. Нарисуйте (якісно) графіки розподілу     температури вдовж стержня у різні характерні послідовні моменти часу. Що таке «малий» і «великий» проміжок часу для цієї задачі? Як характерні часи залежать від розмірів системи?}

\begin{center}
    \textbf{Розв'язок}
\end{center}

\begin{wrapfigure}{r}{0.31\textwidth}
    \centering
    %Graph under comment
    \begin{tikzpicture}
    \begin{axis}
        [width = 0.34\textwidth, height = 0.32\textwidth,
         axis x line = center, axis y line = center,
         ylabel = $T(x)$, xlabel = $x$,
         xmin = -0.3, xmax = 3.6, ymin = -0.5, ymax = 3.7,
         axis line style = thin, xtick = {0}, ytick = {0},
         label style={font=\footnotesize}]   
        
        \tikzmath{\T2 = 2.5; \T1 = 2*\T2/3; \T0 = \T2/3; \l = 3;}
        
        \addplot [black, domain=0:\l, samples = 1000] {0}
        node[anchor=130, pos=0] {\footnotesize$0$} 
        node[anchor=90, pos=0.5] {\footnotesize$l/2$} 
        node[anchor=130, pos=1] {\footnotesize$l$};
        
        \addplot [black, domain=0:0.05, samples = 10] {\T2};
        % вертикальні пунктирні лінії для x > 0
        \addplot[black, dashed, samples=50, domain=-0.2:3.3, name path=three] coordinates {(0,0)(0,\T2)}
        node[anchor=130, pos=1/3] {\footnotesize$T_0$}
        node[anchor=130, pos=2/3] {\footnotesize$T_1$}
        node[anchor=130, pos=1] {\footnotesize$T_2$};
        \addplot[black, dashed, samples=50, domain=-0.2:3.3, name path=three] coordinates {(\l,0)(\l,\T2)};
        \addplot[black, dashed, samples=50, domain=-0.2:3.3, name path=three] coordinates {(\l/2,0)(\l/2,\T2)};


        \addplot [black, domain=0.02:\l/2, samples = 300] {\T1};
        \addplot [black, domain=\l/2-0.02:\l-0.02, samples = 300] {\T2};


        \node (mark) [draw, black, fill=black, circle, minimum size = 2pt, inner sep=0.5pt] at (axis cs: 0, \T0) {};
        \node (mark) [draw, black, circle, minimum size = 2pt, inner sep=0.5pt] at (axis cs: 0, \T0) {};
        \node (mark) [draw, black, circle, minimum size = 2pt, inner sep=0.5pt] at (axis cs: 0, \T1) {};
        \node (mark) [draw, black, fill=black, circle, minimum size = 2pt, inner sep=0.5pt] at (axis cs: \l/2, \T1) {};
        \node (mark) [draw, black, circle, minimum size = 2pt, inner sep=0.5pt] at (axis cs: \l/2, \T2) {};
        \node (mark) [draw, black, circle, minimum size = 2pt, inner sep=0.5pt] at (axis cs: \l, \T2) {};
        \node (mark) [draw, black, fill=black, circle, minimum size = 2pt, inner sep=0.5pt] at (axis cs: \l, \T0) {};
    \end{axis}
\end{tikzpicture}  % main compilation
    %\begin{tikzpicture}
    \begin{axis}
        [width = 0.34\textwidth, height = 0.32\textwidth,
         axis x line = center, axis y line = center,
         ylabel = $T(x)$, xlabel = $x$,
         xmin = -0.3, xmax = 3.6, ymin = -0.5, ymax = 3.7,
         axis line style = thin, xtick = {0}, ytick = {0},
         label style={font=\footnotesize}]   
        
        \tikzmath{\T2 = 2.5; \T1 = 2*\T2/3; \T0 = \T2/3; \l = 3;}
        
        \addplot [black, domain=0:\l, samples = 1000] {0}
        node[anchor=130, pos=0] {\footnotesize$0$} 
        node[anchor=90, pos=0.5] {\footnotesize$l/2$} 
        node[anchor=130, pos=1] {\footnotesize$l$};
        
        \addplot [black, domain=0:0.05, samples = 10] {\T2};
        % вертикальні пунктирні лінії для x > 0
        \addplot[black, dashed, samples=50, domain=-0.2:3.3, name path=three] coordinates {(0,0)(0,\T2)}
        node[anchor=130, pos=1/3] {\footnotesize$T_0$}
        node[anchor=130, pos=2/3] {\footnotesize$T_1$}
        node[anchor=130, pos=1] {\footnotesize$T_2$};
        \addplot[black, dashed, samples=50, domain=-0.2:3.3, name path=three] coordinates {(\l,0)(\l,\T2)};
        \addplot[black, dashed, samples=50, domain=-0.2:3.3, name path=three] coordinates {(\l/2,0)(\l/2,\T2)};


        \addplot [black, domain=0.02:\l/2, samples = 300] {\T1};
        \addplot [black, domain=\l/2-0.02:\l-0.02, samples = 300] {\T2};


        \node (mark) [draw, black, fill=black, circle, minimum size = 2pt, inner sep=0.5pt] at (axis cs: 0, \T0) {};
        \node (mark) [draw, black, circle, minimum size = 2pt, inner sep=0.5pt] at (axis cs: 0, \T0) {};
        \node (mark) [draw, black, circle, minimum size = 2pt, inner sep=0.5pt] at (axis cs: 0, \T1) {};
        \node (mark) [draw, black, fill=black, circle, minimum size = 2pt, inner sep=0.5pt] at (axis cs: \l/2, \T1) {};
        \node (mark) [draw, black, circle, minimum size = 2pt, inner sep=0.5pt] at (axis cs: \l/2, \T2) {};
        \node (mark) [draw, black, circle, minimum size = 2pt, inner sep=0.5pt] at (axis cs: \l, \T2) {};
        \node (mark) [draw, black, fill=black, circle, minimum size = 2pt, inner sep=0.5pt] at (axis cs: \l, \T0) {};
    \end{axis}
\end{tikzpicture}  % this compilation
\end{wrapfigure}


Формальна постановка задачі:
\begin{equation} \label{cond4,2}
    \left\{ \begin{aligned}
        &\;u = u(x,t), \\
        &\;u_{t} = D u_{xx}, \\
        &\;0 \leq x \leq l, t \geq 0, \\
        &\;u(0,t) = u(l,t) = T_0,\\
        &u(x,0) = \varphi(x) = T_1 - (T_1 - T_2)\Theta(x - l/2).
    \end{aligned} \right.
\end{equation}
\vspace{1cm}

Тут використана тета-функція Хевісайта (або функція сходинки). Вона задається таким чином:
\begin{equation*}
    \Theta(x - x_0) = 
    \begin{cases}
        \;0, \text{ при } x < x_0\\
        \;1, \text{ при } x > x_0
    \end{cases}    
\end{equation*}
Зробимо заміну, яка приведе до однорідних крайових умовами\[u(x,t) = \widetilde{u}(x,t) + T_0 \] і тепер можемо скористатися розділенням змінних.
\begin{equation} \label{subst4,2}
    \widetilde{u}(x,t) = \widetilde{X}(x)\cdot\widetilde{T}(t) 
\end{equation}
Рівняння для нової функції не змінить свого виду, тому процедура відокремлення змінних виконується аналогічно до проведених раніше. Результат відокремлення змінних:
\begin{equation} 
    \left\{ \begin{aligned}
        \;&\widetilde{X} = \widetilde{X}(x), \\  &\widetilde{X}'' = -\lambda \widetilde{X}, \\ &0 \leq x \leq l, \\  &\widetilde{X}(0) = 0, \\ &\widetilde{X}(l) = 0. 
    \end{aligned} \right.
    \qquad\qquad
    \widetilde{T}' + \lambda D \widetilde{T} = 0
\end{equation}
Розв'язок цієї задачі Штурма-Ліувілля (\ref{ShLsol1,1}) отриманий в задачі №1,1.
\begin{equation} 
    \left\{ \begin{aligned}
        \;&\lambda_n = \frac{\pi^2 n^2}{l^2},\\ 
        &\widetilde{X}_n(x) = \sin\left(\frac{\pi n x}{l}\right),
    \end{aligned} \right.
    \quad \text{де } n \in \mathbb{N}.
\end{equation}

У часовому рівнянні змінні розділяються, тому його легко проінтегруємо:
\begin{equation} \label{time-eq4,2}
    \begin{aligned}
        \frac{\widetilde{T}'}{\widetilde{T}} = -\lambda_n D = -\tau_n^{-1}
        \;&\Rightarrow\;
        \int \frac{\mathrm{d} \widetilde{T}}{\widetilde{T}} = - \int \frac{\mathrm{d} t}{\tau_n}
        \;\Rightarrow\\
        \Rightarrow\;
        \ln{\widetilde{T}_n} = \ln{C_n} - t/\tau_n
        \;&\Rightarrow\;
        \widetilde{T}_n = C_n e^{-t/\tau_n}, \text{ де } n \in \mathbb{N}
    \end{aligned}
\end{equation}

Отже, отримаємо остаточний розв'язок
\begin{equation}
    \begin{aligned}
        \;&u_n(x,t) = T_0 + \widetilde{X}_n \cdot \widetilde{T}_n = T_0 + C_n e^{-t/\tau_n} \sin(k_n x),\\
        &k_n = \frac{\pi n}{l} - \text{ хвильові вектори}, \\
        &\tau_n = \frac{1}{D k_n^2} = \frac{l^2}{D \pi^2 n^2} - \text{ характерний час зміни температури}, \\
        &n = 1, 2,\ldots
    \end{aligned}
\end{equation}

Запишемо загальний розв'язок задачі
\begin{equation}
    u(x,n) = T_0 + \sum_{n=1}^{\infty}C_n e^{-t/\tau_n} \sin(k_n x)
\end{equation}

Із початкової умови (\ref{cond4,2}) визначимо коефіцієнти $C_n$:
\begin{equation}
    u(x,0) = T_0 + \sum_{n=1}^{\infty}C_n \sin(k_n x) =  T_1 - (T_1 - T_2)\Theta(x - l/2)
\end{equation}

Розкладемо за синусами модифіковану початкову умову $u(x,0) - T_0$ 
\begin{equation*}
    \begin{gathered}
        T_1 - T_0 - (T_1 - T_2)\Theta(x - l/2) = - \sum_{n=1}^{\infty}C_n \sin(k_n x)\\
        C_n = -\frac{2}{l}\left[\int\limits_0^{l/2} (T_1 - T_0)\sin(k_n x) \;\mathrm{d}x + \int\limits_{l/2}^l (T_2 - T_0)\sin(k_n x) \;\mathrm{d}x\right] =\\
        = \frac{2}{lk_n}\left[(T_1 - T_0)\cos(k_n x)\biggr\rvert_0^{l/2} + (T_2 - T_0)\cos(k_n x)\biggr\rvert_0^{l/2}\right] =\\
        = \frac{2}{lk_n}\left[(T_1 - T_0)(1 - \cos(k_n l/2)) + (T_2 - T_0)(\cos(k_n l/2) - (-1)^n)\right] =\\
        = \frac{2}{lk_n}\left[T_1(1 - \cos(k_n l/2)) + T_2(\cos(k_n l/2) - (-1)^n) + T_0 ((-1)^n - 1)\right]
    \end{gathered}
\end{equation*}
Випишемо декілька перших множників $C_n$ для визначення поведінки їх множини
\begin{equation*}
    \begin{aligned}
        &n=1: \left[T_1 + T_2 - 2T_0 \right] &\Rightarrow&\ C_{4m+1} = \frac{2}{k_{4m+1}l}\left[T_1 + T_2 - 2T_0 \right]\\
        &n=2: \left[T_1 - T_2 \right] &\Rightarrow&\ C_{4m+2} = \frac{2}{k_{4m+2}l}\left[T_1 - T_2 \right]\\
        &n=3: \left[T_1 + T_2 - 2T_0 \right] &\Rightarrow&\ C_{4m+3} = \frac{2}{k_{4m+3}l}\left[T_1 + T_2 - 2T_0 \right]\\
        &n=4: 0 &\Rightarrow&\ C_{4m} = 0
    \end{aligned}    
\end{equation*}
Покладемо тут $T_0 = 0$ та $T_1 = T_2$ з чого видно, що коефіцієнти з парними індексами зануляються. Для випадку $T_1 = -T_2$ навпаки -- з непарними.

Отже, розв'язком буде
\begin{equation}
    \begin{aligned}
        u(x,t) = T_0 + \frac{2}{l}\sum_{n=1}^{\infty}\big[&T_1(1 - \cos(k_n l/2)) +\\
        + T_2(\cos(k_n l/2) - (-1)^n) + &T_0 ((-1)^n - 1)\big] e^{-t/\tau_n} \frac{\sin(k_n x)}{k_n}
    \end{aligned}
\end{equation}

\begin{center}
    \textbf{Графіки розв'язків}
\end{center}

Поведінку знайдених груп доданків можна побачити на наступному графіку. Тут наведено по одному з кожної групи. %% графік не хоче підтягуватись вгоду сторінки, бо на новій сторінці немає тексту, а на стару він не вміщається 
\begin{figure}[h]
    \centering
    %\large \textbf{Graph under comment}%
    \begin{tikzpicture}
    \begin{axis}
        [width = 0.95\textwidth, height = 0.6\textwidth,
         axis x line = center, axis y line = center,
         ylabel = $u(x_0; t)$, xlabel = $t$,
         xmin = -0.3, xmax = 5.7, ymin = -1.5, ymax = 3.8,
         axis line style = thin, xtick = {0}, ytick = {0}]   


        \tikzmath{\T2 = 3.5; \T1 = 4*\T2/5; \T0 = \T2/2; \l = 5.5; \t = (\l/pi)^2;}

        \addplot [black, domain=0:\l, samples = 300] {(\T1 + \T2 - 2*\T0) * exp(-x/\t)}
        node[anchor=50, pos=0] {$C_1$}
        node[anchor=south, pos=1/5] {$u_1$};

        \addplot [red, domain=0:\l, samples = 300] {(\T1 - \T2) * exp(-x*2^2/\t)}
        node[anchor=50, pos=0] {$C_2$}
        node[anchor=130, pos=1/4] {$u_2$};

        \addplot [green, domain=0:\l, samples = 300] {0}
        node[anchor=60, pos=0] {$\textcolor{black}{0}$}
        node[anchor=south, pos=1/3] {$u_4$};

        \addplot [blue, domain=0:\l, samples = 300] {\T0}
        node[anchor=50, pos=0] {$T_0$}
        node[anchor=south, pos=2/5] {$u_0$};

    \end{axis}
\end{tikzpicture} %%for compiling main 
    %\begin{tikzpicture}
    \begin{axis}
        [width = 0.95\textwidth, height = 0.6\textwidth,
         axis x line = center, axis y line = center,
         ylabel = $u(x_0; t)$, xlabel = $t$,
         xmin = -0.3, xmax = 5.7, ymin = -1.5, ymax = 3.8,
         axis line style = thin, xtick = {0}, ytick = {0}]   


        \tikzmath{\T2 = 3.5; \T1 = 4*\T2/5; \T0 = \T2/2; \l = 5.5; \t = (\l/pi)^2;}

        \addplot [black, domain=0:\l, samples = 300] {(\T1 + \T2 - 2*\T0) * exp(-x/\t)}
        node[anchor=50, pos=0] {$C_1$}
        node[anchor=south, pos=1/5] {$u_1$};

        \addplot [red, domain=0:\l, samples = 300] {(\T1 - \T2) * exp(-x*2^2/\t)}
        node[anchor=50, pos=0] {$C_2$}
        node[anchor=130, pos=1/4] {$u_2$};

        \addplot [green, domain=0:\l, samples = 300] {0}
        node[anchor=60, pos=0] {$\textcolor{black}{0}$}
        node[anchor=south, pos=1/3] {$u_4$};

        \addplot [blue, domain=0:\l, samples = 300] {\T0}
        node[anchor=50, pos=0] {$T_0$}
        node[anchor=south, pos=2/5] {$u_0$};

    \end{axis}
\end{tikzpicture} %%for compiling this file
\caption{Перші чотири розв'язки}
\end{figure}

Проміжок часу називається малим, коли $t \ll \tau$, а великим -- $t > \tau$

%\end{document}
%\documentclass[a4paper, 14pt]{extreport}

%\usepackage{StyleMMF}

%\begin{document}

%\chapter{Рівняння теплопровідності з однорідними межовими умовами}

\section[Задача №4.4]{4.4}

\textit{Початкова температура повністю теплоізольованого тонкого стержня\\ $0 \leq x \leq l$ дорівнює $T_1 \cos(\pi x/2l) + T_2 \cos(2\pi x/l)$ . Знайти поле температур при $t > 0$. Перевірити виконання початкових умови при $T_1 = 0$ і $T_2 = 0$.}

\begin{center}
    \large{\textbf{Розв'язок}}
\end{center}

\noindent Формальна постановка задачі:
\begin{equation} \label{probcond8}
    \left\{ \begin{aligned} %%
            \;&u = u(x,t), \\
            &u_t = D u_{xx}, \\
            &0 \leq x \leq l, t \geq 0, \\
            &u_x(0,t) = 0, \, u_x(l,t) = 0,\\ 
            &u(x,0) = T_1 \cos(\pi x/2l) + T_2 \cos(2\pi x/l).
    \end{aligned} \right.
\end{equation}

Виконуючи розділення змінних ми отримаємо дві попереднбо розв'язані задачі. Задачу Штурма-Ліувілля з задачі №2.1 та часове диференціальне рівняня з задачі №4.2. Отже, загальний розв'язок можна одразу записати комбінуюці відомі.

\begin{equation} \label{gen-sol8}
    u(x,t) = C_0 + \sum_{n=1}^{\infty}C_n e^{-t/\tau_n} \cos k_n x,
\end{equation}
\begin{equation*}
    \begin{aligned}
        &k_n = \frac{\pi n}{l} - \text{ хвильові вектори}, \\
        &\tau_n = \frac{1}{D k_n^2} - \text{ характерний час зміни температури}, \\
        &n = 0, 1, 2,\ldots
    \end{aligned}
\end{equation*}

З початковї умови визначимо невіомі коефіцієнти. Для цього треба розкласти $\cos(\pi x/2l)$ по набору власних функцій задачі Ш.-Л. 
\begin{equation*}
    \begin{gathered}
        \cos(\pi x/2l) = a_0 + \sum_{n=1}^{\infty} a_n \cos k_nx \\
        a_0 = \frac{1}{l}\int\limits_0^l \cos(\pi x/2l) \;\mathrm{d}x = \frac{2}{\pi} \sin(\pi x/2l) \bigg|_0^l = \frac{2}{\pi}\\
        a_n = \frac{2}{l}\int\limits_0^l \cos(\pi x/2l)\cos k_nx \;\mathrm{d}x = \frac{1}{l}\bigg(\int\limits_0^l \cos((k_n + \pi/2l)x) \;\mathrm{d}x +\\
        + \int\limits_0^l \cos((k_n - \pi/2l)x) \;\mathrm{d}x\bigg) = \frac{1}{l}\left(\frac{\sin((k_n + \pi/2l)x)}{k_n + \pi/2l}\bigg|_0^l + \frac{\sin((k_n - \pi/2l)x)}{k_n - \pi/2l}\bigg|_0^l\right) =\\
        = \left(\frac{\sin(k_nl + \pi/2)}{k_nl + \pi/2} + \frac{\sin(k_nl - \pi/2)}{k_nl - \pi/2}\right) = \left(\frac{1}{k_nl + \pi/2} - \frac{1}{k_nl - \pi/2}\right)\cos k_nl =\\
        = \big|\cos k_nl = (-1)^n\big| = \frac{(-1)^{n+1} \pi}{(k_nl + \pi/2)(k_nl - \pi/2)} =\\
        = (-1)^{n+1} \cdot \frac{4\pi}{4k_n^2l^2 - \pi^2} = \frac{4}{\pi} \cdot \frac{(-1)^{n+1}}{4n^2 - 1}
    \end{gathered}
\end{equation*}
Тепер підставимо (\ref{gen-sol8}) в початкову умову і отримаємо:
\begin{equation}
    \begin{gathered}
        u(x,0) = C_0 + \sum_{n=1}^{\infty}C_n \cos k_n x = T_1 \cos(\pi x/2l) + T_2 \cos(2\pi x/l) =\\
        = \frac{2T_1}{\pi} + T_2 \cos k_2x + \frac{4T_1}{\pi} \sum_{n=1}^{\infty} \frac{(-1)^{n+1} \cos k_nx}{4n^2 - 1}
    \end{gathered}
\end{equation} 
З чого слідує 
\begin{equation*}
    C_0 = \frac{2T_1}{\pi},\, C_2 = T_2 - \frac{4T_1}{15\pi},\, C_n = \frac{4T_1}{\pi} \cdot \frac{(-1)^{n+1}}{4n^2 - 1}, \text{ де } n \neq 2
\end{equation*}
Отже, остаточним розв'язком буде 
\begin{equation}
    u(x,t) = \frac{2T_1}{\pi} + T_2 e^{-t/\tau_2}\cos k_2x + \frac{4T_1}{\pi} \sum_{n=1}^{\infty} \frac{(-1)^{n+1} \cos k_nx}{4n^2 - 1}
\end{equation}

Прямою підстановкою можна переконатися, що при $T_1 = 0$ та $T_2 = 0$ початкові умови виконуються.


%\end{document}

\part{МЕТОД ЧАСТИННИХ РОЗВ’ЯЗКІВ ТА МЕТОД РОЗКЛАДАННЯ ЗА ВЛАСНИМИ ФУНКЦІЯМИ.}

\chapter{Еволюційні задачі з неоднорідним рівнянням або неоднорідними межовими умовами: стаціонарні неоднорідності}
%\documentclass[a4paper, 14pt]{extreport}

%\usepackage{StyleMMF}

%\setcounter{chapter}{4}

%\begin{document}

%\chapter{Еволюційні задачі з неоднорідним рівнянням або неоднорідними межовими умовами: стаціонарні неоднорідності}

\section[Задача №5.1]{5.1}

\textit{Знайти коливання вертикально розташованого пружного стержня під дією сили тяжіння для $t > 0$. Верхній кінець стержня закріплений, а нижній вільний. При $t < 0$ стержень був нерухомим і деформацій не було. Знайти спочатку стаціонарний розв’язок, що відповідає положенню рівноваги стержня в полі тяжіння, а потім знайти відхилення від нього, що відповідає коливанням навколо нового положення рівноваги. Намалювати графіки розподілу поля зміщень та поля напружень у положенні рівноваги.}

\begin{center}
    \large{\textbf{Розв'язок}}
\end{center}

\noindent Формальна постановка задачі:
\begin{equation} %\label{probcond12}
    \left\{ \begin{aligned}
            \;&u = u(x,t), \\
            &u_{tt} = v^2 u_{xx} + g, \\
            &0 \leq x \leq l, t \geq 0 \\
            &u(0,t) = , \, u_x(l,t) = 0, \\
            &u(x,0) = u_t(x,0) = 0.
    \end{aligned} \right.
\end{equation}

Шукаємо розв'язок у вигляді:
\begin{equation}
    u = u_{\text{ст}}(x) + w(x,t)
\end{equation}

Перепишемо задачу для стаціонарної частини розв'язку
\begin{equation} 
    \left\{ \begin{aligned}
            \;&u = u_{\text{ст}}(x), \\
            &v^2 u_{xx} + g = 0, \\
            &0 \leq x \leq l, t \geq 0 \\
            &u(0) = 0, \, u_x(l) = 0
    \end{aligned} \right.
\end{equation}
Рівняння двічі інтегрується і маємо:
\begin{equation}
    u_{\text{ст}}(x) = - \frac{g x^2}{2v^2} + C_1 x + C_2,
\end{equation}
а з межових умов визначимо константи інтегрування
\begin{equation}
    \begin{aligned}
        u_{\text{ст}}(0) = C_2 = 0,\\
        (u_{\text{ст}})_x(l) = -\frac{gl}{v^2} + C_1 = 0;
    \end{aligned}
    \;\Rightarrow\quad
    C_2 = 0, \quad C_1 = \frac{gl}{v^2}
\end{equation}
Отже, стаціонарний розв'язок
\begin{equation}
    u_{\text{ст}}(x) = - \frac{g x^2}{2v^2} + \frac{glx}{v^2} = - \frac{gl^2}{2v^2} \cdot \frac{x^2 - 2lx}{l^2} = -A \cdot \frac{x^2 - 2lx}{l^2}
\end{equation}

Тепер необхідно записати задачу для нестаціонарної частини $w(x,t)$
Рівняння:
\begin{equation*}
    u_{tt} = v^2 u_{xx} + g
    \;\Rightarrow\;
    w_{tt} = v^2w_{xx} - v^2 \frac{gl^2}{2v^2}\frac{2}{l^2} + g = v^2w_{xx}
\end{equation*}
Межові умови:
\begin{equation*}
    \begin{aligned}
        u(0,t) = -\frac{gl^2}{2v^2} \cdot \frac{x^2 - 2lx}{l^2}\bigg|_{x=0} + w(l,t) = 0,\\
        u_x(l,t) = -\frac{gl^2}{v^2} \cdot \frac{x - l}{l^2}\bigg|_{x=l} + w_x(l,t) = 0
    \end{aligned}
        \quad\Rightarrow\; 
        w(0,t) = 0,\, w_x(l,t) = 0
\end{equation*}
Початкові умови:
\begin{equation*}
    u(x,0) = -A \cdot \frac{x^2 - 2lx}{l^2} + w(x,t) = 0
    \;\Rightarrow\;
    w(x,0) = A \cdot \frac{x^2 - 2lx}{l^2}
\end{equation*}
\begin{equation*}
    u_t(x,0) = 0
    \;\Rightarrow\;
    w_t(x,0) = 0
\end{equation*}
Отже, отримуємо задачу для $w(x,t)$ з однорідними межовими умова, але з неоднорідними рівнянням та початковими умовами.
\begin{equation} 
    \left\{ \begin{aligned} 
            \;&w = w(x,t), \\
            &w_{tt} = v^2 w_{xx}, \\
            &0 \leq x \leq l, t \geq 0 \\
            &w(0,t) = 0, \, w_x(l,t) = 0, \\
            &w(x,0) = A \cdot \frac{x^2 - 2lx}{l^2},\\
            &w_t(x,0) = 0.
    \end{aligned} \right.
\end{equation}

Розв'язок такої задачі був знайдений раніше (див. задачу 1.2):
\begin{equation}
    \begin{aligned}
        &w(x,t) = \sum_{n=0}^{\infty} \big(A_n\sin\omega_nt + B_n\cos\omega_nt \big) \sin k_nx,\\
        &k_n = \frac{\pi}{l}(n + 1/2) \text{ -- хвильове число},\\
        &\omega_n = v k_n \text{ -- частота коливання},\\
        &n = 0, 1, 2, \ldots        
    \end{aligned}
\end{equation}

Залишається визначити коефіцієнти $A_n$ та $B_n$ з початкових умов:
\begin{equation*}
    w_t(x,0) = \sum_{n=0}^{\infty} A_n\omega_n\sin k_nx = 0
    \;\Rightarrow\;
    A_n = 0,\, \forall n
\end{equation*}
\begin{equation*}
    w_t(x,0) = \sum_{n=0}^{\infty} B_n\sin k_nx = A \cdot \frac{x^2 - 2lx}{l^2} 
    \;\Rightarrow\;
    B_n = \frac{2A}{l^3} \int\limits_0^l (x^2 - 2lx) \sin k_nx \;\mathrm{d}x
\end{equation*}
Обчислимо отриманий інтеграл 
\begin{equation*}
    \begin{gathered}
        \int\limits_0^l (x^2 - 2lx) \sin k_nx \;\mathrm{d}x = \frac{1}{k_n} (x^2 - 2lx) \cos k_nx \bigg|_0^l -\\
        - \frac{2}{k_n} \int\limits_0^l (x-l) \cos k_nx \;\mathrm{d}x = \frac{2}{k_n^2} \bigg[(x-l)\sin k_nx \bigg|_0^l - \int\limits_0^l \sin k_nx \;\mathrm{d}x \bigg] =\\
        = \frac{2}{k_n^3} \big(\cos k_nl - \cos 0\big) = \bigg|\cos k_nl = 0\bigg| = -\frac{2}{k_n^3}
    \end{gathered}
\end{equation*}
Прим.: \textit{можна було спростити інтегрування, зсунувши межі інтегрування на $-l/2$, адже тоді можна скористатися тим фактом, що непарна підінтегральна функція по симетричним межам дає нуль}

Отже, розв'язок 
\begin{equation}
    \begin{gathered}
        u(x,t) = - A \cdot \frac{x^2 - 2lx}{l^2} - 4A\sum_{n=0}^{\infty} \frac{\cos\omega_nt \sin k_nx}{l^3k_n^3} =\\
        = -A \left(\frac{x^2 - 2lx}{l^2} + 4\sum_{n=0}^{\infty} \frac{\cos\omega_nt \sin k_nx}{(lk_n)^3}\right)
    \end{gathered}
\end{equation}

%\end{document}
%\documentclass[a4paper, 14pt]{extreport}

%\usepackage{StyleMMF}

%\begin{document}

%\setcounter{chapter}{4}

%\chapter{Еволюційні задачі з неоднорідним рівнянням або неоднорідними межовими умовами: стаціонарні неоднорідності}

\section[Задача №5.3]{5.3}


\textbf{У стержні довжиною $l$ з непроникною бічною поверхнею відбувається дифузія частинок (коефіцієнт дифузії $D$), що мають час життя $\tau$. Через правий кінець всередину стержня подається постійний потік частинок $I_0$. Знайти стаціонарний розподіл концентрації та розв’язок, що задовольняє нульову початкову умову, якщо через лівий кінець частинки вільно виходять назовні й назад не вертаються. Знайти вигляд стаціонарного розв’язку в граничних випадках великих і малих $\tau$ та нарисувати графіки.}
\textit{ Указівка. Рівняння дифузії частинок зі скінченним часом життя має вигляд: $u_t = Du_{xx} - \frac{1}{\tau}u$. Його зручно переписати через так звану довжину дифузійного зміщення $L=\sqrt{D\tau}$:}
\begin{equation*}
    \tau u_t=L^2u_{xx}-u.
\end{equation*}
\textit{Величина $L$ має смисл характерної відстані, на яку частинки встигають зміститися (в середньому) за час свого життя. «Великі» й «малі» $\tau$ означають у дійсності $L\;\gg l$ i $L\;\ll l$ відповідно. Останній випадок фактично означає перехід до наближення півнескінченного стержня $-\infty < x \leq l$.}


\begin{center}
    \large{\textbf{Розв'язок}}
\end{center}

\noindent Формальна постановка задачі:
\begin{equation} \label{cond5,3}
    \left\{ \begin{aligned} %%
            \;&u = u(x,t), \\
            &\tau u_t=L^2u_{xx}-u, \\
            &0 \leq x \leq l, t \geq 0, \\
            &u(x,0)=0,\\
            &u(0,t) = 0, \\
            &u_x(l,t) = I_0. 
    \end{aligned} \right.
\end{equation}

Спочатку спробуємо знайти неоднорідну частину розв'язку $u_{\text{неодн}}$, яка задовільнить рівняння і межові умови (але не обов'язково задовільнятиме початкові). Шукатимемо її методом підбору. Оскільки межові умови не залежать від часу, то намагатимемося знайти його у вигляді функції $f(x)$ лише від $x$. Підставляючи такий вигляд неоднорідного розв'язку у рівняння, отримуємо:

\begin{equation*}
L^2\frac{d^2f}{dx^2} - f = 0
\end{equation*}

Звідси маємо $u_{\text{неодн}} = f(x) = C_1\sinh(x/L) + C_2\cosh(x/L)$. Підставляючи отриманий розв'язок у межові умови, отримуємо \[C_1 = \frac{I_0L}{\cosh(l/L)}, \quad C_2 = 0.\] Остаточно маємо:
\begin{equation}
u_{\text{неодн}}=\frac{I_0L\sinh(x/L)}{\cosh(l/L)}
\end{equation}

Далі шукаємо поний розв'язок у вигляді $u = u_{\text{неодн}} + u_{\text{одн}}$. Підставивши $u$ в задачу (\ref{cond5,3}) i отримаємо однорідну задачу на $u_{\text{одн}}$:

\begin{equation*} 
    \left\{ \begin{aligned}
            \;&u_{\text{одн}} = u_{\text{одн}}(x,t), \\
            &\tau (u_{\text{одн}})_t=L^2(u_{\text{одн}})_{xx}-u_{\text{одн}}, \\
            &u_{\text{одн}}(x,0)=-\frac{I_0L\sinh{(\frac{x}{L})}}{\cosh{(\frac{l}{L})}},\\
            &u_{\text{одн}}(0,t) = 0, \\
            &(u_{\text{одн}})_x(l,t) = 0. 
    \end{aligned} \right.
\end{equation*}

Перш ніж розділити змінні, спростимо рівняння класичним прийомом, який непогано було б запам'ятати. Якщо ми маємо рівння типу 
\begin{equation}
U_t = D U_{xx} + aU    
\end{equation}
з однорідними межовими умовами, то підстановка $U = u\exp(at)$ перетворить його рівняння на $u_t = D u_{xx}$, а межові та початкові умови для $u$ будуть такими самими як були на $U$.\\

Тепер використаємо цей прийом. Підставимо $u_{\text{одн}} = \tilde{u}\exp(-\frac{t}{\tau})$ та отримаэмо задачу на $\tilde{u}$:

\begin{equation} \label{new-cond5,3}
    \left\{ \begin{aligned}
            \;&\tilde{u} = \tilde{u}(x,t), \\
            &\tilde{u}_t = D \tilde{u}_{xx}, \\
            &0 \leq x \leq l, t \geq 0, \\
            &\tilde{u}(x,0) = -\frac{I_0L \sinh(x/L)}{\cosh(l/L)},\\
            &\tilde{u}(0,t) = 0, \\
            &\tilde{u}_x(l,t) = 0. 
    \end{aligned} \right.
\end{equation}
\\

А такі задачі ми вже добре вміємо розв'язувати. Потрібно знайти розв'язки (\ref{new-cond5,3})
наступного вигляду:
\begin{equation} \label{subst5,3}
    \tilde{u}(x,t) = X(x) \cdot T(t) \neq 0 
\end{equation}

Після підстановки відокремлення змінних (див. задачу 1.1) маємо:
\begin{equation}
    \frac{T'}{DT} = \frac{X^{\prime\prime}}{X} = -\lambda
\end{equation}

Отримуємо задачу Штурма-Ліувіля на $X(x)$ яка розв'язана у домашній задачі №1,2 з посібника. Випишемо результат:
  \begin{equation} 
        \left\{ \begin{aligned}
            \;&\lambda_n = \frac{\pi^2 (2n-1)^2}{4l^2} - \text{власнi числа},\\
            &k_n = \frac{\pi (2n-1)}{2l} - \text{власнi хвильові вектори},\\
            &\text{де } n \in \mathbb{N},\\ 
            &X_n(x) = \sin k_nx - \text{власнi функції}.
        \end{aligned} \right.
    \end{equation}

\textit{Оскільки всі власні числа додатні, то після підстановки $\tilde{u}$ в $u_{\text{одн}}$ показники часових експонент не зможуть скомпенсуватися, адже будуть одного знаку, отже $u_{\text{одн}}$ є залежним від часу і виділити незалежну частину неможливо. А це означає, що знайдене раніше $u_{\text{неодн}}$ і є шуканим стаціонарним розв'язком задачі.}

Тепер розв'яжемо рівняння на $T_n(t)$:
\begin{equation} 
    \left\{ \begin{aligned}
        &\dot{T_n}+\lambda_n DT_n=0,\\
        &\frac1{\tau_n} = D\lambda_n - \text{власний характерний час},\\
        &T_n=C_n\exp{\left(-\frac t{\tau_n}\right)}.
    \end{aligned} \right.
\end{equation}

Збираючи по отриманим функціям $T(t)$ і $X(x)$ власні функції задачі й просумувавши їх, отримуємо загальний розв'язок:
\begin{equation} \label{mode5,3}
    \tilde{u}(x,t) = \sum_{n=1}^{\infty} C_n\exp{\left(-\frac t{\tau_n}\right)}\sin\left(k_n x\right)
\end{equation}

Тепер підставимо (\ref{mode5,3}) у початкові умови:
\begin{equation} \label{init-con5,3}
    \tilde{u}(x,0) = \sum_{n=1}^{\infty} C_n\sin\left(k_n x\right) = -\frac{I_0L\sinh{(\frac{x}{L})}}{\cosh{(\frac{l}{L})}}
\end{equation}

Ліва сторона рівності є розкладом Фур'є правої сторони по синусах. Знайдемо коефіцієнти цього розкладу:

\begin{equation*}
    \begin{aligned} 
        &C_n\int_{0}^{l} \left( \sin\left(k_n x\right)\right)^2\;dx= \frac{lC_n}{2}
        =-\frac{I_0L}{\cosh{(\frac{l}{L})}}\int_{0}^{l}\sinh{\left(\frac{x}{L}\right)}\sin\left(k_n x\right)\;dx=\\
        &=\left[\cos(k_n l)=(-1)^n\right]=-\frac{I_0L}{\cosh{(\frac{l}{L})}}\frac{4l^2L(-1)^{n+1}\cosh{(\frac{l}{L})}}{4l^2+\pi^2L^2(2n-1)^2}=\frac{L^2I_0(-1)^n}{1+L^2\lambda_n}.
    \end{aligned}
\end{equation*}

Підставляючи отримані коефіцієнти у (\ref{init-con5,3}), а його у $u_{\text{одн}}$ і сумуючи з $u_{\text{неодн}}$, отримуємо відповідь:
\begin{equation*} 
    %\begin{aligned}
        u(x,t)=\frac{I_0L\sinh{(\frac{x}{L})}}{\cosh{(\frac{l}{L})}}+\frac{2L^2I_0}{l}\sum_{n=1}^{\infty}\frac{L^2I_0(-1)^n}{1+L^2\lambda_n}\exp{\left(-\left(\frac t{\tau_n}+\frac{t}{\tau}\right) \right)}\sin\left(k_n x\right)
    %\end{aligned}
\end{equation*}

%\end{document}

\chapter{Задачі з неоднорідним рівнянням або неоднорідними межовими умовами}
%\documentclass[a4paper, 14pt]{extreport}

%\usepackage{StyleMMF}

%\begin{document}

%\setcounter{chapter}{5}

%\chapter{Задачі з неоднорідним рівнянням або неоднорідними межовими умовами}

\textbf{\large Джерела з гармонічною залежністю від часу.}

\section[Задача №6.1]{6.1}

\textit{Знайти коливання струни $0 \leq x \leq l$, лівий кінець якої закріплений, а правий вільний, при $t > 0$ під дією розподіленої сили $f(x,t) = f(x)\cos\omega t$. При $t < 0$ струна перебувала в положенні рівноваги. Розглянути окремий випадок $f(x) = f_0$. Виділити складову розв’язку, яка відповідає усталеним вимушеним коливанням і проаналізувати картину резонансу. Перевірити, чи переходить одержаний розв’язок у розв’язок задачі 5.1 за відповідних умов.}

\begin{center}
    \large{\textbf{Розв'язок}}
\end{center}

\noindent Спершу розберемося з розмірностями. Під розподіленою силою слід розуміти величину $\vec{f}=\frac{\Delta \vec{F}}{\Delta m} = \frac{d\vec{F}}{dV} \nu$, де $\nu =\frac 1\rho$ - питомий об'єм. 

\noindent Формальна постановка задачі:
\begin{equation} \label{cond6,1}
    \left\{ \begin{aligned} %%
            \;&u = u(x,t), \\
            &0 \leq x \leq l, t \geq 0, \\
            &u_{tt}=v^2u_{xx}+f(x,t), \\
            &f(x,t)=f(x)\cos\omega t\\
            &\text{окремий випадок: }\;\; f(x)= f_0,\\
            &u(x,0)=0,\\
            &u_t(x,0)=0,\\
            &u(0,t) =0, \\
            &u_x(l,t) =0 . 
    \end{aligned} \right.
\end{equation}

Виділимо складову розв'язку $\tilde{u}$, яка відповідає вимушеним коливанням. Шукатимемо її у формі $\tilde{u}(x,t) = \widetilde{X}(x) \cos\omega t$. Після підстановки отримуємо наступну задачу (нам не принципово, щоб вона задовільняла ще й початкові умови):
\begin{equation} 
    \left\{ \begin{aligned} 
        &v^2\widetilde{X}'' + \omega^2\widetilde{X} + f(x) = 0,\\
        &\widetilde{X}(0)=0,\\
        &\widetilde{X}'(l)=0.
    \end{aligned} \right.
\end{equation}

Найпростішим методом її розв'язання є метод функцій Гріна, який ви вивчали на курсі диференціальних рівнянь. Спочатку знаходимо функцію Гріна $G(x,s)$ до цієї задачі:

\begin{enumerate} 
  \item При $x \neq s$, вона має задовільняти однорідній частині рівняння, тобто 
  \begin{equation*}
    v^2 G'' + \omega^2G = 0.    
  \end{equation*}
  Тоді, при вказаних вище умовах, маємо для кожної з областей $(x<s) \;\;\text{і} \;\;(x>s)$:
  \begin{equation*}
    G(x,s) = C_i(s) \cos\left(\frac{\omega x}{v}\right) + C_j(s) \sin\left(\frac{\omega x}{v}\right),    
  \end{equation*}
  \begin{equation*} 
    \left\{ \begin{aligned}
            & i=1,\, j=2,\, x<s;\\
            & i=3,\, j=4,\, x>s.
    \end{aligned} \right.
\end{equation*}

  \item Вона має задовільняти крайовим умовам.
  \begin{equation*}
    G(x,s) = C_2(s) \sin\left(\frac{\omega x}{v}\right), \, x<s 
  \end{equation*}
  \begin{equation*}
    G(x,s) = C_3(s) \left[\cos\left(\frac{\omega x}{v}\right) + \sin\left(\frac{\omega x}{v}\right) \tan\left(\frac{\omega l}{v}\right)\right], \, x>s 
  \end{equation*}
  
  \item При $x=s$ вона неперервна по $x$, а її похідна має скачок, що дорівнює оберненій величині до коефіцієнта при другій похідній та залежить від $s$, замість $x$. У нашому випадку цей скачок дорівнює $\frac{1}{v^2}$.
  \begin{equation*}
    C_2(s) \sin\left(\frac{\omega s}{v}\right) = C_3(s) \left[\cos\left(\frac{\omega s}{v}\right) + \sin\left(\frac{\omega s}{v}\right) \tan\left(\frac{\omega l}{v}\right)\right] 
  \end{equation*}

  \begin{equation*}
    C_2(s) \cos\left(\frac{\omega s}{v}\right) + \frac{1}{v\omega} = C_3(s) \left[\cos\left(\frac{\omega s}{v}\right) \tan\left(\frac{\omega l}{v}\right) - \sin\left(\frac{\omega s}{v}\right)\right]
  \end{equation*}
  Після спрощень отримуємо:
  \begin{equation*}
    C_3 = -\frac{\sin\left(\frac{\omega s}{v}\right)}{v\omega}
  \end{equation*}
  \begin{equation*}
    C_2 = -\frac{\cos\left(\frac{\omega(s-l)}{v}\right)}{v\omega \cos\left(\frac{\omega l}{v}\right)}
  \end{equation*}
  \begin{equation*}
    G(x,s) = -\frac{\cos\left(\frac{\omega(s-l)}{v}\right)}{v\omega\cos\left(\frac{\omega l}{v}\right)} \sin\left(\frac{\omega x}{v}\right), \, x<s 
  \end{equation*}
  \begin{equation*}
    G(x,s) = -\frac{\cos\left(\frac{\omega(x-l)}{v}\right)}{v\omega\cos\left(\frac{\omega l}{v}\right)} \sin\left(\frac{\omega s}{v}\right), \, x>s 
  \end{equation*}
\end{enumerate}

Звідси отримуємо просторову складову, що відповідає усталеним коливанням, для загальної функції $f$:
\begin{equation} \label{Green-sol}
    \tilde{u} = \int\limits_0^x \frac{\cos\left(\frac{\omega(x-l)}{v}\right) \sin\left(\frac{\omega s}{v}\right) f(s)}{v\omega\cos\left(\frac{\omega l}{v}\right)} \; \mathrm{d}s + \int\limits_x^l \frac{\cos\left(\frac{\omega (s-l)}{v}\right)\sin\left(\frac{\omega x}{v}\right) f(s)}{v\omega\cos\left(\frac{\omega l}{v}\right)} \; \mathrm{d}s
\end{equation}

Підставивши у (\ref{Green-sol}) окремий випадок і виконавши інтегрування, отримаємо:
\begin{equation}
    \tilde{u} = \frac {f_0\cos(\omega t)}{\omega^2}\left(\frac{\cos\left(\frac{\omega (x-l)}{v}\right)}{\cos\left(\frac{\omega l}{v}\right)}-1\right)
\end{equation}

Повний розв'язок є комбінацією однорідної $u_0$ і вимушеної $\tilde{u}$ частини: $u = u_0 + \tilde{u}$. Задача на $u_0$ отримується підстановкою цієї комбінації у (\ref{cond6,1}) і виглядає наступним чином: 
\begin{equation} 
    \left\{ \begin{aligned} %%
            \;&u_0 = u_0(x,t), \\
            &0 \leq x \leq l, t \geq 0, \\
            &{(u_0)}_{tt}=v^2{(u_0)}_{xx}, \\
            &u_0(0,t) = 0, {(u_0)}_x(l,t) = 0,\\  
            &u_0(x,0) = -\tilde{u}(x,0),\\
            &{(u_0)}_t(x,0) = 0.\\
    \end{aligned} \right.
\end{equation}

Маємо просту задачу на хвильове рівняння з початковими умовами. Пошук власних мод для таких крайових умов вже виконаний у домашній задачі №1,2. Випишемо результат:

\begin{equation*} 
    \left\{ \begin{aligned}
            \;&\lambda_n = \frac{\pi^2 (2n-1)^2}{4l^2} - \text{власнi числа},\\
            &k_n = \frac{\pi (2n-1)}{2l} - \text{власнi хвильові вектори},\\
            &\omega_n=vk_n - \text{власнi частоти},\\
            &\text{де } n \in \mathbb{N},\\ 
            &{(u_0)}_n = \left(A_n\sin\omega_nt + B_n\cos\omega_nt\right)\sin k_nx - \text{власнi моди}.       
    \end{aligned} \right.
\end{equation*}

Одразу можемо побачити, що оскільки початковий розподіл швидкостей нульовий, то $\forall n \in \mathbb{N}, A_n = 0$. Тоді однорідна частина розв'язку матиме вигляд:

\begin{equation} \label{hom-modes6,1}
    u = \sum_{n=1}^\infty B_n\cos\omega_nt\sin k_nx
\end{equation}

Залишається підставити (\ref{hom-modes6,1}) в умову на початкове зміщення:

\begin{equation} 
    u_0(x,0) = \sum_{n=1}^\infty B_n\sin k_nx = -\tilde{u} = \frac{f_0}{\omega^2}\left(1 - \frac{\cos\left(\frac{\omega (x-l)}{v}\right)}{\cos\left(\frac{\omega l}{v}\right)}\right)
\end{equation}
 
Коефіцієнти $B_n$ для просторового розподілу сили загального вигляду знаходяться наступним чином:

\begin{equation} 
    B_n = -\frac{2}{l} \int\limits_0^l \tilde{u} \sin k_nx \;\mathrm{d}x
\end{equation}

І після всіх підстановок загальна задача буде розв'язана. Знайдемо ці коефіцієнти для окремого випадку:

\begin{equation} 
    \begin{aligned}
        &B_n = \frac{2f_0}{l\omega^2} \int\limits_0^l \left(1 - \frac{\cos\left(\frac{\omega(x-l)}{v}\right)}{\cos\left(\frac{\omega l}{v}\right)}\right)\sin(k_nx)dx = \frac{2f_0}{l\omega^2} \left(\frac{1}{k_n} - \frac{k_n}{k_n^2 - \left(\frac{\omega}{v}\right)^2}\right)=\\
        &= -\frac{2f_0}{lk_n \left(\omega_n^2 - \omega^2\right)}
    \end{aligned}    
\end{equation}

Повний розв'язок задачі матиме вигляд:

\begin{equation}
    u = \tilde{u}\cos{\omega t} + \sum_{n=1}^\infty B_n\cos\omega_nt\sin k_nx
\end{equation}

Для окремого випадку:

\begin{equation} 
    u = \frac{f_0}{\omega^2} \left(\frac{\cos\left(\frac{\omega (x-l)}{v}\right)}{\cos\left(\frac{\omega l}{v}\right)}-1\right) \cos{\omega t} - \sum_{n=1}^\infty \frac{2f_0}{lk_n\left(\omega_n^2-\omega^2\right)} \cos\omega_nt\sin k_nx
\end{equation}

\textit{Добре видно, що при наближенні частоти $\omega$, з якою діє сила, до власної частоти $\omega_n$, амплітуда відповідної власної моди нескінченно зростатиме. Це явище відповідає визначенню резонансу.}

Тепер спрямуємо $\omega$ до 0. При цьому вимушений розв'язок матиме вигляд:

\begin{equation*}
    \begin{aligned}
        &\frac {f_0}{\omega^2}\left(\frac{\cos\left(\frac{\omega (x-l)}{v}\right)}{\cos\left(\frac{\omega l}{v}\right)}-1\right)\cos{\omega t}=\left[ \cos\left(\frac{\omega (x-l)}{v}\right)=1-\frac{\omega^2 (x-l)^2}{v^2}+O(\omega^4)\right]=\\
        &\frac {f_0\left(1-\frac{\omega^2 (x-l)^2}{v^2}-1+\frac{\omega^2 l^2}{v^2}+O(\omega^4)\right)}{\omega^2-\frac{\omega^4 l^2}{v^2}+O(\omega^6)}=\frac {f_0\left( 2xl-x^2+O(\omega^2)\right)}{v^2(1+O(\omega^2))}.
    \end{aligned}
\end{equation*}

І повний розв'язок матиме вигляд:
\begin{equation} 
    u = \frac{f_0\left( 2xl-x^2\right)}{v^2} - \sum_{n=1}^\infty \frac{2f_0}{lk_n\omega_n^2}\cos(\omega_n t)\sin(k_n x)
\end{equation}

Що повністю відповідає розв'язку (\ref{Cauchy-sol5,1}) задачі  №5,1.

%\end{document}
\vspace{2cm}
%\documentclass[a4paper, 14pt]{extreport}

%\usepackage{StyleMMF}

%\begin{document}

%\chapter{Задачі з неоднорідним рівнянням або неоднорідними межовими умовами}

\textbf{\large Метод розкладання по власних функціях в задачах з неоднорідним рівнянням}

\section[Задача №6.3]{6.3}

\textit{Знайти коливання струни із закріпленими кінцями під дією сили $f(x,t) = f_0 t^N, \, N > 0$ однорідно розподіленої по довжині струни. У початковий момент струна нерухома, і зміщення дорівнює нулю. Остаточні обчислення виконати
для $N=2$.}

%\end{document}

\chapter{Задачі з неоднорідними межовими умовами загального вигляду}
\documentclass[a4paper, 14pt]{extreport}

\usepackage{StyleMMF}

\begin{document}

\section{Задачі з неоднорідними межовими умовами загального вигляду}

\subsubsection{Задача №1}

\textit{Знайти коливання пружного стержня, якщо правий кінець його закріплений нерухомо, а до лівого при $t > 0$ прикладена сила $F(t)$ , шляхом зведення до задачі з неоднорідним рівнянням. Відповідь одержати для частинного випадку $F(t) = F_0 e^{-\alpha t}$. При $t < 0$ стержень перебував у положенні рівноваги.}

\end{document}
%\documentclass[a4paper, 14pt]{extreport}

%\usepackage{StyleMMF}
%\usepackage{bookmark}

%\begin{document}

%\setcounter{chapter}{6}
%\chapter{Задачі з неоднорідними межовими умовами загального вигляду}

\section[Задача №7.2]{7.2}

\textit{Розв’язати задачу №7.1 методом розкладання по власних функціях.}

\begin{center}
    \large{\textbf{Розв'язок}}
\end{center}

\noindent Формальна постановка задачі:
\begin{equation} \label{cond7,2}
    \left\{ \begin{aligned} 
            \;&u = u(x,t), \\
            &u_{tt} = v^2 u_{xx}, \\
            &0 \leq x \leq l, t \geq 0 \\
            &u_x(0,t) = \frac{F_0}{\beta} e^{-\alpha t} = f_0 e^{-\alpha t},\\
            &u(l,t) = 0, \\
            &u(x,0) = 0, \, u_t(x,0) = 0.
    \end{aligned} \right.
\end{equation}

Розв'язок шукаємо у вигляді:
\begin{equation}
    u(x,t) = \sum\limits_{n = 0}^\infty T_n(t) X_n(x),
\end{equation}
де $X_n(x)$ -- власні функції задачі. Їх визначаємо, розв'язуючи задачу Штурма-Ліувілля з однорідними межовими умовами.
\begin{equation*}
    \begin{gathered}
        \left\{ \begin{aligned} 
            \;&u_{tt} = v^2 u_{xx}, \\
            &u_x(0,t) = 0, \, u(l,t) = 0.
        \end{aligned} \right.
        \;\Rightarrow\;
        \left\{ \begin{aligned} 
            \;&X_m'' + k_m^2 X_m = 0, \\
            &X_m'(0) = 0, \, X_m(l) = 0.
        \end{aligned} \right.
        \;\Rightarrow\\
        \vspace{10pt}
        \Rightarrow\;
        \left\{ \begin{aligned} 
            \;&k_m = \frac{\pi}{l}(m + 1/2) \text{ -- хвильві числа,} \\
            &X_m(x) = \cos k_mx,\, m = 0,1,2\ldots
        \end{aligned} \right.
    \end{gathered}
\end{equation*}

Розкладемо рівняння по власним функціям. Для цього домножимо йього на $X_m / \left\lVert X_m\right\rVert^2$ та проінтегруємо по $x$.
\begin{equation} \label{series-exp7,2}
    \frac{1}{\left\lVert X_m\right\rVert^2} \int\limits_0^l u_{tt} X_m \;\mathrm{d}x = \frac{v^2}{\left\lVert X_m\right\rVert^2} \int\limits_0^l u_{xx} X_m \;\mathrm{d}x
\end{equation}
Випишемо окремо перетворення для лівої та правої частини рівняня. Почнемо з лівої
\begin{equation*}
    \frac{1}{\left\lVert X_m\right\rVert^2} \int\limits_0^l u_{tt} X_m \;\mathrm{d}x =  \sum\limits_{n=0}^\infty \frac{T_n''(t)}{\left\lVert X_m\right\rVert^2} \int\limits_0^l X_n(x) X_m(x) \;\mathrm{d}x = \sum\limits_{n=0}^\infty T_n''(t) \delta_{n,m} = T_m''(t)
\end{equation*}

Тепер розглянемо праву частину рівняння, двічі використаємо інтегрування частинами
\begin{equation*}
    \frac{1}{\left\lVert X_m\right\rVert^2} \int\limits_0^l u_{xx} X_m \;\mathrm{d}x = \sum\limits_{n=0}^\infty \frac{T_n(t)}{\left\lVert X_m\right\rVert^2} \int\limits_0^l X_n''(x) X_m(x) \;\mathrm{d}x \  \textcolor{red}{=}
\end{equation*}
Випишемо окремо інтегрування під сумою
\begin{equation*}
    \begin{gathered}
        \int\limits_0^l X_n''(x) X_m(x) \;\mathrm{d}x = X_n'(x) X_m(x) \bigg|_0^l - \int\limits_0^l X_n'(x) X_m'(x) \;\mathrm{d}x = \\
        = X_n'(x) X_m(x) \bigg|_0^l - X_n(x) X_m'(x) \bigg|_0^l + \int\limits_0^l X_n(x) X_m''(x) \;\mathrm{d}x
    \end{gathered}
\end{equation*}
Скористаємося рівнянням для $X_n(x)$:
\begin{equation*}
    \begin{gathered}
        X_n'(x) X_m(x) \bigg|_0^l - X_n(x) X_m'(x) \bigg|_0^l - k_n^2 \int\limits_0^l X_n(x) X_m(x) \;\mathrm{d}x =\\
        = X_n'(x) X_m(x) \bigg|_0^l - X_n(x) X_m'(x) \bigg|_0^l - k_n^2 \left\lVert X_m\right\rVert^2 \delta_{n,m}
    \end{gathered}
\end{equation*}
І повернемося до початкового виразу та запишемо його через зміщення $u(x,t)$
\begin{equation*}
    \begin{gathered}
        \textcolor{red}{=}\ \sum\limits_{n=0}^\infty \frac{T_n(t)}{\left\lVert X_m\right\rVert^2} \bigg[X_n'(x) X_m(x) \bigg|_0^l - X_n(x) X_m'(x) \bigg|_0^l - k_n^2 \left\lVert X_m\right\rVert^2 \delta_{n,m}\bigg] =\\
        = - k_m^2 T_m(t) + \frac{1}{\left\lVert X_m\right\rVert^2} \left[ u_x(x,t) X_m(x) \bigg|_0^l - u(x,t) X_m'(x) \bigg|_0^l \right] \ \textcolor{red}{=}
    \end{gathered}
\end{equation*}
Скористаємося межовими умовами початкової задачі та задачі Штурма-Ліувілля для власних функцій системи при обчисленні значень отриманих виразів після інтегрування
\begin{equation*}
    \begin{gathered}
        \textcolor{red}{= } -k_n^2 T_m(t) + \frac{2}{l} \bigg[ u_x(l,t) X_m(l) - u_x(0,t) X_m(0) -\\
        - u(x,l) X_m'(l) + u(x,0) X_m'(0) \bigg] = - k_n^2 T_m(t) - \frac{2 f_0}{l} e^{-\alpha t},
    \end{gathered}
\end{equation*}
тут для $X_m(x)$ дивимось на межові умови задачі Штурма-Ліувілля, а для $u(x,t)$ -- межові умови початкової задачі.

Повертаємося до розкладу рівняння (\ref{series-exp7,2}), збираючи разом результати обчислень для лівої та правої частинами
\begin{equation}
    T_m''(t) = - k_m^2v^2 T_m(t)  - \frac{2v^2 f_0}{l} e^{-\alpha t}, \quad \text{або} \quad T_n''(t) + \omega_n^2 T_n(t) = - \kappa_n \omega_n^2  e^{-\alpha t}
\end{equation}

Отримали лінійне неожнорідне рівняння для $T_n(t)$, його розв'язок шукаємо у вигляді
\begin{equation}
    T_n(t) = A_n \cos\omega_nt + B_n \sin\omega_nt + \gamma e^{-\alpha t},
\end{equation}
а $\gamma$, коефіцієнт частинного розв'язку, визначаємо підстановкою в рівняння
\begin{equation*}
    \widetilde{T}_n(t) = \gamma e^{-\alpha t} 
    \quad\Rightarrow\;
    \alpha^2\gamma + \omega_n^2\gamma = - \kappa_n \omega_n^2
    \;\Rightarrow\;
    \gamma = - \frac{\kappa_n \omega_n^2}{\omega_n^2 + \alpha^2}
\end{equation*} 
Маємо
\begin{equation}
    T_n(t) = A_n \cos\omega_nt + B_n \sin\omega_nt - \frac{\kappa_n \omega_n^2}{\omega_n^2 + \alpha^2} e^{-\alpha t}
\end{equation}

Залишається визначити коефіцієнти $A_n$ та $B_n$. Оскільки, обидві умови однорідні можна одразу записати:
\begin{equation*}
    \left\{ \begin{aligned}
        \;&T_n(0) = 0,\\
        &T_n'(0) = 0; 
    \end{aligned} \right.
    \quad\Rightarrow\;
    \left\{ \begin{aligned}
        \;&T_n(0) = A_n - \frac{\kappa_n \omega_n^2}{\omega_n^2 + \alpha^2} = 0,\\
        &T_n'(0) = B_n\omega_n + \frac{\alpha \kappa_n \omega_n^2}{\omega_n^2 + \alpha^2} = 0; 
    \end{aligned} \right.
    \;\Rightarrow\;
    \left\{ \begin{aligned}
        \;&A_n = \frac{\kappa_n \omega_n^2}{\omega_n^2 + \alpha^2},\\
        &B_n = - \frac{\kappa_n \alpha \omega_n}{\omega_n^2 + \alpha^2}. 
    \end{aligned} \right.
\end{equation*}

Наш остаточний розв'язок 
\begin{equation}
    u(x,t) = \sum\limits_0^l \frac{\kappa_n \cos k_nx}{\omega_n^2 + \alpha^2} \left(\omega_n^2 \cos\omega_nt - \alpha \omega_n \sin\omega_nt - \omega_n^2 e^{-\alpha t} \right)
\end{equation}

%\end{document}


\part{??}
\chapter{Метод характеристик і формула Даламбера: нескінченна пряма, півнескінченна пряма та відрізок. Метод непарного продовження.}
%\documentclass[a4paper, 14pt]{extreport}

%\usepackage{../../main/StyleMMF}

%\setcounter{chapter}{7}

%\begin{document}

%\chapter{Метод характеристик і формула Даламбера: нескінченна пряма, півнескінченна пряма та відрізок. Метод непарного продовження.}

\textbf{\large Вільні коливання нескінченної струни.}

\section[Задача №8.1]{8.1}

\textit{Зобразити графічно поле зміщень і поле швидкостей нескінченної струни в характерні послідовні моменти часу, якщо початковий відхил (зміщення) має форму рівнобедреного трикутника висотою $h$ і основою $2L$, а початкова швидкість дорівнює нулю. Чи всі частини трикутника приходять у рух одразу? Відповідь поясніть.}

\begin{center}
    \large{\textbf{Розв'язок}}
\end{center}

\begin{figure}[h!]
    \centering

    \begin{tikzpicture}

        \tikzmath{\l = 3; \h = 2; \v = 3/20; \y0 = -31.5; \dt = \l/\v/4;}

        \begin{axis} %% t6
            [width = \textwidth,
            %axis lines = center,
            axis y line = none, axis x line = center,
            ylabel = $t$, xlabel = $x$,
            xmin = -10, xmax = 10, ymin = \y0, ymax = 4,
            axis line style = thin, ticks = none]   
            
            \filldraw[color = white] (-9,\dt/2) circle (0.1cm) node[]{\textcolor{black}{$t_6 = \frac{3l}{2v}$}};

        \end{axis}

        \begin{axis} %% t5
            [width = \textwidth,
            axis y line = none, axis x line = center,
            xlabel = $x$,
            xmin = -10, xmax = 10, ymin = \y0 + \dt, ymax = 4 + \dt,
            axis line style = thin, ticks = none]   
            
            \filldraw[color = white] (-9,\dt/2) circle (0.1cm) node[]{\textcolor{black}{$t_5 = \frac{5l}{4v}$}};
            
            % струна
            \addplot[red, thick, domain=-10:-9*\l/4] {0};            
            \addplot[red, thick] (-9*\l/4,0) -- (-5*\l/4,\h);            
            \addplot[red, thick] (-5*\l/4,\h) -- (-\l/4,0);            
            \addplot[red, thick] {0};
            \addplot[red, thick] (\l/4,0) -- (5*\l/4,\h);
            \addplot[red, thick] (5*\l/4,\h) -- (9*\l/4,0);
            \addplot[red, thick, domain=9*\l/4:9.75] {0};

        \end{axis}

        \begin{axis} %% t4
            [width = \textwidth,
            axis y line = none, axis x line = center,
            xlabel = $x$,
            xmin = -10, xmax = 10, ymin = \y0 + 2*\dt,ymax = 4 + 2*\dt,
            axis line style = thin, ticks = none]   
            
            \filldraw[color = white] (-9,\dt/2) circle (0.1cm) node[]{\textcolor{black}{$t_4 = \frac{l}{v}$}};
            
            % пунктирні трикутники
            \addplot[blue, dashed] (0,0) -- (\l,\h);            
            \addplot[blue, dashed] (\l,\h) -- (2*\l,0);

            \addplot[orange!75!black, dashed] (-2*\l,0) -- (-\l,\h);            
            \addplot[orange!75!black, dashed] (-\l,\h) -- (0,0);


            % струна
            \addplot[red, thick, samples=50, domain=-10:-2*\l] {0};            
            \addplot[red, thick] (-2*\l,0) -- (-\l,\h);            
            \addplot[red, thick] (-\l,\h) -- (0,0);            
            \addplot[red, thick] (0,0) -- (\l,\h);
            \addplot[red, thick] (\l,\h) -- (2*\l,0);
            \addplot[red, thick, samples=50, domain=2*\l:9.75] {0};

        \end{axis}

        \begin{axis} %% t3
            [width = \textwidth,
            axis y line = none, axis x line = center,
            xlabel = $x$,
            xmin = -10, xmax = 10, ymin = \y0 + 3*\dt, ymax = 4 + 3*\dt,
            axis line style = thin, ticks = none]   
            
            \filldraw[color = white] (-9,\dt/2) circle (0.1cm) node[] {\textcolor{black}{$t_3 = \frac{3l}{4v}$}};
            
            % пунктирні трикутники
            \addplot[blue, dashed] (-\l/4,0) -- (3*\l/4,\h);            
            \addplot[blue, dashed] (3*\l/4,\h) -- (7*\l/4,0);

            \addplot[orange!75!black, dashed] (-7*\l/4,0) -- (-3*\l/4,\h);            
            \addplot[orange!75!black, dashed] (-3*\l/4,\h) -- (\l/4,0);


            % струна
            \addplot[red, thick, samples=50, domain=-10:-7*\l/4] {0};            
            \addplot[red, thick] (-7*\l/4,0) -- (-3*\l/4,\h);            
            \addplot[red, thick] (-3*\l/4,\h) -- (-\l/4,\h/2);            
            \addplot[red, thick] (-\l/4,\h/2) -- (\l/4,\h/2);            
            \addplot[red, thick] (\l/4,\h/2) -- (3*\l/4,\h);
            \addplot[red, thick] (3*\l/4,\h) -- (7*\l/4,0);
            \addplot[red, thick, samples=50, domain=7*\l/4:9.75] {0};

        \end{axis}

        \begin{axis} %% t2
            [width = \textwidth,
            axis y line = none, axis x line = center,
            xlabel = $x$,
            xmin = -10, xmax = 10, ymin = \y0 + 4*\dt, ymax = 4 + 4*\dt,
            axis line style = thin, ticks = none]   
            
            \filldraw[color = white] (-9,\dt/2) circle (0.1cm) node[]{\textcolor{black}{$t_2 = \frac{l}{2v}$}};
            
            % пунктирні трикутники
            \addplot[blue, dashed] (-2*\l/4,0) -- (2*\l/4,\h);            
            \addplot[blue, dashed] (2*\l/4,\h) -- (6*\l/4,0);

            \addplot[orange!75!black, dashed] (-6*\l/4,0) -- (-2*\l/4,\h);            
            \addplot[orange!75!black, dashed] (-2*\l/4,\h) -- (4*\l/4,0);


            % струна
            \addplot[red, thick, samples=50, domain=-10:-6*\l/4] {0};            
            \addplot[red, thick] (-6*\l/4,0) -- (-2*\l/4,\h);            
            \addplot[red, thick] (-2*\l/4,\h) -- (2*\l/4,\h);     
            \addplot[red, thick] (2*\l/4,\h) -- (6*\l/4,0);
            \addplot[red, thick, samples=50, domain=6*\l/4:9.75] {0};
            
        \end{axis}

        \begin{axis} %% t1
            [width = \textwidth,
            axis y line = none, axis x line = center,
            xlabel = $x$,
            xmin = -10, xmax = 10, ymin = \y0 + 5*\dt, ymax = 4 + 5*\dt,
            axis line style = thin, ticks = none]   
            
            \filldraw[color = white] (-9,\dt/2) circle (0.1cm) node[]{\textcolor{black}{$t_1 = \frac{l}{4v}$}};

            % пунктирні трикутники
            \addplot[blue, dashed] (-3*\l/4,0) -- (\l/4,\h);            
            \addplot[blue, dashed] (\l/4,\h) -- (5*\l/4,0);

            \addplot[orange!75!black, dashed] (-5*\l/4,0) -- (-\l/4,\h);            
            \addplot[orange!75!black, dashed] (-\l/4,\h) -- (3*\l/4,0);


            % струна
            \addplot[red, thick, samples=50, domain=-10:-5*\l/4] {0};            
            \addplot[red, thick] (-5*\l/4,0) -- (-3*\l/4,\h/2);            
            \addplot[red, thick] (-3*\l/4,\h/2) -- (-\l/4,3*\h/2);            
            \addplot[red, thick] (-\l/4,3*\h/2) -- (\l/4,3*\h/2);            
            \addplot[red, thick] (\l/4,3*\h/2) -- (3*\l/4,\h/2);
            \addplot[red, thick] (3*\l/4,\h/2) -- (5*\l/4,0);
            \addplot[red, thick, samples=50, domain=5*\l/4:9.75] {0};
            
        \end{axis}

        \begin{axis} %% t0
            [width = \textwidth,
            axis lines = center,
            %axis y line = none, axis x line = center,
            xlabel = $x$, ylabel = $t$,
            xmin = -10, xmax = 10, ymin = \y0 + 6*\dt, ymax = 4 + 6*\dt,
            axis line style = thin, ticks = none]   
            
            %відмітки на осі Ox 
            \addplot[black, samples=10, domain=-10:10, name path=three] coordinates {(0,-0.1)(0,0.1)}
            node[anchor=130, pos=0.5] {\footnotesize{0}};
            \addplot[black, samples=10, domain=-10:10, name path=three] coordinates {(-\l,-0.1)(-\l,0.1)}
            node[anchor=70, pos=0.5] {\footnotesize{$-l$}};
            \addplot[black, samples=10, domain=-10:10, name path=three] coordinates {(\l,-0.1)(\l,0.1)}
            node[anchor=90, pos=0.5] {\footnotesize{$l$}};
            
            \filldraw[color = white] (-9,\dt/2) circle (0.1cm) node[]{\textcolor{black}{$t_0 = 0$}};
        
            %характеристики
            \addplot[gray, samples=50, domain=\l:2.7*\l] {(-\l + x)/\v};            
            \addplot[gray, samples=50, domain=-0.7*\l:\l] {(\l - x)/\v};            

            \addplot[gray, samples=50, domain=-\l:1.7*\l] {(\l + x)/\v};            
            \addplot[gray, samples=50, domain=-3.7*\l:-\l] {(-\l - x)/\v}; 

            % пунктирний трикутник
            \addplot[green!50!black, dashed, samples=50, domain=-\l:0] {\h*(1 + x/\l)};            
            \addplot[green!50!black, dashed, samples=50, domain=0:\l] {\h*(1 - x/\l)};

            % струна
            \addplot[red, thick, samples=50, domain=-10:-\l] {0};            
            \addplot[red, thick, samples=50, domain=-\l:0] {2*\h*(1 + x/\l)};            
            \addplot[red, thick, samples=50, domain=0:\l] {2*\h*(1 - x/\l)};
            \addplot[red, thick, samples=50, domain=\l:9.75] {0};            

        \end{axis}

    \end{tikzpicture}
\end{figure}

\clearpage

%\end{document}
%\documentclass[a4paper, 14pt]{extreport}

%\usepackage{StyleMMF}

%\setcounter{chapter}{7}

%\begin{document}

%\chapter{Метод характеристик і формула Даламбера: нескінченна пряма, півнескінченна пряма та відрізок. Метод непарного продовження.}

\section[Задача №8.2]{8.2}

\textit{Зобразити графічно поле зміщень і поле швидкостей нескінченної струни в характерні послідовні моменти часу, якщо початкове відхилення (зміщення) дорівнює нулю, початкова швидкість всіх точок струни на деякому відрізку довжиною $2l$ однакова і дорівнює $\nu_0$, а в усіх інших точках дорівнює нулю. У який кінцевий стан переходить струна в результаті такого процесу? З точки зору механіки системи частинок результат є парадоксальним: у початковий момент тілу був переданий імпульс (у поперечному напрямі до струни), а в кінцевому стані струна нерухома, замість того щоб рухатись рівномірно. Зобразіть також вигляд поля зміщень і поля швидкостей при наближенні до границі: $l \to 0$, $\nu_0 \to \infty$ при фіксованому $t$, якщо переданий струні імпульс залишається сталим.}


%\end{document}
%\documentclass[a4paper, 14pt]{extreport}

%\usepackage{../../main/StyleMMF}

%\setcounter{chapter}{7}

%\begin{document}

%\chapter{Метод характеристик і формула Даламбера: нескінченна пряма, півнескінченна пряма та відрізок. Метод непарного продовження.}

\textbf{\large Метод непарного продовження для півнескінченної та скінченної струни}

\section[Задача №8.3]{8.3}

\textit{Зобразити графічно поле зміщень півнескінченної струни у характерні послідовні моменти часу. Початкове відхилення має форму прямокутного трикутника, більшим катетом служить положення рівноваги струни, а вершина гострого кута орієнтована в бік кінця струни. Початкова швидкість дорівнює нулю. Кінець струни а) вільний (відносно поперечних зміщень), б) закріплений нерухомо.}

\begin{center}
    \large{\textbf{Розв'язок}}
\end{center}

\begin{figure}[h!]
    \centering

    \begin{tikzpicture}

        \tikzmath{\l = 5; \h = 2.75; \v = 1/4; \y0 = -31.5; \dt = 5;}

        \begin{axis} %% t6
            [width = \textwidth,
            %axis lines = center,
            axis y line = none, axis x line = center,
            ylabel = $t$, xlabel = $x$,
            xmin = -10, xmax = 10, ymin = \y0, ymax = 4,
            axis line style = thin, ticks = none]   
            
        \end{axis}

        \begin{axis} %% t5
            [width = \textwidth,
            axis y line = none, axis x line = center,
            xlabel = $x$,
            xmin = -10, xmax = 10, ymin = \y0 + \dt, ymax = 4 + \dt,
            axis line style = thin, ticks = none]   
            
            \filldraw[color = white] (-9,\dt/2) circle (0.1cm) node[]{\textcolor{black}{$t_5 = \frac{5l}{4v}$}};
            
            % парне продовження
            \addplot[blue!75!black, dashed] coordinates {(0.25*\l,0)(0.25*\l,\h)};
            \addplot[blue!75!black, dashed] coordinates {(0.25*\l,\h)(1.25*\l,0)};            

            % збурення струни
            \addplot[yellow!50!black, dashed] coordinates {(-1.25*\l,0)(-0.25*\l,\h)};            
            \addplot[yellow!50!black, dashed] coordinates {(-0.25*\l,\h)(-0.25*\l,0)};

            %струна
            \addplot[red, thick] coordinates {(0.25*\l,0)(0.25*\l,\h)};            
            \addplot[red, thick] coordinates {(0.25*\l,\h)(1.25*\l,0)};

        \end{axis}

        \begin{axis} %% t4
            [width = \textwidth,
            axis y line = none, axis x line = center,
            xlabel = $x$,
            xmin = -10, xmax = 10, ymin = \y0 + 2*\dt,ymax = 4 + 2*\dt,
            axis line style = thin, ticks = none]   
            
            \filldraw[color = white] (-9,\dt/2) circle (0.1cm) node[]{\textcolor{black}{$t_4 = \frac{l}{v}$}};
            
            % парне продовження
            \addplot[blue!75!black, dashed] coordinates {(0,0)(0,\h)};
            \addplot[blue!75!black, dashed] coordinates {(0,\h)(\l,0)};  

            % збурення струни
            \addplot[yellow!50!black, dashed] coordinates {(0,0)(0,\h)};            
            \addplot[yellow!50!black, dashed] coordinates {(0,\h)(-\l,0)};

            %струна
            \addplot[red, thick] coordinates {(0,0)(0,\h)};            
            \addplot[red, thick] coordinates {(0,\h)(\l,0)};

        \end{axis}

        \begin{axis} %% t3
            [width = \textwidth,
            axis y line = none, axis x line = center,
            xlabel = $x$,
            xmin = -10, xmax = 10, ymin = \y0 + 3*\dt, ymax = 4 + 3*\dt,
            axis line style = thin, ticks = none]   
            
            \filldraw[color = white] (-9,\dt/2) circle (0.1cm) node[] {\textcolor{black}{$t_3 = \frac{3l}{4v}$}};
            
            % парне продовження
            \addplot[blue!75!black, dashed] coordinates {(-0.25*\l,0)(-0.25*\l,\h)};
            \addplot[blue!75!black, dashed] coordinates {(-0.25*\l,\h)(0.75*\l,0)};            

            % збурення струни
            \addplot[yellow!50!black, dashed] coordinates {(-0.75*\l,0)(0.25*\l,\h)};            
            \addplot[yellow!50!black, dashed] coordinates {(0.25*\l,\h)(0.25*\l,0)};

            %струна
            \addplot[red, thick] coordinates {(0,3*\h/2)(0.25*\l,3*\h/2)};            
            \addplot[red, thick] coordinates {(0.25*\l,3*\h/2)(0.25*\l,\h/2)};
            \addplot[red, thick] coordinates {(0.25*\l,\h/2)(0.75*\l,0)};

        \end{axis}

        \begin{axis} %% t2
            [width = \textwidth,
            axis y line = none, axis x line = center,
            xlabel = $x$,
            xmin = -10, xmax = 10, ymin = \y0 + 4*\dt, ymax = 4 + 4*\dt,
            axis line style = thin, ticks = none]   
            
            \filldraw[color = white] (-9,\dt/2) circle (0.1cm) node[]{\textcolor{black}{$t_2 = \frac{l}{2v}$}};
            
            % парне продовження
            \addplot[blue!75!black, dashed] coordinates {(-0.5*\l,0)(-0.5*\l,\h)};
            \addplot[blue!75!black, dashed] coordinates {(-0.5*\l,\h)(0.5*\l,0)};            

            % збурення струни
            \addplot[yellow!50!black, dashed] coordinates {(-0.5*\l,0)(0.5*\l,\h)};            
            \addplot[yellow!50!black, dashed] coordinates {(0.5*\l,\h)(0.5*\l,0)};

            %струна
            \addplot[red, thick] coordinates {(0,\h)(0.5*\l,\h)};            
            \addplot[red, thick] coordinates {(0.5*\l,\h)(0.5*\l,0)};
            
        \end{axis}

        \begin{axis} %% t1
            [width = \textwidth,
            axis y line = none, axis x line = center,
            xlabel = $x$,
            xmin = -10, xmax = 10, ymin = \y0 + 5*\dt, ymax = 4 + 5*\dt,
            axis line style = thin, ticks = none]   
            
            \filldraw[color = white] (-9,\dt/2) circle (0.1cm) node[]{\textcolor{black}{$t_1 = \frac{l}{4v}$}};

            % парне продовження
            \addplot[blue!75!black, dashed] coordinates {(-0.75*\l,0)(-0.75*\l,\h)};
            \addplot[blue!75!black, dashed] coordinates {(-0.75*\l,\h)(0.25*\l,0)};            

            % збурення струни
            \addplot[yellow!50!black, dashed] coordinates {(-0.25*\l,0)(0.75*\l,\h)};            
            \addplot[yellow!50!black, dashed] coordinates {(0.75*\l,\h)(0.75*\l,0)};

            %струна
            \addplot[red, thick] coordinates {(0,\h/2)(0.25*\l,\h/2)};            
            \addplot[red, thick] coordinates {(0.25*\l,\h/2)(0.75*\l,\h)};            
            \addplot[red, thick] coordinates {(0.75*\l,\h)(0.75*\l,0)};
            
        \end{axis}

        \begin{axis} %% t0
            [width = \textwidth,
            axis lines = center,
            %axis y line = none, axis x line = center,
            xlabel = $x$, ylabel = $t$,
            xmin = -10, xmax = 10, ymin = \y0 + 6*\dt, ymax = 4 + 6*\dt,
            axis line style = thin, ticks = none]   
            
            %відмітки на осі Ox 
            \addplot[black, samples=10, domain=-10:10, name path=three] coordinates {(0,-0.1)(0,0.1)}
            node[anchor=130, pos=0.5] {\footnotesize{0}};
            \addplot[black, samples=10, domain=-10:10, name path=three] coordinates {(-\l,-0.1)(-\l,0.1)}
            node[anchor=70, pos=0.5] {\footnotesize{$-l$}};
            \addplot[black, samples=10, domain=-10:10, name path=three] coordinates {(\l,-0.1)(\l,0.1)}
            node[anchor=90, pos=0.5] {\footnotesize{$l$}};
            
            \filldraw[color = white] (-9,\dt/2) circle (0.1cm) node[]{\textcolor{black}{$t_0 = 0$}};
        
            %характеристики
            \addplot[gray, samples=50, domain=0:1.7*\l] {x/\v};            
            \addplot[gray, samples=50, domain=-0.7*\l:\l] {(\l - x)/\v};            

            \addplot[gray, samples=50, domain=-\l:1.7*\l] {(\l + x)/\v};            
            \addplot[gray, samples=50, domain=-2.7*\l:0] {-x/\v}; 

            % парне продовження
            \addplot[blue!75!black, dashed] coordinates {(-\l,0)(-\l,\h)};
            \addplot[blue!75!black, dashed] coordinates {(-\l,\h)(0,0)};  

            % збурення струни
            \addplot[yellow!50!black, dashed] coordinates {(0,0)(\l,\h)};            
            \addplot[yellow!50!black, dashed] coordinates {(\l,\h)(\l,0)};

            %струна
            \addplot[red, thick] coordinates {(0,0)(\l,\h)};            
            \addplot[red, thick] coordinates {(\l,\h)(\l,0)};

        \end{axis}
    \end{tikzpicture}

    \caption{Парне продовження}
\end{figure}

\begin{figure}[h]
    \centering

    \begin{tikzpicture}

        \tikzmath{\l = 5; \h = 2.25; \v = 1/4; \y0 = -33; \dt = 5;}

        \begin{axis} %% t6
            [width = \textwidth,
            %axis lines = center,
            axis y line = none, axis x line = center,
            ylabel = $t$, xlabel = $x$,
            xmin = -10, xmax = 10, ymin = \y0, ymax = 2,
            axis line style = thin, ticks = none]   
            
        \end{axis}

        \begin{axis} %% t5
            [width = \textwidth,
            axis y line = none, axis x line = center,
            xlabel = $x$,
            xmin = -10, xmax = 10, ymin = \y0 + \dt, ymax = 2 + \dt,
            axis line style = thin, ticks = none]   
            
            \filldraw[color = white] (-9,\dt/2) circle (0.1cm) node[]{\textcolor{black}{$t_5 = \frac{5l}{4v}$}};
            
            % непарне продовження
            \addplot[blue!75!black, dashed] coordinates {(0.25*\l,0)(0.25*\l,-\h)};
            \addplot[blue!75!black, dashed] coordinates {(0.25*\l,-\h)(1.25*\l,0)};            

            % збурення струни
            \addplot[yellow!50!black, dashed] coordinates {(-1.25*\l,0)(-0.25*\l,\h)};            
            \addplot[yellow!50!black, dashed] coordinates {(-0.25*\l,\h)(-0.25*\l,0)};

            %струна
            \addplot[red, thick] coordinates {(0.25*\l,0)(0.25*\l,\h)};            
            \addplot[red, thick] coordinates {(0.25*\l,\h)(1.25*\l,0)};

        \end{axis}

        \begin{axis} %% t4
            [width = \textwidth,
            axis y line = none, axis x line = center,
            xlabel = $x$,
            xmin = -10, xmax = 10, ymin = \y0 + 2*\dt,ymax = 2 + 2*\dt,
            axis line style = thin, ticks = none]   
            
            \filldraw[color = white] (-9,\dt/2) circle (0.1cm) node[]{\textcolor{black}{$t_4 = \frac{l}{v}$}};
            
            % непарне продовження
            \addplot[blue!75!black, dashed] coordinates {(0,0)(0,-\h)};
            \addplot[blue!75!black, dashed] coordinates {(0,-\h)(\l,0)};  

            % збурення струни
            \addplot[yellow!50!black, dashed] coordinates {(0,0)(0,\h)};            
            \addplot[yellow!50!black, dashed] coordinates {(0,\h)(-\l,0)};

            %струна
            \addplot[red, thick] coordinates {(0,0)(0,\h)};            
            \addplot[red, thick] coordinates {(0,\h)(\l,0)};

        \end{axis}

        \begin{axis} %% t3
            [width = \textwidth,
            axis y line = none, axis x line = center,
            xlabel = $x$,
            xmin = -10, xmax = 10, ymin = \y0 + 3*\dt, ymax = 2 + 3*\dt,
            axis line style = thin, ticks = none]   
            
            \filldraw[color = white] (-9,\dt/2) circle (0.1cm) node[] {\textcolor{black}{$t_3 = \frac{3l}{4v}$}};
            
            % непарне продовження
            \addplot[blue!75!black, dashed] coordinates {(-0.25*\l,0)(-0.25*\l,-\h)};
            \addplot[blue!75!black, dashed] coordinates {(-0.25*\l,-\h)(0.75*\l,0)};            

            % збурення струни
            \addplot[yellow!50!black, dashed] coordinates {(-0.75*\l,0)(0.25*\l,\h)};            
            \addplot[yellow!50!black, dashed] coordinates {(0.25*\l,\h)(0.25*\l,0)};

            %струна
            \addplot[red, thick] coordinates {(0,0)(0.25*\l,\h/2)};            
            \addplot[red, thick] coordinates {(0.25*\l,\h/2)(0.75*\l,0)};

        \end{axis}

        \begin{axis} %% t2
            [width = \textwidth,
            axis y line = none, axis x line = center,
            xlabel = $x$,
            xmin = -10, xmax = 10, ymin = \y0 + 4*\dt, ymax = 2 + 4*\dt,
            axis line style = thin, ticks = none]   
            
            \filldraw[color = white] (-9,\dt/2) circle (0.1cm) node[]{\textcolor{black}{$t_2 = \frac{l}{2v}$}};
            
            % непарне продовження
            \addplot[blue!75!black, dashed] coordinates {(-0.5*\l,0)(-0.5*\l,-\h)};
            \addplot[blue!75!black, dashed] coordinates {(-0.5*\l,-\h)(0.5*\l,0)};            

            % збурення струни
            \addplot[yellow!50!black, dashed] coordinates {(-0.5*\l,0)(0.5*\l,\h)};            
            \addplot[yellow!50!black, dashed] coordinates {(0.5*\l,\h)(0.5*\l,0)};

            %струна
            \addplot[red, thick] coordinates {(0,0)(0.5*\l,\h)};            
            \addplot[red, thick] coordinates {(0.5*\l,\h)(0.5*\l,0)};
            
        \end{axis}

        \begin{axis} %% t1
            [width = \textwidth,
            axis y line = none, axis x line = center,
            xlabel = $x$,
            xmin = -10, xmax = 10, ymin = \y0 + 5*\dt, ymax = 2 + 5*\dt,
            axis line style = thin, ticks = none]   
            
            \filldraw[color = white] (-9,\dt/2) circle (0.1cm) node[]{\textcolor{black}{$t_1 = \frac{l}{4v}$}};

            % непарне продовження
            \addplot[blue!75!black, dashed] coordinates {(-0.75*\l,0)(-0.75*\l,-\h)};
            \addplot[blue!75!black, dashed] coordinates {(-0.75*\l,-\h)(0.25*\l,0)};            

            % збурення струни
            \addplot[yellow!50!black, dashed] coordinates {(-0.25*\l,0)(0.75*\l,\h)};            
            \addplot[yellow!50!black, dashed] coordinates {(0.75*\l,\h)(0.75*\l,0)};

            %струна
            \addplot[red, thick] coordinates {(0,0)(0.25*\l,\h/2)};            
            \addplot[red, thick] coordinates {(0.25*\l,\h/2)(0.75*\l,\h)};            
            \addplot[red, thick] coordinates {(0.75*\l,\h)(0.75*\l,0)};
            
        \end{axis}

        \begin{axis} %% t0
            [width = \textwidth,
            axis lines = center,
            %axis y line = none, axis x line = center,
            xlabel = $x$, ylabel = $t$,
            xmin = -10, xmax = 10, ymin = \y0 + 6*\dt, ymax = 2 + 6*\dt,
            axis line style = thin, ticks = none]   
            
            %відмітки на осі Ox 
            \addplot[black, samples=10, domain=-10:10, name path=three] coordinates {(0,-0.1)(0,0.1)}
            node[anchor=130, pos=0.5] {\footnotesize{0}};
            \addplot[black, samples=10, domain=-10:10, name path=three] coordinates {(-\l,-0.1)(-\l,0.1)}
            node[anchor=70, pos=0.5] {\footnotesize{$-l$}};
            \addplot[black, samples=10, domain=-10:10, name path=three] coordinates {(\l,-0.1)(\l,0.1)}
            node[anchor=90, pos=0.5] {\footnotesize{$l$}};
            
            \filldraw[color = white] (-9,\dt/2) circle (0.1cm) node[]{\textcolor{black}{$t_0 = 0$}};
        
            %характеристики
            \addplot[gray, samples=50, domain=0:1.7*\l] {x/\v};            
            \addplot[gray, samples=50, domain=-0.7*\l:\l] {(\l - x)/\v};            

            \addplot[gray, samples=50, domain=-\l:1.7*\l] {(\l + x)/\v};            
            \addplot[gray, samples=50, domain=-2.7*\l:0] {-x/\v}; 

            % непарне продовження
            \addplot[blue!75!black, dashed] coordinates {(-\l,0)(-\l,-\h)};
            \addplot[blue!75!black, dashed] coordinates {(-\l,-\h)(0,0)};  

            % збурення струни
            \addplot[yellow!50!black, dashed] coordinates {(0,0)(\l,\h)};            
            \addplot[yellow!50!black, dashed] coordinates {(\l,\h)(\l,0)};

            %струна
            \addplot[red, thick] coordinates {(0,0)(\l,\h)};            
            \addplot[red, thick] coordinates {(\l,\h)(\l,0)};

        \end{axis}
    \end{tikzpicture}

    \caption{Непарне продовження}
\end{figure}

%\end{document}

\chapter{Використання загального розв’язку хвильового рівняння у вигляді суперпозиції зустрічних хвиль. Нестаціонарна задача розсіяння.}
%\documentclass[a4paper, 14pt]{extreport}

%\usepackage{StyleMMF}

%\setcounter{chapter}{8}

%\begin{document}

%\chapter{Використання загального розв’язку хвильового рівняння у вигляді суперпозиції зустрічних хвиль. Нестаціонарна задача розсіяння.}

\section[Задача №9.1]{9.1}

\textit{Півнескінченна струна (сила натягу $T_0$, швидкість хвиль $v$) з вільним кінцем перебувала у стані рівноваги. Починаючи з моменту часу $t=0$, на її кінець діє у поперечному напрямі задана сила $F(t)$. Знайти розв’язок задачі про вимушені коливання струни у квадратурах, а також знайти поле зміщень у явному вигляді і зобразити графічно форму струни, якщо: а) $F(t) = F_0$, б) $F(t) = F_0 \cos\omega t$, в) $F(t) = F_0 \sin\omega t$\\
Задача є прикладом так званої задачі про поширення межового режиму: задачі для півнескінченної струни з неоднорідною межовою умовою. Указівка: задача відшукання форми хвилі, створеної таким джерелом, зводиться до диференціального рівняння першого порядку; проблема знаходження сталої інтегрування вирішується, якщо врахувати умову неперервності хвильового поля на передньому фронті хвилі, тобто на межі областей $x > vt$ й $x < vt$.}

\begin{center}
    \large{\textbf{Розв'язок}}
\end{center}

\noindent Формальна постановка задачі:
\begin{equation} %\label{probcond12}
    \left\{ \begin{aligned} %%
            \;&u = u(x,t), \\
            &u_{tt} = v^2 u_{xx}, \\
            &0 \leq x < \infty, t \geq 0 \\
            &u_x(0,t) = F(t)/\beta = f(t),\\
            &u(x,0) = 0, \, u_t(x,0) = 0.
    \end{aligned} \right.
\end{equation}

Шукаємо розв'язок у вигляді:
\begin{equation}
    u(x,t) = g(t - x/v) + h(t + x/v)
\end{equation}

Із початкових умов маємо систему:
\begin{equation}
    \left\{ \begin{aligned}
        \;&u(x,0) = g(-x/v) + h(x/v) = 0,\\
        &u_t(x,0) = g'(-x/v) + h'(x/v) = 0.
\end{aligned} \right.
\end{equation}
Інтегруємо друге рівняння та розв'язуємо лінійну систему
\begin{equation*}
    \begin{gathered}
        \left\{ \begin{aligned}
            \;&g(-\xi) + h(\xi) = 0,\\
            &\int g'(-\xi)\;\mathrm{d}\xi + \int h'(\xi)\;\mathrm{d}\xi = 0;
        \end{aligned} \right.
        \quad\Rightarrow\quad
        \left\{ \begin{aligned}
            \;&g(-\xi) + h(\xi) = 0,\\
            &h(\xi) + C_2 - g(-\xi) + C_1 = 0;
        \end{aligned} \right.
        \quad\Rightarrow\\
        \Rightarrow\quad
        \left\{ \begin{aligned}
            \;&g(-\xi) + h(\xi) = 0,\\
            &-g(-\xi) + h(\xi) = 2\widetilde{C};
        \end{aligned} \right.
        \quad\Rightarrow\quad
        \left\{ \begin{aligned}
            \;&g(-\xi) = -\widetilde{C},\\
            &h(\xi) = \widetilde{C}.
        \end{aligned} \right.
    \end{gathered}
\end{equation*}
Обираємо константу інтегрування $\widetilde{C}$ рівною нулю, тоді 
\begin{equation*}
    \left\{ \begin{aligned}
        \;&g(-x/v) = 0,\\
        &h(x/v) = 0;
    \end{aligned} \right.
    \quad\Rightarrow\quad
    \left\{ \begin{aligned}
        \;&g(\xi) = 0, \text{ при } \xi < 0,\\
        &h(\eta)= 0, \text{ при } \eta > 0.
    \end{aligned} \right.
\end{equation*}

Отже, $h(t + x/v) = 0$ в нашій задачі, адже $x \geq 0$ та $t > 0$, що фізично означає відсутність хвилі, яка поширюється з нескінченность до краю струни (падаючої хвилі).\\
Маємо розв'язок у виді біжучої хвилі, яка створюється межовою умовою.
\begin{equation}
    u(x,t) = g(t - x/v)
\end{equation}

З межової умови визначимо розв'язок
\begin{equation}
    u_x(0,t) = f(t)
    \quad\Rightarrow\quad
    -\frac{1}{v}g'(t) = f(t)
    \quad\Rightarrow\quad
    g(t) = - v \int\limits_0^t f(\tau) \;\mathrm{d}\tau
\end{equation}

Загальний вид розв'язку
\begin{equation}
    u(x,t) = - v \int\limits_0^{t-x/v} f(\tau) \;\mathrm{d}\tau
\end{equation}

Обчислимо розв'язки для визначених межових умов:
\begin{enumerate}
    \item[\text{а})] \[u(x,t) = - \frac{vF_0}{\beta} \int\limits_0^{t-x/v}  \;\mathrm{d}\tau = -\frac{vF_0}{\beta} \left(t - \frac{x}{v}\right),\]
    \item[\text{б})] \[u(x,t) = - \frac{vF_0}{\beta} \int\limits_0^{t-x/v} \sin\omega\tau \;\mathrm{d}\tau = -\frac{vF_0}{\omega\beta} \cos\omega(t - x/v),\]
    \item[\text{в})] \[u(x,t) = - \frac{vF_0}{\beta} \int\limits_0^{t-x/v} \cos\omega\tau \;\mathrm{d}\tau = \frac{vF_0}{\omega\beta} \sin\omega(t - x/v).\]
\end{enumerate} 

%\end{document}
%\documentclass[a4paper, 14pt]{extreport}

%\usepackage{StyleMMF}

%\setcounter{chapter}{8}

%\begin{document}

%\chapter{Використання загального розв’язку хвильового рівняння у вигляді суперпозиції зустрічних хвиль. Нестаціонарна задача розсіяння.}

\section[Задача №9.2]{9.2}

\textit{При $t < t_0$ по півнескінченній струні $0 \geq x < \infty$ у напрямі її кінця поширюється хвиля заданої форми (падаючий «імпульс»), причому передній фронт хвилі при $t \geq t_0$ не досягає кінця струни. Знайти коливання струни при $t > t_0$ і форму відбитого імпульсу для скінченного $t_0$ і $t_0 \to -\infty$. Кінець струни: а) закріплений жорстко; б) зазнає дії сили тертя, пропорційної швидкості. Як пояснити відсутність відбивання при певному значенні коефіцієнта тертя?\\
Указівка: звести до задачі про поширення межового режиму типу задачі 9.1, використати вказівку до цієї задачі та умову, що при $t < t_0$ фронт хвилі не досягає кінця струни.}

\begin{center}
    \large{\textbf{Розв'язок}}
\end{center}

\noindent Формальна постановка задачі:
\begin{equation} \label{withered}
    \left\{ \begin{aligned} 
            \;&u = u(x,t), \\
            &u_{tt} = v^2 u_{xx}, \\
            &0 \leq x \leq \infty, t \geq t_0 \\
            &\text{а)}\; u(0,t) = 0,\\
            &\text{б)}\; \mu u_x(0,t) = u_t(0,t),\\
            &u(x,t) = F_0(vt + x) - \text{падаюча хвиля, для  } t<t_0\\
            &F(t) = 0,\;t < t_0 \text{  - падаюча хвиля не досягне кінця струни до }t_0.\\
    \end{aligned} \right.
\end{equation}
Де $\mu$ - коефіцієнт тертя. 

Повний розв'язок хвильового рівняння представляється комбінацією двох збурень, що поширюються у протилежних напрямках та є фукнціями однієї змінної. У нашому випадку це падаюча та відбита хвиля у відповідному порядку:

\begin{equation}
    u(x,t) = u_{\text{пад}}(x,t) + u_{\text{від}}(x,t) = F(x + tv) + f(x - tv)
\end{equation}

З постановки (\ref{withered}) ми знаємо частину, що відповідає падаючій хвилі:
\begin{equation*}
    u_{\text{пад}}(x,t) = F_0(x + tv)
\end{equation*}

Звідси маємо умову на $u_{\text{від}}$ при $t<t_0$:

\begin{equation} \label{hunger}
    u_{\text{від}}(x,t)=0
\end{equation}

Для випадку \textit{а)} можна було б cкористатися доведенною в лекціях вимогою на непарність $u(x,t)$ по змінній $x$ для приведенної граничної умови. Але розв'яжемо обидва випадки одним методом. Розглянемо точку $x=0$ у момент часу $t>t_0$:

\begin{equation}  \label{slave}
    \begin{aligned} 
            &\text{а)}\;u_{\text{від}}(0,t) + u_{\text{пад}}(0,t) = 0 \quad\Rightarrow\\
            &\Rightarrow\quad u_{\text{від}}(0,t) = f(x - tv)|_{x=0} = f(-tv) = -F_0(tv), \\
            & \\
            &\text{б)}\; \mu u_x(0,t) - u_t(0,t) = 0 \quad\Rightarrow\\
            &\Rightarrow\quad \mu \frac{\partial u_{\text{від}}}{\partial x}\bigg|_{x=0} +\frac{\partial u_{\text{від}}}{\partial t}\bigg|_{x=0} = (\mu + v)f'(-tv) =\\
            &= -(\mu - v)F_0'(tv)
    \end{aligned} 
\end{equation}

З отриманих рівнянь (\ref{slave}) та умови (\ref{hunger}) можна одразу отримати відповіді:

\begin{equation} 
    \begin{aligned} 
            &\text{а)}\;u_{\text{від}}(x,t) =
                \begin{cases}
                    0 & \text{, якщо } t < t_0, \text{ або } x > vt \\
                    -F_0(tv - x) & \text{, якщо } t > t_0,\;x < vt.
                \end{cases}
            & \\
            & \\
            &\text{б)}\;u_{\text{від}}(x,t)=
                \begin{cases}
                    0 & \text{, якщо } t < t_0, \text{ або } x > vt \\
                    \frac{\mu - v}{\mu + v}(F_0(tv - x) + c) & \text{, якщо } t > t_0,\;x < vt.
                \end{cases}
    \end{aligned} 
\end{equation}

Де \textit{c} - константа інтегрування. Накладаючи вимогу неперервності $u_{\text{від}}(0,t_0) = 0$ маємо $c = -F_0(0)$. Якщо $F(\xi)$ є неперервною функцією, то $c = 0$. Ці результати можна переписати у більш винтонченній формі за допомогою тета-функції:

\begin{equation} 
    \begin{aligned} 
            &\text{а)}\;u_{\text{від}}(x,t)=-F_0(tv-x) \Theta\big(v(t - t_0) - x\big)\\
            & \\
            &\text{б)}\;u_{\text{від}}(x,t)=\frac{\mu - v}{\mu + v} \big(F_0(tv-x)-F_0(0)\big) \Theta\big(v(t - t_0) - x\big).
    \end{aligned} 
\end{equation}

При $t_0 \rightarrow - \infty, \quad \Theta\big(v(t - t_0) - x\big) = 1$, тож поле матиме форму: 

\begin{equation} 
    \begin{aligned} 
            &\text{а)}\;u_{\text{від}}(x,t)=-F_0(tv - x)\\
            & \\
            &\text{б)}\;u_{\text{від}}(x,t) = \frac{\mu - v}{\mu + v} (F_0(tv - x) - F_0(0)).
    \end{aligned} 
\end{equation}


%\end{document}

\chapter{Приведення лінійних рівнянь у частинних похідних 2-го порядку з двома змінними до заданого вигляду}
%\documentclass[a4paper, 14pt]{extreport}

%\usepackage{../StyleMMF}

%\setcounter{chapter}{9}

%\begin{document}

%\chapter{Приведення лінійних рівнянь у частинних похідних 2-го порядку з двома змінними до заданого вигляду}

\section[Задача №10.1]{10.1}

\textit{Визначити тип рівняння $u_{xx} + 4u_{xy} + cu_{yy} + u_x = 0$, привести його до канонічного вигляду для $c = 0$ і знайти загальний розв’язок.}

\begin{center}
    \large{\textbf{Розв'язок}}
\end{center}

Загальний вид рівняння:
\begin{equation}
    a_{11}u_{xx} + 2a_{12}u_{xy} + a_{22}u_{yy} + b_1u_x + b_2u_y + cu = 0, \quad \text{або} \quad \hat{L}u + cu = 0
\end{equation}
Тип рівняння визначається визначником матриці, яка складається з коефіцієнтів перед другими похідними. \textit{Фактично оператор $\hat{L}$ є білінійною формою з лінійної алгебри, де замість змінних будуть похідні.}
\begin{equation}
    \Delta = -
    \begin{vmatrix}
        a_{11} & a_{12}\\
        a_{12} & a_{22}
    \end{vmatrix} 
    = a_{12}^2 - a_{11}a_{22} = 2^2 - 1\cdot c = 4 - c 
\end{equation}
При $c = 0$ визначник $\Delta > 0$, тому маємо рівняння гіперболічного типу. 

Суть канонізації -- перейти до нових змінних для яких рівняння прийматиме канонічний вид. Для визначення таких змінних записуємо  спочатку характеристичне рівняння:
\begin{equation}
    a_{11} (\mathrm{d}y)^2 + 2a_{12} \mathrm{d}x\mathrm{d}y + a_{22} (\mathrm{d}x)^2 = 0, \quad \text{або} \quad \frac{\mathrm{d}y}{\mathrm{d}x} = \frac{a_{12} \pm \sqrt{a_{12}^2 - a_{11}a_{22}}}{a_{12}}
\end{equation}
Обидва рівняння приводять до 
\begin{equation}
    y'_1 = 4, \qquad y'_2 = 0.
\end{equation}
Звідси маємо перші інтеграли
\begin{equation}
    \begin{gathered}
        y'_1 = 4 
        \quad\Rightarrow\quad
        y_1 = 4x
        \quad\Rightarrow\quad
        \Phi(x,y) = y - 4x = C_1\\
        y'_2 = 0 
        \quad\Rightarrow\quad
        \Psi(x,y) = y = C_2
    \end{gathered}
\end{equation}
З теорії нові змінні отримаємо формальною заміною $C_1 \to \xi$, $C_2 \to \eta$. Отже, нові змінні
\begin{equation}
    \left\{ \begin{aligned}
        \xi = y - 4x,\\
        \eta = y.
    \end{aligned} \right.
\end{equation}

Далі треба зробити заміну змінних. Для цього окремо випишемо похідні від нових змінних
\begin{equation*}
    \xi_x = -4,\, \xi_y = 1,\, \eta_x = 0,\, \eta_y = 1,\, \xi_{xy} = \eta_{xy} = 0,\, \xi_{xx} = \eta_{xx} = 0,\, \xi_{yy} = \eta_{yy} = 0.
\end{equation*}  

Тепер не важко виконати заміну змінних 
\begin{equation*}
    \begin{gathered}
        u_x = u_\xi \xi_x + u_\eta \eta_x = -4 u_\xi,\\
        u_y = u_\xi \xi_y + u_\eta \eta_y = u_\xi + u_\eta,\\
        u_{xx} = (-4 u_\xi)'_x = -4(u_{\xi\xi} \xi_x + u_{\eta\eta} \eta_x) = 16 u_{\xi\xi},\\
        u_{xy} = (-4 u_\xi)'_y = -4(u_{\xi\xi} \xi_y + u_{\eta\eta} \eta_y) = -4(u_{\xi\xi} + u_{\xi\eta}).
    \end{gathered}
\end{equation*}
Підставляємо отримані вирази в рівняння 
\begin{equation*}
    u_{xx} + 4u_{xy} + u_x = 16 u_{\xi\xi} - 16(u_{\xi\xi} + u_{\xi\eta}) - 4u_\xi = 0
    \quad\Rightarrow\quad
    u_{\xi\eta} + \frac{1}{4}u_\xi = 0
\end{equation*}
Отже, отримали рівняння в канонічному виді
\begin{equation}
    u_{\xi\eta} + \frac{1}{4}u_\xi = 0
\end{equation}

Розв'яжемо отримане рівняння. Легко побачити, що по $\xi$ можна проінтегрувати. 
\begin{equation*}
    u_{\xi\eta} + \frac{1}{4}u_\xi = 0
    \quad\Rightarrow\quad
    \left(u_\eta + \frac{1}{4}u\right)'_\xi = 0
    \quad\Rightarrow\quad
    u_\eta + \frac{1}{4}u = f(\eta)
\end{equation*}
Звідки ми отримали лінійне неоднорідне диференційне рівняння однієї змінної. Розв'яжемо спочатку однорідне рівняння
\begin{equation*}
    \tilde{u}_\eta + \frac{1}{4}\tilde{u} = 0
    \quad\Rightarrow\quad
    \ln\tilde{u} = -\frac{1}{4}\eta + \ln C 
    \quad\Rightarrow\quad
    \tilde{u} = Ce^{-\eta/4} 
\end{equation*}
Варіюєму змінну $C \to C(\eta)$
\begin{equation*}
    u = C(\eta)e^{-\eta/4} 
    \quad\Rightarrow\quad
    C'e^{-\eta/4} = f(\eta)
    \quad\Rightarrow\quad
    C(\eta) = \int f(\eta) e^{\eta/4} \;\mathrm{d}\eta + \gamma
\end{equation*}

Отже, маємо розв'язок рівняння
\begin{equation}
    u(\xi,\eta) = \gamma e^{-\eta/4} + e^{-\eta/4} \cdot \int^\eta f(z) e^{z/4} \;\mathrm{d}z
\end{equation} 


%\end{document}
%\documentclass[a4paper, 14pt]{extreport}

%\usepackage{StyleMMF}

%\setcounter{chapter}{9}

%\begin{document}

%\chapter{Приведення лінійних рівнянь у частинних похідних 2-го порядку з двома змінними до заданого вигляду}

\section[Задача №10.5]{10.5}

\textit{Привести до простішого вигляду рівняння $u_t = a^2(u_{xx} + \alpha u_x) + cu$.}


%\end{document}
%\documentclass[a4paper, 14pt]{extreport}

%\usepackage{StyleMMF}

%\setcounter{chapter}{9}

%\begin{document}

%\chapter{Приведення лінійних рівнянь у частинних похідних 2-го порядку з двома змінними до заданого вигляду}

\section[Задача №10.8]{10.8}

\textit{Привести рівняння $u_{tt} = v^2 (u_{rr} + (2/r) u_r) + cu$ до самоспряженого вигляду: \[\rho(r)u_{tt} = \frac{\partial\;}{\partial r} \left(k(r) \frac{\partial u}{\partial r}\right) - q(r)u.\]}


%\end{document}

\part{РІВНЯННЯ ЛАПЛАСА І ПУАССОНА.}

\chapter{Рівняння Лапласа в прямокутній області.}
%\documentclass[a4paper, 14pt]{extreport}
%
%\usepackage{../../main/StyleMMF}
%
%\setcounter{chapter}{10}
%
%\begin{document}
%
%\chapter{Рівняння Лапласа в прямокутній області.}

\section[Задача №11.1]{11.1}

\textit{Знайти стаціонарний розподіл температури в однорідній прямокутній пластині, якщо вздовж лівої її сторони (довжиною $b$) підтримується заданий розподіл температури, права сторона теплоізольована, а верхня і нижня (довжиною $a$) підтримуються при нульовій температурі. Відповідь запишіть через коефіцієнти Фур’є розподілу температури на лівій стороні, вважаючи їх відомими. Які якісні зміни відбуваються у розв’язку при переході від довгої і вузької пластини ($a \gg b$) до короткої і широкої ($a \ll b$)? Намалюйте для цих випадків графіки функцій, що описують зміну температури в повздовжньому напрямку для кількох перших поперечних мод; функції нормуйте так, щоб на лівій стороні пластини вони приймали однакове значення одиниця. Як змінюється в залежності від співвідношення сторін відносна роль внесків різних поперечних мод у розподіл температури на правій стороні пластини?}

\begin{center}
    \large{\textbf{Розв'язок}}
\end{center}

Для стаціонарної задачі $u \neq u(t)$ рівняння параболічного типу, яке відповідає задачі теплопровідності, перетворюється на на еліптичне. Тобто нам потрібно розглянути задачу, де є дві незалежні просторові змінні. Запишемо постановку задачі:
\begin{equation} \label{cond11,1}
    \left\{ \begin{aligned} 
        \;&u = u(x,y), \\
          &\Delta u = u_{xx} + u_{yy} = 0, \\
          &0 \leq x \leq a,\\ &0 \leq y \leq b, \\
          &u(0,y) = \varphi(y),\\ &u_x(a,y) = 0,\\
          &u(x,0) = 0,\\ &u(x,b) = 0.
    \end{aligned} \right.
\end{equation}

Виконаємо розділення змінних $u(x,y) = X(x)\cdot Y(y)$. Для $Y(y)$ отримаємо багато разів розв'язану задачу Штурма-Ліувілля, а для $X(x)$ -- лінійне рівняння.
\begin{equation} 
    \left\{ \begin{aligned}
        \;&Y'' + \lambda Y = 0, \\ 
          &0 \leq y \leq b, \\
          &Y(0) = 0,\, Y(b) = 0 
    \end{aligned} \right.
    \quad\Rightarrow\quad
    \left\{ \begin{aligned}
        \;& Y_n(y) = \sin k_ny, \\
          & k_n = \sqrt{\lambda_n} = \pi n/b, n \in \mathbb{N} 
    \end{aligned} \right.
\end{equation}
Запишемо розв'язок рівняння для $X(x)$
\begin{equation}
    X'' - k_n^2 X = 0
    \quad\Rightarrow\quad
    X_n(x) = A_n\mathrm{sh}k_nx + B_n\mathrm{ch}k_nx
\end{equation}

Виконуємо зворотню заміну та отримаємо, виконуючи підсумовування по всім модам, загальний розв'язок задачі.
\begin{equation}
    u(x,y) = \sum_{n=1}^{\infty} \left(A_n\mathrm{sh}k_nx + B_n\mathrm{ch}k_nx\right) \sin k_ny
\end{equation}

Залишається із межових умов для змінної $x$ визначити невідомі константи $A_n$ та $B_n$. Маємо
\begin{equation}
    \left\{ \begin{aligned}
        \;&u(0,y) = \sum_{n=1}^{\infty} B_n \sin k_ny = \varphi(y),\\
          &u_x(a,y) = \sum_{n=1}^{\infty} \left(A_nk_n\mathrm{ch}k_na + B_nk_n\mathrm{sh}k_na\right) \sin k_ny = 0.
    \end{aligned} \right.
\end{equation}
В правій частині першого рівняння підставимо розклад межової умови в ряд Фур'є, який вважається відомим, та визначимо $B_n$
\begin{equation}
    \sum_{n=1}^{\infty} B_n \sin k_ny = \frac{2}{b} \sum_{n=1}^{\infty} \varphi_n \sin k_ny
    \quad\Rightarrow\quad 
    B_n = \frac{2}{b} \varphi_n
\end{equation} 
З другої, однорідної, межової умови маємо
\begin{equation}
    A_n\mathrm{ch}k_na + B_n\mathrm{sh}k_na = 0
    \quad\Rightarrow\quad
    A_n = - B_n \mathrm{th}k_na = -\frac{2}{b}\varphi_n \mathrm{th}k_na
\end{equation}
Підставляємо отримані значення в загальний розв'язок
\begin{equation} \label{gensol11,1}
    u(x,y) = \frac{2}{b}\sum_{n=1}^{\infty} \varphi_n \big(\mathrm{ch}k_nx - \mathrm{th}(k_na)\mathrm{sh}k_nx\big) \sin k_ny,
\end{equation}
або, скориставшись однією з властивостей гіперболічних функцій, запишемо
\begin{equation}
    u(x,y) = \frac{2}{b}\sum_{n=1}^{\infty} \varphi_n \cdot \frac{\mathrm{ch}\big(k_n(x-a)\big)}{\mathrm{ch}k_na} \sin k_ny
\end{equation}

Розглянемо, використовуючи формулу (\ref{gensol11,1}), граничні випадки: а) довгої і вузької пластини, б) короткої і широкої.\\
a) $a \gg b$
\begin{equation*}
    \mathrm{th}k_na = \mathrm{th}(\pi na/b)\bigg|_{a \gg b} \to 1
\end{equation*}
Таким чином розв'язок переходить в 
\begin{equation}
    u(x,y) = \frac{2}{b}\sum_{n=1}^{\infty} \varphi_n \big(\mathrm{ch}k_nx - \mathrm{sh}k_nx\big) \sin k_ny = \frac{2}{b}\sum_{n=1}^{\infty} \varphi_n e^{-k_nx} \sin k_ny,
\end{equation}
що відповідає рівнянню теплопровідності для одновимірного випадку; температура спадає за експоненційним законом при віддалені від джерела   

б) $a \ll b$
\begin{equation*}
    \mathrm{th}(\pi na/b)\bigg|_{a \ll b} \to 0, \quad \mathrm{ch}(\pi nx/b)\bigg|_{b\to\infty} \to 1,
\end{equation*}
оскільки $x \leq a$, то умову $a \ll b$ можна замінити на $b\to\infty$

Отже, при зменшенні $b$ зменшуються втрати теплоти, оскільки ширина пластинки набагато менша за довжину джерела.\\
Таким чином розв'язок переходить в 
\begin{equation}
    u(x,y) = \frac{2}{b}\sum_{n=1}^{\infty} \mathrm{ch}k_nx\sin k_ny
\end{equation}

%\end{document}
%\documentclass[a4paper, 14pt]{extreport}

%\usepackage{StyleMMF}

%\setcounter{chapter}{10}

%\begin{document}

%\chapter{Рівняння Лапласа в прямокутній області.}

\section[Задача №11.3]{11.3}

\textit{Знайти електростатичний потенціал всередині області, обмеженої провідними пластинами $y=0, y=b, x=0$, якщо пластина $x=0$ заряджена до потенціалу $V$, а інші -- заземлені. Заряди всередині області відсутні. Розв’язком якої задачі є знайдена функція у півпросторі $x>0$?\\
Вказівка. Це приклад задачі для рівняння Лапласа в необмеженій області. Подумайте, яку умову слід накласти на розв’язок при
$x \to +\infty$, щоб для $V=0$ задача мала лише нульовий розв’язок (в іншому разі розв’язок задачі не буде єдиним).\\
Ряд просумувати.\\
Указівка: скористайтесь формулою для суми геометричної прогресії.}


%\end{document}

\chapter{Функції Гріна звичайних диференціальних задач}
%\documentclass[a4paper, 14pt]{extreport}

%\usepackage{StyleMMF}

%\setcounter{chapter}{11}

%\begin{document}

%\chapter{Функції Гріна звичайних диференціальних задач}

\section[Задача №12.1]{12.1}

\textit{Функція Гріна $G(t)$ задачі Коші для рівняння гармонічного осцилятора \[y'' + \omega^2y = f(t), \, t \geq 0, \, y(0)=y_0, \, y'(0)=\nu_0\] є розв’язком цієї задачі при $\nu_0 = 1 , \, y_0 = 1$ і $f(t) = 0$. Тобто $y = G(t)$ задовольняє умови \[y'' + \omega^2y = 0, \, t \geq 0, \, y(0)=1, \, y'(0)=1\] Знайдіть явний вигляд функції Гріна сцилятора; чи зберігає вона смисл при $\omega \to 0$?}


%\end{document}
%\documentclass[a4paper, 14pt]{extreport}
%
%\usepackage{../../main/StyleMMF}
%
%\setcounter{chapter}{11}
%
%\begin{document}
%
%\chapter{Функції Гріна звичайних диференціальних задач}

\section[Задача №12.2]{12.2}

\textit{Користуючись означенням функції Гріна $G(t)$, але не використовуючи її явного вигляду, показати безпосередньою підстановкою в умови задачі, що функція \[y(t) = \int\limits_0^t G(t - t') f(t') \;\mathrm{d}t' + y'(0) G(t) + y(0) G'(t)\] є розв’язком задачі про вимушені коливання гармонічного осцилятора при $t>0$ під дією узагальненої сили $f(t)$ з початковими умовами $y(0)=y_0, \, y'(0)=\nu_0$. Розв’язками яких частинних задач є окремі доданки цього виразу?}

\begin{center}
    \large{\textbf{Розв'язок}}
\end{center}

Закон руху  
\begin{equation} \label{sol12,2}
    y(t) = \int\limits_0^t G(t - t') f(t') \;\mathrm{d}t' + y'(0) G(t) + y(0) G'(t)
\end{equation}
є розв'язком задачі:
\begin{equation} \label{cond12,2}
    \left\{ \begin{aligned}
        \;&y'' + \omega^2y = 0,\, t \geq 0,\\
          &y(0) = y_0,\, y'(0) = v_0.
    \end{aligned} \right.
\end{equation}

Обчислимо першу похідну по часу від розв'язку (\ref{sol12,2})
\begin{equation*} 
    y'(t) = G(0)f(t) + \int\limits_0^t G'(t - t') f(t') \;\mathrm{d}t' + y'(0) G'(t) + y(0) G''(t) \textcolor{red}{=}
\end{equation*}
За означенням функції Гріна $G(t)$ (\ref{cond12,1})
\begin{equation*}
    G''(t) = -\omega^2 G(t), \quad G(0) = 0, \quad G'(0) = 1
\end{equation*}
Підставимо $G''(t)$ та $G(0)$
\begin{equation*} 
    \textcolor{red}{=} \int\limits_0^t G'(t - t') f(t') \;\mathrm{d}t' + y'(0) G'(t) - \omega^2 y(0)G(t) 
\end{equation*}

Аналогічно друга похідна
\begin{equation*} 
    \begin{gathered}
        y''(t) = G'(0) f(t) + \int\limits_0^t G''(t - t') f(t') \;\mathrm{d}t' + y'(0) G''(t) - \omega^2 y(0) G'(t) =\\
        = f(t) - \omega^2 \left(\int\limits_0^t G(t - t') f(t') \;\mathrm{d}t' + y'(0) G(t) + y(0) G'(t) \right) = f(t) - \omega^2 y(t)
    \end{gathered}
\end{equation*}

Підставимо другу похідну в рівняння
\begin{equation*}
    y'' + \omega^2 y = f(t) - \omega^2 y + \omega^2 y \equiv f(t)
\end{equation*}
Таким чином (\ref{sol12,2}) задовільняє рівняння (\ref{cond12,2}) 

Визначимо для яких задач є розв'язками кожен з доданків (\ref{sol12,2}). Для цього треба покласти 2 з 3 параметрів (зовнішня сила та початкові умови) рівними нулю.
\begin{enumerate}
    \item $y(0) = 0, y'(0) = 0$
    \[y(t) = \int\limits_0^t G(t - t') f(t') \;\mathrm{d}t' \quad \text{є розв'язком задачі:}\] 
    \begin{equation}
        \left\{ \begin{aligned}
            \;&y'' + \omega^2y = f(t),\, t \geq 0,\\
              &y(0) = 0,\, y'(0) = 0.
        \end{aligned} \right.
    \end{equation}
    \item $f(t) = 0, y(0) = 0$
    \[y(t) = y'(0) G(t) \quad \text{є розв'язком задачі:}\] 
    \begin{equation}
        \left\{ \begin{aligned}
            \;&y'' + \omega^2y = 0,\, t \geq 0,\\
              &y(0) = 0,\, y'(0) = 1.
        \end{aligned} \right.
    \end{equation}
    \item $f(t) = 0, y'(0) = 0$
    \[y(t) = y(0) G'(t) \quad \text{є розв'язком задачі:}\] 
    \begin{equation}
        \left\{ \begin{aligned}
            \;&y'' + \omega^2y = 0,\, t \geq 0,\\
              &y(0) = 1,\, y'(0) = 0.
        \end{aligned} \right.
    \end{equation}
\end{enumerate}

%\end{document}
%\documentclass[a4paper, 14pt]{extreport}
%
%\usepackage{../../main/StyleMMF}
%
%\setcounter{chapter}{11}

%\begin{document}

%\chapter{Функції Гріна звичайних диференціальних задач}

\section[Задача №12.5]{12.5}

\textit{Функція Гріна $G(x,x')$ крайової задачі для одновимірного рівняння Гельмгольца $u'' - \mu^2u = -f(x), \, u(0) = 0, \, |u| < \infty $ при $x \to \infty$ за означенням є неперервним розв’язком цієї задачі для $f(x) = \delta(x-x'), \, 0 < x' < \infty$.\\
а) Знайти функцію Гріна цієї задачі шляхом зшивання розв’язків однорідного рівняння і подальшого нормування (для даної задачі можливі принаймні три різні способи нормування розв’язку, які?).\\
б) Знайти функцію Гріна $G(x,x')$ крайової задачі для одновимірного рівняння Гельмгольца $u'' - \mu^2u = -f(x), \, x \in \mathbb{R}, \, |u| < \pm\infty $ при $x \to \pm\infty$ – шляхом граничного переходу $x, x' \to \infty$ при сталому $x-x'$ у $G(x,x')$, одержаній у пункті а) цієї задачі.\\
Дайте фізичну інтерпретацію знайдених функцій Гріна у термінах стаціонарної дифузії частинок зі скінченним часом життя. Якою є залежність від кожного з аргументів функції Гріна та симетрія відносно їх перестановки? Чому в одних випадках функція Гріна залежить від кожного з аргументів окремо, а в інших – тільки від їх різниці?}

\begin{center}
    \large{\textbf{Розв'язок}}
\end{center}

Постановка задачі:
\begin{equation} \label{cond12,5}
    \left\{ \begin{aligned}
        \;&u'' + \mu^2u = -f(x),\, t \geq 0,\\
          &u(0) = 0,\\
          & |u| < \infty \text{ при } x \to +\infty.
    \end{aligned} \right.
\end{equation}

Розв'язок однорідного рівняння:
\begin{equation}
    u_0(x) = C_1 e^{-\mu x} + C_2 e^{\mu x}
\end{equation}
Знайдемо вигляд розв'язків, що задовольняють також і межовим умовам. Перший розв'язок знаходимо прямою підстановкою:
\begin{equation*}
    u(0) = C_1 + C_2 = 0 \quad\Rightarrow\quad C_1 = - C_2 = 1 \quad\Rightarrow\quad u_1(x) = e^{-\mu x} - e^{\mu x}
\end{equation*}
Для другого розв'язку треба покласти $C_2$ рівним нулю, щоб позбавитись розбіжного доданку.
\begin{equation*}
    |u(+\infty)| < \infty \quad\Rightarrow\quad u_2(x) = e^{-\mu x}
\end{equation*}
Отже, маємо два розв'язки з яких зшиванням побудуємо функцію Гріна
\begin{equation}
    u_1(x) = e^{-\mu x} - e^{\mu x}, \quad u_2(x) = e^{-\mu x}
\end{equation}

Визначимо функцію Гріна за формулою
\begin{equation} \label{green-draft}
    G(x,x') = 
    \left\{ \begin{aligned}
        \;& \varphi(x') \big( e^{-\mu x} - e^{\mu x} \big), \; 0 \leq x \leq x' \\
          & \psi(x') e^{-\mu x}, \; x' \leq x \leq \infty,
    \end{aligned} \right.
\end{equation}
де $\varphi(x)$ та $\psi(x)$ -- гарні функції, та задовольняє умовам
\begin{equation} \label{green-cond}
    \left\{ \begin{aligned}
        \;& G(x,x')\bigg|_{x = x'+0} = G(x,x')\bigg|_{x = x'-0},\\
          & \frac{\partial \;}{\partial x}G(x,x')\bigg|_{x = x'+0} - \frac{\partial \;}{\partial x} G(x,x')\bigg|_{x = x'-0} = 1.
    \end{aligned} \right.
\end{equation}

Підставимо (\ref{green-draft}) в (\ref{green-cond}) та розв'яжемо отриману ситему рівнянь відносно невідомих функцій.
\begin{equation} 
    \left\{ \begin{aligned}
        \;& \psi(x') e^{-\mu x'} - \varphi(x') (e^{-\mu x'} - e^{\mu x'}) = 0,\\
          & -\mu \psi(x') e^{-\mu x'} + \mu \varphi(x') (e^{-\mu x'} + e^{\mu x'}) = 1.
    \end{aligned} \right.
\end{equation}
Поділимо на $\mu$ друге рівняння та додамо до першого
\begin{equation*} 
    \left\{ \begin{aligned}
        \;& \varphi(x') (e^{-\mu x'} + e^{\mu x'} - e^{-\mu x'} + e^{\mu x'}) = \frac{1}{\mu},\\
          & \psi(x') e^{-\mu x'} = \varphi(x') (e^{-\mu x'} - e^{\mu x'});
    \end{aligned} \right.
    \quad\Rightarrow\;
    \left\{ \begin{aligned}
        \;& 2\varphi(x') e^{\mu x'} = \frac{1}{\mu},\\
          & \psi(x')  = \varphi(x') (1 - e^{2\mu x'});
    \end{aligned} \right.
    \;\Rightarrow
\end{equation*}
\begin{equation*} 
    \Rightarrow\quad
    \left\{ \begin{aligned}
        \;& \varphi(x') = \frac{1}{2\mu}e^{-\mu x'},\\
          & \psi(x')  = \frac{1}{2\mu}e^{-\mu x'} (1 - e^{2\mu x'}) = \frac{1}{2\mu} (e^{-\mu x'} - e^{\mu x'}).
    \end{aligned} \right.
\end{equation*}
Отже, функція Гріна (\ref{green-draft}) має вигляд
\begin{equation} 
    G(x,x') = 
    \left\{ \begin{aligned}
        \;& \frac{1}{2\mu}e^{-\mu x'} \big( e^{-\mu x} - e^{\mu x} \big), \; 0 \leq x \leq x' \\
          & \frac{1}{2\mu} \big(e^{-\mu x'} - e^{\mu x'}\big) e^{-\mu x}, \; x' \leq x \leq \infty,
    \end{aligned} \right.
\end{equation}
або якщо розкрити дужки та врахувати невід'ємність різниці $(x - x')$ в аргументі другої експоненти модулем
\begin{equation} 
    G(x,x') = \frac{1}{2\mu} \big( e^{-\mu (x + x')} - e^{-\mu |x - x'|} \big)
\end{equation}

Розв'язок для задачі на нескінченному інтервалі (випадок б) ) знайдемо граничним переходом $x, x' \to \infty$ при сталій різниці $x - x'$
\begin{equation}
    \lim_{\substack{|x-x'| = \mathrm{c}\\ x,x'\to\infty}} G(x,x') = \lim_{\substack{|x-x'| = \mathrm{c}\\ x,x'\to\infty}} \frac{1}{2\mu} \big( e^{-\mu (x + x')} - e^{-\mu |x - x'|} \big) = -\frac{1}{2\mu} e^{-\mu |x - x'|}
\end{equation}  

%\end{document}

\end{document}