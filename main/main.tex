\documentclass[a4paper, 14pt]{extreport}

\usepackage{StyleMMF}

\begin{document}

\tableofcontents
\setcounter{page}{2}

%%Создать отдельный файлик на задачу 3.1

\part{ЗАСТОСУВАННЯ ПРОЦЕДУРИ ФУР’Є БЕЗПОСЕРЕДНЬОГО ВІДОКРЕМЛЕННЯ ЗМІННИХ}

\chapter{Відокремлення змінних, задача Штурма-Ліувілля і власні моди коливань струни для різних межових умов}
\documentclass[a4paper, 14pt]{extreport}
\usepackage[top=2cm, bottom=2cm, left=2.5cm, right=1.5cm]{geometry}

\usepackage[utf8]{inputenc}
\usepackage[english, russian, ukrainian]{babel}
\usepackage{amssymb,amsfonts,amsmath,amsthm}

\usepackage[pdftex, unicode, colorlinks=true, linkcolor=black]{hyperref}

%\usepackage{relsize} %%позволяет пользоваться функцией \mathlarger{}
\usepackage{xcolor}
\usepackage[pdf]{xy}

\usepackage{wrapfig}
\usepackage{tikz} 
\usetikzlibrary{math}

\usepackage{pgfplots}
\pgfplotsset{compat=1.18}


\usepackage{titlesec}
\titleformat{\chapter}[display]{\normalfont\Large\bfseries}{\chaptertitlename\ \thechapter}{24pt}{\large\bfseries}
\titleformat{\section}{\normalfont\Large\bfseries}{\thesection}{20pt}{\large\bfseries}

\renewcommand{\labelenumii}{\theenumii)} %% заменяем счёчтик 2 уровня вида (a), (b), (c) и т.д. на русский алфавит а), б), в), и т.д. 


\begin{document}

\tableofcontents
\setcounter{page}{2}

\chapter{ЗАСТОСУВАННЯ ПРОЦЕДУРИ ФУР’Є БЕЗПОСЕРЕДНЬОГО ВІДОКРЕМЛЕННЯ ЗМІННИХ}

\section{Відокремлення змінних, задача Штурма-Ліувілля і власні моди коливань струни для різних межових умов}

\subsection*{Задача №1.1}

\textit{\textbf{Знайти власні моди коливань струни завдовжки $l$ із закріпленими кінцями (знайти функції вигляду $u(x,t) = X(x) \cdot T(t)$, визначені і достатньо гладкі в області $0 \leq x \leq l, -\infty \leq t \leq \infty$, не рівні тотожно нулю, які задовольняють одновимірне хвильове рівняння $u_{tt} = v^2 u_{xx}$ на проміжку $0 \leq x \leq l$ і межові умови $u(0,t) = 0, u(l,t) = 0$ на його кінцях).} Результат перевірити аналітично й графічно (див. текст до модульної контрольної роботи №1, с. 25) та проаналізувати його фізичний смисл. Знайти початкові умови (початкове відхилення і початкову швидкість) для кожної з мод.}

\begin{center}
    \large{\textbf{Розв'язок}}
\end{center}

\noindent Постановка задачі:
\begin{equation}
    \left\{ \begin{aligned} %%
        \;&u = u(x,t), \\  &u_{tt} = v^2 u_{xx}, \\ &0 \leq x \leq l, t \in \mathbb{R}, \\  &u(0,t) = 0, \\ &u(l,t) = 0. 
    \end{aligned} \right.
\end{equation}
Шукаємо нетривіальні розв'язки рівняння у виді:
\begin{equation} \label{subst}
    u(x,t) = X(x) \cdot T(t) \neq 0 
\end{equation}

Тепер можливе відокремлення змінних в задачі. Почнемо з межових умов:
\begin{equation*}
    \begin{aligned}
        \;u(0,t) = X(0) \cdot T(t) = 0
        \;\Rightarrow\;
        \left\{ \begin{aligned}
            &T(t) \neq 0, \forall t, \\  &X(0) = 0; 
        \end{aligned} \right.\\
        u(l,t) = X(l) \cdot T(t) = 0
        \;\Rightarrow\;
        \left\{ \begin{aligned}
            &T(t) \neq 0, \forall t, \\  &X(l) = 0; 
        \end{aligned} \right.\\
    \end{aligned}
\end{equation*}
Далі підставимо (\ref{subst}) в рівняння та виконаємо ряд перетворень:
\begin{equation*}
    \frac{\partial^2}{\partial t^2}\left[X(x)T(t)\right] = v^2 \frac{\partial^2}{\partial x^2}\left[X(x)T(t)\right]
    \;\to\; 
    X T^{\prime\prime} = v^2 X^{\prime\prime} T 
    \;\to\; 
    \frac{T^{\prime\prime}}{v^2T} = \frac{X^{\prime\prime}}{X} = - \lambda,
\end{equation*}
де $\lambda$ -- стала відокремлення.\\
Виписуємо результат відокремлення змінних:
\begin{equation} \label{sepvar}
    \left\{ \begin{aligned}
        \;&X = X(x), \\  &X^{\prime\prime} = -\lambda X, \\ &0 \leq x \leq l, \\  &X(0) = 0, \\ &X(l) = 0. 
    \end{aligned} \right.
    \qquad\qquad
    \begin{aligned}
        T^{\prime\prime} + \lambda v^2 T = 0\\
        \lambda \text{ -- невідома}
    \end{aligned}
\end{equation}

Для $X = X(x)$ отримуємо задачу Штурма-Ліувілля. Розв'яжемо її:
\begin{enumerate}
    \item[] \begin{enumerate}
        \item Розглянемо випадок $\lambda = 0$:
        \begin{equation*}
            X^{\prime\prime} = -\lambda X
            \;\to\;
            X^{\prime\prime} = 0
            \;\to\;
            X(x) = C_1 + C_2 x
        \end{equation*}
        Знаходимо константи з межових умов:
        \begin{equation*}
            \begin{aligned}
                &\left\{ \begin{aligned}
                    &X(0) = C_1 = 0, \\ 
                    &X(l) = C_1 + C_2 l = 0;
                \end{aligned} \right.
                \\   
                &\left\{ \begin{aligned}
                    C_1 = 0, \\ 
                    C_2 = 0;
                \end{aligned} \right.
            \end{aligned}
            \quad\Rightarrow\;
            X(x) = 0 \text{ -- розв'язок тривівльний}
        \end{equation*}
    
        \item Розглянемо випадок $\lambda < 0$. Розв'язок рівняння шукаємо у виді $X(x) = e^{\alpha x}$, підставимо це в рівняння: 
        \begin{equation*}
            \begin{aligned}
                &X^{\prime\prime} = -\lambda X
                \quad\to\quad
                \alpha^2 \textcolor{red}{\begin{xy}*{\textcolor{black}{e^{\alpha x}}};p+LU;+RD**h@{}+/\jot/**h@{-}\end{xy}} = +|\lambda| \textcolor{red}{\begin{xy}*{\textcolor{black}{e^{\alpha x}}};p+LU;+RD**h@{}+/\jot/**h@{-}\end{xy}}
                \quad\to\quad
                \alpha = \pm \sqrt{|\lambda|}
                \;\Rightarrow\\
                \Rightarrow\;
                &X(x) = \widetilde{C}_1 e^{\sqrt{|\lambda|}x} + \widetilde{C}_2 e^{-\sqrt{|\lambda|}x} \equiv C_1 sh(\sqrt{|\lambda|}x) + C_2 ch({\sqrt{|\lambda|}x})
            \end{aligned}
        \end{equation*}
        Знаходимо константи з межових умов:
        \begin{equation*}
            \begin{aligned}
                &\left\{ \begin{aligned}
                    &X(0) = C_2 = 0, \\ 
                    &X(l) = C_1 sh(\sqrt{|\lambda|}l) + C_2 ch({\sqrt{|\lambda|}l}) = 0;
                \end{aligned} \right.
                \;\to\\
                \to\;
                &\left\{ \begin{aligned}
                    &C_2 = 0, \\ 
                    &C_1 sh(\sqrt{|\lambda|}l) = 0, \\
                    &sh(\sqrt{|\lambda|}l) \neq 0;
                \end{aligned} \right.
                \;\to\;
                \left\{ \begin{aligned}
                    C_1 = 0, \\ 
                    C_2 = 0;
                \end{aligned} \right.
                \quad\Rightarrow\;
                \text{розв'язок тривівльний}
            \end{aligned}
        \end{equation*}
    
        \item Розглянемо випадок $\lambda > 0$. Розв'язок рівняння шукаємо у виді $X(x) = e^{\alpha x}$, підставимо це в рівняння: 
        \begin{equation*}
            \begin{aligned}
                &X^{\prime\prime} = -\lambda X
                \quad\to\quad
                \alpha^2 \textcolor{red}{\begin{xy}*{\textcolor{black}{e^{\alpha x}}};p+LD;+RU**h@{}+/\jot/**h@{-}\end{xy}} = -\lambda \textcolor{red}{\begin{xy}*{\textcolor{black}{e^{\alpha x}}};p+LD;+RU**h@{}+/\jot/**h@{-}\end{xy}}
                \quad\to\quad
                \alpha = \pm i\sqrt{\lambda}
                \;\Rightarrow\\
                \Rightarrow\;
                &X(x) = \widetilde{C}_1 e^{i\sqrt{\lambda}x} + \widetilde{C}_2 e^{-\sqrt{\lambda}x} \equiv C_1 \sin(\sqrt{\lambda}x) + C_2 \cos({\sqrt{\lambda}x})
            \end{aligned}
        \end{equation*}
        Знаходимо константи з межових умов:
        \begin{equation*}
            \begin{aligned}
                \left\{ \begin{aligned}
                    &X(0) = C_2 = 0, \\ 
                    &X(l) = C_1 \sin(\sqrt{\lambda}l) + \textcolor{red}{\begin{xy}*{\textcolor{black}{C_2}};p+LU;+RD**h@{}+//**h@{-}*h@{>}*h!LD{\scriptstyle 0}\end{xy}} \cos({\sqrt{\lambda}l}) = 0;
                \end{aligned} \right.
                \;\to\;
                \left\{ \begin{aligned}
                    &C_1 \neq 0, \\ 
                    &\sin(\sqrt{\lambda}l) = 0;
                \end{aligned} \right.
            \end{aligned}
        \end{equation*}
        Маємо нетривіальний розв'язок. Визначимо з характерисичного рівняння при яких значеннях $\lambda$ він можливий:
        \begin{equation*}
            \sin(\sqrt{\lambda}l) = 0
            \;\to\;
            \sqrt{\lambda}l = \pi n, \, n \in \mathbb{Z}
            \;\to\;
            \lambda_n = \frac{\pi^2 n^2}{l^2}, \, n \in \mathbb{N}.
        \end{equation*}
    \end{enumerate}
\end{enumerate} 
Отже, ми визначили всі власні значення та відповідні їм власні функції.
    \begin{equation} \label{ShLsol}
        \left\{ \begin{aligned}
            \;&\lambda_n = \frac{\pi^2 n^2}{l^2},\\ 
            &X_n(x) = C_n \sin\left(\frac{\pi n x}{l}\right),
        \end{aligned} \right.
        \quad \text{де } n \in \mathbb{N}.
    \end{equation}
    
Повертаємося до рівняння для $T(t)$ - (\ref{sepvar}). Підставляємо знайдені значення та знаходимо $T_n(t)$:
\begin{equation*}
    \left. \begin{aligned}
        \lambda_n = \frac{\pi^2 n^2}{l^2},&\;\\ 
        T^{\prime\prime} + \lambda v^2T = 0,&
    \end{aligned} \right\}
    \;\Rightarrow\;
    T_n(t) = A\cos(\omega_n t) + B\sin(\omega_n t),
\end{equation*}
де $\omega_n^2 = \lambda_n v^2, \, n \in \mathbb{N}.$\\
Власними модами коливань струни будуть всі розв'язки вигляду:
\begin{equation*}
    u_n(x,t) = X_n(x) \cdot T_n(t)
\end{equation*}
Виконаємо перепозначення і запишемо остаточний розв'язок:
\begin{equation}
    \left\{ \begin{aligned}
        \;&u_n(x,t) = \left(A_n\cos(\omega_n t) + B_n\sin(\omega_n t)\right) \sin(k_n x), \\
        &k_n = \frac{\pi n}{l} - \text{ хвильові вектори}, \\
        &\omega_n = vk_n = \frac{v \pi n}{l} - \text{ власні частоти}, \\
        &n = 1, 2,\ldots
    \end{aligned}\right.
\end{equation}
\newpage

\begin{center}
    \large{\textbf{Перевірка розв'язку задачі Штурма-Ліувілля}}
\end{center}

\noindent Щоб перевірити правильність отриманого результату, (\ref{ShLsol}), треба використовувати постановки самої задачі -- (\ref{sepvar}). 
\begin{enumerate}
    \item Аналітична перевірка (пряма підстановка результату в рівняння).
    \begin{enumerate}
        \item[1)] Перевіряємо рівняння, для цього обчислимо другу похідну власної функції. 
        \begin{equation*}
            X_n^{''} = \frac{\pi n}{l} \left(C_n\cos\left(\frac{\pi n x}{l}\right)\right)^{'} = -\left(\frac{\pi n}{l}\right)^2 C_n\sin\left(\frac{\pi n x}{l}\right) = -\left(\frac{\pi n}{l}\right)^2 X_n
        \end{equation*}
        Порівнюємо з вихідним рівнянням і робимо перший висновок: кожна з функцій $X_n(x)$ дійсно є розв’язком рівняння (\ref{sepvar}) для значення спектрального параметра $\lambda = \left(\frac{\pi n}{l}\right)^2$. Далі, ці значення $\lambda$ збігаються зі знайденими раніше власними значеннями (\ref{ShLsol}), отже робимо другий висновок: знайдені власні значення дійсно відповідають знайденим власним функціям.
        \item[2)] Перевіряємо виконання межових умов, підставляємо власні функції в межові умови.
        \begin{equation*}
            \begin{aligned}
                X_n(0) &=  0:\\
                &\begin{aligned}
                    X_n(0) = C_n \sin(\sqrt{\lambda_n} \cdot 0) = 0 &\text{ -- виконується,}\\
                    &\text{ причому незалежно від }\lambda_n
                \end{aligned}\\
                \\
                X_n(l) &= 0:\\
                &\begin{aligned}
                    X_n(l) = C_n \sin\left(\frac{\pi n}{l} \cdot l\right) = C_n &\sin(\pi n) = 0 \text{ -- виконується,  але}\\
                    &\text{ саме для знайдених значень }\lambda_n
                \end{aligned}\\
            \end{aligned}
        \end{equation*}
    \end{enumerate}
    \item Графічна перевірка:
    \begin{center}
        \begin{tikzpicture}
            \begin{axis}
                [width = 0.85\textwidth, height = 0.4\textwidth,
                 axis x line = center, axis y line = center,
                 ylabel = $X(x)$, xlabel = $x$,
                 xmin = 0.7, xmax = 6.7, ymin = -3.3, ymax = 5.3,
                 axis line style = thin, xtick = {0}, ytick = {0}]   
                
                \tikzmath{\A1 = 5; \l = 5; \k = pi/\l;}
                
                \addplot [black, domain = 1:(\l+1), samples = 1000] {\A1 * sin(deg(\k*(x-1)))}
                node[anchor=north, pos=0] {0} 
                node[pos=0.75, fill=white] {$X_0(x)$} 
                node[anchor=north, pos=1] {$l$};
                
                \addplot [black, domain = 1:(\l+1), samples = 1000] {\A1/2.5 * sin(deg(2*\k*(x-1)))}
                node[pos=0.83, fill=white] {$X_1(x)$};
                
                \addplot [black, domain = 1:(\l+1), samples = 1000] {\A1/3.7 * sin(deg(3*\k*(x-1)))}
                node[pos=0.74, fill=white] {$X_2(x)$};
                
            \end{axis}
        \end{tikzpicture}
    \end{center}
\end{enumerate}




\end{document}

\chapter{Власні моди інших систем. Вільні коливання для заданих початкових умов.}
\documentclass[a4paper, 14pt]{extreport}

\usepackage{StyleMMF}

\begin{document}

\subsection{Стержень з вільними та пружно закріпленими кінцями; системи, описувані іншими рівняннями.}

\subsubsection{Задача №1}

\textit{Знайти власні моди повздовжніх рухів тонкого стержня $0 \leq x \leq l$ із вільними кінцями  (задача для хвильового рівняння з межовими умовами $u_x(0,t) = 0, u_x(l,t) = 0$).\\
Результат перевірити аналітично й графічно (див. заняття №6, зразок модульної контрольної роботи №1) та проаналізувати його фізичний смисл. Чим відрізняється від інших основна (нульова) мода? Якому рухові стержня вона відповідає?}

\begin{center}
    \large{\textbf{Розв'язок}}
\end{center}

\noindent Формальна постановка задачі:
\begin{equation} \label{probcond2}
    \left\{ \begin{aligned} %%
            \;&u = u(x,t), \\
            &u_{tt} = v^2 u_{xx}, \\
            &0 \leq x \leq l, t \in \mathbb{R} \\
            &u_x(0,t) = 0, \\
            &u_x(l,t) = 0. 
    \end{aligned} \right.
\end{equation}
Необхідно знайти розв'язки (\ref{probcond2}) вигляду:
\begin{equation} \label{subst2}
    u(x,t) = X(x) \cdot T(t) \neq 0 
\end{equation}

Від задачі №1 попереднього заняття задача відрізняється тільки межовою умовою, тому підставляємо розв'язок у вигляді добутку (\ref{subst2}) тільки у межові умови (\ref{probcond2}):
\begin{equation*}
    \begin{aligned}
        \;u_x(0,t) = X'(0) \cdot T(t) = 0
        \;\Rightarrow\;
        \left\{ \begin{aligned}
            &T(t) \neq 0, \forall t, \\  &X'(0) = 0; 
        \end{aligned} \right.\\
        u_x(l,t) = X'(l) \cdot T(t) = 0
        \;\Rightarrow\;
        \left\{ \begin{aligned}
            &T(t) \neq 0, \forall t, \\  &X'(l) = 0; 
        \end{aligned} \right.\\
    \end{aligned}
\end{equation*}
Тут ми врахували, що умови на кінцях струни виконуються при всіх $t$, тому $T(t)$ не може бути рівним нулю.\\

Виписуємо результат відокремлення змінних:
\begin{equation} \label{sepvar2}
    \left\{ \begin{aligned}
        \;&X = X(x), \\
          &X^{\prime\prime} = -\lambda X, \\
          &0 \leq x \leq l, \\
          &X'(0) = 0, \\ 
          &X'(l) = 0. 
    \end{aligned} \right.
    \qquad\qquad
    T^{\prime\prime} + \lambda v^2 T = 0
\end{equation}

\begin{enumerate}
    \item[] Розв'язуємо задачу Штурма-Ліувілля (\ref{sepvar2}). Знову скористаємося результатами попередньої задачі та одразу запишемо якого типу отримуємо розв'язки для різних $\lambda$.
    \begin{enumerate}[wide, labelindent=0pt]
        
        \item Випадок $\lambda < 0$. 
        \begin{equation*}
            X(x) = C_1 sh(\sqrt{|\lambda|}x) + C_2 ch({\sqrt{|\lambda|}x})
        \end{equation*}
        Знаходимо константи з межових умов:
        \begin{equation*}
            \begin{aligned}
                &X'(0) = C_1\sqrt{|\lambda|}
                \;\Rightarrow\;
                X(x) = C_2 &ch(\sqrt{|\lambda|}x)\\
                &\left\{ \begin{aligned}
                    &X'(l) = C_2\sqrt{|\lambda|} sh(\sqrt{|\lambda|}l) = 0, \\
                    &sh(\sqrt{|\lambda|}l) \neq 0;
                \end{aligned} \right.&\\
                &\left\{ \begin{aligned}
                    C_1 = 0, \\ 
                    C_2 = 0;
                \end{aligned} \right. \qquad\qquad\qquad\qquad&
            \end{aligned}
            \;\Rightarrow\;
            \begin{aligned}
                \text{розв'язок тривівльний,}\\
                \text{немає від'ємних}\\
                \text{власних значень.}
            \end{aligned}
        \end{equation*}

        \item Випадок $\lambda = 0$:
        \begin{equation*}
            X(x) = C_1 + C_2 x
        \end{equation*}
        Знаходимо константи з межових умов:
        \begin{equation*}
            \begin{aligned}
                &\left\{ \begin{aligned}
                    &X'(0) = C_2 = 0, \\ 
                    &X'(l) = C_2 = 0;
                \end{aligned} \right.
                \\   
                &\left\{ \begin{aligned}
                    C_1 \in \mathbb{R}, \\ 
                    C_2 = 0;
                \end{aligned} \right.
            \end{aligned}
            \quad\Rightarrow\;
            \begin{aligned}
                X(x) = 0 \text{ -- розв'язок нетривівльний,}\\
                \lambda = 0 \text{ є власним значенням.}
            \end{aligned}
        \end{equation*}

        \item Випадок $\lambda > 0$
        \begin{equation*}
            X(x) = C_1 \sin(\sqrt{\lambda}x) + C_2 \cos({\sqrt{\lambda}x})
        \end{equation*}
        Знаходимо константи з межових умов:
        \begin{equation*}
            \begin{aligned}
                \left\{ \begin{aligned}
                    &X'(0) = C_1\sqrt{\lambda} = 0, \\ 
                    &X'(l) = \sqrt{\lambda}\left(\textcolor{red}{\begin{xy}*{\textcolor{black}{C_1}};p+LU;+RD**h@{}+//**h@{-}*h@{>}*h!LD{\scriptstyle 0}\end{xy}} \cos({\sqrt{\lambda}l}) - C_2 \sin(\sqrt{\lambda}l)\right) = 0;
                \end{aligned} \right.
                \;\Rightarrow\;
                \left\{ \begin{aligned}
                    &C_2 \neq 0, \\ 
                    &\sin(\sqrt{\lambda}l) = 0;
                \end{aligned} \right.
            \end{aligned}
        \end{equation*}
        Отже, нетривіальні розв'язки існують при значеннях параметра $\lambda$, які задовольняють характеристичне рівняння :
        \begin{equation*}
            \sin(\sqrt{\lambda}l) = 0
            \;\Rightarrow\;
            \sqrt{\lambda_n}l = \pi n, \, n \in \mathbb{Z}
            \;\Rightarrow\;
            \lambda_n = \frac{\pi^2 n^2}{l^2}.
        \end{equation*}
    \end{enumerate}
\end{enumerate} 
Випишемо тепер розв'язки для всіх $n$ і визначимо, які з них необхідно залишити:
    \begin{equation*}
        \left\{ \begin{aligned}
            &X_0(x) = C_0,\\
            &\lambda_0 = 0;
        \end{aligned} \right.
        \qquad
        \left\{ \begin{aligned}
            &X_n(x) = C_n \sin\left(\frac{\pi n x}{l}\right),\\
            &\lambda_n = \frac{\pi^2 n^2}{l^2}, n \in \mathbb{N}.
        \end{aligned} \right.
    \end{equation*}

    \textbf{\Large Необхідно відредагувати наступний текст}
Видно, що $n = 0$ відповідає тривіальному розв'язку. Видно також, що всі інші розв'язки визначені з точністю до довільного множника.\\
Тому власні функції, які співпадають з точністю до множника, вважають однаковими. У загальному випадку різними вважають лише лінійно незалежні власні функції, а розвя'зати задачу Штурма-Ліувілля означає знайти всі різні власні функції і відповідні власні значення. Отже, різним власним функціям відповідають лише натуральні $n$, а коефіцієнти $C_n$ можна покласти рівними одиниці.\\
    Власними значеннями і власними функціями є
    \begin{equation} %\label{ShLsol}
        \left\{ \begin{aligned}
            \;&\lambda_n = \frac{\pi^2 n^2}{l^2},\\ 
            &X_n(x) = \sin\left(\frac{\pi n x}{l}\right),
        \end{aligned} \right.
        \quad \text{де } n \in \mathbb{N}.
    \end{equation}

Повертаємося до рівняння для $T(t)$ (\ref{sepvar2}). Підставляємо знайдені власні значення та знаходимо $T_n(t)$:
\begin{equation*}
    \left. \begin{aligned}
        \lambda_n = \frac{\pi^2 n^2}{l^2},&\;\\ 
        T^{\prime\prime} + \lambda v^2T = 0,&
    \end{aligned} \right\}
    \;\Rightarrow\;
    T_n(t) = A\cos(\omega_n t) + B\sin(\omega_n t),
\end{equation*}
де $\omega_n^2 = \lambda_n v^2, \, n \in \mathbb{N}.$\\
\begin{equation*}
    \left. \begin{aligned}
        \lambda_0 = 0,&\;\\ 
        T^{\prime\prime} = 0,&
    \end{aligned} \right\}
    \;\Rightarrow\;
    T_0(t) = A_0 + B_0 t,
\end{equation*}
Власними модами коливань струни будуть всі розв'язки вигляду:
\begin{equation*}
    u_n(x,t) = X_n(x) \cdot T_n(t)
\end{equation*}
Виконаємо перепозначення і запишемо остаточний розв'язок:
\begin{equation}
    \left\{ \begin{aligned} \label{sol2}
        \;&u_0(x,t) = A_0 + B_0 t, \\
        &u_n(x,t) = \left[A_n\cos(\omega_n t) + B_n\sin(\omega_n t)\right] \sin(k_n x), \\
        &k_n = \frac{\pi n}{l} - \text{ хвильові вектори}, \\
        &\omega_n = vk_n = \frac{v \pi n}{l} - \text{ власні частоти}, \\
        &n = 1, 2,\ldots
    \end{aligned}\right.
\end{equation}

\end{document}
\vspace{2cm}
\subsection{Вільні коливання поля в резонаторі для заданих початкових умов. Ряд Фур'є по системі ортогональних функцій.}

\subsubsection{Задача №3}

\textit{Знайти коливання струни завдовжки $0 \leq x \leq l$ із закріпленими кінцями, якщо початкове відхил є $\varphi(x) = hx/l$, а початкова швидкість $\psi(x) = \nu_0$. Обчислити інтеграл ортогональності власних функцій і знайти квадрат норми. Чи є рух струни періодичним (тобто повторюється початковий стан струни через деякий проміжок часу?) Чи буде рух періодичним, якщо він описується рівнянням $u_{tt} = v^2 u_{xx} - \omega_0^2 u$}

\begin{center}
    \textbf{Розв'язок}
\end{center}
Формальна постановка задачі:
\begin{equation} \label{probcond3}
    \left\{ \begin{aligned} %%
        &\;u = u(x,t), \\
        &\;u_{tt} = v^2 u_{xx}, \\
        &\;0 \leq x \leq l, t \in \mathbb{R}, \\
        &\;u(0,t) = u(l,t) = 0,\\
        &\left.\begin{aligned}
            &u(x,0) = \varphi(x) = \frac{hx}{l}, \\ 
            &u_t(x,0) = \psi(x) = \nu_0.
        \end{aligned}\right\} \; 
        \begin{aligned}
            &\text{ початкові умови задають} \\
          - &\text{ механічний стан} \\
            &\text{ системи при } t = 0
        \end{aligned}
    \end{aligned} \right.
\end{equation}

Задача з заданими початковими умовами має єдиний розв'язок. Скористаємося результатами задачі 1 попередньго заняття (\ref{sol1}).
\begin{equation*}
    \left\{ \begin{aligned} \label{fullsol}
        \;&u_n(x,t) = \left[A_n\cos(\omega_n t) + B_n\sin(\omega_n t)\right] \sin(k_n x), \\
        &k_n = \frac{\pi n}{l}, \, n = 1, 2,\ldots\\
        &\omega_n = vk_n = \frac{v \pi n}{l} - \text{ власні частоти}.
    \end{aligned}\right.
\end{equation*}

Запишемо загальний розв'язок задачі:
\begin{equation} \label{gensol}
    u(x,t) = \sum^{\infty}_{n=1} u_n(x,t) = \sum^{\infty}_{n=1} \left[A_n\cos(\omega_n t) + B_n\sin(\omega_n t)\right] \sin(k_n x)
\end{equation}
Коефіцієнти $A_n$ та $B_n$ визначаємо із початкових умов. Підставляємо (\ref{gensol}) в початкові умови (\ref{probcond3}):
\begin{equation} \label{sol-init-cond}
    \begin{aligned}
        &u(x,0) = \varphi(x)
        \;\Rightarrow\;
        \sum^{\infty}_{n=1} A_n\sin(k_n x) = \varphi(x)\\
        &\begin{aligned}
            u_t(x,0) = \psi(x)
            \;&\Rightarrow\\
            \Rightarrow \left(\sum^{\infty}_{n=1}\right.&\left.\left. \left[-A_n\omega_n\sin(\omega_n t) + B_n\omega_n\cos(\omega_n t)\right] \sin(k_n x)\right)\right|_{t=0} =\\
            &= \sum^{\infty}_{n=1} B_n\omega_n\sin(k_n x) = \psi(x)
        \end{aligned}
    \end{aligned}
\end{equation}
Отже, ми отримали дві умови для визначення $A_n$, $B_n$.\\
Далі скористаємося ортогональністю власних функцій задачі Штурма-Ліувілля.
\begin{equation} \label{orth}
    \int_0^l X_n(x) \cdot X_m(x) \,\mathrm{d}x = ||X_n||^2\delta_{n,m},
\end{equation}
де $||X_n||$ -- норма власної функції.

Доможуємо отримані вирази в (\ref{sol-init-cond}) на $m$-ту власну функції $\sin(k_m x)$ та інтегруємо від $0$ до $l$. 
\begin{equation*}
    \begin{aligned}
        \begin{aligned}
            &\int\limits_0^l \varphi(x) \sin(k_m x) \,\mathrm{d}x = \sum^{\infty}_{n=1} A_n \int\limits_0^l \sin(k_n x) \sin(k_m x) \,\mathrm{d}x =\\
            &= \sum^{\infty}_{n=1} A_n \cdot \frac{l}{2} \delta_{n,m} = \frac{A_m l}{2}
            \;\Rightarrow\;
            A_n = \frac{2}{l} \int\limits_0^l \varphi(x) \sin(k_n x) \,\mathrm{d}x =\\
            &= \frac{2}{l} \int\limits_0^l \frac{hx}{l} \sin(k_n x) \,\mathrm{d}x = \frac{2h}{l^2} \left(\left.-\frac{1}{k_n} x \cos(k_n x)\right|_0^l + \int\limits_0^l \frac{\cos(k_n x)}{k_n} \,\mathrm{d}t\right) =\\
            &= \left| k_n l = \frac{\pi n}{l} l = \pi n \Rightarrow \sin(k_n l) = 0,\, \cos(k_n l) = (-1)^n \right| =\\
            &= \frac{2h}{l^2} \left(-\frac{l}{k_n}(-1)^n + \left.\frac{\sin(k_n x)}{k_n^2}\right|_0^l \right) = \frac{2h}{l} \frac{(-1)^{n+1}}{k_n} \equiv A_n
        \end{aligned}\\
        \begin{aligned}
            \int\limits_0^l \psi(x) \sin(k_m x) \,\mathrm{d}x &= \sum^{\infty}_{n=1} B_n\omega_n \cdot \frac{l}{2} \delta_{n,m} = \frac{B_m \omega_m l}{2}
            \;\Rightarrow\\
            \Rightarrow\;
            B_n =&\ \frac{2}{\omega_n l} \int\limits_0^l \psi(x) \sin(k_n x) \,\mathrm{d}x = \frac{2\nu_0}{\omega_n l} \int\limits_0^l \sin(k_n x) \,\mathrm{d}x =\\
            =&\ \left.\frac{2\nu_0}{k_n\omega_n l} \cos(k_n x)\right|_l^0 = \frac{2\nu_0}{l} \frac{1 - (-1)^n}{k_n\omega_n} \equiv B_n
        \end{aligned}
    \end{aligned}
\end{equation*} 
Підставляємо визначені константи в (\ref{gensol})
\begin{equation} \label{sol3}
    u(x,t) = \frac{2}{l}\sum^{\infty}_{n=1} \left[\nu_0 (1 - (-1)^n)\frac{\sin(\omega_n t)}{\omega_n} - h(-1)^n\cos(\omega_n t)\right] \frac{\sin(k_n x)}{k_n}
\end{equation}

Перевіримо періодичність розв'язку. Період коливання визначається за відомою формулою \[T_n = \frac{2\pi}{\omega_n},\] де $n$ - номер власної моди. Підставимо в (\ref{sol3}) $t = t + T_n$
\begin{equation*}
    \begin{aligned}
        u(x,t+T_n) & = \frac{2}{l}\sum^{\infty}_{n=1} \left[\nu_0 (1 - (-1)^n)\frac{\sin(\omega_n t + \omega_n \cdot \frac{2\pi}{\omega_n})}{\omega_n} -\right.\\
        &\qquad\qquad\qquad\qquad\quad\left.- h(-1)^n\cos(\omega_n t + \omega_n \cdot \frac{2\pi}{\omega_n})\right] \frac{\sin(k_n x)}{k_n} =\\
        = \frac{2}{l}\sum^{\infty}_{n=1} & \left[\nu_0 (1 - (-1)^n)\frac{\sin(\omega_n t + 2\pi)}{\omega_n} - h(-1)^n\cos(\omega_n t + 2\pi)\right] \frac{\sin(k_n x)}{k_n} =\\ 
        = \frac{2}{l}\sum^{\infty}_{n=1} & \left[\nu_0 (1 - (-1)^n)\frac{\sin(\omega_n t)}{\omega_n} - h(-1)^n\cos(\omega_n t)\right] \frac{\sin(k_n x)}{k_n} = u(x,t)
    \end{aligned}
\end{equation*} 
Тобто коливання струни буде періодичним.

\chapter{Другий спосіб знаходження коефіцієнтів. Коливання стержня з вільними кінцями, неповнота базису.}
%\documentclass[a4paper, 14pt]{extreport}

%\usepackage{StyleMMF}

%\begin{document}

%\setcounter{chapter} {2}
%\chapter{Другий спосіб знаходження коефіцієнтів. Коливання стержня з вільними кінцями, неповнота базису.}

\section[Задача №3.1]{3.1}

\textit{Знайти коливання пружного стержня $0 \leq x \leq l$, лівий кінець якого закріплений, а правий вільний, якщо початкове відхилення $\varphi(x) = h \sin(3\pi x/2l)$, а початкова швидкість $\psi(x) = \nu_0 \sin(\pi x/2l)$.}

\begin{center}
    \textbf{Розв'язок}
\end{center}
Формальна постановка задачі:
\begin{equation} \label{cond3,1}
    \left\{ \begin{aligned} %%
        &\;u = u(x,t), \\
        &\;u_{tt} = v^2 u_{xx}, \\
        &\;0 \leq x \leq l, t \geq 0, \\
        &\;u(0,t) = 0,\\
        &\;u_x(0,t) = 0,\\
        &\left.\begin{aligned}
            &u(x,0) = \varphi(x) = h \sin \left(\frac{3 \pi x}{2 l} \right), \\ 
            &u_t(x,0) = \psi(x) = v_0 \sin \left(\frac{\pi x}{2 l}\right).
        \end{aligned}\right\} \; 
        \begin{aligned}
            &\text{ специфіка задачі} \\
          - &\text{ у вигляді } \\
            &\text{ початкових умов } 
        \end{aligned}
    \end{aligned} \right.
\end{equation}

Це задача із заданими початковими умовами (а саме - початковим розподілом зміщення та швидкостей), яка має єдиний розв'язок.

Для початку скористаємося розв'язком задачі 1.2:

\begin{equation}
    \left\{ \begin{aligned} \label{mode3.1}
        \;&u_n(x,t) = \left[A_n\cos(\omega_n t) + B_n\sin(\omega_n t)\right] \sin(k_n x), \\
        &k_n = (n + \frac{1}{2})\frac{\pi}{l}, \, n = 1, 2,\ldots\\
        &\omega_n = vk_n = (n + \frac{1}{2})\frac{\pi v}{l} - \text{ власні частоти}.
    \end{aligned}\right.
\end{equation}

І запишемо загальний розв'язок:

\begin{equation} \label{gensol3,1}
    u(x,t) = \sum^{\infty}_{n=0} \left[A_n\cos(\omega_n t) + B_n\sin(\omega_n t)\right] \sin(k_n x)
\end{equation}

\begin{equation} \label{(sol3,1)_t}
    \begin{aligned}
        u_t(x,t) &= 
   \sum^{\infty}_{n=0}\left[-A_n\omega_n\sin(\omega_n t) + B_n\omega_n\cos(\omega_n t)\right] \sin(k_n x)  
    \end{aligned}
\end{equation}

Підставляємо (\ref{gensol3,1}) у початкові умови (\ref{cond3,1}):

\begin{equation}
    u(x,0) = \varphi(x) \;\Rightarrow\; \sum^{\infty}_{n=0} A_n\sin\left((n + \frac{1}{2}) \frac{\pi x}{l} \right) = h \sin \left( \frac{3 \pi x}{2l} \right)
\end{equation}


Підставляємо (\ref{(sol3,1)_t}) у початкові умови (\ref{cond3,1}):

\begin{equation}
    u_t(x,0) = \psi(x) \;\Rightarrow\; \sum^{\infty}_{n=0} B_n \omega_n \sin\left((n + \frac{1}{2}) \frac{\pi x}{l} \right) = v_0 \sin \left( \frac{ \pi x}{2l} \right)
\end{equation}

Особливі ситуації: функції у правій частині є однією з власних функцій задачі Штурма-Ліувіля. Це дозволяє знайти коефіцієнти $A_n, B_n$ простіше, порівнюючи з загальнім знаходженням з вихідних функцій $\varphi(x)$ і  $\psi(x)$ загального вигляду! Тобто брати інтеграл у цій особливій ситуації не потрібно.


Якщо 2 ряда Фур'є по одній системі функцій рівні, то і відповідні коефіцієнти цих рядів рівні.

\begin{equation}
    \begin{aligned}
        \sum_{n=0} A_n \sin \left( (n + \frac{1}{2}) \frac{\pi x}{l} \right) = A_0 \sin \left(\frac{\pi x}{2 l} \right) &+\\
        + A_1 \sin \left(\frac{3 \pi x}{2 l} \right) + A_2 \sin \left(\frac{5 \pi x}{2 l} \right) + &... = h \sin \left( \frac{3 \pi x}{2 l} \right)
    \end{aligned}
\end{equation}

Результат $A_1 = h, A_0 = A_2 = A_3 = ... = 0$

Аналогічно робимо з \ref{(sol3,1)_t}: 

\begin{equation}
    \begin{aligned}
        \sum_{n=0} \omega_n B_n \sin \left( (n + \frac{1}{2}) \frac{\pi x}{l} \right) = \omega_0 B_0 \sin \left(\frac{\pi x}{2 l} \right) &+\\
        + \omega_1 B_1 \sin \left(\frac{3 \pi x}{2 l} \right) + \omega_2 B_2 \sin \left(\frac{5 \pi x}{2 l} \right) + &... = v_0 \sin \left( \frac{\pi x}{2 l} \right)
    \end{aligned}
\end{equation}

Результат $B_0 = \frac{v_0}{\omega_0}, B_1 = B_2 = B_3 = ... = 0$

Тепер треба правильно написати відповідь через знайдені коефіцієнти $A_n, B_n$! Підставляємо знайдені коефіцієнти у загальний розв'язок (тільки два коефіцієнти - $A_1$ і $B_0$ не дорівнюють нулю, тож членів у розв'язку всего два!)

Фінальна відповідь:

\begin{equation}
    u (x,t) = h \cos (\omega_1 t) \sin (k_1 x) + \frac{v_0}{\omega_0} \sin (\omega_0 t) \sin (k_0 x)
\end{equation}

де $k_0 = \frac{\pi x}{2 l}, k_1 = \frac{3 \pi x}{2 l}, \omega_0 = v k_0, \omega_1 = v k_1 $. Можемо помітити, що у кожної моди своя частота.

Перевіряємо відповідь

\begin{itemize}
    \item Власні функції перевірені в задачі 1.2
    \item Постановка задачі містить два неоднорідних члени у початкових умовах. Один пропорційний $\sim h$, інший пропорційний $\sim v_0$. Перевірити наявність цих множників у загальному розв'язку.
    \item Перевіряємо початкові умови - виконуються?

\end{itemize}

Альтернативний шлях -- знайти за означенням коефіцієнти розкладу у ряд Фур'є.

\begin{equation}
A_n = \frac{2}{l} \int_{0}^{l} \varphi (x)  \sin \left( (\frac{1}{2} + n) \frac{\pi x}{l} \right) dx    
\end{equation}


Одержали інтеграл ортогональності 

\begin{equation}
    \int^{l}_0 \sin \left( \frac{3 \pi x}{2 l} \right) \sin \left( (\frac{1}{2} + n) \frac{\pi x}{l} \right) dx = \int^{l}_0 \chi_1 (x) \chi_n (x) dx = \delta_{1n}
\end{equation}

Якщо ви не побачите що інтеграл є інтегралом ортогональності, і будете його обчилювати, то втратите час і можете помилитися і одержати неправильну відповідь (що часто і буває).

Результат $A_1 = h, A_0 = A_2 = A_3 = ... = 0$ та для швидкостей $B_0 = \frac{v_0}{\omega_0}, B_1 = B_2 = B_3 = ... = 0$

Отримали теж саме, але складнішим шляхом!

%\end{document}
%\documentclass[a4paper, 14pt]{extreport}

%\usepackage{StyleMMF}

%\begin{document}

%\chapter{Другий спосіб знаходження коефіцієнтів. Коливання стержня з вільними кінцями, неповнота базису.}

\section[Задача №3.3]{3.3}

\textit{Знайти коливання пружного стержня довжиною $l$ з вільними кінцями, якщо початкове відхилення дорівнює нулю, а початкова швидкість $\psi(x) = \nu_0$. Якщо всі знайдені вами коефіцієнти Фур'є (коефіцієнти загального\\ розв’язку) дорівнюють нулю, поясніть, що це означає, і знайдіть, де була допущена помилка.}

%\end{document}

\chapter{Рівняння теплопровідності з однорідними межовими умовами}
\documentclass[a4paper, 14pt]{extreport}

\usepackage{StyleMMF}

\begin{document}

\section{Рівняння теплопровідності з однорідними межовими умовами}

\subsubsection{Задача №1}

\textit{Одну і ту ж функцію, наприклад $f(x) = \alpha x$, можна представити на проміжку $0 \leq x \leq l$ узагальненим рядом Фур’є по кожній із систем власних функцій чотирьох задач Штурма-Ліувілля, одержаних у задачах 1.1, 1.2, 1.3, 2.1. Користуючись явним виглядом власних функцій і не обчислюючи коєфіцієнтів рядів, дайте відповіді на такі запитання.
\begin{enumerate}
    \item Який вигляд матиме графік суми кожного з таких рядів на всій числовій осі? Якою є парність суми ряду відносно точок $x = nl$, де $n$ – ціле число, і як це пов’язано з виглядом крайових умов задачі Штурма-Ліувілля?
    \item Покажіть, що кожний з рядів є частинним випадком класичного тригонометричного ряду Фур’є, сума якого є періодичною функцією. Які саме періоди відповідають кожному з рядів? Яка саме частина повного тригонометричного базису використовується в кожному з розкладань, а які коефіцієнти Фур’є дорівнюють нулю і чому?
    \item Як пов’язаний характер збіжності вказаних рядів з крайовими умовами, які задовольняє функція $f(x)$ у точках $x = 0, l$ ? Чи дорівнює сума ряду Фур’є функції $f(x)$ на відкритому проміжку $0 < x < l$? на закритому проміжку
    $0 \leq x \leq l$
\end{enumerate}}


\end{document}
\vspace{2cm}
\documentclass[a4paper, 14pt]{extreport}

\usepackage{StyleMMF}

\begin{document}

\section{Рівняння теплопровідності з однорідними межовими умовами}

\subsubsection{Задача №2}

\textit{У початковий момент часу ліва половина стержня з теплоізольованою бічною поверхнею має температуру $T_1$ , а права -- температуру $T_2$ . Знайти розподіл температури при $t> 0$, якщо кінці стержня підтримуються при температурі $T_0$. Указівка: подумайте, що означає «температура дорівнює нулю», що це за нуль? Покладіть у кінцевому результаті $T_0 = 0$ і розгляньте частинні випадки: $T_1 = T_2$ та $T_1 = -T_2$. Які члени ряду при цьому обертаються в нуль? Чому? Нарисуйте графіки та порівняйте часову залежність температури для     різних мод. Нарисуйте (якісно) графіки розподілу     температури вдовж стержня у різні характерні послідовні моменти часу. Що таке «малий» і «великий» проміжок часу для цієї задачі? Як характерні часи залежать від розмірів системи?}

\end{document}
%\documentclass[a4paper, 14pt]{extreport}

%\usepackage{StyleMMF}

%\begin{document}

%\chapter{Рівняння теплопровідності з однорідними межовими умовами}

\section[Задача №4.4]{4.4}

\textit{Початкова температура повністю теплоізольованого тонкого стержня\\ $0 \leq x \leq l$ дорівнює $T_1 \cos(\pi x/2l) + T_2 \cos(2\pi x/l)$ . Знайти поле температур при $t > 0$. Перевірити виконання початкових умови при $T_1 = 0$ і $T_2 = 0$.}

\begin{center}
    \large{\textbf{Розв'язок}}
\end{center}

\noindent Формальна постановка задачі:
\begin{equation} \label{cond4,4}
    \left\{ \begin{aligned} %%
            \;&u = u(x,t), \\
            &u_t = D u_{xx}, \\
            &0 \leq x \leq l, t \geq 0, \\
            &u_x(0,t) = 0, \, u_x(l,t) = 0,\\ 
            &u(x,0) = T_1 \cos(\pi x/2l) + T_2 \cos(2\pi x/l).
    \end{aligned} \right.
\end{equation}

Виконуючи розділення змінних ми отримаємо дві вже розв'язані задачі. Задачу Штурма-Ліувілля (\ref{sepvar2,1}) з задачі №2.1 та часове диференціальне рівняня (\ref{time-eq4,2}) з задачі №4.2. Отже, загальний розв'язок можна одразу записати комбінуюці відомі.

\begin{equation} \label{gensol4,4}
    u(x,t) = C_0 + \sum_{n=1}^{\infty}C_n e^{-t/\tau_n} \cos k_n x,
\end{equation}
\begin{equation*}
    \begin{aligned}
        &k_n = \frac{\pi n}{l} - \text{ хвильові вектори}, \\
        &\tau_n = \frac{1}{D k_n^2} - \text{ характерний час зміни температури}, \\
        &n = 0, 1, 2,\ldots
    \end{aligned}
\end{equation*}

З початковї умови визначимо невіомі коефіцієнти. Для цього треба розкласти $\cos(\pi x/2l)$ по набору власних функцій задачі Ш.-Л. 
\begin{equation*}
    \begin{gathered}
        \cos(\pi x/2l) = a_0 + \sum_{n=1}^{\infty} a_n \cos k_nx \\
        a_0 = \frac{1}{l}\int\limits_0^l \cos(\pi x/2l) \;\mathrm{d}x = \frac{2}{\pi} \sin(\pi x/2l) \bigg|_0^l = \frac{2}{\pi}\\
        a_n = \frac{2}{l}\int\limits_0^l \cos(\pi x/2l)\cos k_nx \;\mathrm{d}x = \frac{1}{l}\bigg(\int\limits_0^l \cos((k_n + \pi/2l)x) \;\mathrm{d}x +\\
        + \int\limits_0^l \cos((k_n - \pi/2l)x) \;\mathrm{d}x\bigg) = \frac{1}{l}\left(\frac{\sin((k_n + \pi/2l)x)}{k_n + \pi/2l}\bigg|_0^l + \frac{\sin((k_n - \pi/2l)x)}{k_n - \pi/2l}\bigg|_0^l\right) =\\
        = \left(\frac{\sin(k_nl + \pi/2)}{k_nl + \pi/2} + \frac{\sin(k_nl - \pi/2)}{k_nl - \pi/2}\right) = \left(\frac{1}{k_nl + \pi/2} - \frac{1}{k_nl - \pi/2}\right)\cos k_nl =\\
        = \big|\cos k_nl = (-1)^n\big| = \frac{(-1)^{n+1} \pi}{(k_nl + \pi/2)(k_nl - \pi/2)} =\\
        = (-1)^{n+1} \cdot \frac{4\pi}{4k_n^2l^2 - \pi^2} = \frac{4}{\pi} \cdot \frac{(-1)^{n+1}}{4n^2 - 1}
    \end{gathered}
\end{equation*}
Тепер підставимо (\ref{gensol4,4}) в початкову умову (\ref{cond4,4}) і отримаємо:
\begin{equation}
    \begin{gathered}
        u(x,0) = C_0 + \sum_{n=1}^{\infty}C_n \cos k_n x = T_1 \cos(\pi x/2l) + T_2 \cos(2\pi x/l) =\\
        = \frac{2T_1}{\pi} + T_2 \cos k_2x + \frac{4T_1}{\pi} \sum_{n=1}^{\infty} \frac{(-1)^{n+1} \cos k_nx}{4n^2 - 1}
    \end{gathered}
\end{equation} 
З чого слідує 
\begin{equation*}
    C_0 = \frac{2T_1}{\pi},\, C_2 = T_2 - \frac{4T_1}{15\pi},\, C_n = \frac{4T_1}{\pi} \cdot \frac{(-1)^{n+1}}{4n^2 - 1}, \text{ де } n \neq 2
\end{equation*}
Отже, остаточним розв'язком буде 
\begin{equation}
    u(x,t) = \frac{2T_1}{\pi} + T_2 e^{-t/\tau_2}\cos k_2x + \frac{4T_1}{\pi} \sum_{n=1}^{\infty} \frac{(-1)^{n+1} \cos k_nx}{4n^2 - 1}
\end{equation}

Прямою підстановкою можна переконатися, що при $T_1 = 0$ та $T_2 = 0$ початкові умови виконуються.


%\end{document}

\part{МЕТОД ЧАСТИННИХ РОЗВ’ЯЗКІВ ТА МЕТОД РОЗКЛАДАННЯ ЗА ВЛАСНИМИ ФУНКЦІЯМИ.}

\chapter{Еволюційні задачі з неоднорідним рівнянням або неоднорідними межовими умовами: стаціонарні неоднорідності}
\documentclass[a4paper, 14pt]{extreport}

\usepackage{StyleMMF}

\begin{document}

\section{Еволюційні задачі з неоднорідним рівнянням або неоднорідними межовими умовами: стаціонарні неоднорідності}

\subsubsection{Задача №1}

\textit{Знайти коливання вертикально розташованого пружного стержня під дією сили тяжіння для $t > 0$. Верхній кінець стержня закріплений, а нижній вільний. При $t < 0$ стержень був нерухомим і деформацій не було. Знайти спочатку стаціонарний розв’язок, що відповідає положенню рівноваги стержня в полі тяжіння, а потім знайти відхилення від нього, що відповідає коливанням навколо нового положення рівноваги. Намалювати графіки розподілу поля зміщень та поля напружень у положенні рівноваги.}

\end{document}
\documentclass[a4paper, 14pt]{extreport}

\usepackage{StyleMMF}

\begin{document}

\section{Еволюційні задачі з неоднорідним рівнянням або неоднорідними межовими умовами: стаціонарні неоднорідності}

\subsubsection{Задача №3}

\textit{У стержні довжиною $l$ з непроникною бічною поверхнею відбувається дифузія частинок (коефіцієнт дифузії $D$), що мають час життя $\tau$. Через правий кінець всередину стержня подається постійний потік частинок $I_0$. Знайти стаціонарний розподіл концентрації та розв’язок, що задовольняє нульову початкову умову, якщо через лівий кінець частинки вільно виходять назовні й назад не вертаються. Знайти вигляд стаціонарного розв’язку в граничних випадках великих і малих $\tau$ та нарисувати графіки.\\
Указівка. Рівняння дифузії частинок зі скінченним часом життя має вигляд:
$u_t = D u_{xx} - u/\tau$. Його зручно переписати через так звану
довжину дифузійного зміщення $L = \sqrt{D\tau}$: \[\tau u_t = L^2 u_{xx} - u\] Величина $L$ має смисл характерної відстані, на яку частинки встигають зміститися (в середньому) за час свого життя. «Великі» й «малі» $\tau$ означають у дійсності $L \gg l$ і $L \ll l$ відповідно. Останній випадок фактично означає перехід до наближення півнескінченного стержня $-\infty < x \leq l$}

\end{document}

\chapter{Задачі з неоднорідним рівнянням або неоднорідними межовими умовами}
\documentclass[a4paper, 14pt]{extreport}

\usepackage{StyleMMF}

\begin{document}

\section{Задачі з неоднорідним рівнянням або неоднорідними межовими умовами}

\subsection{Джерела з гармонічною залежністю від часу.}

\subsubsection{Задача №1}

\textit{Знайти коливання струни $0 \leq x \leq l$, лівий кінець якої закріплений, а правий вільний, при $t > 0$ під дією розподіленої сили $f(x,t) = f(x)\cos\omega t$. При $t < 0$ струна перебувала в положенні рівноваги. Розглянути окремий випадок $f(x) = f_0$. Виділити складову розв’язку, яка відповідає усталеним вимушеним коливанням і проаналізувати картину резонансу. Перевірити, чи переходить одержаний розв’язок у розв’язок задачі 5.1 за відповідних умов.}

\end{document}
\vspace{2cm}
\documentclass[a4paper, 14pt]{extreport}

\usepackage{StyleMMF}

\begin{document}

\section{Задачі з неоднорідним рівнянням або неоднорідними межовими умовами}

\subsection{Метод розкладання по власних функціях в задачах з неоднорідним рівнянням}

\subsubsection{Задача №3}

\textit{Знайти коливання струни із закріпленими кінцями під дією сили $f(x,t) = f_0 t^N, \, N > 0$ однорідно розподіленої по довжині струни. У початковий момент струна нерухома, і зміщення дорівнює нулю. Остаточні обчислення виконати
для $N=2$.}

\end{document}

\chapter{Задачі з неоднорідними межовими умовами загального вигляду}
\documentclass[a4paper, 14pt]{extreport}

\usepackage{StyleMMF}

\begin{document}

\section{Задачі з неоднорідними межовими умовами загального вигляду}

\subsubsection{Задача №1}

\textit{Знайти коливання пружного стержня, якщо правий кінець його закріплений нерухомо, а до лівого при $t > 0$ прикладена сила $F(t)$ , шляхом зведення до задачі з неоднорідним рівнянням. Відповідь одержати для частинного випадку $F(t) = F_0 e^{-\alpha t}$. При $t < 0$ стержень перебував у положенні рівноваги.}

\end{document}
%\documentclass[a4paper, 14pt]{extreport}

%\usepackage{StyleMMF}

%\begin{document}

%\chapter{Задачі з неоднорідними межовими умовами загального вигляду}

\section[Задача №7.2]{7.2}

\textit{Розв’язати задачу №7.1 методом розкладання по власних функціях.}

%\end{document}


\part{??}
\chapter{Метод характеристик і формула Даламбера: нескінченна пряма, півнескінченна пряма та відрізок. Метод непарного продовження.}
%\documentclass[a4paper, 14pt]{extreport}

%\usepackage{StyleMMF}

%\setcounter{chapter}{7}

%\begin{document}

%\chapter{Метод характеристик і формула Даламбера: нескінченна пряма, півнескінченна пряма та відрізок. Метод непарного продовження.}

\textbf{\large Вільні коливання нескінченної струни.}

\section[Задача №8.1]{8.1}

\textit{Зобразити графічно поле зміщень і поле швидкостей нескінченної струни в характерні послідовні моменти часу, якщо початковий відхил (зміщення) має форму рівнобедреного трикутника висотою $h$ і основою $2L$, а початкова швидкість дорівнює нулю. Чи всі частини трикутника приходять у рух одразу? Відповідь поясніть.}


%\end{document}
%\documentclass[a4paper, 14pt]{extreport}

%\usepackage{StyleMMF}

%\setcounter{chapter}{7}

%\begin{document}

%\chapter{Метод характеристик і формула Даламбера: нескінченна пряма, півнескінченна пряма та відрізок. Метод непарного продовження.}

\section[Задача №8.2]{8.2}

\textit{Зобразити графічно поле зміщень і поле швидкостей нескінченної струни в характерні послідовні моменти часу, якщо початкове відхилення (зміщення) дорівнює нулю, початкова швидкість всіх точок струни на деякому відрізку довжиною $2l$ однакова і дорівнює $\nu_0$, а в усіх інших точках дорівнює нулю. У який кінцевий стан переходить струна в результаті такого процесу? З точки зору механіки системи частинок результат є парадоксальним: у початковий момент тілу був переданий імпульс (у поперечному напрямі до струни), а в кінцевому стані струна нерухома, замість того щоб рухатись рівномірно. Зобразіть також вигляд поля зміщень і поля швидкостей при наближенні до границі: $l \to 0$, $\nu_0 \to \infty$ при фіксованому $t$, якщо переданий струні імпульс залишається сталим.}


%\end{document}
%\documentclass[a4paper, 14pt]{extreport}

%\usepackage{../../main/StyleMMF}

%\setcounter{chapter}{7}

%\begin{document}

%\chapter{Метод характеристик і формула Даламбера: нескінченна пряма, півнескінченна пряма та відрізок. Метод непарного продовження.}

\textbf{\large Метод непарного продовження для півнескінченної та скінченної струни}

\section[Задача №8.3]{8.3}

\textit{Зобразити графічно поле зміщень півнескінченної струни у характерні послідовні моменти часу. Початкове відхилення має форму прямокутного трикутника, більшим катетом служить положення рівноваги струни, а вершина гострого кута орієнтована в бік кінця струни. Початкова швидкість дорівнює нулю. Кінець струни а) вільний (відносно поперечних зміщень), б) закріплений нерухомо.}

\begin{center}
    \large{\textbf{Розв'язок}}
\end{center}

\begin{figure}[h!]
    \centering

    \begin{tikzpicture}

        \tikzmath{\l = 5; \h = 2.75; \v = 1/4; \y0 = -31.5; \dt = 5;}

        \begin{axis} %% t6
            [width = \textwidth,
            %axis lines = center,
            axis y line = none, axis x line = center,
            ylabel = $t$, xlabel = $x$,
            xmin = -10, xmax = 10, ymin = \y0, ymax = 4,
            axis line style = thin, ticks = none]   
            
        \end{axis}

        \begin{axis} %% t5
            [width = \textwidth,
            axis y line = none, axis x line = center,
            xlabel = $x$,
            xmin = -10, xmax = 10, ymin = \y0 + \dt, ymax = 4 + \dt,
            axis line style = thin, ticks = none]   
            
            \filldraw[color = white] (-9,\dt/2) circle (0.1cm) node[]{\textcolor{black}{$t_5 = \frac{5l}{4v}$}};
            
            % парне продовження
            \addplot[blue!75!black, dashed] coordinates {(0.25*\l,0)(0.25*\l,\h)};
            \addplot[blue!75!black, dashed] coordinates {(0.25*\l,\h)(1.25*\l,0)};            

            % збурення струни
            \addplot[yellow!50!black, dashed] coordinates {(-1.25*\l,0)(-0.25*\l,\h)};            
            \addplot[yellow!50!black, dashed] coordinates {(-0.25*\l,\h)(-0.25*\l,0)};

            %струна
            \addplot[red, thick] coordinates {(0.25*\l,0)(0.25*\l,\h)};            
            \addplot[red, thick] coordinates {(0.25*\l,\h)(1.25*\l,0)};

        \end{axis}

        \begin{axis} %% t4
            [width = \textwidth,
            axis y line = none, axis x line = center,
            xlabel = $x$,
            xmin = -10, xmax = 10, ymin = \y0 + 2*\dt,ymax = 4 + 2*\dt,
            axis line style = thin, ticks = none]   
            
            \filldraw[color = white] (-9,\dt/2) circle (0.1cm) node[]{\textcolor{black}{$t_4 = \frac{l}{v}$}};
            
            % парне продовження
            \addplot[blue!75!black, dashed] coordinates {(0,0)(0,\h)};
            \addplot[blue!75!black, dashed] coordinates {(0,\h)(\l,0)};  

            % збурення струни
            \addplot[yellow!50!black, dashed] coordinates {(0,0)(0,\h)};            
            \addplot[yellow!50!black, dashed] coordinates {(0,\h)(-\l,0)};

            %струна
            \addplot[red, thick] coordinates {(0,0)(0,\h)};            
            \addplot[red, thick] coordinates {(0,\h)(\l,0)};

        \end{axis}

        \begin{axis} %% t3
            [width = \textwidth,
            axis y line = none, axis x line = center,
            xlabel = $x$,
            xmin = -10, xmax = 10, ymin = \y0 + 3*\dt, ymax = 4 + 3*\dt,
            axis line style = thin, ticks = none]   
            
            \filldraw[color = white] (-9,\dt/2) circle (0.1cm) node[] {\textcolor{black}{$t_3 = \frac{3l}{4v}$}};
            
            % парне продовження
            \addplot[blue!75!black, dashed] coordinates {(-0.25*\l,0)(-0.25*\l,\h)};
            \addplot[blue!75!black, dashed] coordinates {(-0.25*\l,\h)(0.75*\l,0)};            

            % збурення струни
            \addplot[yellow!50!black, dashed] coordinates {(-0.75*\l,0)(0.25*\l,\h)};            
            \addplot[yellow!50!black, dashed] coordinates {(0.25*\l,\h)(0.25*\l,0)};

            %струна
            \addplot[red, thick] coordinates {(0,3*\h/2)(0.25*\l,3*\h/2)};            
            \addplot[red, thick] coordinates {(0.25*\l,3*\h/2)(0.25*\l,\h/2)};
            \addplot[red, thick] coordinates {(0.25*\l,\h/2)(0.75*\l,0)};

        \end{axis}

        \begin{axis} %% t2
            [width = \textwidth,
            axis y line = none, axis x line = center,
            xlabel = $x$,
            xmin = -10, xmax = 10, ymin = \y0 + 4*\dt, ymax = 4 + 4*\dt,
            axis line style = thin, ticks = none]   
            
            \filldraw[color = white] (-9,\dt/2) circle (0.1cm) node[]{\textcolor{black}{$t_2 = \frac{l}{2v}$}};
            
            % парне продовження
            \addplot[blue!75!black, dashed] coordinates {(-0.5*\l,0)(-0.5*\l,\h)};
            \addplot[blue!75!black, dashed] coordinates {(-0.5*\l,\h)(0.5*\l,0)};            

            % збурення струни
            \addplot[yellow!50!black, dashed] coordinates {(-0.5*\l,0)(0.5*\l,\h)};            
            \addplot[yellow!50!black, dashed] coordinates {(0.5*\l,\h)(0.5*\l,0)};

            %струна
            \addplot[red, thick] coordinates {(0,\h)(0.5*\l,\h)};            
            \addplot[red, thick] coordinates {(0.5*\l,\h)(0.5*\l,0)};
            
        \end{axis}

        \begin{axis} %% t1
            [width = \textwidth,
            axis y line = none, axis x line = center,
            xlabel = $x$,
            xmin = -10, xmax = 10, ymin = \y0 + 5*\dt, ymax = 4 + 5*\dt,
            axis line style = thin, ticks = none]   
            
            \filldraw[color = white] (-9,\dt/2) circle (0.1cm) node[]{\textcolor{black}{$t_1 = \frac{l}{4v}$}};

            % парне продовження
            \addplot[blue!75!black, dashed] coordinates {(-0.75*\l,0)(-0.75*\l,\h)};
            \addplot[blue!75!black, dashed] coordinates {(-0.75*\l,\h)(0.25*\l,0)};            

            % збурення струни
            \addplot[yellow!50!black, dashed] coordinates {(-0.25*\l,0)(0.75*\l,\h)};            
            \addplot[yellow!50!black, dashed] coordinates {(0.75*\l,\h)(0.75*\l,0)};

            %струна
            \addplot[red, thick] coordinates {(0,\h/2)(0.25*\l,\h/2)};            
            \addplot[red, thick] coordinates {(0.25*\l,\h/2)(0.75*\l,\h)};            
            \addplot[red, thick] coordinates {(0.75*\l,\h)(0.75*\l,0)};
            
        \end{axis}

        \begin{axis} %% t0
            [width = \textwidth,
            axis lines = center,
            %axis y line = none, axis x line = center,
            xlabel = $x$, ylabel = $t$,
            xmin = -10, xmax = 10, ymin = \y0 + 6*\dt, ymax = 4 + 6*\dt,
            axis line style = thin, ticks = none]   
            
            %відмітки на осі Ox 
            \addplot[black, samples=10, domain=-10:10, name path=three] coordinates {(0,-0.1)(0,0.1)}
            node[anchor=130, pos=0.5] {\footnotesize{0}};
            \addplot[black, samples=10, domain=-10:10, name path=three] coordinates {(-\l,-0.1)(-\l,0.1)}
            node[anchor=70, pos=0.5] {\footnotesize{$-l$}};
            \addplot[black, samples=10, domain=-10:10, name path=three] coordinates {(\l,-0.1)(\l,0.1)}
            node[anchor=90, pos=0.5] {\footnotesize{$l$}};
            
            \filldraw[color = white] (-9,\dt/2) circle (0.1cm) node[]{\textcolor{black}{$t_0 = 0$}};
        
            %характеристики
            \addplot[gray, samples=50, domain=0:1.7*\l] {x/\v};            
            \addplot[gray, samples=50, domain=-0.7*\l:\l] {(\l - x)/\v};            

            \addplot[gray, samples=50, domain=-\l:1.7*\l] {(\l + x)/\v};            
            \addplot[gray, samples=50, domain=-2.7*\l:0] {-x/\v}; 

            % парне продовження
            \addplot[blue!75!black, dashed] coordinates {(-\l,0)(-\l,\h)};
            \addplot[blue!75!black, dashed] coordinates {(-\l,\h)(0,0)};  

            % збурення струни
            \addplot[yellow!50!black, dashed] coordinates {(0,0)(\l,\h)};            
            \addplot[yellow!50!black, dashed] coordinates {(\l,\h)(\l,0)};

            %струна
            \addplot[red, thick] coordinates {(0,0)(\l,\h)};            
            \addplot[red, thick] coordinates {(\l,\h)(\l,0)};

        \end{axis}
    \end{tikzpicture}

    \caption{Парне продовження}
\end{figure}

\begin{figure}[h]
    \centering

    \begin{tikzpicture}

        \tikzmath{\l = 5; \h = 2.25; \v = 1/4; \y0 = -33; \dt = 5;}

        \begin{axis} %% t6
            [width = \textwidth,
            %axis lines = center,
            axis y line = none, axis x line = center,
            ylabel = $t$, xlabel = $x$,
            xmin = -10, xmax = 10, ymin = \y0, ymax = 2,
            axis line style = thin, ticks = none]   
            
        \end{axis}

        \begin{axis} %% t5
            [width = \textwidth,
            axis y line = none, axis x line = center,
            xlabel = $x$,
            xmin = -10, xmax = 10, ymin = \y0 + \dt, ymax = 2 + \dt,
            axis line style = thin, ticks = none]   
            
            \filldraw[color = white] (-9,\dt/2) circle (0.1cm) node[]{\textcolor{black}{$t_5 = \frac{5l}{4v}$}};
            
            % непарне продовження
            \addplot[blue!75!black, dashed] coordinates {(0.25*\l,0)(0.25*\l,-\h)};
            \addplot[blue!75!black, dashed] coordinates {(0.25*\l,-\h)(1.25*\l,0)};            

            % збурення струни
            \addplot[yellow!50!black, dashed] coordinates {(-1.25*\l,0)(-0.25*\l,\h)};            
            \addplot[yellow!50!black, dashed] coordinates {(-0.25*\l,\h)(-0.25*\l,0)};

            %струна
            \addplot[red, thick] coordinates {(0.25*\l,0)(0.25*\l,\h)};            
            \addplot[red, thick] coordinates {(0.25*\l,\h)(1.25*\l,0)};

        \end{axis}

        \begin{axis} %% t4
            [width = \textwidth,
            axis y line = none, axis x line = center,
            xlabel = $x$,
            xmin = -10, xmax = 10, ymin = \y0 + 2*\dt,ymax = 2 + 2*\dt,
            axis line style = thin, ticks = none]   
            
            \filldraw[color = white] (-9,\dt/2) circle (0.1cm) node[]{\textcolor{black}{$t_4 = \frac{l}{v}$}};
            
            % непарне продовження
            \addplot[blue!75!black, dashed] coordinates {(0,0)(0,-\h)};
            \addplot[blue!75!black, dashed] coordinates {(0,-\h)(\l,0)};  

            % збурення струни
            \addplot[yellow!50!black, dashed] coordinates {(0,0)(0,\h)};            
            \addplot[yellow!50!black, dashed] coordinates {(0,\h)(-\l,0)};

            %струна
            \addplot[red, thick] coordinates {(0,0)(0,\h)};            
            \addplot[red, thick] coordinates {(0,\h)(\l,0)};

        \end{axis}

        \begin{axis} %% t3
            [width = \textwidth,
            axis y line = none, axis x line = center,
            xlabel = $x$,
            xmin = -10, xmax = 10, ymin = \y0 + 3*\dt, ymax = 2 + 3*\dt,
            axis line style = thin, ticks = none]   
            
            \filldraw[color = white] (-9,\dt/2) circle (0.1cm) node[] {\textcolor{black}{$t_3 = \frac{3l}{4v}$}};
            
            % непарне продовження
            \addplot[blue!75!black, dashed] coordinates {(-0.25*\l,0)(-0.25*\l,-\h)};
            \addplot[blue!75!black, dashed] coordinates {(-0.25*\l,-\h)(0.75*\l,0)};            

            % збурення струни
            \addplot[yellow!50!black, dashed] coordinates {(-0.75*\l,0)(0.25*\l,\h)};            
            \addplot[yellow!50!black, dashed] coordinates {(0.25*\l,\h)(0.25*\l,0)};

            %струна
            \addplot[red, thick] coordinates {(0,0)(0.25*\l,\h/2)};            
            \addplot[red, thick] coordinates {(0.25*\l,\h/2)(0.75*\l,0)};

        \end{axis}

        \begin{axis} %% t2
            [width = \textwidth,
            axis y line = none, axis x line = center,
            xlabel = $x$,
            xmin = -10, xmax = 10, ymin = \y0 + 4*\dt, ymax = 2 + 4*\dt,
            axis line style = thin, ticks = none]   
            
            \filldraw[color = white] (-9,\dt/2) circle (0.1cm) node[]{\textcolor{black}{$t_2 = \frac{l}{2v}$}};
            
            % непарне продовження
            \addplot[blue!75!black, dashed] coordinates {(-0.5*\l,0)(-0.5*\l,-\h)};
            \addplot[blue!75!black, dashed] coordinates {(-0.5*\l,-\h)(0.5*\l,0)};            

            % збурення струни
            \addplot[yellow!50!black, dashed] coordinates {(-0.5*\l,0)(0.5*\l,\h)};            
            \addplot[yellow!50!black, dashed] coordinates {(0.5*\l,\h)(0.5*\l,0)};

            %струна
            \addplot[red, thick] coordinates {(0,0)(0.5*\l,\h)};            
            \addplot[red, thick] coordinates {(0.5*\l,\h)(0.5*\l,0)};
            
        \end{axis}

        \begin{axis} %% t1
            [width = \textwidth,
            axis y line = none, axis x line = center,
            xlabel = $x$,
            xmin = -10, xmax = 10, ymin = \y0 + 5*\dt, ymax = 2 + 5*\dt,
            axis line style = thin, ticks = none]   
            
            \filldraw[color = white] (-9,\dt/2) circle (0.1cm) node[]{\textcolor{black}{$t_1 = \frac{l}{4v}$}};

            % непарне продовження
            \addplot[blue!75!black, dashed] coordinates {(-0.75*\l,0)(-0.75*\l,-\h)};
            \addplot[blue!75!black, dashed] coordinates {(-0.75*\l,-\h)(0.25*\l,0)};            

            % збурення струни
            \addplot[yellow!50!black, dashed] coordinates {(-0.25*\l,0)(0.75*\l,\h)};            
            \addplot[yellow!50!black, dashed] coordinates {(0.75*\l,\h)(0.75*\l,0)};

            %струна
            \addplot[red, thick] coordinates {(0,0)(0.25*\l,\h/2)};            
            \addplot[red, thick] coordinates {(0.25*\l,\h/2)(0.75*\l,\h)};            
            \addplot[red, thick] coordinates {(0.75*\l,\h)(0.75*\l,0)};
            
        \end{axis}

        \begin{axis} %% t0
            [width = \textwidth,
            axis lines = center,
            %axis y line = none, axis x line = center,
            xlabel = $x$, ylabel = $t$,
            xmin = -10, xmax = 10, ymin = \y0 + 6*\dt, ymax = 2 + 6*\dt,
            axis line style = thin, ticks = none]   
            
            %відмітки на осі Ox 
            \addplot[black, samples=10, domain=-10:10, name path=three] coordinates {(0,-0.1)(0,0.1)}
            node[anchor=130, pos=0.5] {\footnotesize{0}};
            \addplot[black, samples=10, domain=-10:10, name path=three] coordinates {(-\l,-0.1)(-\l,0.1)}
            node[anchor=70, pos=0.5] {\footnotesize{$-l$}};
            \addplot[black, samples=10, domain=-10:10, name path=three] coordinates {(\l,-0.1)(\l,0.1)}
            node[anchor=90, pos=0.5] {\footnotesize{$l$}};
            
            \filldraw[color = white] (-9,\dt/2) circle (0.1cm) node[]{\textcolor{black}{$t_0 = 0$}};
        
            %характеристики
            \addplot[gray, samples=50, domain=0:1.7*\l] {x/\v};            
            \addplot[gray, samples=50, domain=-0.7*\l:\l] {(\l - x)/\v};            

            \addplot[gray, samples=50, domain=-\l:1.7*\l] {(\l + x)/\v};            
            \addplot[gray, samples=50, domain=-2.7*\l:0] {-x/\v}; 

            % непарне продовження
            \addplot[blue!75!black, dashed] coordinates {(-\l,0)(-\l,-\h)};
            \addplot[blue!75!black, dashed] coordinates {(-\l,-\h)(0,0)};  

            % збурення струни
            \addplot[yellow!50!black, dashed] coordinates {(0,0)(\l,\h)};            
            \addplot[yellow!50!black, dashed] coordinates {(\l,\h)(\l,0)};

            %струна
            \addplot[red, thick] coordinates {(0,0)(\l,\h)};            
            \addplot[red, thick] coordinates {(\l,\h)(\l,0)};

        \end{axis}
    \end{tikzpicture}

    \caption{Непарне продовження}
\end{figure}

%\end{document}

\chapter{Використання загального розв’язку хвильового рівняння у вигляді суперпозиції зустрічних хвиль. Нестаціонарна задача розсіяння.}
%\documentclass[a4paper, 14pt]{extreport}

%\usepackage{StyleMMF}

%\setcounter{chapter}{8}

%\begin{document}

%\chapter{Використання загального розв’язку хвильового рівняння у вигляді суперпозиції зустрічних хвиль. Нестаціонарна задача розсіяння.}

\section[Задача №9.1]{9.1}

\textit{Півнескінченна струна (сила натягу $T_0$, швидкість хвиль $v$) з вільним кінцем перебувала у стані рівноваги. Починаючи з моменту часу $t=0$, на її кінець діє у поперечному напрямі задана сила $F(t)$. Знайти розв’язок задачі про вимушені коливання струни у квадратурах, а також знайти поле зміщень у явному вигляді і зобразити графічно форму струни, якщо: а) $F(t) = F_0$, б) $F(t) = F_0 \cos\omega t$, в) $F(t) = F_0 \sin\omega t$\\
Задача є прикладом так званої задачі про поширення межового режиму: задачі для півнескінченної струни з неоднорідною межовою умовою. Указівка: задача відшукання форми хвилі, створеної таким джерелом, зводиться до диференціального рівняння першого порядку; проблема знаходження сталої інтегрування вирішується, якщо врахувати умову неперервності хвильового поля на передньому фронті хвилі, тобто на межі областей $x > vt$ й $x < vt$.}

\begin{center}
    \large{\textbf{Розв'язок}}
\end{center}

\noindent Формальна постановка задачі:
\begin{equation} %\label{probcond12}
    \left\{ \begin{aligned} %%
            \;&u = u(x,t), \\
            &u_{tt} = v^2 u_{xx}, \\
            &0 \leq x < \infty, t \geq 0 \\
            &u_x(0,t) = F(t)/\beta = f(t),\\
            &u(x,0) = 0, \, u_t(x,0) = 0.
    \end{aligned} \right.
\end{equation}

Шукаємо розв'язок у вигляді:
\begin{equation}
    u(x,t) = g(t - x/v) + h(t + x/v)
\end{equation}

Із початкових умов маємо систему:
\begin{equation}
    \left\{ \begin{aligned}
        \;&u(x,0) = g(-x/v) + h(x/v) = 0,\\
        &u_t(x,0) = g'(-x/v) + h'(x/v) = 0.
\end{aligned} \right.
\end{equation}
Інтегруємо друге рівняння та розв'язуємо лінійну систему
\begin{equation*}
    \begin{gathered}
        \left\{ \begin{aligned}
            \;&g(-\xi) + h(\xi) = 0,\\
            &\int g'(-\xi)\;\mathrm{d}\xi + \int h'(\xi)\;\mathrm{d}\xi = 0;
        \end{aligned} \right.
        \quad\Rightarrow\quad
        \left\{ \begin{aligned}
            \;&g(-\xi) + h(\xi) = 0,\\
            &h(\xi) + C_2 - g(-\xi) + C_1 = 0;
        \end{aligned} \right.
        \quad\Rightarrow\\
        \Rightarrow\quad
        \left\{ \begin{aligned}
            \;&g(-\xi) + h(\xi) = 0,\\
            &-g(-\xi) + h(\xi) = 2\widetilde{C};
        \end{aligned} \right.
        \quad\Rightarrow\quad
        \left\{ \begin{aligned}
            \;&g(-\xi) = -\widetilde{C},\\
            &h(\xi) = \widetilde{C}.
        \end{aligned} \right.
    \end{gathered}
\end{equation*}
Обираємо константу інтегрування $\widetilde{C}$ рівною нулю, тоді 
\begin{equation*}
    \left\{ \begin{aligned}
        \;&g(-x/v) = 0,\\
        &h(x/v) = 0;
    \end{aligned} \right.
    \quad\Rightarrow\quad
    \left\{ \begin{aligned}
        \;&g(\xi) = 0, \text{ при } \xi < 0,\\
        &h(\eta)= 0, \text{ при } \eta > 0.
    \end{aligned} \right.
\end{equation*}

Отже, $h(t + x/v) = 0$ в нашій задачі, адже $x \geq 0$ та $t > 0$, що фізично означає відсутність хвилі, яка поширюється з нескінченность до краю струни (падаючої хвилі).\\
Маємо розв'язок у виді біжучої хвилі, яка створюється межовою умовою.
\begin{equation}
    u(x,t) = g(t - x/v)
\end{equation}

З межової умови визначимо розв'язок
\begin{equation}
    u_x(0,t) = f(t)
    \quad\Rightarrow\quad
    -\frac{1}{v}g'(t) = f(t)
    \quad\Rightarrow\quad
    g(t) = - v \int\limits_0^t f(\tau) \;\mathrm{d}\tau
\end{equation}

Загальний вид розв'язку
\begin{equation}
    u(x,t) = - v \int\limits_0^{t-x/v} f(\tau) \;\mathrm{d}\tau
\end{equation}

Обчислимо розв'язки для визначених межових умов:
\begin{enumerate}
    \item[\text{а})] \[u(x,t) = - \frac{vF_0}{\beta} \int\limits_0^{t-x/v}  \;\mathrm{d}\tau = -\frac{vF_0}{\beta} \left(t - \frac{x}{v}\right),\]
    \item[\text{б})] \[u(x,t) = - \frac{vF_0}{\beta} \int\limits_0^{t-x/v} \sin\omega\tau \;\mathrm{d}\tau = -\frac{vF_0}{\omega\beta} \cos\omega(t - x/v),\]
    \item[\text{в})] \[u(x,t) = - \frac{vF_0}{\beta} \int\limits_0^{t-x/v} \cos\omega\tau \;\mathrm{d}\tau = \frac{vF_0}{\omega\beta} \sin\omega(t - x/v).\]
\end{enumerate} 

%\end{document}
%\documentclass[a4paper, 14pt]{extreport}

%\usepackage{StyleMMF}

%\setcounter{chapter}{8}

%\begin{document}

%\chapter{Використання загального розв’язку хвильового рівняння у вигляді суперпозиції зустрічних хвиль. Нестаціонарна задача розсіяння.}

\section[Задача №9.2]{9.2}

\textit{При $t < t_0$ по півнескінченній струні $0 \geq x < \infty$ у напрямі її кінця поширюється хвиля заданої форми (падаючий «імпульс»), причому передній фронт хвилі при $t \geq t_0$ не досягає кінця струни. Знайти коливання струни при $t > t_0$ і форму відбитого імпульсу для скінченного $t_0$ і $t_0 \to -\infty$. Кінець струни: а) закріплений жорстко; б) зазнає дії сили тертя, пропорційної швидкості. Як пояснити відсутність відбивання при певному значенні коефіцієнта тертя?\\
Указівка: звести до задачі про поширення межового режиму типу задачі 9.1, використати вказівку до цієї задачі та умову, що при $t < t_0$ фронт хвилі не досягає кінця струни.}

\begin{center}
    \large{\textbf{Розв'язок}}
\end{center}

\noindent Формальна постановка задачі:
\begin{equation} \label{withered}
    \left\{ \begin{aligned} 
            \;&u = u(x,t), \\
            &u_{tt} = v^2 u_{xx}, \\
            &0 \leq x \leq \infty, t \geq t_0 \\
            &\text{а)}\; u(0,t) = 0,\\
            &\text{б)}\; \mu u_x(0,t) = u_t(0,t),\\
            &u(x,t) = F_0(vt + x) - \text{падаюча хвиля, для  } t<t_0\\
            &F(t) = 0,\;t < t_0 \text{  - падаюча хвиля не досягне кінця струни до }t_0.\\
    \end{aligned} \right.
\end{equation}
Де $\mu$ - коефіцієнт тертя. 

Повний розв'язок хвильового рівняння представляється комбінацією двох збурень, що поширюються у протилежних напрямках та є фукнціями однієї змінної. У нашому випадку це падаюча та відбита хвиля у відповідному порядку:

\begin{equation}
    u(x,t) = u_{\text{пад}}(x,t) + u_{\text{від}}(x,t) = F(x + tv) + f(x - tv)
\end{equation}

З постановки (\ref{withered}) ми знаємо частину, що відповідає падаючій хвилі:
\begin{equation*}
    u_{\text{пад}}(x,t) = F_0(x + tv)
\end{equation*}

Звідси маємо умову на $u_{\text{від}}$ при $t<t_0$:

\begin{equation} \label{hunger}
    u_{\text{від}}(x,t)=0
\end{equation}

Для випадку \textit{а)} можна було б cкористатися доведенною в лекціях вимогою на непарність $u(x,t)$ по змінній $x$ для приведенної граничної умови. Але розв'яжемо обидва випадки одним методом. Розглянемо точку $x=0$ у момент часу $t>t_0$:

\begin{equation}  \label{slave}
    \begin{aligned} 
            &\text{а)}\;u_{\text{від}}(0,t) + u_{\text{пад}}(0,t) = 0 \quad\Rightarrow\\
            &\Rightarrow\quad u_{\text{від}}(0,t) = f(x - tv)|_{x=0} = f(-tv) = -F_0(tv), \\
            & \\
            &\text{б)}\; \mu u_x(0,t) - u_t(0,t) = 0 \quad\Rightarrow\\
            &\Rightarrow\quad \mu \frac{\partial u_{\text{від}}}{\partial x}\bigg|_{x=0} +\frac{\partial u_{\text{від}}}{\partial t}\bigg|_{x=0} = (\mu + v)f'(-tv) =\\
            &= -(\mu - v)F_0'(tv)
    \end{aligned} 
\end{equation}

З отриманих рівнянь (\ref{slave}) та умови (\ref{hunger}) можна одразу отримати відповіді:

\begin{equation} 
    \begin{aligned} 
            &\text{а)}\;u_{\text{від}}(x,t) =
                \begin{cases}
                    0 & \text{, якщо } t < t_0, \text{ або } x > vt \\
                    -F_0(tv - x) & \text{, якщо } t > t_0,\;x < vt.
                \end{cases}
            & \\
            & \\
            &\text{б)}\;u_{\text{від}}(x,t)=
                \begin{cases}
                    0 & \text{, якщо } t < t_0, \text{ або } x > vt \\
                    \frac{\mu - v}{\mu + v}(F_0(tv - x) + c) & \text{, якщо } t > t_0,\;x < vt.
                \end{cases}
    \end{aligned} 
\end{equation}

Де \textit{c} - константа інтегрування. Накладаючи вимогу неперервності $u_{\text{від}}(0,t_0) = 0$ маємо $c = -F_0(0)$. Якщо $F(\xi)$ є неперервною функцією, то $c = 0$. Ці результати можна переписати у більш винтонченній формі за допомогою тета-функції:

\begin{equation} 
    \begin{aligned} 
            &\text{а)}\;u_{\text{від}}(x,t)=-F_0(tv-x) \Theta\big(v(t - t_0) - x\big)\\
            & \\
            &\text{б)}\;u_{\text{від}}(x,t)=\frac{\mu - v}{\mu + v} \big(F_0(tv-x)-F_0(0)\big) \Theta\big(v(t - t_0) - x\big).
    \end{aligned} 
\end{equation}

При $t_0 \rightarrow - \infty, \quad \Theta\big(v(t - t_0) - x\big) = 1$, тож поле матиме форму: 

\begin{equation} 
    \begin{aligned} 
            &\text{а)}\;u_{\text{від}}(x,t)=-F_0(tv - x)\\
            & \\
            &\text{б)}\;u_{\text{від}}(x,t) = \frac{\mu - v}{\mu + v} (F_0(tv - x) - F_0(0)).
    \end{aligned} 
\end{equation}


%\end{document}

\chapter{Приведення лінійних рівнянь у частинних похідних 2-го порядку з двома змінними до заданого вигляду}
%\documentclass[a4paper, 14pt]{extreport}

%\usepackage{StyleMMF}

%\setcounter{chapter}{9}

%\begin{document}

%\chapter{Приведення лінійних рівнянь у частинних похідних 2-го порядку з двома змінними до заданого вигляду}

\section[Задача №10.1]{10.1}

\textit{Визначити тип рівняння $u_{xx} + 4u_{xy} + cu_{yy} + u_x = 0$, привести його до канонічного вигляду для $c = 0$ і знайти загальний розв’язок.}

\begin{center}
    \large{\textbf{Розв'язок}}
\end{center}

Загальний вид рівняння:
\begin{equation}
    a_{11}u_{xx} + 2a_{12}u_{xy} + a_{22}u_{yy} + b_1u_x + b_2u_y + cu = 0, \quad \text{або} \quad \hat{L}u + cu = 0
\end{equation}
Тип рівняння визначається визначником матриці, яка складається з коефіцієнтів перед другими похідними. \textit{Фактично оператор $\hat{L}$ є білінійною формою з лінійної алгебри, де замість змінних будуть похідні.}
\begin{equation}
    \Delta = -
    \begin{vmatrix}
        a_{11} & a_{12}\\
        a_{12} & a_{22}
    \end{vmatrix} 
    = a_{12}^2 - a_{11}a_{22} = 2^2 - 1\cdot c = 4 - c 
\end{equation}
При $c = 0$ визначник $\Delta > 0$, тому маємо рівняння гіперболічного типу. 

Суть канонізації -- перейти до нових змінних для яких рівняння прийматиме канонічний вид. Для визначення таких змінних записуємо  спочатку характеристичне рівняння:
\begin{equation}
    a_{11} (\mathrm{d}y)^2 + 2a_{12} \mathrm{d}x\mathrm{d}y + a_{22} (\mathrm{d}x)^2 = 0, \quad \text{або} \quad \frac{\mathrm{d}y}{\mathrm{d}x} = \frac{a_{12} \pm \sqrt{a_{12}^2 - a_{11}a_{22}}}{a_{12}}
\end{equation}
Обидва рівняння приводять до 
\begin{equation}
    y'_1 = 4, \qquad y'_2 = 0.
\end{equation}
Звідси маємо перші інтеграли
\begin{equation}
    \begin{gathered}
        y'_1 = 4 
        \quad\Rightarrow\quad
        y_1 = 4x
        \quad\Rightarrow\quad
        \Phi(x,y) = y - 4x = C_1\\
        y'_2 = 0 
        \quad\Rightarrow\quad
        \Psi(x,y) = y = C_2
    \end{gathered}
\end{equation}
З теорії нові змінні отримаємо формальною заміною $C_1 \to \xi$, $C_2 \to \eta$. Отже, нові змінні
\begin{equation}
    \left\{ \begin{aligned}
        \xi = y - 4x,\\
        \eta = y.
    \end{aligned} \right.
\end{equation}

Далі треба зробити заміну змінних. Для цього окремо випишемо похідні від нових змінних
\begin{equation*}
    \xi_x = -4,\, \xi_y = 1,\, \eta_x = 0,\, \eta_y = 1,\, \xi_{xy} = \eta_{xy} = 0,\, \xi_{xx} = \eta_{xx} = 0,\, \xi_{yy} = \eta_{yy} = 0.
\end{equation*}  

Тепер не важко виконати заміну змінних 
\begin{equation*}
    \begin{gathered}
        u_x = u_\xi \xi_x + u_\eta \eta_x = -4 u_\xi,\\
        u_y = u_\xi \xi_y + u_\eta \eta_y = u_\xi + u_\eta,\\
        u_{xx} = (-4 u_\xi)'_x = -4(u_{\xi\xi} \xi_x + u_{\eta\eta} \eta_x) = 16 u_{\xi\xi},\\
        u_{xy} = (-4 u_\xi)'_y = -4(u_{\xi\xi} \xi_y + u_{\eta\eta} \eta_y) = -4(u_{\xi\xi} + u_{\xi\eta}).
    \end{gathered}
\end{equation*}
Підставляємо отримані вирази в рівняння 
\begin{equation*}
    u_{xx} + 4u_{xy} + u_x = 16 u_{\xi\xi} - 16(u_{\xi\xi} + u_{\xi\eta}) - 4u_\xi = 0
    \quad\Rightarrow\quad
    u_{\xi\eta} + \frac{1}{4}u_\xi = 0
\end{equation*}
Отже, отримали рівняння в канонічному виді
\begin{equation}
    u_{\xi\eta} + \frac{1}{4}u_\xi = 0
\end{equation}

Розв'яжемо отримане рівняння. Легко побачити, що по $\xi$ можна проінтегрувати. 
\begin{equation*}
    u_{\xi\eta} + \frac{1}{4}u_\xi = 0
    \quad\Rightarrow\quad
    \left(u_\eta + \frac{1}{4}u\right)'_\xi = 0
    \quad\Rightarrow\quad
    u_\eta + \frac{1}{4}u = f(\eta)
\end{equation*}
Звідки ми отримали лінійне неоднорідне диференційне рівняння однієї змінної. Розв'яжемо спочатку однорідне рівняння
\begin{equation*}
    \tilde{u}_\eta + \frac{1}{4}\tilde{u} = 0
    \quad\Rightarrow\quad
    \ln\tilde{u} = -\frac{1}{4}\eta + \ln C 
    \quad\Rightarrow\quad
    \tilde{u} = Ce^{-\eta/4} 
\end{equation*}
Варіюєму змінну $C \to C(\eta)$
\begin{equation*}
    u = C(\eta)e^{-\eta/4} 
    \quad\Rightarrow\quad
    C'e^{-\eta/4} = f(\eta)
    \quad\Rightarrow\quad
    C(\eta) = \int f(\eta) e^{\eta/4} \;\mathrm{d}\eta + \gamma
\end{equation*}

Отже, маємо розв'язок рівняння
\begin{equation}
    u(\xi,\eta) = \gamma e^{-\eta/4} + e^{-\eta/4} \cdot \int^\eta f(z) e^{z/4} \;\mathrm{d}z
\end{equation} 


%\end{document}
%\documentclass[a4paper, 14pt]{extreport}
%
%\usepackage{../../main/StyleMMF}
%
%\setcounter{chapter}{9}
%
%\begin{document}
%
%\chapter{Приведення лінійних рівнянь у частинних похідних 2-го порядку з двома змінними до заданого вигляду}

\section[Задача №10.5]{10.5}

\textit{Привести до простішого вигляду рівняння $u_t = a^2(u_{xx} + \alpha u_x) + cu$.}

\begin{center}
    \large{\textbf{Розв'язок}}
\end{center}

Перенесемо всі доданки на одну сторону та поділимо на $a^2$
\begin{equation} \label{eq10,5}
    u_{xx} + \alpha u_x - a^{-2}u_t + a^{-2}cu = 0
\end{equation}
Маємо рівняння параболічного типу в канонічному виді.\\
Спростимо його позбавившись якнайбільше від похідних першого порядку. Це зробимо використовуючи наступну заміну змінних та функції
\begin{equation}
    u(x,t) =  e^{\lambda x + \mu t} v(x,t)
\end{equation}

Обчислимо перші похідні та другу похідну по просторовый змінній
\begin{equation*}
    u_x =  e^{\lambda x + \mu t} (v_x + \lambda v), \; u_t =  e^{\lambda x + \mu t} (v_t + \mu v), \; u_{xx} =  e^{\lambda x + \mu t} (v_{xx} + 2\lambda v_x + \lambda^2 v)
\end{equation*}
Підставляємо їх в рівняння (\ref{eq10,5}) та ділимо його на експоненту
\begin{equation*}
    v_{xx} + 2\lambda v_x + \lambda^2 v + \alpha (v_x + \lambda v) - a^{-2}(v_t + \mu v) + a^{-2}cv = 0
\end{equation*}
Зводимо подібні доданки 
\begin{equation}
    v_{xx} + (2\lambda + \alpha)v_x - a^{-2}v_t + (\lambda^2 + \alpha \lambda - a^{-2}\mu  + a^{-2}c)v = 0
\end{equation}
Звідси визначимо $\lambda$ та $\mu$, прирівнючи вирази перед $v_x$ та $v$ до нуля.
\begin{equation}
    \left\{ \begin{aligned}
        \;& 2\lambda + \alpha = 0,\\
          & \lambda^2 + \alpha \lambda - a^{-2}\mu  + a^{-2}c = 0;
    \end{aligned} \right.
    \quad\Rightarrow\quad
    \left\{ \begin{aligned}
        \;& \lambda = -\alpha/2,\\
          & \mu = c - a^2\alpha^2/4;
    \end{aligned} \right.
\end{equation}
Таким чином, заміна
\begin{equation}
    u(x,t) =  \exp\bigg[\bigg(c - \frac{a^2\alpha^2}{4}\bigg)t - \frac{\alpha t}{2}\bigg] v(x,t)
\end{equation}
спрощує вихідне рівняння (\ref{eq10,5}) до вигляду
\begin{equation}
    v_t = a^2v_{xx}
\end{equation}

%\end{document}
%\documentclass[a4paper, 14pt]{extreport}
%
%\usepackage{../../main/StyleMMF}
%
%\setcounter{chapter}{9}
%
%\begin{document}
%
%\chapter{Приведення лінійних рівнянь у частинних похідних 2-го порядку з двома змінними до заданого вигляду}

\section[Задача №10.8]{10.8}

\textit{Привести рівняння $u_{tt} = v^2 \big(u_{rr} + (2/r) u_r\big) + cu$ до самоспряженого вигляду: \[\rho(r)u_{tt} = \frac{\partial\;}{\partial r} \left(k(r) \frac{\partial u}{\partial r}\right) - q(r)u.\]}

\begin{center}
    \large{\textbf{Розв'язок}}
\end{center}

Запишемо рівняння, розкриваючи дужки 
\begin{equation} \label{eq10,8}
    u_{tt} = v^2u_{rr} + \frac{2v^2}{r}u_r + cu
\end{equation}
Домножимо його на довільну функцію $\rho(r)$, яку ми визначимо далі
\begin{equation*}
    \rho(r)u_{tt} = v^2\rho(r)u_{rr} + \frac{2v^2\rho(r)}{r}u_r + c\rho(r)u
\end{equation*}

Порівнюючи його з рівнянням у самоспряженому вигляді, позначимо \[k(r) = v^2\rho(r),\quad k'(r) = \frac{2v^2\rho(r)}{r},\quad q(r) = - c\rho(r).\] Ми отримали вирази для функції $k(r)$ та її похідної. Звідси і знайдемо рівняння для $\rho(r)$: диференцюємо вираз для $k(r)$ і прирівнюємо до $k'(r)$.
\begin{equation}
    \big(v^2\rho(r)\big)' = \frac{2v^2\rho(r)}{r}
\end{equation}
Розв'яжемо отримане рівняння
\begin{equation*}
    \rho'(r) = \frac{2\rho(r)}{r}
    \;\Rightarrow\;
    \frac{\mathrm{d}\rho}{\rho} = \frac{2\mathrm{d}r}{r}
    \;\Rightarrow\;
    \ln\rho = 2\ln r + \ln K = \ln(Kr^2)
    \;\Rightarrow\;
    \rho(r) = Kr^2
\end{equation*}
Оскільки на $\rho(r)$ ми домножили все рівняння і враховуючи, що рівняння є однорідним, то можна покласти $K = 1$.

Маємо вихідне рівняння (\ref{eq10,8}) в самоспряженому вигляді:
\begin{equation}
    r^2u_{tt} = \frac{\partial\;}{\partial r} \left(v^2r^2 \frac{\partial u}{\partial r}\right) + cr^2u
\end{equation}

%\end{document}

\part{РІВНЯННЯ ЛАПЛАСА І ПУАССОНА.}

\chapter{Рівняння Лапласа в прямокутній області.}
%\documentclass[a4paper, 14pt]{extreport}

%\usepackage{StyleMMF}

%\setcounter{chapter}{10}

%\begin{document}

%\chapter{Рівняння Лапласа в прямокутній області.}

\section[Задача №11.1]{11.1}

\textit{Знайти стаціонарний розподіл температури в однорідній прямокутній пластині, якщо вздовж лівої її сторони (довжиною $b$) підтримується заданий розподіл температури, права сторона теплоізольована, а верхня і нижня (довжиною $a$) підтримуються при нульовій температурі. Відповідь запишіть через коефіцієнти Фур’є розподілу температури на лівій стороні, вважаючи їх відомими. Які якісні зміни відбуваються у розв’язку при переході від довгої і вузької пластини ($a \gg b$) до короткої і широкої ($a \ll b$)? Намалюйте для цих випадків графіки функцій, що описують зміну температури в повздовжньому напрямку для кількох перших поперечних мод; функції нормуйте так, щоб на лівій стороні пластини вони приймали однакове значення одиниця. Як змінюється в залежності від співвідношення сторін відносна роль внесків різних поперечних мод у розподіл температури на правій стороні пластини?}


%\end{document}
%\documentclass[a4paper, 14pt]{extreport}

%\usepackage{StyleMMF}

%\setcounter{chapter}{10}

%\begin{document}

%\chapter{Рівняння Лапласа в прямокутній області.}

\section[Задача №11.3]{11.3}

\textit{Знайти електростатичний потенціал всередині області, обмеженої провідними пластинами $y=0, y=b, x=0$, якщо пластина $x=0$ заряджена до потенціалу $V$, а інші -- заземлені. Заряди всередині області відсутні. Розв’язком якої задачі є знайдена функція у півпросторі $x>0$?\\
Вказівка. Це приклад задачі для рівняння Лапласа в необмеженій області. Подумайте, яку умову слід накласти на розв’язок при
$x \to +\infty$, щоб для $V=0$ задача мала лише нульовий розв’язок (в іншому разі розв’язок задачі не буде єдиним).\\
Ряд просумувати.\\
Указівка: скористайтесь формулою для суми геометричної прогресії.}


%\end{document}

\chapter{Функції Гріна звичайних диференціальних задач}
%\documentclass[a4paper, 14pt]{extreport}

%\usepackage{../../main/StyleMMF}

%\setcounter{chapter}{11}

%\begin{document}

%\chapter{Функції Гріна звичайних диференціальних задач}

\section[Задача №12.1]{12.1}

\textit{Функція Гріна $G(t)$ задачі Коші для рівняння гармонічного осцилятора \[y'' + \omega^2y = f(t), \, t \geq 0, \, y(0)=y_0, \, y'(0)=\nu_0\] є розв’язком цієї задачі при $\nu_0 = 1 , \, y_0 = 1$ і $f(t) = 0$. Тобто $y = G(t)$ задовольняє умови \[y'' + \omega^2y = 0, \, t \geq 0, \, y(0)=1, \, y'(0)=1\] Знайдіть явний вигляд функції Гріна сцилятора; чи зберігає вона смисл при $\omega \to 0$?}

\begin{center}
    \large{\textbf{Розв'язок}}
\end{center}

Запишимо постановку задачі для визначення функції Гріна $G(t)$
\begin{equation}
    \left\{ \begin{aligned} \label{cond12,1}
        \;&y'' + \omega^2y = 0, t \geq 0,\\
          &y(0) = 0, y'(0) = 1.
    \end{aligned} \right.
\end{equation}

Функція Гріна є розв'язком рівняння, тому
\begin{equation*}
    y = G(t) \;\to\; G'' + \omega^2G = 0,
\end{equation*}
а розв'язком буде
\begin{equation}
    G(t) = A\cos\omega t + B\sin\omega t
\end{equation}
З початкових умов визначаємо константи
\begin{equation*}
    \begin{aligned}
        &G(0) = A = 0 \;\Rightarrow\; A = 0\\
        &G'(0) = B\omega = 1 \;\Rightarrow\; B = \frac{1}{\omega}
    \end{aligned}
\end{equation*}
Таким чином функція Гріна для осцилятора має вигляд
\begin{equation}
    G(t) = \frac{\sin\omega t}{\omega}
\end{equation}

Перевіримо чи зберігає функція Гріна сенс при $\omega \to 0$.
\begin{equation*}
    \lim_{\omega\to0} \frac{\sin\omega t}{\omega} = t \cdot \lim_{\omega\to0} \frac{\sin\omega t}{\omega t} = t
\end{equation*}
Зрозуміло, що $\omega = 0$ відповідає відсутності повертаючої сили в системі, тобто тоді ми отримаємо задачу про вільний рух. Повертаючись до умови задачі Коші, яка визначає функцію Гріна, бачимо, що початкова умова описує частинку, яка в початковий момент часу в точці $y_0 = 0$ має швидкість $v_0 = 1$. При вільному русі закон руху буде лінійною функцією і, використовуючи початкові умови, отримаємо $G(t) = t$. 

%\end{document}
%\documentclass[a4paper, 14pt]{extreport}
%
%\usepackage{../../main/StyleMMF}
%
%\setcounter{chapter}{11}
%
%\begin{document}
%
%\chapter{Функції Гріна звичайних диференціальних задач}

\section[Задача №12.2]{12.2}

\textit{Користуючись означенням функції Гріна $G(t)$, але не використовуючи її явного вигляду, показати безпосередньою підстановкою в умови задачі, що функція \[y(t) = \int\limits_0^t G(t - t') f(t') \;\mathrm{d}t' + y'(0) G(t) + y(0) G'(t)\] є розв’язком задачі про вимушені коливання гармонічного осцилятора при $t>0$ під дією узагальненої сили $f(t)$ з початковими умовами $y(0)=y_0, \, y'(0)=\nu_0$. Розв’язками яких частинних задач є окремі доданки цього виразу?}

\begin{center}
    \large{\textbf{Розв'язок}}
\end{center}

Закон руху  
\begin{equation} \label{sol12,2}
    y(t) = \int\limits_0^t G(t - t') f(t') \;\mathrm{d}t' + y'(0) G(t) + y(0) G'(t)
\end{equation}
є розв'язком задачі:
\begin{equation} \label{cond12,2}
    \left\{ \begin{aligned}
        \;&y'' + \omega^2y = 0,\, t \geq 0,\\
          &y(0) = y_0,\, y'(0) = v_0.
    \end{aligned} \right.
\end{equation}

Обчислимо першу похідну по часу від розв'язку (\ref{sol12,2})
\begin{equation*} 
    y'(t) = G(0)f(t) + \int\limits_0^t G'(t - t') f(t') \;\mathrm{d}t' + y'(0) G'(t) + y(0) G''(t) \textcolor{red}{=}
\end{equation*}
За означенням функції Гріна $G(t)$ (\ref{cond12,1})
\begin{equation*}
    G''(t) = -\omega^2 G(t), \quad G(0) = 0, \quad G'(0) = 1
\end{equation*}
Підставимо $G''(t)$ та $G(0)$
\begin{equation*} 
    \textcolor{red}{=} \int\limits_0^t G'(t - t') f(t') \;\mathrm{d}t' + y'(0) G'(t) - \omega^2 y(0)G(t) 
\end{equation*}

Аналогічно друга похідна
\begin{equation*} 
    \begin{gathered}
        y''(t) = G'(0) f(t) + \int\limits_0^t G''(t - t') f(t') \;\mathrm{d}t' + y'(0) G''(t) - \omega^2 y(0) G'(t) =\\
        = f(t) - \omega^2 \left(\int\limits_0^t G(t - t') f(t') \;\mathrm{d}t' + y'(0) G(t) + y(0) G'(t) \right) = f(t) - \omega^2 y(t)
    \end{gathered}
\end{equation*}

Підставимо другу похідну в рівняння
\begin{equation*}
    y'' + \omega^2 y = f(t) - \omega^2 y + \omega^2 y \equiv f(t)
\end{equation*}
Таким чином (\ref{sol12,2}) задовільняє рівняння (\ref{cond12,2}) 

Визначимо для яких задач є розв'язками кожен з доданків (\ref{sol12,2}). Для цього треба покласти 2 з 3 параметрів (зовнішня сила та початкові умови) рівними нулю.
\begin{enumerate}
    \item $y(0) = 0, y'(0) = 0$
    \[y(t) = \int\limits_0^t G(t - t') f(t') \;\mathrm{d}t' \quad \text{є розв'язком задачі:}\] 
    \begin{equation}
        \left\{ \begin{aligned}
            \;&y'' + \omega^2y = f(t),\, t \geq 0,\\
              &y(0) = 0,\, y'(0) = 0.
        \end{aligned} \right.
    \end{equation}
    \item $f(t) = 0, y(0) = 0$
    \[y(t) = y'(0) G(t) \quad \text{є розв'язком задачі:}\] 
    \begin{equation}
        \left\{ \begin{aligned}
            \;&y'' + \omega^2y = 0,\, t \geq 0,\\
              &y(0) = 0,\, y'(0) = 1.
        \end{aligned} \right.
    \end{equation}
    \item $f(t) = 0, y'(0) = 0$
    \[y(t) = y(0) G'(t) \quad \text{є розв'язком задачі:}\] 
    \begin{equation}
        \left\{ \begin{aligned}
            \;&y'' + \omega^2y = 0,\, t \geq 0,\\
              &y(0) = 1,\, y'(0) = 0.
        \end{aligned} \right.
    \end{equation}
\end{enumerate}

%\end{document}
%%\documentclass[a4paper, 14pt]{extreport}

%\usepackage{StyleMMF}

%\setcounter{chapter}{11}

%\begin{document}

%\chapter{Функції Гріна звичайних диференціальних задач}

\section[Задача №12.5]{12.5}

\textit{Функція Гріна $G(x,x')$ крайової задачі для одновимірного рівняння Гельмгольца $u'' - \mu^2u = -f(x), \, u(0) = 0, \, |u| < \infty $ при $x \to \infty$ за означенням є неперервним розв’язком цієї задачі для $f(x) = \delta(x-x'), \, 0 < x' < \infty$.\\
а) Знайти функцію Гріна цієї задачі шляхом зшивання розв’язків однорідного рівняння і подальшого нормування (для даної задачі можливі принаймні три різні способи нормування розв’язку, які?).\\
б) Знайти функцію Гріна $G(x,x')$ крайової задачі для одновимірного рівняння Гельмгольца $u'' - \mu^2u = -f(x), \, x \in \mathbb{R}, \, |u| < \pm\infty $ при $x \to \pm\infty$ – шляхом граничного переходу $x, x' \to \infty$ при сталому $x-x'$ у $G(x,x')$, одержаній у пункті а) цієї задачі.\\
Дайте фізичну інтерпретацію знайдених функцій Гріна у термінах стаціонарної дифузії частинок зі скінченним часом життя. Якою є залежність від кожного з аргументів функції Гріна та симетрія відносно їх перестановки? Чому в одних випадках функція Гріна залежить від кожного з аргументів окремо, а в інших – тільки від їх різниці?}


%\end{document}

\chapter{Функції Гріна і розв’язки задач для рівнянь у частинних похідних з однорідними межовими умовами}
%\documentclass[a4paper, 14pt]{extreport}

%\usepackage{StyleMMF}

%\setcounter{chapter}{11}

%\begin{document}

%\chapter{Функції Гріна звичайних диференціальних задач}

\section[Задача №12.5]{12.5}

\textit{Функція Гріна $G(x,x')$ крайової задачі для одновимірного рівняння Гельмгольца $u'' - \mu^2u = -f(x), \, u(0) = 0, \, |u| < \infty $ при $x \to \infty$ за означенням є неперервним розв’язком цієї задачі для $f(x) = \delta(x-x'), \, 0 < x' < \infty$.\\
а) Знайти функцію Гріна цієї задачі шляхом зшивання розв’язків однорідного рівняння і подальшого нормування (для даної задачі можливі принаймні три різні способи нормування розв’язку, які?).\\
б) Знайти функцію Гріна $G(x,x')$ крайової задачі для одновимірного рівняння Гельмгольца $u'' - \mu^2u = -f(x), \, x \in \mathbb{R}, \, |u| < \pm\infty $ при $x \to \pm\infty$ – шляхом граничного переходу $x, x' \to \infty$ при сталому $x-x'$ у $G(x,x')$, одержаній у пункті а) цієї задачі.\\
Дайте фізичну інтерпретацію знайдених функцій Гріна у термінах стаціонарної дифузії частинок зі скінченним часом життя. Якою є залежність від кожного з аргументів функції Гріна та симетрія відносно їх перестановки? Чому в одних випадках функція Гріна залежить від кожного з аргументів окремо, а в інших – тільки від їх різниці?}


%\end{document}
% \documentclass[a4paper, 14pt]{extreport}
%
%\usepackage{../../main/StyleMMF}
%\usepackage{subcaption}
%
%\setcounter{chapter}{12}
%
%\begin{document}
%
%\chapter{Функції Гріна і розв’язки задач для рівнянь у частинних похідних з однорідними межовими умовами}

\section[Задача №13.6]{13.6}

\textit{Поставити задачу на функцію Гріна $G(\vec{r},\vec{r}')$ крайової задачі для 3-D рівняння Гельмгольца $\Delta_3 u - \mu^2 u = -f(\vec{r})$ у необмеженому просторі з умовою прямування розв’язку до нуля на нескінченності і розв’язати її за допомогою інтегрального перетворення Фур’є, дати фізичну інтерпретацію розв’язку у термінах стаціонарної дифузії частинок зі скінченним часом життя. Граничним переходом $\mu \to +0$ перейти до функції Гріна рівняння Лапласа. Записати розв’язок задачі з довільним джерелом $f(\vec{r})$ через функцію Гріна.}

\begin{center}
    \large{\textbf{Розв'язок}}
\end{center}

Постановка задачі на фінкцію Гріна 
\begin{equation}
    \left\{ \begin{aligned}
        \,& u = G(\vec{r}, \vec{r}^{\,\prime}),\\
        & \Delta_3 u - \mu^2 u = -\delta(\vec{r} - \vec{r}^{\,\prime})
    \end{aligned} \right.
\end{equation}

Виконаємо перетворення Фур'є рівняння
\begin{equation*}
    - k_x^2 \hat{u} - k_y^2 \hat{u} - k_z^2 \hat{u} - \mu^2 \hat{u} = -e^{i(\vec{k} \cdot (\vec{r} - \vec{r}^{\,\prime}))}
\end{equation*}
Звідки отримаємо Фур'є-образ функції Гріна
\begin{equation}
    \hat{u}(\vec{k}) = \frac{1}{k^2 + \mu^2} \cdot e^{i(\vec{k} \cdot (\vec{r} - \vec{r}^{\,\prime}))}
\end{equation} 

Тепер функції Гріна знайдемо оберненим перетворенням Фур'є
\begin{equation}
    G(\vec{r} - \vec{r}^{\,\prime}) = \frac{1}{(2\pi)^3} \int_{\mathbb{R}^3} \frac{e^{i(\vec{k} \cdot (\vec{r} - \vec{r}^{\,\prime}))}}{k^2 + \mu^2} \;\mathrm{d}\vec{k} \; \textcolor{red}{=}
\end{equation}  
Залишається обчислити отриманий інтеграл. Це не важко зробити використовуючи лемму Жордана та обчислючи лишки.\\
Переходимо в сферичні координати і позначимо $\rho = |\vec{r} - \vec{r}^{\,\prime}|$ та $\theta$ -- кут між векторами $\vec{k}$ та $\vec{r} - \vec{r}^{\,\prime}$ 
\begin{equation*}
    \begin{gathered}
        \textcolor{red}{=} \; \frac{1}{(2\pi)^3} \int_0^{2\pi} \;\mathrm{d}\phi \int_0^\infty k^2 \;\mathrm{d}k \int_0^\pi \frac{e^{ik\rho \cos\theta}}{k^2 + \mu^2} \;\mathrm{d}(\cos\phi) =\\
        = \frac{2\pi}{(2\pi)^3} \int_0^\infty \frac{k^2}{k^2 + \mu^2} \bigg( \frac{e^{ik\rho \cos\theta}}{ik\rho} \bigg) \bigg|_0^\pi \;\mathrm{d}k =\\
        = \frac{1}{8\pi^2i\rho} \bigg[ \int_{-\infty}^\infty \frac{ke^{-ik\rho}}{k^2 + \mu^2} \;\mathrm{d}k - \int_{-\infty}^\infty \frac{ke^{ik\rho}}{k^2 + \mu^2} \;\mathrm{d}k \bigg] \; \textcolor{red}{=}
    \end{gathered}
\end{equation*}  
Підінтегральній вираз має 2 особливі точки $k = \pm i\mu$. Для першого інтегралу потрібно розглянути контур в комплексній півплощині, де $\mathrm{Im}z > 0$, а для другого навпаки -- $\mathrm{Im}z < 0$. 

\begin{figure}[h]
    \centering
    %Graph under comment
    % Зображення контурів на комплексній площині

  \begin{minipage}{.49\textwidth}
    \centering
    \begin{tikzpicture}
      \begin{axis}
          [width = \textwidth,
          axis lines = center,
          ylabel = $\mathrm{Im}z$, xlabel = $\mathrm{Re}z$,
          xmin = -5, xmax = 5, ymin = -5, ymax = 5,
          axis line style = thin, ticks = none]   
          
          \tikzmath{\R = 3.75;}

          \draw[red, thick] (-\R,0) -- (\R,0) arc(0:180:\R) --cycle;
          \addplot[red, thick, domain=-\R:\R] {0};   
          
          \addplot[mark=*, red] coordinates {(0,1)}
          node[anchor=160, pos=0.5] {\footnotesize{$k = i\mu$}};

          \addplot[mark=*] coordinates {(0,-1)}
          node[anchor=160, pos=0.5] {\footnotesize{$k = -i\mu$}};

      \end{axis}
    \end{tikzpicture}

    \captionof{figure}{Контур для 1 інтегралу}
  \end{minipage}
  \begin{minipage}{.49\textwidth}
    \centering
    \begin{tikzpicture}
      \begin{axis} 
          [width = \textwidth,
          axis lines = center,
          ylabel = $\mathrm{Im}z$, xlabel = $\mathrm{Re}z$,
          xmin = -5, xmax = 5, ymin = -5, ymax = 5,
          axis line style = thin, ticks = none]   
          
          \tikzmath{\R = 3.75;}

          \draw[red, thick] (-\R,0) -- (\R,0) arc(0:-180:\R) --cycle;
          \addplot[red, thick, domain=-\R:\R] {0};   

          \addplot[mark=*] coordinates {(0,1)}
          node[anchor=160, pos=0.5] {\footnotesize{$k = i\mu$}};

          \addplot[mark=*, red] coordinates {(0,-1)}
          node[anchor=160, pos=0.5] {\footnotesize{$k = -i\mu$}};

      \end{axis}
    \end{tikzpicture}
    
    \captionof{figure}{Контур для 2 інтегралу}
  \end{minipage}
  % main compilation
    %% Зображення контурів на комплексній площині

  \begin{minipage}{.49\textwidth}
    \centering
    \begin{tikzpicture}
      \begin{axis}
          [width = \textwidth,
          axis lines = center,
          ylabel = $\mathrm{Im}z$, xlabel = $\mathrm{Re}z$,
          xmin = -5, xmax = 5, ymin = -5, ymax = 5,
          axis line style = thin, ticks = none]   
          
          \tikzmath{\R = 3.75;}

          \draw[red, thick] (-\R,0) -- (\R,0) arc(0:180:\R) --cycle;
          \addplot[red, thick, domain=-\R:\R] {0};   
          
          \addplot[mark=*, red] coordinates {(0,1)}
          node[anchor=160, pos=0.5] {\footnotesize{$k = i\mu$}};

          \addplot[mark=*] coordinates {(0,-1)}
          node[anchor=160, pos=0.5] {\footnotesize{$k = -i\mu$}};

      \end{axis}
    \end{tikzpicture}

    \captionof{figure}{Контур для 1 інтегралу}
  \end{minipage}
  \begin{minipage}{.49\textwidth}
    \centering
    \begin{tikzpicture}
      \begin{axis} 
          [width = \textwidth,
          axis lines = center,
          ylabel = $\mathrm{Im}z$, xlabel = $\mathrm{Re}z$,
          xmin = -5, xmax = 5, ymin = -5, ymax = 5,
          axis line style = thin, ticks = none]   
          
          \tikzmath{\R = 3.75;}

          \draw[red, thick] (-\R,0) -- (\R,0) arc(0:-180:\R) --cycle;
          \addplot[red, thick, domain=-\R:\R] {0};   

          \addplot[mark=*] coordinates {(0,1)}
          node[anchor=160, pos=0.5] {\footnotesize{$k = i\mu$}};

          \addplot[mark=*, red] coordinates {(0,-1)}
          node[anchor=160, pos=0.5] {\footnotesize{$k = -i\mu$}};

      \end{axis}
    \end{tikzpicture}
    
    \captionof{figure}{Контур для 2 інтегралу}
  \end{minipage}
  % this compilation
\end{figure}

За лемою Жордана інтеграл вздовж півкола буде прямувати до нуля при прямуванні його радіуса до нескінченності, тому значення інтегралів, які ми отримали раніше дорівнює 
\begin{equation*}
    \begin{gathered}
        \textcolor{red}{=} \; \frac{2\pi i}{8\pi^2i\rho} \bigg[ \mathop{\mathrm{Res}}_{k = i\mu} \frac{ke^{-ik\rho}}{k^2 + \mu^2} + \mathop{\mathrm{Res}}_{k = -i\mu} \frac{ke^{ik\rho}}{k^2 + \mu^2} \bigg] = \frac{1}{4\pi \rho} \bigg[ \lim_{k = i\mu} \frac{ke^{-ik\rho}}{k + i\mu} + \lim_{k = -i\mu} \frac{ke^{ik\rho}}{k - i\mu} \bigg] =\\
        = \frac{1}{4\pi \rho} \bigg[ \frac{i\mu e^{\mu\rho}}{2i\mu} + \frac{-i\mu e^{\mu\rho}}{-2i\mu} \bigg] = \frac{1}{4\pi\rho} e^{\mu\rho} = \frac{e^{\mu|\vec{r} - \vec{r}^{\,\prime}|}}{4\pi |\vec{r} - \vec{r}^{\,\prime}|}
    \end{gathered}
\end{equation*}  
Отже, маємо функцію Гріна для рівняння Гельмгольца
\begin{equation}
    G(\vec{r} - \vec{r}^{\,\prime}) = \frac{e^{\mu|\vec{r} - \vec{r}^{\,\prime}|}}{4\pi |\vec{r} - \vec{r}^{\,\prime}|}
\end{equation} 

Фізична інтерпритація: ???

Знайдемо функцію Гріна для рівняння Лапласа
\begin{equation}
    G(\vec{r} - \vec{r}^{\,\prime}) = \lim_{\mu \to +0} \frac{e^{\mu|\vec{r} - \vec{r}^{\,\prime}|}}{4\pi |\vec{r} - \vec{r}^{\,\prime}|} = \frac{1}{4\pi |\vec{r} - \vec{r}^{\,\prime}|}
\end{equation} 
Розв'язок задачі для довільного джерела
\begin{equation}
    u(\vec{r}) = \int_{-\infty}^{+\infty} G(\vec{r} - \vec{r}^{\,\prime}) f(\vec{r}^{\,\prime}) \; \mathrm{d}\vec{r}^{\,\prime} = \frac{1}{4\pi} \int_{-\infty}^{+\infty} \frac{f(\vec{r}^{\,\prime})}{|\vec{r} - \vec{r}^{\,\prime}|} \; \mathrm{d}\vec{r}^{\,\prime}
\end{equation}

%\end{document}
%\documentclass[a4paper, 14pt]{extreport}
%
%\usepackage{../../main/StyleMMF}
%
%\setcounter{chapter}{12}
%
%\begin{document}
%
%\chapter{Функції Гріна і розв’язки задач для рівнянь у частинних похідних з однорідними межовими умовами}

\section[Задача №13.7]{13.7}

\textit{Знайти функцію Гріна $G(x,x')$ крайової задачі для одновимірного рівняння Гельмгольца \[u'' - \mu^2 u = - f(x), \quad -\infty < x < +\infty, \, |u| < \infty \text{ при } x \to \pm\infty \] за допомогою інтегрального перетворення Фур’є. Порівняти результат з розв’язком задачі 12.5б.}

\begin{center}
    \large{\textbf{Розв'язок}}
\end{center}

Постановка задачі на фінкцію Гріна 
\begin{equation}
    \left\{ \begin{aligned}
        \,& u = G(x, x'),\\
          & u'' - \mu^2 u = -\delta(x)
    \end{aligned} \right.
\end{equation}

Перетворимо рівняння за Фур'є 
\begin{equation*}
    - k^2 \hat{u} - \mu^2 \hat{u} = -e^{ikx}
\end{equation*}
Маємо Фур'є-образ
\begin{equation}
    \hat{u}(k) = \frac{e^{ikx}}{k^2 + \mu^2} 
\end{equation} 
Тепер функції Гріна знайдемо оберненим перетворенням Фур'є
\begin{equation}
    G(x) = \frac{1}{2\pi} \int_{-\infty}^{+\infty} \frac{e^{ikx}}{k^2 + \mu^2} \;\mathrm{d}k 
\end{equation}  
Аналогічно до задачі №13,6 будемо шукати лишки, але в цій задачі треба розглянути окремо дві області $x > 0$ та $x < 0$ і зшити їх.\\
При $x > 0$ розглядаємо контур з $\mathrm{Im}z < 0$:
\begin{equation*}
    G(x>0) = \frac{1}{2\pi} \int_{-\infty}^{+\infty} \frac{e^{ikx}}{k^2 + \mu^2} \;\mathrm{d}k = i \mathop{\mathrm{Res}}_{k = -i\mu} \frac{ke^{-ikx}}{k^2 + \mu^2} = i \lim_{k = -i\mu} \frac{ke^{-ikx}}{k - i\mu} = -\frac{e^{\mu x}}{2\mu}
\end{equation*}
При $x < 0$ розглядаємо контур з $\mathrm{Im}z > 0$:
\begin{equation*}
    G(x<0) = i \mathop{\mathrm{Res}}_{k = -i\mu} \frac{ke^{-ikx}}{k^2 + \mu^2} = i \lim_{k = -i\mu} \frac{ke^{-ikx}}{k - i\mu} = -\frac{e^{-\mu x}}{2\mu}
\end{equation*}
Отже, функція Гріна має вигляд
\begin{equation}
    G(x, x') = 
    \left\{ \begin{aligned}
        \;& -\frac{1}{2\mu} e^{-\mu (x - x')} , \; x < 0 \\
          & -\frac{1}{2\mu} e^{\mu (x - x')} , \; x > 0,
    \end{aligned} \right.
\end{equation}
або 
\begin{equation}
    G(x,x') = -\frac{1}{2\mu} e^{-\mu |x - x'|}
\end{equation}

Отриманий вираз выдповідає результату пунтку б) задачі №12,5 

%\end{document}


\end{document}