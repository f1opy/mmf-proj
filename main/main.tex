\documentclass[a4paper, 14pt]{extreport}
\usepackage[top=2cm, bottom=2cm, left=2.5cm, right=1.5cm]{geometry}

\usepackage[utf8]{inputenc}
\usepackage[english, russian, ukrainian]{babel}
\usepackage{amssymb,amsfonts,amsmath,amsthm}

\usepackage{enumitem}

\usepackage[pdftex, unicode, colorlinks=true, linkcolor=black]{hyperref}

%\usepackage{relsize} %%позволяет пользоваться функцией \mathlarger{}
\usepackage{xcolor}
\usepackage[pdf]{xy}

\usepackage{wrapfig}
\usepackage{tikz} 
\usetikzlibrary{math}
\usetikzlibrary{arrows.meta}

\usepackage{pgfplots}
\pgfplotsset{compat=1.18}

\usepackage{titlesec}
\titleformat{\chapter}[display]{\normalfont\Large\bfseries}{\chaptertitlename\ \thechapter}{10pt}{\large\bfseries}
\titleformat{\section}{\normalfont\Large\bfseries}{\thesection}{12pt}{\large\bfseries}
\titleformat{\subsection}{\normalfont\large\bfseries}{\thesubsection}{12pt}{\large\bfseries}

\renewcommand{\labelenumii}{\theenumii)} %% заменяем счёчтик 2 уровня вида (a), (b), (c) и т.д. на русский алфавит а), б), в), и т.д. 


\begin{document}

%\tableofcontents
\setcounter{page}{2}

\chapter{ЗАСТОСУВАННЯ ПРОЦЕДУРИ ФУР’Є БЕЗПОСЕРЕДНЬОГО ВІДОКРЕМЛЕННЯ ЗМІННИХ}

%\section{Відокремлення змінних, задача Штурма-Ліувілля і власні моди коливань струни для різних межових умов}

%\documentclass[a4paper, 14pt]{extreport}
\usepackage[top=2cm, bottom=2cm, left=2.5cm, right=1.5cm]{geometry}

\usepackage[utf8]{inputenc}
\usepackage[english, russian, ukrainian]{babel}
\usepackage{amssymb,amsfonts,amsmath,amsthm}

\usepackage[pdftex, unicode, colorlinks=true, linkcolor=black]{hyperref}

%\usepackage{relsize} %%позволяет пользоваться функцией \mathlarger{}
\usepackage{xcolor}
\usepackage[pdf]{xy}

\usepackage{wrapfig}
\usepackage{tikz} 
\usetikzlibrary{math}

\usepackage{pgfplots}
\pgfplotsset{compat=1.18}


\usepackage{titlesec}
\titleformat{\chapter}[display]{\normalfont\Large\bfseries}{\chaptertitlename\ \thechapter}{24pt}{\large\bfseries}
\titleformat{\section}{\normalfont\Large\bfseries}{\thesection}{20pt}{\large\bfseries}

\renewcommand{\labelenumii}{\theenumii)} %% заменяем счёчтик 2 уровня вида (a), (b), (c) и т.д. на русский алфавит а), б), в), и т.д. 


\begin{document}

\tableofcontents
\setcounter{page}{2}

\chapter{ЗАСТОСУВАННЯ ПРОЦЕДУРИ ФУР’Є БЕЗПОСЕРЕДНЬОГО ВІДОКРЕМЛЕННЯ ЗМІННИХ}

\section{Відокремлення змінних, задача Штурма-Ліувілля і власні моди коливань струни для різних межових умов}

\subsection*{Задача №1.1}

\textit{\textbf{Знайти власні моди коливань струни завдовжки $l$ із закріпленими кінцями (знайти функції вигляду $u(x,t) = X(x) \cdot T(t)$, визначені і достатньо гладкі в області $0 \leq x \leq l, -\infty \leq t \leq \infty$, не рівні тотожно нулю, які задовольняють одновимірне хвильове рівняння $u_{tt} = v^2 u_{xx}$ на проміжку $0 \leq x \leq l$ і межові умови $u(0,t) = 0, u(l,t) = 0$ на його кінцях).} Результат перевірити аналітично й графічно (див. текст до модульної контрольної роботи №1, с. 25) та проаналізувати його фізичний смисл. Знайти початкові умови (початкове відхилення і початкову швидкість) для кожної з мод.}

\begin{center}
    \large{\textbf{Розв'язок}}
\end{center}

\noindent Постановка задачі:
\begin{equation}
    \left\{ \begin{aligned} %%
        \;&u = u(x,t), \\  &u_{tt} = v^2 u_{xx}, \\ &0 \leq x \leq l, t \in \mathbb{R}, \\  &u(0,t) = 0, \\ &u(l,t) = 0. 
    \end{aligned} \right.
\end{equation}
Шукаємо нетривіальні розв'язки рівняння у виді:
\begin{equation} \label{subst}
    u(x,t) = X(x) \cdot T(t) \neq 0 
\end{equation}

Тепер можливе відокремлення змінних в задачі. Почнемо з межових умов:
\begin{equation*}
    \begin{aligned}
        \;u(0,t) = X(0) \cdot T(t) = 0
        \;\Rightarrow\;
        \left\{ \begin{aligned}
            &T(t) \neq 0, \forall t, \\  &X(0) = 0; 
        \end{aligned} \right.\\
        u(l,t) = X(l) \cdot T(t) = 0
        \;\Rightarrow\;
        \left\{ \begin{aligned}
            &T(t) \neq 0, \forall t, \\  &X(l) = 0; 
        \end{aligned} \right.\\
    \end{aligned}
\end{equation*}
Далі підставимо (\ref{subst}) в рівняння та виконаємо ряд перетворень:
\begin{equation*}
    \frac{\partial^2}{\partial t^2}\left[X(x)T(t)\right] = v^2 \frac{\partial^2}{\partial x^2}\left[X(x)T(t)\right]
    \;\to\; 
    X T^{\prime\prime} = v^2 X^{\prime\prime} T 
    \;\to\; 
    \frac{T^{\prime\prime}}{v^2T} = \frac{X^{\prime\prime}}{X} = - \lambda,
\end{equation*}
де $\lambda$ -- стала відокремлення.\\
Виписуємо результат відокремлення змінних:
\begin{equation} \label{sepvar}
    \left\{ \begin{aligned}
        \;&X = X(x), \\  &X^{\prime\prime} = -\lambda X, \\ &0 \leq x \leq l, \\  &X(0) = 0, \\ &X(l) = 0. 
    \end{aligned} \right.
    \qquad\qquad
    \begin{aligned}
        T^{\prime\prime} + \lambda v^2 T = 0\\
        \lambda \text{ -- невідома}
    \end{aligned}
\end{equation}

Для $X = X(x)$ отримуємо задачу Штурма-Ліувілля. Розв'яжемо її:
\begin{enumerate}
    \item[] \begin{enumerate}
        \item Розглянемо випадок $\lambda = 0$:
        \begin{equation*}
            X^{\prime\prime} = -\lambda X
            \;\to\;
            X^{\prime\prime} = 0
            \;\to\;
            X(x) = C_1 + C_2 x
        \end{equation*}
        Знаходимо константи з межових умов:
        \begin{equation*}
            \begin{aligned}
                &\left\{ \begin{aligned}
                    &X(0) = C_1 = 0, \\ 
                    &X(l) = C_1 + C_2 l = 0;
                \end{aligned} \right.
                \\   
                &\left\{ \begin{aligned}
                    C_1 = 0, \\ 
                    C_2 = 0;
                \end{aligned} \right.
            \end{aligned}
            \quad\Rightarrow\;
            X(x) = 0 \text{ -- розв'язок тривівльний}
        \end{equation*}
    
        \item Розглянемо випадок $\lambda < 0$. Розв'язок рівняння шукаємо у виді $X(x) = e^{\alpha x}$, підставимо це в рівняння: 
        \begin{equation*}
            \begin{aligned}
                &X^{\prime\prime} = -\lambda X
                \quad\to\quad
                \alpha^2 \textcolor{red}{\begin{xy}*{\textcolor{black}{e^{\alpha x}}};p+LU;+RD**h@{}+/\jot/**h@{-}\end{xy}} = +|\lambda| \textcolor{red}{\begin{xy}*{\textcolor{black}{e^{\alpha x}}};p+LU;+RD**h@{}+/\jot/**h@{-}\end{xy}}
                \quad\to\quad
                \alpha = \pm \sqrt{|\lambda|}
                \;\Rightarrow\\
                \Rightarrow\;
                &X(x) = \widetilde{C}_1 e^{\sqrt{|\lambda|}x} + \widetilde{C}_2 e^{-\sqrt{|\lambda|}x} \equiv C_1 sh(\sqrt{|\lambda|}x) + C_2 ch({\sqrt{|\lambda|}x})
            \end{aligned}
        \end{equation*}
        Знаходимо константи з межових умов:
        \begin{equation*}
            \begin{aligned}
                &\left\{ \begin{aligned}
                    &X(0) = C_2 = 0, \\ 
                    &X(l) = C_1 sh(\sqrt{|\lambda|}l) + C_2 ch({\sqrt{|\lambda|}l}) = 0;
                \end{aligned} \right.
                \;\to\\
                \to\;
                &\left\{ \begin{aligned}
                    &C_2 = 0, \\ 
                    &C_1 sh(\sqrt{|\lambda|}l) = 0, \\
                    &sh(\sqrt{|\lambda|}l) \neq 0;
                \end{aligned} \right.
                \;\to\;
                \left\{ \begin{aligned}
                    C_1 = 0, \\ 
                    C_2 = 0;
                \end{aligned} \right.
                \quad\Rightarrow\;
                \text{розв'язок тривівльний}
            \end{aligned}
        \end{equation*}
    
        \item Розглянемо випадок $\lambda > 0$. Розв'язок рівняння шукаємо у виді $X(x) = e^{\alpha x}$, підставимо це в рівняння: 
        \begin{equation*}
            \begin{aligned}
                &X^{\prime\prime} = -\lambda X
                \quad\to\quad
                \alpha^2 \textcolor{red}{\begin{xy}*{\textcolor{black}{e^{\alpha x}}};p+LD;+RU**h@{}+/\jot/**h@{-}\end{xy}} = -\lambda \textcolor{red}{\begin{xy}*{\textcolor{black}{e^{\alpha x}}};p+LD;+RU**h@{}+/\jot/**h@{-}\end{xy}}
                \quad\to\quad
                \alpha = \pm i\sqrt{\lambda}
                \;\Rightarrow\\
                \Rightarrow\;
                &X(x) = \widetilde{C}_1 e^{i\sqrt{\lambda}x} + \widetilde{C}_2 e^{-\sqrt{\lambda}x} \equiv C_1 \sin(\sqrt{\lambda}x) + C_2 \cos({\sqrt{\lambda}x})
            \end{aligned}
        \end{equation*}
        Знаходимо константи з межових умов:
        \begin{equation*}
            \begin{aligned}
                \left\{ \begin{aligned}
                    &X(0) = C_2 = 0, \\ 
                    &X(l) = C_1 \sin(\sqrt{\lambda}l) + \textcolor{red}{\begin{xy}*{\textcolor{black}{C_2}};p+LU;+RD**h@{}+//**h@{-}*h@{>}*h!LD{\scriptstyle 0}\end{xy}} \cos({\sqrt{\lambda}l}) = 0;
                \end{aligned} \right.
                \;\to\;
                \left\{ \begin{aligned}
                    &C_1 \neq 0, \\ 
                    &\sin(\sqrt{\lambda}l) = 0;
                \end{aligned} \right.
            \end{aligned}
        \end{equation*}
        Маємо нетривіальний розв'язок. Визначимо з характерисичного рівняння при яких значеннях $\lambda$ він можливий:
        \begin{equation*}
            \sin(\sqrt{\lambda}l) = 0
            \;\to\;
            \sqrt{\lambda}l = \pi n, \, n \in \mathbb{Z}
            \;\to\;
            \lambda_n = \frac{\pi^2 n^2}{l^2}, \, n \in \mathbb{N}.
        \end{equation*}
    \end{enumerate}
\end{enumerate} 
Отже, ми визначили всі власні значення та відповідні їм власні функції.
    \begin{equation} \label{ShLsol}
        \left\{ \begin{aligned}
            \;&\lambda_n = \frac{\pi^2 n^2}{l^2},\\ 
            &X_n(x) = C_n \sin\left(\frac{\pi n x}{l}\right),
        \end{aligned} \right.
        \quad \text{де } n \in \mathbb{N}.
    \end{equation}
    
Повертаємося до рівняння для $T(t)$ - (\ref{sepvar}). Підставляємо знайдені значення та знаходимо $T_n(t)$:
\begin{equation*}
    \left. \begin{aligned}
        \lambda_n = \frac{\pi^2 n^2}{l^2},&\;\\ 
        T^{\prime\prime} + \lambda v^2T = 0,&
    \end{aligned} \right\}
    \;\Rightarrow\;
    T_n(t) = A\cos(\omega_n t) + B\sin(\omega_n t),
\end{equation*}
де $\omega_n^2 = \lambda_n v^2, \, n \in \mathbb{N}.$\\
Власними модами коливань струни будуть всі розв'язки вигляду:
\begin{equation*}
    u_n(x,t) = X_n(x) \cdot T_n(t)
\end{equation*}
Виконаємо перепозначення і запишемо остаточний розв'язок:
\begin{equation}
    \left\{ \begin{aligned}
        \;&u_n(x,t) = \left(A_n\cos(\omega_n t) + B_n\sin(\omega_n t)\right) \sin(k_n x), \\
        &k_n = \frac{\pi n}{l} - \text{ хвильові вектори}, \\
        &\omega_n = vk_n = \frac{v \pi n}{l} - \text{ власні частоти}, \\
        &n = 1, 2,\ldots
    \end{aligned}\right.
\end{equation}
\newpage

\begin{center}
    \large{\textbf{Перевірка розв'язку задачі Штурма-Ліувілля}}
\end{center}

\noindent Щоб перевірити правильність отриманого результату, (\ref{ShLsol}), треба використовувати постановки самої задачі -- (\ref{sepvar}). 
\begin{enumerate}
    \item Аналітична перевірка (пряма підстановка результату в рівняння).
    \begin{enumerate}
        \item[1)] Перевіряємо рівняння, для цього обчислимо другу похідну власної функції. 
        \begin{equation*}
            X_n^{''} = \frac{\pi n}{l} \left(C_n\cos\left(\frac{\pi n x}{l}\right)\right)^{'} = -\left(\frac{\pi n}{l}\right)^2 C_n\sin\left(\frac{\pi n x}{l}\right) = -\left(\frac{\pi n}{l}\right)^2 X_n
        \end{equation*}
        Порівнюємо з вихідним рівнянням і робимо перший висновок: кожна з функцій $X_n(x)$ дійсно є розв’язком рівняння (\ref{sepvar}) для значення спектрального параметра $\lambda = \left(\frac{\pi n}{l}\right)^2$. Далі, ці значення $\lambda$ збігаються зі знайденими раніше власними значеннями (\ref{ShLsol}), отже робимо другий висновок: знайдені власні значення дійсно відповідають знайденим власним функціям.
        \item[2)] Перевіряємо виконання межових умов, підставляємо власні функції в межові умови.
        \begin{equation*}
            \begin{aligned}
                X_n(0) &=  0:\\
                &\begin{aligned}
                    X_n(0) = C_n \sin(\sqrt{\lambda_n} \cdot 0) = 0 &\text{ -- виконується,}\\
                    &\text{ причому незалежно від }\lambda_n
                \end{aligned}\\
                \\
                X_n(l) &= 0:\\
                &\begin{aligned}
                    X_n(l) = C_n \sin\left(\frac{\pi n}{l} \cdot l\right) = C_n &\sin(\pi n) = 0 \text{ -- виконується,  але}\\
                    &\text{ саме для знайдених значень }\lambda_n
                \end{aligned}\\
            \end{aligned}
        \end{equation*}
    \end{enumerate}
    \item Графічна перевірка:
    \begin{center}
        \begin{tikzpicture}
            \begin{axis}
                [width = 0.85\textwidth, height = 0.4\textwidth,
                 axis x line = center, axis y line = center,
                 ylabel = $X(x)$, xlabel = $x$,
                 xmin = 0.7, xmax = 6.7, ymin = -3.3, ymax = 5.3,
                 axis line style = thin, xtick = {0}, ytick = {0}]   
                
                \tikzmath{\A1 = 5; \l = 5; \k = pi/\l;}
                
                \addplot [black, domain = 1:(\l+1), samples = 1000] {\A1 * sin(deg(\k*(x-1)))}
                node[anchor=north, pos=0] {0} 
                node[pos=0.75, fill=white] {$X_0(x)$} 
                node[anchor=north, pos=1] {$l$};
                
                \addplot [black, domain = 1:(\l+1), samples = 1000] {\A1/2.5 * sin(deg(2*\k*(x-1)))}
                node[pos=0.83, fill=white] {$X_1(x)$};
                
                \addplot [black, domain = 1:(\l+1), samples = 1000] {\A1/3.7 * sin(deg(3*\k*(x-1)))}
                node[pos=0.74, fill=white] {$X_2(x)$};
                
            \end{axis}
        \end{tikzpicture}
    \end{center}
\end{enumerate}




\end{document}

\section{Власні моди інших систем. Вільні коливання для заданих початкових умов.}

\documentclass[a4paper, 14pt]{extreport}

\usepackage{StyleMMF}

\begin{document}

\subsection{Стержень з вільними та пружно закріпленими кінцями; системи, описувані іншими рівняннями.}

\subsubsection{Задача №1}

\textit{Знайти власні моди повздовжніх рухів тонкого стержня $0 \leq x \leq l$ із вільними кінцями  (задача для хвильового рівняння з межовими умовами $u_x(0,t) = 0, u_x(l,t) = 0$).\\
Результат перевірити аналітично й графічно (див. заняття №6, зразок модульної контрольної роботи №1) та проаналізувати його фізичний смисл. Чим відрізняється від інших основна (нульова) мода? Якому рухові стержня вона відповідає?}

\begin{center}
    \large{\textbf{Розв'язок}}
\end{center}

\noindent Формальна постановка задачі:
\begin{equation} \label{probcond2}
    \left\{ \begin{aligned} %%
            \;&u = u(x,t), \\
            &u_{tt} = v^2 u_{xx}, \\
            &0 \leq x \leq l, t \in \mathbb{R} \\
            &u_x(0,t) = 0, \\
            &u_x(l,t) = 0. 
    \end{aligned} \right.
\end{equation}
Необхідно знайти розв'язки (\ref{probcond2}) вигляду:
\begin{equation} \label{subst2}
    u(x,t) = X(x) \cdot T(t) \neq 0 
\end{equation}

Від задачі №1 попереднього заняття задача відрізняється тільки межовою умовою, тому підставляємо розв'язок у вигляді добутку (\ref{subst2}) тільки у межові умови (\ref{probcond2}):
\begin{equation*}
    \begin{aligned}
        \;u_x(0,t) = X'(0) \cdot T(t) = 0
        \;\Rightarrow\;
        \left\{ \begin{aligned}
            &T(t) \neq 0, \forall t, \\  &X'(0) = 0; 
        \end{aligned} \right.\\
        u_x(l,t) = X'(l) \cdot T(t) = 0
        \;\Rightarrow\;
        \left\{ \begin{aligned}
            &T(t) \neq 0, \forall t, \\  &X'(l) = 0; 
        \end{aligned} \right.\\
    \end{aligned}
\end{equation*}
Тут ми врахували, що умови на кінцях струни виконуються при всіх $t$, тому $T(t)$ не може бути рівним нулю.\\

Виписуємо результат відокремлення змінних:
\begin{equation} \label{sepvar2}
    \left\{ \begin{aligned}
        \;&X = X(x), \\
          &X^{\prime\prime} = -\lambda X, \\
          &0 \leq x \leq l, \\
          &X'(0) = 0, \\ 
          &X'(l) = 0. 
    \end{aligned} \right.
    \qquad\qquad
    T^{\prime\prime} + \lambda v^2 T = 0
\end{equation}

\begin{enumerate}
    \item[] Розв'язуємо задачу Штурма-Ліувілля (\ref{sepvar2}). Знову скористаємося результатами попередньої задачі та одразу запишемо якого типу отримуємо розв'язки для різних $\lambda$.
    \begin{enumerate}[wide, labelindent=0pt]
        
        \item Випадок $\lambda < 0$. 
        \begin{equation*}
            X(x) = C_1 sh(\sqrt{|\lambda|}x) + C_2 ch({\sqrt{|\lambda|}x})
        \end{equation*}
        Знаходимо константи з межових умов:
        \begin{equation*}
            \begin{aligned}
                &X'(0) = C_1\sqrt{|\lambda|}
                \;\Rightarrow\;
                X(x) = C_2 &ch(\sqrt{|\lambda|}x)\\
                &\left\{ \begin{aligned}
                    &X'(l) = C_2\sqrt{|\lambda|} sh(\sqrt{|\lambda|}l) = 0, \\
                    &sh(\sqrt{|\lambda|}l) \neq 0;
                \end{aligned} \right.&\\
                &\left\{ \begin{aligned}
                    C_1 = 0, \\ 
                    C_2 = 0;
                \end{aligned} \right. \qquad\qquad\qquad\qquad&
            \end{aligned}
            \;\Rightarrow\;
            \begin{aligned}
                \text{розв'язок тривівльний,}\\
                \text{немає від'ємних}\\
                \text{власних значень.}
            \end{aligned}
        \end{equation*}

        \item Випадок $\lambda = 0$:
        \begin{equation*}
            X(x) = C_1 + C_2 x
        \end{equation*}
        Знаходимо константи з межових умов:
        \begin{equation*}
            \begin{aligned}
                &\left\{ \begin{aligned}
                    &X'(0) = C_2 = 0, \\ 
                    &X'(l) = C_2 = 0;
                \end{aligned} \right.
                \\   
                &\left\{ \begin{aligned}
                    C_1 \in \mathbb{R}, \\ 
                    C_2 = 0;
                \end{aligned} \right.
            \end{aligned}
            \quad\Rightarrow\;
            \begin{aligned}
                X(x) = 0 \text{ -- розв'язок нетривівльний,}\\
                \lambda = 0 \text{ є власним значенням.}
            \end{aligned}
        \end{equation*}

        \item Випадок $\lambda > 0$
        \begin{equation*}
            X(x) = C_1 \sin(\sqrt{\lambda}x) + C_2 \cos({\sqrt{\lambda}x})
        \end{equation*}
        Знаходимо константи з межових умов:
        \begin{equation*}
            \begin{aligned}
                \left\{ \begin{aligned}
                    &X'(0) = C_1\sqrt{\lambda} = 0, \\ 
                    &X'(l) = \sqrt{\lambda}\left(\textcolor{red}{\begin{xy}*{\textcolor{black}{C_1}};p+LU;+RD**h@{}+//**h@{-}*h@{>}*h!LD{\scriptstyle 0}\end{xy}} \cos({\sqrt{\lambda}l}) - C_2 \sin(\sqrt{\lambda}l)\right) = 0;
                \end{aligned} \right.
                \;\Rightarrow\;
                \left\{ \begin{aligned}
                    &C_2 \neq 0, \\ 
                    &\sin(\sqrt{\lambda}l) = 0;
                \end{aligned} \right.
            \end{aligned}
        \end{equation*}
        Отже, нетривіальні розв'язки існують при значеннях параметра $\lambda$, які задовольняють характеристичне рівняння :
        \begin{equation*}
            \sin(\sqrt{\lambda}l) = 0
            \;\Rightarrow\;
            \sqrt{\lambda_n}l = \pi n, \, n \in \mathbb{Z}
            \;\Rightarrow\;
            \lambda_n = \frac{\pi^2 n^2}{l^2}.
        \end{equation*}
    \end{enumerate}
\end{enumerate} 
Випишемо тепер розв'язки для всіх $n$ і визначимо, які з них необхідно залишити:
    \begin{equation*}
        \left\{ \begin{aligned}
            &X_0(x) = C_0,\\
            &\lambda_0 = 0;
        \end{aligned} \right.
        \qquad
        \left\{ \begin{aligned}
            &X_n(x) = C_n \sin\left(\frac{\pi n x}{l}\right),\\
            &\lambda_n = \frac{\pi^2 n^2}{l^2}, n \in \mathbb{N}.
        \end{aligned} \right.
    \end{equation*}

    \textbf{\Large Необхідно відредагувати наступний текст}
Видно, що $n = 0$ відповідає тривіальному розв'язку. Видно також, що всі інші розв'язки визначені з точністю до довільного множника.\\
Тому власні функції, які співпадають з точністю до множника, вважають однаковими. У загальному випадку різними вважають лише лінійно незалежні власні функції, а розвя'зати задачу Штурма-Ліувілля означає знайти всі різні власні функції і відповідні власні значення. Отже, різним власним функціям відповідають лише натуральні $n$, а коефіцієнти $C_n$ можна покласти рівними одиниці.\\
    Власними значеннями і власними функціями є
    \begin{equation} %\label{ShLsol}
        \left\{ \begin{aligned}
            \;&\lambda_n = \frac{\pi^2 n^2}{l^2},\\ 
            &X_n(x) = \sin\left(\frac{\pi n x}{l}\right),
        \end{aligned} \right.
        \quad \text{де } n \in \mathbb{N}.
    \end{equation}

Повертаємося до рівняння для $T(t)$ (\ref{sepvar2}). Підставляємо знайдені власні значення та знаходимо $T_n(t)$:
\begin{equation*}
    \left. \begin{aligned}
        \lambda_n = \frac{\pi^2 n^2}{l^2},&\;\\ 
        T^{\prime\prime} + \lambda v^2T = 0,&
    \end{aligned} \right\}
    \;\Rightarrow\;
    T_n(t) = A\cos(\omega_n t) + B\sin(\omega_n t),
\end{equation*}
де $\omega_n^2 = \lambda_n v^2, \, n \in \mathbb{N}.$\\
\begin{equation*}
    \left. \begin{aligned}
        \lambda_0 = 0,&\;\\ 
        T^{\prime\prime} = 0,&
    \end{aligned} \right\}
    \;\Rightarrow\;
    T_0(t) = A_0 + B_0 t,
\end{equation*}
Власними модами коливань струни будуть всі розв'язки вигляду:
\begin{equation*}
    u_n(x,t) = X_n(x) \cdot T_n(t)
\end{equation*}
Виконаємо перепозначення і запишемо остаточний розв'язок:
\begin{equation}
    \left\{ \begin{aligned} \label{sol2}
        \;&u_0(x,t) = A_0 + B_0 t, \\
        &u_n(x,t) = \left[A_n\cos(\omega_n t) + B_n\sin(\omega_n t)\right] \sin(k_n x), \\
        &k_n = \frac{\pi n}{l} - \text{ хвильові вектори}, \\
        &\omega_n = vk_n = \frac{v \pi n}{l} - \text{ власні частоти}, \\
        &n = 1, 2,\ldots
    \end{aligned}\right.
\end{equation}

\end{document}

\subsection{Вільні коливання поля в резонаторі для заданих початкових умов. Ряд Фур'є по системі ортогональних функцій.}

\subsubsection{Задача №3}

\textit{Знайти коливання струни завдовжки $0 \leq x \leq l$ із закріпленими кінцями, якщо початкове відхил є $\varphi(x) = hx/l$, а початкова швидкість $\psi(x) = \nu_0$. Обчислити інтеграл ортогональності власних функцій і знайти квадрат норми. Чи є рух струни періодичним (тобто повторюється початковий стан струни через деякий проміжок часу?) Чи буде рух періодичним, якщо він описується рівнянням $u_{tt} = v^2 u_{xx} - \omega_0^2 u$}

\begin{center}
    \textbf{Розв'язок}
\end{center}
Формальна постановка задачі:
\begin{equation} \label{probcond3}
    \left\{ \begin{aligned} %%
        &\;u = u(x,t), \\
        &\;u_{tt} = v^2 u_{xx}, \\
        &\;0 \leq x \leq l, t \in \mathbb{R}, \\
        &\;u(0,t) = u(l,t) = 0,\\
        &\left.\begin{aligned}
            &u(x,0) = \varphi(x) = \frac{hx}{l}, \\ 
            &u_t(x,0) = \psi(x) = \nu_0.
        \end{aligned}\right\} \; 
        \begin{aligned}
            &\text{ початкові умови задають} \\
          - &\text{ механічний стан} \\
            &\text{ системи при } t = 0
        \end{aligned}
    \end{aligned} \right.
\end{equation}

Задача з заданими початковими умовами має єдиний розв'язок. Скористаємося результатами задачі 1 попередньго заняття (\ref{sol1}).
\begin{equation*}
    \left\{ \begin{aligned} \label{fullsol}
        \;&u_n(x,t) = \left[A_n\cos(\omega_n t) + B_n\sin(\omega_n t)\right] \sin(k_n x), \\
        &k_n = \frac{\pi n}{l}, \, n = 1, 2,\ldots\\
        &\omega_n = vk_n = \frac{v \pi n}{l} - \text{ власні частоти}.
    \end{aligned}\right.
\end{equation*}

Запишемо загальний розв'язок задачі:
\begin{equation} \label{gensol}
    u(x,t) = \sum^{\infty}_{n=1} u_n(x,t) = \sum^{\infty}_{n=1} \left[A_n\cos(\omega_n t) + B_n\sin(\omega_n t)\right] \sin(k_n x)
\end{equation}
Коефіцієнти $A_n$ та $B_n$ визначаємо із початкових умов. Підставляємо (\ref{gensol}) в початкові умови (\ref{probcond3}):
\begin{equation} \label{sol-init-cond}
    \begin{aligned}
        &u(x,0) = \varphi(x)
        \;\Rightarrow\;
        \sum^{\infty}_{n=1} A_n\sin(k_n x) = \varphi(x)\\
        &\begin{aligned}
            u_t(x,0) = \psi(x)
            \;&\Rightarrow\\
            \Rightarrow \left(\sum^{\infty}_{n=1}\right.&\left.\left. \left[-A_n\omega_n\sin(\omega_n t) + B_n\omega_n\cos(\omega_n t)\right] \sin(k_n x)\right)\right|_{t=0} =\\
            &= \sum^{\infty}_{n=1} B_n\omega_n\sin(k_n x) = \psi(x)
        \end{aligned}
    \end{aligned}
\end{equation}
Отже, ми отримали дві умови для визначення $A_n$, $B_n$.\\
Далі скористаємося ортогональністю власних функцій задачі Штурма-Ліувілля.
\begin{equation} \label{orth}
    \int_0^l X_n(x) \cdot X_m(x) \,\mathrm{d}x = ||X_n||^2\delta_{n,m},
\end{equation}
де $||X_n||$ -- норма власної функції.

Доможуємо отримані вирази в (\ref{sol-init-cond}) на $m$-ту власну функції $\sin(k_m x)$ та інтегруємо від $0$ до $l$. 
\begin{equation*}
    \begin{aligned}
        \begin{aligned}
            &\int\limits_0^l \varphi(x) \sin(k_m x) \,\mathrm{d}x = \sum^{\infty}_{n=1} A_n \int\limits_0^l \sin(k_n x) \sin(k_m x) \,\mathrm{d}x =\\
            &= \sum^{\infty}_{n=1} A_n \cdot \frac{l}{2} \delta_{n,m} = \frac{A_m l}{2}
            \;\Rightarrow\;
            A_n = \frac{2}{l} \int\limits_0^l \varphi(x) \sin(k_n x) \,\mathrm{d}x =\\
            &= \frac{2}{l} \int\limits_0^l \frac{hx}{l} \sin(k_n x) \,\mathrm{d}x = \frac{2h}{l^2} \left(\left.-\frac{1}{k_n} x \cos(k_n x)\right|_0^l + \int\limits_0^l \frac{\cos(k_n x)}{k_n} \,\mathrm{d}t\right) =\\
            &= \left| k_n l = \frac{\pi n}{l} l = \pi n \Rightarrow \sin(k_n l) = 0,\, \cos(k_n l) = (-1)^n \right| =\\
            &= \frac{2h}{l^2} \left(-\frac{l}{k_n}(-1)^n + \left.\frac{\sin(k_n x)}{k_n^2}\right|_0^l \right) = \frac{2h}{l} \frac{(-1)^{n+1}}{k_n} \equiv A_n
        \end{aligned}\\
        \begin{aligned}
            \int\limits_0^l \psi(x) \sin(k_m x) \,\mathrm{d}x &= \sum^{\infty}_{n=1} B_n\omega_n \cdot \frac{l}{2} \delta_{n,m} = \frac{B_m \omega_m l}{2}
            \;\Rightarrow\\
            \Rightarrow\;
            B_n =&\ \frac{2}{\omega_n l} \int\limits_0^l \psi(x) \sin(k_n x) \,\mathrm{d}x = \frac{2\nu_0}{\omega_n l} \int\limits_0^l \sin(k_n x) \,\mathrm{d}x =\\
            =&\ \left.\frac{2\nu_0}{k_n\omega_n l} \cos(k_n x)\right|_l^0 = \frac{2\nu_0}{l} \frac{1 - (-1)^n}{k_n\omega_n} \equiv B_n
        \end{aligned}
    \end{aligned}
\end{equation*} 
Підставляємо визначені константи в (\ref{gensol})
\begin{equation} \label{sol3}
    u(x,t) = \frac{2}{l}\sum^{\infty}_{n=1} \left[\nu_0 (1 - (-1)^n)\frac{\sin(\omega_n t)}{\omega_n} - h(-1)^n\cos(\omega_n t)\right] \frac{\sin(k_n x)}{k_n}
\end{equation}

Перевіримо періодичність розв'язку. Період коливання визначається за відомою формулою \[T_n = \frac{2\pi}{\omega_n},\] де $n$ - номер власної моди. Підставимо в (\ref{sol3}) $t = t + T_n$
\begin{equation*}
    \begin{aligned}
        u(x,t+T_n) & = \frac{2}{l}\sum^{\infty}_{n=1} \left[\nu_0 (1 - (-1)^n)\frac{\sin(\omega_n t + \omega_n \cdot \frac{2\pi}{\omega_n})}{\omega_n} -\right.\\
        &\qquad\qquad\qquad\qquad\quad\left.- h(-1)^n\cos(\omega_n t + \omega_n \cdot \frac{2\pi}{\omega_n})\right] \frac{\sin(k_n x)}{k_n} =\\
        = \frac{2}{l}\sum^{\infty}_{n=1} & \left[\nu_0 (1 - (-1)^n)\frac{\sin(\omega_n t + 2\pi)}{\omega_n} - h(-1)^n\cos(\omega_n t + 2\pi)\right] \frac{\sin(k_n x)}{k_n} =\\ 
        = \frac{2}{l}\sum^{\infty}_{n=1} & \left[\nu_0 (1 - (-1)^n)\frac{\sin(\omega_n t)}{\omega_n} - h(-1)^n\cos(\omega_n t)\right] \frac{\sin(k_n x)}{k_n} = u(x,t)
    \end{aligned}
\end{equation*} 
Тобто коливання струни буде періодичним.


%\chapter{МЕТОД ЧАСТИННИХ РОЗВ’ЯЗКІВ ТА МЕТОД РОЗКЛАДАННЯ ЗА ВЛАСНИМИ ФУНКЦІЯМИ.}

%\chapter{??}

%\chapter{РІВНЯННЯ ЛАПЛАСА І ПУАССОНА.}

\end{document}