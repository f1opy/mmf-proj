\begin{tikzpicture}
    \begin{axis}
        [width = \textwidth, height = 0.7\textwidth,
         axis x line = center, axis y line = center,
         ylabel = $X(x)$, xlabel = $x$,
         xmin = -19, xmax = 19, ymin = -3, ymax = 3,
         axis line style = thin, xtick = {0}, ytick = {0}]   
        
        \tikzmath{\l = 5; \alp = 2/5;}
        
        \addplot[gray, dashed, samples=50, domain=-20:20, name path=three] coordinates {(0,-3)(0,3)}
        node[anchor=130, pos=0.5] {\footnotesize\textcolor{black}{0}};

        % вертикальні пунктирні лінії для x > 0
        \addplot[gray, dashed, samples=50, domain=-20:20, name path=three] coordinates {(\l,-3)(\l,3)}
        node[anchor=130, pos=0.5] {\footnotesize$\textcolor{black}{l}$};
        \addplot[gray, dashed, samples=50, domain=-20:20, name path=three] coordinates {(2*\l,-3)(2*\l,3)}
        node[anchor=130, pos=0.5] {\footnotesize$\textcolor{black}{2l}$};
        \addplot[gray, dashed, samples=50, domain=-20:20, name path=three] coordinates {(3*\l,-3)(3*\l,3)}
        node[anchor=130, pos=0.5] {\footnotesize$\textcolor{black}{3l}$};
        
        % вертикальні пунктирні лінії для x < 0
        \addplot[gray, dashed, samples=50, domain=-20:20, name path=three] coordinates {(-\l,-3)(-\l,3)}
        node[anchor=100, pos=0.5] {\footnotesize$\textcolor{black}{-l}$};
        \addplot[gray, dashed, samples=50, domain=-20:20, name path=three] coordinates {(-2*\l,-3)(-2*\l,3)}
        node[anchor=100, pos=0.5] {\footnotesize$\textcolor{black}{-2l}$};
        \addplot[gray, dashed, samples=50, domain=-20:20, name path=three] coordinates {(-3*\l,-3)(-3*\l,3)}
        node[anchor=100, pos=0.5] {\footnotesize$\textcolor{black}{-3l}$};
        
        % функція, яку розкладаємо
        \addplot [red, thick, domain=0:0.99*\l, samples=150] {\alp*x};
        % точки закріплення струни
        \node (mark) [draw, red, fill=red, circle, minimum size = 2pt, inner sep=0.5pt] at (axis cs: 0, 0) {};
        \node (mark) [draw, red, fill=red, circle, minimum size = 2pt, inner sep=0.5pt] at (axis cs: \l, 0) {};
        % виколата точка (l,αl)
        \node (mark) [draw, red, circle, minimum size = 2pt, inner sep=0.5pt] at (axis cs: \l, \alp*\l) {};

        % непарне продовження відносно нуля
        \addplot [red, dashed, thick, domain=-0.99*\l:0, samples=150] {\alp*x};
        % виколата точка (-l,-αl)
        \node (mark) [draw, red, circle, minimum size = 2pt, inner sep=0.5pt] at (axis cs: -\l, -\alp*\l) {};
        % ізольована точка 
        \node (mark) [draw, black, fill=black, circle, minimum size = 2pt, inner sep=0.5pt] at (axis cs: -\l, 0) {};

        % непарне продовження відносно точки x = l 
        \addplot [black, thick, dashed, domain=1.01*\l:3*0.998*\l, samples=150] {\alp*(x - 2*\l)};
        % виколоті точки
        \node (mark) [draw, black, circle, minimum size = 2pt, inner sep=0.5pt] at (axis cs: \l, -\alp*\l) {};
        \node (mark) [draw, black, circle, minimum size = 2pt, inner sep=0.5pt] at (axis cs: 3*\l, \alp*\l) {};
        % ізольована точка
        \node (mark) [draw, black, fill=black, circle, minimum size = 2pt, inner sep=0.5pt] at (axis cs: 3*\l, 0) {};
        
        % непарне продовження відносно точки x = -l 
        \addplot [black, thick, dashed, domain=-3*0.998*\l:-0.99*\l, samples=150] {\alp*(x + 2*\l)};
        \node (mark) [draw, black, circle, minimum size = 2pt, inner sep=0.5pt] at (axis cs: -3*\l, -\alp*\l) {};
        \node (mark) [draw, black, fill=black, circle, minimum size = 2pt, inner sep=0.5pt] at (axis cs: -3*\l, 0) {};
        \node (mark) [draw, black, circle, minimum size = 2pt, inner sep=0.5pt] at (axis cs: -\l, \alp*\l) {};
        
        \tikzmath{\k = pi/\l; \A = 2*\alp/\k;}

        \addplot [black, domain=-19:19, samples = 1500] {\A*sin(deg(\k*x))};

    \end{axis}
\end{tikzpicture}